\subsection{Acceptance Test Report}\label{reporting:atr}
The Acceptance Test Report (ATR) provides an overview of the testing efforts
made towards executing the ``Acceptance Test Plan for \progname{}: A
Computational Model of Emotion for Enhancing Non-Player Character Believability
in Games''. The document's content and organization is a combination of IEEE
Std 829-2008 Clause 15: Level Interim Test Report and 16: Level Test
Report~\citep{vvDocIEEE}. The ATR must include:
\begin{enumerate}

    \item Document Revision History

    \item Reference Material (e.g. symbols, acronyms, definitions)

    \item Introduction \\
    Identify who prepared the report, its purpose, and its current status (e.g.
    Draft, Final)
    \begin{enumerate}

        \item Scope \\
        Describe the contents and organization of the ATR and references to
        information captured by automated tools that are not contained in the
        ATR

        \item Relevant Documentation \\
        Provide references to documents necessary for understanding the ATR,
        including the: Master Test Plan (MTP), Acceptance Test Plan (ATP),
        relevant Issue Reports (Section~\ref{reporting:issues}), and any
        additional relevant V \& V documentation; and Software Development
        Artifacts (SDAs) and standards (e.g. \citet{vvDocIEEE})

    \end{enumerate}

    \item Details of the ATR \\
    Preamble is ``This section provides an overview of the test status and
    results, detailed test results, the status of issues related to these
    testing efforts, and any changes made from the Acceptance Test Plan
    (ATP).''. If all testing is complete, append ``...rationale for all
    decisions, and the final conclusions and recommendations.''
    \begin{enumerate}

        \item Overview of Test Status and Results \\
        Provide a summary of the status of planned tests and results of testing
        efforts with references to the ATP, record the version and components
        of tested SDAs and the impact that the testing environment might have
        had on the results

        \item Detailed Test Results \\
        Provide a summary of the testing activities and events, relevant
        metrics collected, variances of test items from their specifications
        and from tests to their documentation with explanations for the
        variances, and an evaluation of the comprehensiveness of testing
        efforts (e.g. coverage metrics)

        \item Status of Related Issues \\
        Provide a summary of all related issues with references to their
        associated report (Section~\ref{reporting:issues}) with a description
        of their desired and actual status:
        \begin{itemize}
            \item Resolved with a summary of their resolution
            \item Unresolved (``Open'' or ``In Process'' with a progress
            summary)
            \item Deferred with explanations to address them
        \end{itemize}

        \clearpage

        \item Changes From Plans \\
        Provide a summary of test activities that have not yet been executed
        with justification and describe any changes to testing activities such
        as tests that will not be done and tests that must be rerun

        \item Decision Rationale \\
        When all testing activities are complete, Describe any issues
        considered during decision-making and justify the conclusions drawn
        from test results

        \item Conclusions and Recommendations \\
        When all testing activities are complete, describe the overall
        evaluation of each test activity and its limitations, recommendations
        concerning their readiness for use and under what circumstances, and
        lessons learned during testing which might include the identification
        of issue categories and/or root cause analysis

    \end{enumerate}

\end{enumerate}