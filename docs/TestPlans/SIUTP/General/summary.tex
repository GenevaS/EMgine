\subsection{Summary of \progname{}'s Purpose and Design Goals}
\progname{} is a Computational Model of Emotion (CME) for Non-Player Characters
(NPCs) to enhance their believability, with the goal of improving long-term
player engagement. \progname{} is for \textit{emotion generation}, accepting
user-defined information from a game environment to determines what emotion and
intensity a NPC is ``experiencing''. How the emotion is expressed and what
other effects it could have on game entities is left for game
designers/developers to decide.

\progname{} aims to provide a feasible and easy-to-use method for game
designers/developers to include emotion in their NPCs, they perceive to be
challenging with the current tools and
restrictions~\citep{broekens2016emotional}. \progname{} should be modular and
portable such that game designers/developers can use it in their regular
development environment, and should not require knowledge of affective science,
psychology, and/or emotion theories. Therefore, it is a library of components
to maximize a game designer/developer's control over how and when \progname{}
functions.