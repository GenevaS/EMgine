\subsection{Testing and Verification Tools}\label{sec:tools}
\progname{}'s development uses the C\# programming language because it is one
of the languages supported in Unity, a well-known game development
platform~\citep{unity3Dcsharp}. The supporting Integrated Development
Environment (IDE), Microsoft Visual Studio (VS), is the default script editor
in Unity. \progname{} development uses VS 2022 (Community Edition), which can
access the following tools:
\begin{itemize}

    \item \textbf{NUnit Unit Testing Framework} \\
    This supports the bulk of the automated testing approach for unit,
    integration, system, and regression testing. The IDE is configured to
    automatically run existing unit tests when it is compiling the code base.
    Unity Testing Framework uses custom integration of NUnit
    3.5~\citep{unity3Dtestingfw}.

    \item \textbf{Moq Library for .NET} \\
    This supports tests that rely on components that do not have a concrete
    implementation, such as the user-defined data types
    %(Section~\ref{sec_sysUserDataTypes}).
    It allows the definition of mocked
    interface calls within unit tests that are type-safe~\citep{moq}.

    \item \textbf{Performance Analysis} \\
    \progname{} uses the performance tools built into VS 2022, which includes
    CPU, memory, and time usage tools~\citep{vs2022perf}.

    \item \textbf{Code Style and Quality Analyzers} \\
    \progname{}'s development uses the official .NET Compiler Platform
    (Roslyn)~\citep{roslyn} and the third-party Roslynator~\citep{roslynator}
    analyzers to help adhere to good code quality and style practices. The
    Unity documentation also references Roslyn analyzers for code style and
    quality~\citep{unity3Droslyn}.

\end{itemize}