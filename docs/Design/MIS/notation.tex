\section{Notation}

The module structure in this MIS comes from \citet{HoffmanAndStrooper1995},
with the addition that template modules have been adapted from
\cite{ghezzifundamentals2003}.  The mathematical notation comes from Chapter 3
of \citet{HoffmanAndStrooper1995}.  For instance, the symbol := represents a
multiple assignment statement and conditional rules follow the form $(c_1
\Rightarrow r_1 \, | \, c_2 \Rightarrow r_2 \, | \, ... \, | \, c_n \Rightarrow
r_n )$.

The following table summarizes the primitive data types used by \progname.

\begin{center}
    \renewcommand{\arraystretch}{1.2}
    \noindent
    \begin{tabular}{l c p{0.5\linewidth}}
        \toprule
        \textbf{Data Type} & \textbf{Notation} & \textbf{Description}\\
        \midrule

        \rowcolor[gray]{0.9}Boolean & $\mathbb{B}$ & An element in the set of
        $\{ \True, \False \}$ \\

        Character & char & A single symbol or digit \\

        \rowcolor[gray]{0.9}Integer & $\mathbb{Z}$ & A number without a
        fractional component in (-$\infty$, $\infty$) \\

        Natural & $\mathbb{N}$ & A number without a fractional component in [0,
        $\infty$) \\

        \rowcolor[gray]{0.9}Real & $\mathbb{R}$ & Any number in (-$\infty$,
        $\infty$)\\

        \bottomrule
    \end{tabular}
\end{center}

\noindent
The specification of \progname{} uses some derived data types:
\begin{itemize}

    \item \textit{Sets}: Unordered lists filled with elements of the same data
    type

    \item \textit{Sequences}: Ordered lists filled with elements of the same
    data type

    \item \textit{Strings}: Sequences of characters

    \item \textit{Tuples}: Contain a list of labelled values, potentially of
    different data types

\end{itemize}

\progname{} also uses functions, defined by the data types of their inputs and
outputs. The MIS describes local functions by giving their type signature,
followed by their specification.