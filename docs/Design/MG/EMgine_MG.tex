\documentclass[11pt, titlepage]{article}

\usepackage[letterpaper, hmargin=1in, bmargin=1in, tmargin=1.3in,
headheight=26pt]{geometry}

\usepackage{graphicx}
\usepackage{hyperref}
\usepackage{array}
\newcolumntype{C}[1]{>{\centering\arraybackslash}m{#1}}
\newcolumntype{N}[1]{>{\centering\arraybackslash}p{#1}}
\usepackage{longtable}
\usepackage{multirow}
\usepackage{tabularx}
\usepackage{float}
\usepackage{booktabs}
\usepackage[usenames,dvipsnames,table]{xcolor}
\usepackage{txfonts}
\usepackage[shortcuts]{extdash}

\usepackage[nottoc,notlof,notlot,numbib]{tocbibind}

\usepackage{natbib}

\usepackage[page]{totalcount}

\usepackage{fancyhdr}
\fancyhf{}
\lhead{Module Guide \\ \progname{}}
\rhead{Geneva M. Smith \\ Dept. of Computing and Software---McMaster University}
\lfoot{Ver.~\ref{current_version_MG}}
\rfoot{\thepage\ of \thepagesMG}

\renewcommand{\footrule}{\vbox to 0pt
    {\makebox[\textwidth]{\hrulefill}\vss}}

\hypersetup{
    bookmarks=true,
    colorlinks=true,
    linktoc=all,
    linkcolor=Purple,
    citecolor=ForestGreen,
    filecolor=WildStrawberry,
    urlcolor=Cerulean
}

\newcommand{\citepg}[2]{\citeauthor{#1}, \citeyear{#1}, p.~#2}

\makeatletter
\newcommand\newref[1]{#1\def\@currentlabel{#1}}
\makeatother

%% Comments

\usepackage{color}

\newif\ifcomments\commentstrue

\ifcomments
\newcommand{\authornote}[3]{\textcolor{#1}{[#3 ---#2]}}
\newcommand{\todo}[1]{\textcolor{red}{[TODO: #1]}}
\else
\newcommand{\authornote}[3]{}
\newcommand{\todo}[1]{}
\fi

\newcommand{\wss}[1]{\authornote{blue}{SS}{#1}}
\newcommand{\wgs}[1]{\authornote{teal}{GS}{#1}}
\newcommand{\wjc}[1]{\authornote{purple}{JC}{#1}}

\usepackage{xr}
\externaldocument{../../SRS/EMgine_SRS}
\newcommand{\rref}[1]{SRS-R\ref{#1}}
\newcommand{\nfref}[1]{SRS-NF\ref{#1}}
\newcommand{\tyref}[1]{SRS-TY\ref{#1}}
\newcommand{\iref}[1]{SRS-IM\ref{#1}}
\newcommand{\lcref}[1]{SRS-LC\ref{#1}}

\newcounter{acnum}
\newcommand{\actheacnum}{AC\theacnum}
\newcommand{\acref}[1]{AC\ref{#1}}

\newcounter{ucnum}
\newcommand{\uctheucnum}{UC\theucnum}
\newcommand{\uref}[1]{UC\ref{#1}}

\newcounter{mnum}
\newcommand{\mthemnum}{M\themnum}
\newcommand{\mref}[1]{M\ref{#1}}

\newcommand{\progname}{EMgine} % PUT YOUR PROGRAM NAME HERE

\newcommand{\conceptVersion}{1.0}
\newcommand{\srsVersion}{1.5}
\newcommand{\mgversion}{1.5}
\newcommand{\misversion}{0.8}
\newcommand{\codeversion}{0.8}
\newcommand{\mastertestplanVersion}{1.0}
\newcommand{\verifytestplanVersion}{1.0}
\newcommand{\validatetestplanVersion}{1.0}
\newcommand{\manualversion}{?}

\newcommand{\timetype}{\mathbb{T}}
\newcommand{\deltatimetype}{\mathbb{T}_{\Delta}}
\newcommand{\energytype}{\mathbb{J}}
\newcommand{\energychangetype}{\mathbb{J}_{\Delta}}
\newcommand{\stringtype}{\text{String}}
\newcommand{\emotionintensitytype}{\mathbb{I}}
\newcommand{\responsestrength}{\mathbb{I}_{\Delta}}
\newcommand{\emotionstatetype}{\mathbb{ES}}
\newcommand{\emotionstatedecaytype}{\mathbb{ES}_\lambda}
\newcommand{\emotiontype}{\mathbb{E}}
\newcommand{\emotionkindstype}{\mathbb{K}}
\newcommand{\emotionequilibriumtype}{\mathbb{ES_{\rightleftharpoons}}}
\newcommand{\emotiondecaytype}{\mathbb{I_{\lambda}}}
\newcommand{\egotype}{\mathbb{O}}
\newcommand{\egoidentitytype}{\mathbb{ID}}
\newcommand{\goaltype}{\mathbb{G}}
\newcommand{\plantype}{\mathbb{P}}
\newcommand{\padpoint}{P_{\left(P,A,D\right)}}
\newcommand{\goallabeltype}{\mathbb{L}}
\newcommand{\worldtype}{\mathbb{W}}
\newcommand{\worldstatetype}{\mathbb{S}}
\newcommand{\worldstatechangetype}{\mathbb{S}_{\Delta}}
\newcommand{\statedistancetype}{\mathbb{D}}
\newcommand{\statedistancechangetype}{\mathbb{D}_{\Delta}}
\newcommand{\indexsettype}{\mathbb{IS}}
\newcommand{\goalegotype}{\mathbb{GE}}
\newcommand{\socialrelationtype}{{R^{S}}}
\newcommand{\socialattachmenttype}{\mathbb{SA}}
\newcommand{\actiontype}{\mathbb{AC}}
\newcommand{\actortype}{\mathbb{A}}
\newcommand{\attentiontype}{\mathbb{AT}_x}
\newcommand{\probabilitytype}{\left[0,1\right]}
\newcommand{\maxval}{\mathsf{MAX}}
\newcommand{\True}{\mathit{True}}
\newcommand{\False}{\mathit{False}}
\newcommand{\defEq}{\text{ } \mathlarger{\mathlarger\circeq} \text{ }}

\newcommand{\mOtherChange}{\mathit{otherChange}}
\newcommand{\mActor}{\mathit{actor}}
\newcommand{\mDeliberate}{\mathit{deliberate}}
\newcommand{\mImportance}{\mathit{importance}}
\newcommand{\mFear}{\mathtt{Fear}}
\newcommand{\mAnger}{\mathtt{Anger}}
\newcommand{\mSadness}{\mathtt{Sadness}}
\newcommand{\mJoy}{\mathtt{Joy}}
\newcommand{\mInterest}{\mathtt{Interest}}
\newcommand{\mSurprise}{\mathtt{Surprise}}
\newcommand{\mDisgust}{\mathtt{Disgust}}
\newcommand{\mTrust}{\mathtt{Acceptance}}
\newcommand{\mStore}{\mathit{Store}}
\newcommand{\mControl}{\mathit{Control}}
\newcommand{\mGenerate}{\mathit{Generate}}
\newcommand{\mDecay}{\mathit{Decay}}
\newcommand{\mAppraisal}{\mathit{Appraisal}}


\usepackage[shortlabels, inline]{enumitem}

\begin{document}

    \newcounter{pagesMG}
    \setcounter{pagesMG}{\totalpages}
    % Subtracts one from the total page count so that it does not include the
    % title page
    %\addtocounter{pagesMG}{-1}

    \begin{titlepage}
        \thispagestyle{empty}

        \title{Module Guide for \progname{}: A Computational Model of Emotion
        for
        Enhancing Non-Player Character Believability in Games}
        \author{Geneva M. Smith}
        \date{October 15, 2022}

        \maketitle
    \end{titlepage}

    \pagestyle{fancy}

    \vspace*{\fill}
    \section*{Revision History}
    \begin{center}
        \begin{tabular}{m{0.16\linewidth}C{0.15\linewidth}m{0.59\linewidth}}
            \toprule {\bf Date} & {\bf Version} & {\bf Notes}\\
            \midrule

            \vspace*{1mm}October 15, 2022 &
            \vspace*{1mm}\newref{1.0}\label{current_version_MG} & \vspace*{6mm}
            \begin{itemize}[noitemsep, nosep, leftmargin=*]
                \item Completed First Version based on SRS Version~1.0
            \end{itemize} \\ \midrule

            \vspace*{1mm}July 21, 2022 & \vspace*{1mm}0.8 & \vspace*{6mm}
            \begin{itemize}[noitemsep, nosep, leftmargin=*]
                \item Module decomposition and definition based on SRS
                Version~0.5
                \item Missing references to Non-Functional Requirements in
                Section~\ref{SecConnection}
            \end{itemize} \\
            \bottomrule
        \end{tabular}
    \end{center}
    \vspace*{\fill}

    \clearpage

    \tableofcontents

    \clearpage

    \listoftables

    \listoffigures

    \clearpage

    \section{Reference Material}

This section records the symbols, abbreviations, and acronyms that appear in
this document for easy reference.

\subsection*{Table of Symbols}

The table that follows summarizes the symbols used in this document along with
their units, listed in alphabetical order. These correspond to the symbols in
\progname{}'s Software Requirement Specification (SRS) at
\href{https://github.com/GenevaS/EMgine/blob/main/docs/SRS/EMgine_SRS.pdf}{https://github.com/GenevaS/EMgine/blob/main/docs\newline
    /SRS/EMgine\_SRS.pdf}

\noindent \begin{center}
    \renewcommand{\arraystretch}{1.2}
    \begin{tabular}{C{0.1\textwidth} C{0.2\textwidth}
            m{0.6\textwidth}} \toprule
        \textbf{Symbol} & \textbf{Type} & \textbf{Description}\\
        \midrule

        \colourRow$\attentiontype$ & \tyref{TY_Attention} & The time elapsed
        while focusing on $x$ \\

        $\statedistancetype$ & \tyref{TY_DistanceBetweenWorldStates} & The
        difference between two game world states \\

        \colourRow$\statedistancechangetype$ &
        \tyref{TY_DistanceBetweenWorldStatesChange} & The change that a game
        world event makes to a game world state \\

        $\emotiontype$ & \tyref{TY_Emotion} & Emotion states over
        time \\

        \colourRow$\emotionstatetype$ & \tyref{TY_EmotionState} & An emotion
        state \\

        $\emotionstatedecaytype$ & \tyref{TY_EmotionDecayState} & An emotion
        decay state \\

        \colourRow$\goaltype$ & \tyref{TY_Goal} & An entity's goal \\

        $\emotionintensitytype$ & \tyref{TY_EmotionIntensity} & Emotion
        intensity \\

        \colourRow$\responsestrength$ & \tyref{TY_DeltaIntensity} & A change in
        emotion intensity \\

        $\emotiondecaytype$ & \tyref{TY_EmotionDecay} & Emotion decay constant
        \\

        \colourRow$\emotionkindstype$ & \tyref{TY_EmotionKind} & Allowable
        emotion labels/types \\

        $\plantype$ & \tyref{TY_Plan} & An entity's plan \\

        \colourRow$\padpoint$ & \tyref{TY_PAD} & A point in PAD space \\

        $\worldstatetype$ & \tyref{TY_WorldState} & A game world state \\

        \colourRow$\worldstatechangetype$ & \tyref{TY_WorldStateChange} & A
        game action that changes a game world state \\

        $\socialattachmenttype$ & \tyref{TY_Relation-CTE} & A social attachment
        to entity $A$ \\

        \colourRow$\timetype$ & \tyref{TY_Time} & Abstract, linearly ordered
        time \\

        $\deltatimetype$ & \tyref{TY_Time} & The difference between two times \\

        \bottomrule
    \end{tabular}

\end{center}

\subsection*{Abbreviations and Acronyms}

\begin{center}

    \renewcommand{\arraystretch}{1.2}
    \begin{tabular}{C{0.1\textwidth} m{0.7\textwidth}}
        \toprule
        \textbf{Text} & \textbf{Description} \\
        \midrule

        \colourRow AC & Anticipated Change \\

        API & Application Programming Interface \\

        \colourRow CP & Component \\

        DAG & Directed Acyclic Graph \\

        \colourRow M & Module \\

        MG & Module Guide \\

        \colourRow NPC & Non-Player Character (Games) \\

        OS & Operating System \\

        \colourRow SRS-R & Requirement from the Software Requirements
        Specification \\

        SC & Scientific Computing \\

        \colourRow SRS & Software Requirements Specification \\

        SRS-LC & Likely Change from the Software Requirements Specification \\

        \colourRow UC & Unlikely Change \\

        \bottomrule
    \end{tabular}

\end{center}

    \clearpage

    \section{Introduction}
This document details the Module Interface Specifications for \progname{}, a
Computational Model of Emotion (CME) for Non-Player Characters (NPCs) to
enhance their believability, with the goal of improving long-term player
engagement. \progname{} is for \textit{emotion generation}, accepting
user-defined information from a game environment to determines what emotion
and intensity a NPC is ``experiencing''. How the emotion is expressed and what
other effects it could have on game entities is left for game
designers/developers to decide.

Other necessary documents include the Software Requirement Specification and
Module Guide. You can find the full documentation and implementation at
\begin{center}
    \href{https://github.com/GenevaS/EMgine}{https://github.com/GenevaS/EMgine}
\end{center}


    \clearpage

    \section{Anticipated and Unlikely Changes}\label{SecChange}
Possible changes are classified into two categories based on likeliness:
Anticipated (Section~\ref{SecAchange}) and Unlikely (Section~\ref{SecUchange}).

\subsection{Anticipated Changes}\label{SecAchange}
Anticipated changes are decisions and information that are to be hidden
inside modules. Ideally, changing one of the anticipated changes will only
require changing the one module that hides the associated decision. This is an
adaptation of the \textit{design for change} approach to software design. Where
applicable, associated Likely Changes [SRS-LC] from the SRS (Ver.~\srsVersion)
appear in square brackets.

\begin{description}

    \item[\refstepcounter{acnum} \actheacnum \label{acEmotionIntensityMin}:]
    The minimum value of emotion intensity [\lcref{LC_PositiveIntensity}].

    \item[\refstepcounter{acnum} \actheacnum \label{acIntensityAlgo}:] The
    algorithm/method for calculating emotion intensity
    [\lcref{LC_Goal2Intensity}].

    \item[\refstepcounter{acnum} \actheacnum \label{acUpdateIntensityAlgo}:]
    The algorithm/method for updating emotion intensity with an intensity
    change.

    \item[\refstepcounter{acnum} \actheacnum \label{acIntensityEvalAlgo}:]
    The algorithm/method for calculating emotion intensity
    [\lcref{LC_Interest}]. Although each of \progname{}'s supported emotion
    kinds have their own intensity model, end-users do not need to know that
    they are calculated differently. They also collectively share the same
    secret: their method for evaluating emotion intensity. Therefore, these are
    associated with one AC.

    \item[\refstepcounter{acnum} \actheacnum \label{acElicitAcceptanceAlgo}:]
    The algorithm/method for calculating intensity of \textit{Acceptance}.

    \item[\refstepcounter{acnum} \actheacnum \label{acEmotionKindType}:] The
    implementation of emotion kinds data structure [\lcref{LC_EmotionPairs}].

    \item[\refstepcounter{acnum} \actheacnum \label{acAppraisalAlgo}:] The
    algorithm/method for generating emotion kinds [\lcref{LC_Subgoal},
    \lcref{LC_Goal2Emotion}, \lcref{LC_EmotionTypeIntensity}]. Although each of
    \progname{}'s supported emotion kinds have their own elicitation model,
    end-users do not need to know that they are calculated differently. They
    also collectively share the same secret: the method for emotion
    elicitation. Therefore, these are gathered into one module.

    \item[\refstepcounter{acnum} \actheacnum \label{acEmotionStateImpl}:] The
    implementation of emotion state data structure.

    \item[\refstepcounter{acnum} \actheacnum \label{acEmotionDecayStateImpl}:]
    The implementation of emotion state decay data structure.

    \item[\refstepcounter{acnum} \actheacnum \label{acDecayIntensityAlgo}:] The
    algorithm/method for decaying emotion intensity [\lcref{LC_DecaySpeed},
    \lcref{LC_Equilibrium}, \lcref{LC_DecayRate}].

    \item[\refstepcounter{acnum} \actheacnum \label{acDecayStateAlgo}:] The
    algorithm/method for decaying an emotion state.

    \item[\refstepcounter{acnum} \actheacnum \label{acEmotionState}:] The
    implementation of emotion data structure.

    \item[\refstepcounter{acnum} \actheacnum \label{acPADPointType}:] The
    implementation of PAD point data structure.

    \item[\refstepcounter{acnum} \actheacnum \label{acState2PADAlgo}:] The
    algorithm/method for converting an emotion state to a PAD point
    [\lcref{LC_EmotionTerms}, \lcref{LC_PADStats}].

    \item[\refstepcounter{acnum} \actheacnum \label{acGoalImpl}:] The
    implementation of goal data structure.

    \item[\refstepcounter{acnum} \actheacnum \label{acPlanImpl}:] The
    implementation of plan data structure.

    \item[\refstepcounter{acnum} \actheacnum \label{acAttachmentImpl}:] The
    definition and/or implementation of social attachment data structure.

    \item[\refstepcounter{acnum} \actheacnum \label{acAttentionImpl}:] The
    definition and/or implementation of attention data structure.

    \item[\refstepcounter{acnum} \actheacnum \label{acTimeImpl}:] The
    implementation of time and, consequently, delta time.

    \item[\refstepcounter{acnum} \actheacnum \label{acWorldStateImpl}:] The
    implementation of World State View (WSV) data structure.

    \item[\refstepcounter{acnum} \actheacnum \label{acWorldStateChangeImpl}:]
    The implementation of world event data structure.

    \item[\refstepcounter{acnum} \actheacnum \label{acDistanceImpl}:] The
    implementation of distance between WSVs data structure.

    \item[\refstepcounter{acnum} \actheacnum \label{acDistanceChangeImpl}:] The
    implementation of change in distance between WSVs data structure.

\end{description}

\subsection{Unlikely Changes}\label{SecUchange}
Module design should be as general as possible, but a general system is
typically more complex. Sometimes this complexity is not necessary and fixing
some design decisions at the system architecture stage can simplify the
software design. These decisions are \textit{unlikely changes} because they are
not intended to be changed due to the potentially large number of design
elements that need to be modified to accommodate them.

\progname{} does not have any known unlikely changes, as its goal is for users
to choose, organize, modify, and reuse its modules as they see fit. These needs
also motivate its conception as a library of components (see
Section~\ref{SecConnection} for reasoning).

    \clearpage

    \section{Module Hierarchy}\label{SecMH}
This section provides an overview of the module design. Modules are summarized
in a hierarchy decomposed by secrets (Table \ref{TblMH}).The modules listed
below, which are leaves in the hierarchy tree, are the modules that will
actually be implemented.

\begin{description}

    \item [\refstepcounter{mnum} \mthemnum \label{mIntensity}:] Emotion
    Intensity Type Module

    \item [\refstepcounter{mnum} \mthemnum \label{mIntensityFun}:] Emotion
    Intensity Function Module

    \item [\refstepcounter{mnum} \mthemnum \label{mStateType}:] Emotion State
    Type Module

    \item [\refstepcounter{mnum} \mthemnum \label{mGenerate}:] Emotion
    Generation Module

    \item [\refstepcounter{mnum} \mthemnum \label{mDecay}:] Emotion Intensity
    Decay Rate Type Module

    \item [\refstepcounter{mnum} \mthemnum \label{mDecayState}:] Emotion
    Intensity Decay State Type Module

    \item [\refstepcounter{mnum} \mthemnum \label{mDecayFun}:] Emotion Decay
    Function Module

    \item [\refstepcounter{mnum} \mthemnum \label{mPADType}:] PAD Type Module

    \item [\refstepcounter{mnum} \mthemnum \label{mPADFun}:] PAD Function Module

    \item [\refstepcounter{mnum} \mthemnum \label{mEmotionType}:] Emotion Type
    Module

    \item [\refstepcounter{mnum} \mthemnum \label{mEmotionFun}:] Emotion
    Function Module

    \item [\refstepcounter{mnum} \mthemnum \label{mGoal}:] Goal Module

    \item [\refstepcounter{mnum} \mthemnum \label{mPlan}:] Plan Module

    \item [\refstepcounter{mnum} \mthemnum \label{mAttention}:] Attention Module

    \item [\refstepcounter{mnum} \mthemnum \label{mSocial}:] Social Attachment
    Module

    \item [\refstepcounter{mnum} \mthemnum \label{mTime}:] Time Module

    \item [\refstepcounter{mnum} \mthemnum \label{mWorld}:] World State Module

\end{description}

\begin{table}[h!]
    \centering
    \small
    \renewcommand{\arraystretch}{1.2}
    \begin{tabular}{p{0.16\linewidth} p{0.3\linewidth} p{0.44\linewidth}}
        \toprule
        \textbf{Level 1} & \textbf{Level 2} & \textbf{Level 3} \\
        \midrule

        \multirow{14}{\linewidth}{Behaviour\-/Hiding Module} &
        \cellcolor[gray]{0.9} & \cellcolor[gray]{0.9}[\mref{mIntensity}]
        Emotion Intensity Type Module \\
        & \cellcolor[gray]{0.9}\multirow{-2}{\linewidth}{Emotion Intensity
        Module} & \cellcolor[gray]{0.9}[\mref{mIntensityFun}] Emotion Intensity
        Function Module \\

        &  & [\mref{mStateType}] Emotion State Type Module \\
        &  & [\mref{mGenerate}] Emotion Generation Module \\
        &  & [\mref{mDecay}] Emotion Intensity Decay Rate Type Module \\
        &  & [\mref{mDecayState}] Emotion Intensity Decay State Type Module \\
        & \multirow{-5}{\linewidth}{Emotion State Module} &
        [\mref{mDecayFun}] Emotion Decay Function Module \\

        & \cellcolor[gray]{0.9} & \cellcolor[gray]{0.9}[\mref{mPADType}] PAD
        Type Module \\
        & \cellcolor[gray]{0.9}\multirow{-2}{\linewidth}{PAD Module} &
        \cellcolor[gray]{0.9}[\mref{mPADFun}] PAD Function Module \\

        &  & [\mref{mEmotionType}] Emotion Type Module \\
        & \multirow{-2}{\linewidth}{Emotion Module} & [\mref{mEmotionFun}]
        Emotion Function Module \\

        & \cellcolor[gray]{0.9}[\mref{mGoal}] Goal Module &
        \cellcolor[gray]{0.9}-- \\

        & [\mref{mPlan}] Plan Module & -- \\

        & \cellcolor[gray]{0.9}[\mref{mAttention}] Attention Module &
        \cellcolor[gray]{0.9}-- \\

        & [\mref{mSocial}] Social Attachment Module & -- \\

        \midrule

        \multirow{2}{\linewidth}{Software Decision Module} &
        \cellcolor[gray]{0.9}[\mref{mTime}] Time Module &
        \cellcolor[gray]{0.9}-- \\

        & [\mref{mWorld}] World State Module & -- \\

        \bottomrule
    \end{tabular}
    \caption{Module Hierarchy}
    \label{TblMH}
\end{table}

    \clearpage

    \section{Connection Between Requirements and Design} \label{SecConnection}

The design of the system is intended to satisfy the requirements developed in
the SRS (Version~\srsVersion). In this stage, the system is decomposed into
modules. Table~\ref{TblRT} lists the connection between requirements and
modules.

\paragraph{Library of Components}
This MG treats \progname{} as a library of components due to its need to be an 
``understandable black box'' design. \progname{} must not require users to have 
an understanding of affective science and/or emotion research 
(\nfref{N_Knowledge}) and have reasonable compatibility with different agent 
architectures/frameworks (\nfref{N_Arch}), game entity embodiment 
(\nfref{N_Embody}), and development environments (\nfref{N_Env}). Together with 
its requirement to clearly document usage information for user-facing 
(\nfref{N_CodeDoc}), \progname{} already lends itself to a black box as it 
focuses on the relation between inputs and outputs with no claims about the 
underlying process~\citep[p.~601]{wehrle1995potential}. This aligns with the 
description of domain-specific CMEs, where the transformation of inputs into 
affective phenomena is not important, as long as it has the desired effects 
(\citepg{hudlicka2019modeling}{130--131}; 
\citepg{osuna2020seperspective}{4--6}). However, it would be difficult to 
support the Verifiability requirements (\nfref{N_Atomic}, \nfref{N_Trace}) if 
one could not see how \progname{}'s parts passed information to each other. 
Therefore, \progname{}'s components should be black boxes, but not their 
connections so that its overall behaviour can be 
explained~\citep[p.~20]{guimaraes2022fatima}.

This implies that a component-based software 
architecture~\citep[p.~248--261]{qian2010software} is best, where each step in 
emotion generation is a component. This would:
\begin{itemize}
    \item Increase \progname{}'s portability by specifying what each component 
    is guaranteed to provide~\citep[p.~254]{qian2010software} so that designers 
    can use existing, tested and validated systems and enables fast and easy 
    integration of new components~\citep[p.~443]{rodriguez2015computational}, 
    increasing its compatibility with agent architectures/frameworks 
    (\nfref{N_Arch}), game entity embodiment (\nfref{N_Embody}), and 
    development environments (\nfref{N_Env})

    \item Allow designers to call \progname{}'s components as they see fit,     
    supporting customization and modifiability (\nfref{N_Mod}, 
    \nfref{N_ChooseEm}), efficiency (\nfref{N_Complex}, 
    \nfref{N_Efficient})~\citep[p.~466]{carbone2020radically}, enabling users 
    to specify how to use (\nfref{N_Output}) and verify (\nfref{N_Atomic}, 
    \nfref{N_Trace}) \progname{} outputs, and---potentially---demonstrating how 
    \progname{} could improve the player experience (\nfref{N_PX}) and/or 
    create novel game experiences (\nfref{N_Novel})

    \item Have the potential for developing authoring tools that interface with 
    and manage \progname{} components, which could reduce total authorial 
    effort (\nfref{N_AuthorialEffort})
\end{itemize}

Ongoing work on the FAtiMA architecture and its descendants, the FAtiMA Modular 
framework and FAtiMA Toolkit, found this approach 
successful~\citep[p.~8:2, 8:12--8:13]{mascarenhas2022fatima}. After gaining 
feedback from people in the games industry\footnote{As part of the ``Realising 
an Applied Gaming Eco-system'' (RAGE) project 
(\url{https://cordis.europa.eu/project/id/644187}).}, the FAtiMA Toolkit's 
designers realized it as a library of components so that its parts can work 
autonomously. A library-based approach also alleviates some of \progname{}'s 
design requirements, as it no long has to ``...be generic enough to encompass 
all possible forms of a perception-action cycle in an 
agent.''~\citep[p.~8:12]{mascarenhas2022fatima}. This also increased the 
Toolkit's chances of adoption, as it allows game designers to use it in their 
existing systems and/or frameworks and avoiding the complexity and 
accessibility issues of other agent 
architectures~\citep[p.~3]{guimaraes2022fatima}. In the same vein, \progname{} 
would be a \textit{library of components} so it can take advantage of these 
benefits. Users would also be free to focus on what sequence of components for 
emotion processing works for their needs because no one really knows what order 
they truly run in~\citep[p.~142]{moffat1997personality}.

\paragraph{Pre-Built Components}
A library of components creates opportunities for \progname{} to address
additional nonfunctional requirements by pre-building some compound components,
such as a default \textit{``engine''}. 

While each emotion generation component would be a black 
box~\citep[p.~253]{qian2010software}, a user might not know when they should 
use a component---or if it is even necessary, which might require some 
knowledge of affective science and/or emotion research to bridge (violates 
\nfref{N_Knowledge}). A component-based software architecture supports this 
need too by pre-building some compound components. It could come with an 
``engine''---a pre-built component that is itself a ``system'' of 
components~\citep[p.~249]{qian2010software} that minimizes the necessary inputs 
for emotion generation. It would accept data and returns an emotion state, 
without the designer knowing how it 
works~\citep[p.~443]{rodriguez2015computational}. These would also serve as 
usage examples (\nfref{N_CodeDoc}, \nfref{N_Manual}, \nfref{N_Knowledge}) 
and/or as parts for automating tasks (\nfref{N_Auto}).

In this form, \progname{} would be similar to GAMYGDALA which compares itself 
to a physics engine~\citep[p.~32]{popescu2014gamygdala}. Designers have applied 
GAMYGDALA to: an arcade and puzzle game~\citep{broekens2015emotion}; a 
narrative generation framework to drive character 
emotions~\citep{kaptein2015affective}; implement affective decision-making in 
fighting game characters~\citep{yuda2019creating}; and modify it with the fuzzy 
logic, classical conditioning, and learning from another 
CME\footnote{FLAME~\citep{el2000flame}}~\citep[p.~4]{code2015learning}. The 
breadth of games that GAMYGDALA appears in suggests that, even pre-packaged as 
a large component, \progname{} could also find success.

\paragraph{Separating Data Types and Functions}
\progname{}'s module decomposition separates data types (\mref{mIntensity},
\mref{mDecay}, \mref{mDecayState}, \mref{mPADType}, \mref{mEmotionType}) from
functions that use them (\mref{mIntensityFun}, \mref{mDecayFun},
\mref{mPADFun}, \mref{mEmotionFun}). While this increases the total modules, it
also improves an end-user's ability to adopt individual parts of \progname{} as
needed (\nfref{N_Efficient}, \nfref{N_Reuse}, \nfref{N_Mod}) and change the
underlying theories (\nfref{N_Evolve}), and ensures that theory-agnostic
components (e.g. \mref{mStateType}) are separated from theory-specific ones
(e.g. \mref{mGenerate}).

\paragraph{User-Defined APIs}
\progname{} also encapsulates each Application Programming Interface (API) that
end-users must define in their own modules (\mref{mTime}, \mref{mWorld},
\mref{mAttention}, \mref{mSocial}). This allows a team to divide their
implementation between members, reduces concerns regarding compatibility with
othe \progname{} components (\nfref{N_Mod}) and allows end-users to test and
optimize them individually (\nfref{N_Efficient}, \nfref{N_Trace}).

    \clearpage

    \section{Module Decomposition}\label{SecMD}
Modules are decomposed according to the principle of ``information hiding''
proposed by \citet{ParnasEtAl1984}. Each module documents:
\begin{itemize}

    \item Its \emph{Secrets}, a brief statement of the design decision hidden 
    by the module, 

    \item Its \emph{Services}, specifying \emph{what} the module will do 
    without documenting \emph{how} to do it.

    \item A suggestion for what the module should be \emph{Implemented By}. If 
    this is \emph{OS}, it means that the operating system or a standard 
    programming language library provides the module. The system being designed 
    does not need to provide them. If a there is a dash (\emph{--}), the module 
    is \textit{not} a leaf node in the module hierarchy (Section~\ref{SecMH}). 
    It is a ``virtual'' module, and will not have to be implemented unless 
    required by a programming language.

\end{itemize}
%Also indicate if the module will be implemented specifically for the software.

\subsection{Behaviour-Hiding Module}

\begin{description}[font=\scshape]
    \item[Secrets:]The contents of the required behaviours.

    \item[Services:] Includes programs that provide externally visible behaviour
    of the system as specified in the SRS (Ver.~\srsVersion). The programs in
    this module will need to change if the SRS changes.

    \item[Implemented By:] --
\end{description}

\subsubsection{Emotion Intensity Module}

\begin{description}[font=\scshape]
    \item[Secrets:] ``Virtual'' module containing the emotion intensity modules.

    \item[Services:] Using Emotion Intensity data type, separated by data
    structure (\mref{mIntensity}) and methods (\mref{mIntensityFun})

    \item[Implemented By:] --
\end{description}

\subsubsection{Emotion Intensity Type Module (\mref{mIntensity})}

\begin{description}[font=\scshape]
    \item[Secrets:] The internal representation of the Emotion Intensity data
    type.

    \item[Services:] Stores the Emotion Intensity data type, enforces its
    constraints, and provides methods for using it.

    \item[Implemented By:] \progname{}
\end{description}

\subsubsection{Emotion Intensity Function Module (\mref{mIntensityFun})}

\begin{description}[font=\scshape]
    \item[Secrets:] The method for calculating emotion intensity.

    \item[Services:] Evaluates emotion intensity using Goal data, and optional
    Attention and Social Attachment data. Returns the result as Emotion
    Intensity data.

    \item[Implemented By:] \progname{}
\end{description}

\subsubsection{Emotion State Module}

\begin{description}[font=\scshape]
    \item[Secrets:] ``Virtual'' module containing the emotion state, emotion
    decay, and PAD modules.

    \item[Services:] Using emotion state data type and associated emotion decay
    and PAD data types, separated by data structures (\mref{mStateType},
    \mref{mDecay}, \mref{mPADType}) and methods (\mref{mGenerate},
    \mref{mDecayFun}, \mref{mPADFun}).

    \item[Implemented By:] --
\end{description}

\subsubsection{Emotion State Type Module (\mref{mStateType})}

\begin{description}[font=\scshape]
    \item[Secrets:] The internal representation of the Emotion State data type.

    \item[Services:] Stores the Emotion State data type, enforces its
    constraints, and provides methods for using it.

    \item[Implemented By:] \progname{}
\end{description}

\subsubsection{Emotion Generation Module (\mref{mGenerate})}

\begin{description}[font=\scshape]
    \item[Secrets:] The method for calculating which kind of emotion the
    current world state elicits.

    \item[Services:] Evaluates what emotion an entity is ``experiences'' using
    Goal, Plan, and World State data. Returns the result as Emotion Kind data.

    \item[Implemented By:] \progname{}
\end{description}

\subsubsection{Emotion Intensity Decay Rate Type Module (\mref{mDecay})}

\begin{description}[font=\scshape]
    \item[Secrets:] The internal representation of the Emotion Intensity Decay
    Rate data type.

    \item[Services:] Stores the Emotion Intensity Decay Rate data type,
    enforces its constraints, and provides methods for using it.

    \item[Implemented By:] \progname{}
\end{description}

\subsubsection{Emotion Intensity Decay State Type Module (\mref{mDecayState})}

\begin{description}[font=\scshape]
    \item[Secrets:] The internal representation of the Emotion Intensity Decay
    State data type.

    \item[Services:] Stores the Emotion Intensity Decay State data type,
    enforces its constraints, and provides methods for using it.

    \item[Implemented By:] \progname{}
\end{description}

\subsubsection{Emotion Decay Function Module (\mref{mDecayFun})}

\begin{description}[font=\scshape]
    \item[Secrets:] The method for calculating the temporal decay of emotion
    intensity.

    \item[Services:] Evaluates emotion intensity using an Emotion Intensity
    data, Emotion Decay data, decay constants, and Time data. Returns the
    result as Emotion Intensity data.

    \item[Implemented By:] \progname{}
\end{description}

\subsubsection{PAD Type Module (\mref{mPADType})}

\begin{description}[font=\scshape]
    \item[Secrets:] The internal representation of the PAD data type.

    \item[Services:] Stores the PAD data type, enforces its constraints, and
    provides methods for using it.

    \item[Implemented By:] \progname{}
\end{description}

\subsubsection{PAD Function Module (\mref{mPADFun})}

\begin{description}[font=\scshape]
    \item[Secrets:] The method of converting Emotion State data into a PAD data.

    \item[Services:] Evaluates ``equivalent'' PAD data from Emotion State data.
    Returns the result as PAD data.

    \item[Implemented By:] \progname{}
\end{description}

\subsubsection{Emotion Module}

\begin{description}[font=\scshape]
    \item[Secrets:] ``Virtual'' module containing the emotion modules.

    \item[Services:] Using Emotion data type, separated by data structure
    (\mref{mEmotionType}) and methods (\mref{mEmotionFun})

    \item[Implemented By:] --
\end{description}

\subsubsection{Emotion Type Module (\mref{mEmotionType})}

\begin{description}[font=\scshape]
    \item[Secrets:] The internal representation of the Emotion data type.

    \item[Services:] Stores the Emotion data type, enforces its constraints,
    and provides methods for using it.

    \item[Implemented By:] \progname{}
\end{description}

\subsubsection{Emotion Function Module (\mref{mEmotionFun})}

\begin{description}[font=\scshape]
    \item[Secrets:] The method for evaluating Emotion State data from Emotion
    data.

    \item[Services:] Calculates Emotion-data dependant Emotion State data using
    a decay constant, and Emotion Decay and Time data. Returns the result as
    Emotion State.

    \item[Implemented By:] \progname{}
\end{description}

\subsubsection{Goal Module (\mref{mGoal})}

\begin{description}[font=\scshape]
    \item[Secrets:] Implementation of the Goal data structure.

    \item[Services:] Stores the Goal data type, enforces its constraints, and
    provides methods for using it.

    \item[Implemented By:] \progname{}
\end{description}

\subsubsection{Plan Module (\mref{mPlan})}

\begin{description}[font=\scshape]
    \item[Secrets:] Implementation of the Plan data structure.

    \item[Services:] Stores the Plan data type, enforces its constraints,
    and provides methods for using it.

    \item[Implemented By:] \progname{}
\end{description}

\subsubsection{Attention Module (\mref{mAttention})}

\begin{description}[font=\scshape]
    \item[Secrets:] Implementation of the Attention data structure.

    \item[Services:] Provides the required data structures and methods
    associated with the Attention data type.

    \item[Implemented By:] \progname{}
\end{description}

\subsubsection{Social Attachment Module (\mref{mSocial})}

\begin{description}[font=\scshape]
    \item[Secrets:] Implementation of the Social Attachment data structure.

    \item[Services:] Provides the required data structures and methods
    associated with the Social Attachment data type.

    \item[Implemented By:] \progname{}
\end{description}

\subsection{Software Decision Module}

\begin{description}[font=\scshape]
    \item[Secrets:] The design decision based on mathematical theorems,
    physical facts, or programming considerations. The secrets of this module
    are \emph{not} described in the SRS. For \progname{}, these are are
    user-defined modules.

    \item[Services:] \progname{} provides the API describing necessary data
    types and methods, but the end-user must implement them as they are unique
    to their application (e.g. video game).

    \item[Implemented By:] --
\end{description}

\subsubsection{Time Module (\mref{mTime})}

\begin{description}[font=\scshape]
    \item[Secrets:] Implementation of the Time and Delta Time data structures.

    \item[Services:]  Provides the required data structures and methods
    associated with the Time and Delta Time data types.

    \item[Implemented By:] End-User
\end{description}

\subsubsection{World State Module (\mref{mWorld})}

\begin{description}[font=\scshape]
    \item[Secrets:] Implementation of the World State data structure.

    \item[Services:] Provides the required data structures and methods
    associated with the World State data types.

    \item[Implemented By:] End-User
\end{description}

    \clearpage

    \section{Traceability Matrices}
\label{SecTM}
Traceability matrices show the connections modules and the requirements
developed in the SRS, between modules and anticipated changes, and between
components and modules.

% the table should use mref, the requirements should be named, use something
% like fref
\begin{table}[H]
    \centering
    \renewcommand{\arraystretch}{1.2}
    \begin{tabular}{C{0.1\textwidth} m{0.6\textwidth}}
        \toprule
        \textbf{Module} & \textbf{Requirements} \\
        \midrule

        \colourRow\mref{mIntensity} & \rref{R_Types},
        \rref{R_IntensityTypeUse}, \rref{R_IntensityChangeType},
        \rref{R_MixingEmotionsPES}, \rref{R_PartitionEmotions},
        \rref{R_UpdateAnIntensity} \\

        \mref{mIntensityFun} & \rref{R_CalculateIntensity} \\

        \colourRow\mref{mStateType} & \rref{R_Types},
        \rref{R_EmotionKindsType}, \rref{R_EmotionStateType},
        \rref{R_MixingEmotionsPES}, \rref{R_PartitionEmotions},
        \rref{R_MixingEmotionsCTE}, \rref{R_UpdateEmotionState} \\

        \mref{mGenerate} & \rref{R_GenerateEmotionCTE} \\

        \colourRow\mref{mDecay} & \rref{R_Types}, \rref{R_IntensityDecayType},
        \rref{R_DecayIntensity} \\

        \mref{mDecayState} & \rref{R_Types}, \rref{R_EmotionDecayStateType},
        \rref{R_DecayEmotion} \\

        \colourRow\mref{mPADType} & \rref{R_Types}, \rref{R_PADPointType} \\

        \mref{mPADFun} & \rref{R_Convert2PAD} \\

        \colourRow\mref{mEmotionType} & \rref{R_Types}, \rref{R_EmotionType},
        \rref{R_UpdateEmotion}, \rref{R_GetEmotionState} \\

        \mref{mEmotionFun} & \rref{R_NewESFromDecay} \\

        \colourRow\mref{mGoal} & \rref{R_Types}, \rref{R_GoalType} \\

        \mref{mPlan} & \rref{R_Types}, \rref{R_PlanType} \\

        \colourRow\mref{mSocial} & \rref{R_Types}, \rref{R_SocialAttachment} \\

        \mref{mAttention} & \rref{R_Types}, \rref{R_Attention} \\

        \colourRow\mref{mTime} & \rref{R_Types}, \rref{R_TimeType} \\

        \mref{mWorld} & \rref{R_Types}, \rref{R_WorldType},
        \rref{R_WorldChangeType}, \rref{R_DistanceType},
        \rref{R_DistanceChangeType} \\

        \bottomrule
    \end{tabular}
    \caption{Trace Between Requirements and Modules}
    \label{TblRT}
\end{table}

\begin{table}[H]
    \centering
    \renewcommand{\arraystretch}{1.2}
    \begin{tabular}{C{0.1\textwidth} m{0.5\textwidth}}
        \toprule
        \textbf{Module} & \textbf{Anticipated Changes}\\
        \midrule

        \colourRow\mref{mIntensity} & \acref{acEmotionIntensityMin},
        \acref{acUpdateIntensityAlgo} \\

        \mref{mIntensityFun} & \acref{acIntensityEvalAlgo} \\

        \colourRow\mref{mStateType} & \acref{acEmotionKindType},
        \acref{acEmotionStateImpl} \\

        \mref{mGenerate} & \acref{acAppraisalAlgo} \\

        \colourRow\mref{mDecay} & \acref{acDecayIntensityAlgo} \\

        \mref{mDecayState} & \acref{acEmotionDecayStateImpl},
        \acref{acDecayStateAlgo} \\

        \colourRow\mref{mPADType} & \acref{acPADPointType} \\

        \mref{mPADFun} & \acref{acState2PADAlgo} \\

        \colourRow\mref{mEmotionType} & \acref{acEmotionState} \\

        \mref{mEmotionFun} & -- \\

        \colourRow\mref{mGoal} & \acref{acGoalImpl} \\

        \mref{mPlan} & \acref{acPlanImpl} \\

        \colourRow\mref{mSocial} & \acref{acAttachmentImpl} \\

        \mref{mAttention} & \acref{acAttentionImpl} \\

        \colourRow\mref{mTime} & \acref{acTimeImpl} \\

        \mref{mWorld} & \acref{acWorldStateImpl},
        \acref{acWorldStateChangeImpl}, \acref{acDistanceImpl},
        \acref{acDistanceChangeImpl} \\

        \bottomrule
    \end{tabular}
    \caption{Trace Between Anticipated Changes and Modules}
    \label{TblACT}
\end{table}

Generally, each anticipated change should map to one module because it implies
that each module contains information that only it knows about the design.
However, in \progname{}'s design, five modules address multiple anticipated
changes:
\begin{itemize}

    \item \mref{mIntensity} collects the emotion intensity data types and
    function for updating an Emotion Intensity with an Emotion Intensity
    Change. Changing the minimum intensity (\acref{acEmotionIntensityMin}) will
    likely cause changes to the method for ``combining'' the two intensity
    types (\acref{acUpdateIntensityAlgo}). Therefore, these are gathered into
    one module.

    \item \mref{mStateType} contains the types for emotion kinds and emotion
    state. The emotion state type (\acref{acEmotionStateImpl}) directly depends
    on the type of emotion kinds (\acref{acEmotionKindType}), and the function
    for updating emotion intensity (\acref{acUpdateIntensityAlgo}) directly
    depends on the emotion state type. This tight coupling between models
    implies that changes to one likely mean changes to another, so they should
    be in the same module.

    \item \mref{mDecayState} contains the emotion decay state data type and the
    method for decaying an emotion state. The emotion state decay method
    (\acref{acDecayStateAlgo}) directly depends on the emotion state decay data
    type (\acref{acEmotionDecayStateImpl}). This tight coupling between models
    implies that changes to one likely mean changes to another, so they should
    be in the same module.

    \item \mref{mWorld} contains APIs for defining world states. These types
    are interrelated and inseparable, so they should be in the same module.

\end{itemize}

    \clearpage

    \section{Use Hierarchy Between Modules}\label{SecUse}

\citet{Parnas1978} said of two programs, A and B, that A {\em uses} B if the
correct execution of B might be necessary for A to complete the task as
specified. That is, A {\em uses} B if there exist situations in which the
correct functioning of A depends upon the availability of a correct
implementation of B.

The \textit{uses} hierarchy between modules (Figure~\ref{FigUH}) shows that the
graph is a directed acyclic graph (DAG). Each level of the hierarchy offers a
testable and usable subset of the system. Modules in the higher level of the
hierarchy are simpler because they use modules from the lower levels.

\vspace*{\fill}
\begin{figure}[!tbh]
    \centering
    \includegraphics[width=\linewidth]{figures/usesHierarchy.pdf}
    \caption{Use hierarchy among modules}
    \label{FigUH}
\end{figure}
\vspace*{\fill}

    \clearpage

    \bibliographystyle{ACM-Reference-Format}
    \bibliography{../../../refs/references_documentation,
    ../../../refs/references, ../../../refs/references_gamedesign,
    ../../../refs/references_SEPerspective, ../../../refs/references_psych}

\end{document}