\section{Introduction}
This is the Module Guide (MG) for the \progname{}, a Computational Model of
Emotion (CME) for Non-Player Characters (NPCs) to enhance their believability,
with the goal of improving long-term player engagement. \progname{} is for
\textit{emotion generation}, accepting user-defined information from a game
environment to determines what emotion and intensity a NPC is ``experiencing''.
How the emotion is expressed and what other effects it could have on game
entities is left for game designers/developers to decide. For more information
about \progname{} and its requirement specification, see its Software
Requirements Specification (SRS) document (Version~\srsVersion).

\subsection{Purpose of the Document}
After completing an SRS, the MG is developed~\citep{ParnasEtAl1984}. The MG
specifies the modular structure of the system and is intended to allow both
designers and maintainers to easily identify the parts of the software.

Decomposing a system into modules is a commonly accepted approach to developing
software.  A module is a work assignment for a programmer or programming
team~\citep{ParnasEtAl1984}. The module decomposition in this MG is based on
the principle of information hiding~\citep{Parnas1972a}. This principle
supports design for change, because the ``secrets'' that each module hides
represent likely future changes. Design for change is valuable in software
design, where modifications are frequent, especially during initial development
as the solution space is explored. \progname{}'s design follows
\citet{ParnasEtAl1984}:
\begin{itemize}
    \item System details that are likely to change independently should be the
    secrets of separate modules.
    \item Each data structure is implemented in only one module.
    \item Any other program that requires information stored in a module's data
    structures must obtain it by calling access programs belonging to that
    module.
\end{itemize}

\subsection{Intended Readers of the Document}

\begin{itemize}
    \item \textbf{New Project Members} \\
    This document can be a guide for a new project member to aid their
    understanding of the overall structure and quickly find modules that are
    relevant to their work.
    
    \item \textbf{Maintainers} \\
    The MG's hierarchical structure improves the maintainers' understanding of
    the system when they need to make changes. It is crucial for a maintainer
    to update the relevant sections of the document after changes have been
    made.
    
    \item \textbf{Designers} \\
    Once the MG has been written, designers can use it to verify the system in
    various ways, such as consistency among modules, feasibility of the
    decomposition, and flexibility of the design.
\end{itemize}

\subsection{Organization of the Document}
The rest of the document is organized as follows:
\begin{itemize}
    \item Section~\ref{SecChange} lists the software requirements'
    anticipated/likely and unlikely changes

    \item Section~\ref{SecMH} summarizes the module decomposition based on the
    anticipated changes

    \item Section~\ref{SecConnection} documents design decisions made to 
    connect the software requirements to modules

    \item Section~\ref{SecMD} gives a description for each module

    \item Section~\ref{SecTM} includes two traceability matrices: one checks
    the completeness of the design against the requirements provided in the
    SRS, and the other shows the relation between anticipated changes and
    modules

    \item Section~\ref{SecUse} describes the \textit{uses} hierarchy between 
    modules
\end{itemize}