\section{Traceability Matrices}
\label{SecTM}
Traceability matrices show the connections modules and the requirements
developed in the SRS, between modules and anticipated changes, and between
components and modules.

% the table should use mref, the requirements should be named, use something
% like fref
\begin{table}[H]
    \centering
    \renewcommand{\arraystretch}{1.2}
    \begin{tabular}{C{0.1\textwidth} m{0.6\textwidth}}
        \toprule
        \textbf{Module} & \textbf{Requirements} \\
        \midrule

        \colourRow\mref{mIntensity} & \rref{R_Types},
        \rref{R_IntensityTypeUse}, \rref{R_IntensityChangeType},
        \rref{R_MixingEmotionsPES}, \rref{R_PartitionEmotions},
        \rref{R_UpdateAnIntensity} \\

        \mref{mIntensityFun} & \rref{R_CalculateIntensity} \\

        \colourRow\mref{mStateType} & \rref{R_Types},
        \rref{R_EmotionKindsType}, \rref{R_EmotionStateType},
        \rref{R_MixingEmotionsPES}, \rref{R_PartitionEmotions},
        \rref{R_MixingEmotionsCTE}, \rref{R_UpdateEmotionState} \\

        \mref{mGenerate} & \rref{R_GenerateEmotionCTE} \\

        \colourRow\mref{mDecay} & \rref{R_Types}, \rref{R_IntensityDecayType},
        \rref{R_DecayIntensity} \\

        \mref{mDecayState} & \rref{R_Types}, \rref{R_EmotionDecayStateType},
        \rref{R_DecayEmotion} \\

        \colourRow\mref{mPADType} & \rref{R_Types}, \rref{R_PADPointType} \\

        \mref{mPADFun} & \rref{R_Convert2PAD} \\

        \colourRow\mref{mEmotionType} & \rref{R_Types}, \rref{R_EmotionType},
        \rref{R_UpdateEmotion}, \rref{R_GetEmotionState} \\

        \mref{mEmotionFun} & \rref{R_NewESFromDecay} \\

        \colourRow\mref{mGoal} & \rref{R_Types}, \rref{R_GoalType} \\

        \mref{mPlan} & \rref{R_Types}, \rref{R_PlanType} \\

        \colourRow\mref{mSocial} & \rref{R_Types}, \rref{R_SocialAttachment} \\

        \mref{mAttention} & \rref{R_Types}, \rref{R_Attention} \\

        \colourRow\mref{mTime} & \rref{R_Types}, \rref{R_TimeType} \\

        \mref{mWorld} & \rref{R_Types}, \rref{R_WorldType},
        \rref{R_WorldChangeType}, \rref{R_DistanceType},
        \rref{R_DistanceChangeType} \\

        \bottomrule
    \end{tabular}
    \caption{Trace Between Requirements and Modules}
    \label{TblRT}
\end{table}

\begin{table}[H]
    \centering
    \renewcommand{\arraystretch}{1.2}
    \begin{tabular}{C{0.1\textwidth} m{0.5\textwidth}}
        \toprule
        \textbf{Module} & \textbf{Anticipated Changes}\\
        \midrule

        \colourRow\mref{mIntensity} & \acref{acEmotionIntensityMin},
        \acref{acUpdateIntensityAlgo} \\

        \mref{mIntensityFun} & \acref{acIntensityEvalAlgo} \\

        \colourRow\mref{mStateType} & \acref{acEmotionKindType},
        \acref{acEmotionStateImpl} \\

        \mref{mGenerate} & \acref{acAppraisalAlgo} \\

        \colourRow\mref{mDecay} & \acref{acDecayIntensityAlgo} \\

        \mref{mDecayState} & \acref{acEmotionDecayStateImpl},
        \acref{acDecayStateAlgo} \\

        \colourRow\mref{mPADType} & \acref{acPADPointType} \\

        \mref{mPADFun} & \acref{acState2PADAlgo} \\

        \colourRow\mref{mEmotionType} & \acref{acEmotionState} \\

        \mref{mEmotionFun} & -- \\

        \colourRow\mref{mGoal} & \acref{acGoalImpl} \\

        \mref{mPlan} & \acref{acPlanImpl} \\

        \colourRow\mref{mSocial} & \acref{acAttachmentImpl} \\

        \mref{mAttention} & \acref{acAttentionImpl} \\

        \colourRow\mref{mTime} & \acref{acTimeImpl} \\

        \mref{mWorld} & \acref{acWorldStateImpl},
        \acref{acWorldStateChangeImpl}, \acref{acDistanceImpl},
        \acref{acDistanceChangeImpl} \\

        \bottomrule
    \end{tabular}
    \caption{Trace Between Anticipated Changes and Modules}
    \label{TblACT}
\end{table}

Generally, each anticipated change should map to one module because it implies
that each module contains information that only it knows about the design.
However, in \progname{}'s design, five modules address multiple anticipated
changes:
\begin{itemize}

    \item \mref{mIntensity} collects the emotion intensity data types and
    function for updating an Emotion Intensity with an Emotion Intensity
    Change. Changing the minimum intensity (\acref{acEmotionIntensityMin}) will
    likely cause changes to the method for ``combining'' the two intensity
    types (\acref{acUpdateIntensityAlgo}). Therefore, these are gathered into
    one module.

    \item \mref{mStateType} contains the types for emotion kinds and emotion
    state. The emotion state type (\acref{acEmotionStateImpl}) directly depends
    on the type of emotion kinds (\acref{acEmotionKindType}), and the function
    for updating emotion intensity (\acref{acUpdateIntensityAlgo}) directly
    depends on the emotion state type. This tight coupling between models
    implies that changes to one likely mean changes to another, so they should
    be in the same module.

    \item \mref{mDecayState} contains the emotion decay state data type and the
    method for decaying an emotion state. The emotion state decay method
    (\acref{acDecayStateAlgo}) directly depends on the emotion state decay data
    type (\acref{acEmotionDecayStateImpl}). This tight coupling between models
    implies that changes to one likely mean changes to another, so they should
    be in the same module.

    \item \mref{mWorld} contains APIs for defining world states. These types
    are interrelated and inseparable, so they should be in the same module.

\end{itemize}