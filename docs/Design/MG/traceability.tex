\section{Traceability Matrices}
\label{SecTM}
Two traceability matrices are provided to show connections between modules and
the requirements developed in the SRS, and between modules and anticipated
changes.

% the table should use mref, the requirements should be named, use something
% like fref
\begin{table}[H]
    \centering
    \renewcommand{\arraystretch}{1.2}
    \begin{tabular}{C{0.1\textwidth} m{0.6\textwidth}}
        \toprule
        \textbf{Module} & \textbf{Requirements} \\
        \midrule

        \rowcolor[gray]{0.9}\mref{mIntensity} & \rref{R_Types},
        \rref{R_IntensityTypeUse}, \rref{R_IntensityChangeType},
        \rref{R_MixingEmotionsPES}, \rref{R_PartitionEmotions} \\

        \mref{mIntensityFun} & \rref{R_CalculateIntensity} \\

        \rowcolor[gray]{0.9}\mref{mStateType} & \rref{R_Types},
        \rref{R_EmotionKindsType}, \rref{R_EmotionStateType},
        \rref{R_MixingEmotionsPES}, \rref{R_PartitionEmotions},
        \rref{R_MixingEmotionsCTE}, \rref{R_UpdateAnIntensity},
        \rref{R_UpdateEmotionState} \\

        \mref{mGenerate} & \rref{R_GenerateEmotionCTE} \\

        \rowcolor[gray]{0.9}\mref{mDecay} & \rref{R_Types},
        \rref{R_IntensityDecayType} \\

        \mref{mDecayState} & \rref{R_Types}, \rref{R_EmotionDecayStateType} \\

        \rowcolor[gray]{0.9}\mref{mDecayFun} & \rref{R_DecayIntensity} \\

        \mref{mPADType} & \rref{R_Types}, \rref{R_PADPointType} \\

        \rowcolor[gray]{0.9}\mref{mPADFun} & \rref{R_Convert2PAD} \\

        \mref{mEmotionType} & \rref{R_Types}, \rref{R_EmotionType},
        \rref{R_UpdateEmotion}, \rref{R_GetEmotionState} \\

        \rowcolor[gray]{0.9}\mref{mEmotionFun} & \rref{R_DecayEmotion} \\

        \mref{mGoal} & \rref{R_Types}, \rref{R_GoalType} \\

        \rowcolor[gray]{0.9}\mref{mPlan} & \rref{R_Types}, \rref{R_PlanType} \\

        \mref{mAttention} & \rref{R_Types}, \rref{R_Attention} \\

        \rowcolor[gray]{0.9}\mref{mSocial} & \rref{R_Types},
        \rref{R_SocialAttachment} \\

        \mref{mTime} & \rref{R_Types}, \rref{R_TimeType} \\

        \rowcolor[gray]{0.9}\mref{mWorld} & \rref{R_Types}, \rref{R_WorldType},
        \rref{R_WorldChangeType}, \rref{R_DistanceType},
        \rref{R_DistanceChangeType} \\

        \bottomrule
    \end{tabular}
    \caption{Trace Between Requirements and Modules}
    \label{TblRT}
\end{table}

\begin{table}[H]
    \centering
    \renewcommand{\arraystretch}{1.2}
    \begin{tabular}{C{0.1\textwidth} m{0.5\textwidth}}
        \toprule
        \textbf{Module} & \textbf{Anticipated Changes}\\
        \midrule

        \rowcolor[gray]{0.9}\mref{mIntensity} & \acref{acEmotionIntensityMin} \\

        \mref{mIntensityFun} & \acref{acIntensityAlgo},
        \acref{acElicitSurpriseAlgo}, \acref{acElicitInterestAlgo},
        \acref{acElicitAcceptanceAlgo} \\

        \rowcolor[gray]{0.9}\mref{mStateType} & \acref{acUpdateIntensityAlgo},
        \acref{acEmotionKindType}, \acref{acEmotionStateImpl} \\

        \mref{mGenerate} & \acref{acAppraisalAlgo} \\

        \rowcolor[gray]{0.9}\mref{mDecay} & \acref{acDecayIntensityAlgo} \\

        \mref{mDecayState} & \acref{acEmotionDecayStateImpl} \\

        \rowcolor[gray]{0.9}\mref{mDecayFun} & \acref{acDecayIntensityAlgo} \\

        \mref{mPADType} & \acref{acPADPointType} \\

        \rowcolor[gray]{0.9}\mref{mPADFun} & \acref{acState2PADAlgo} \\

        \mref{mEmotionType} & \acref{acEmotionState} \\

        \rowcolor[gray]{0.9}\mref{mEmotionFun} & \acref{acDecayStateAlgo} \\

        \mref{mGoal} & \acref{acGoalImpl} \\

        \rowcolor[gray]{0.9}\mref{mPlan} & \acref{acPlanImpl} \\

        \mref{mAttention} & \acref{acAttentionImpl} \\

        \rowcolor[gray]{0.9}\mref{mSocial} & \acref{acAttachmentImpl} \\

        \mref{mTime} & \acref{acTimeImpl} \\

        \rowcolor[gray]{0.9}\mref{mWorld} & \acref{acWorldStateImpl},
        \acref{acWorldStateChangeImpl}, \acref{acDistanceImpl},
        \acref{acDistanceChangeImpl} \\

        \bottomrule
    \end{tabular}
    \caption{Trace Between Anticipated Changes and Modules}
    \label{TblACT}
\end{table}

Generally, each anticipated change should map to one module because it implies
that each module contains information that only it knows about the design.
However, in \progname{}'s design, three modules address multiple anticipated
changes:
\begin{itemize}

    \item \mref{mIntensityFun} contains all of the functions that calculate
    emotion intensity. The SRS (Ver.~\srsVersion) calculates the intensity for
    some emotion kinds differently. End-users do not need to know that they are
    calculated differently, and there is the potential for consolidating them
    into a single emotion intensity function. They also collectively share the
    same secret: the method for evaluating emotion intensity. Therefore, these
    are encapsulated into one module.

    \item \mref{mStateType} contains the types for emotion kinds and emotion
    state. The emotion state type (\acref{acEmotionStateImpl}) directly depends
    on the type of emotion kinds (\acref{acEmotionKindType}), and the function
    for updating emotion intensity (\acref{acUpdateIntensityAlgo}) directly
    depends on the emotion state type. This tight coupling between components
    implies that changes to one likely mean changes to another, so they should
    be in the same module.

    \item \mref{mWorld} contains APIs for defining world states. These types
    are interrelated and inseparable, so they should be in the same module.

\end{itemize}

There is also one anticipated change, \acref{acDecayIntensityAlgo} covered by
two modules (\mref{mDecay}, \mref{mDecayFun}). This is due to their mutual
reliance on the same underlying model. However, one could feasibly modify only
one of these modules if that model does not change. Therefore, they are
separated to afford users more flexibility in how they use them.