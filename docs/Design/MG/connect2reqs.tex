\section{Connection Between Requirements and Design} \label{SecConnection}

The design of the system is intended to satisfy the requirements developed in
the SRS (Version~\srsVersion). In this stage, the system is decomposed into
modules. Table~\ref{TblRT} lists the connection between requirements and
modules.

\paragraph{Library of Components}
This MG treats \progname{} as a library of components due to its need to be an 
``understandable black box'' design. \progname{} must not require users to have 
an understanding of affective science and/or emotion research 
(\nfref{N_Knowledge}) and have reasonable compatibility with different agent 
architectures/frameworks (\nfref{N_Arch}), game entity embodiment 
(\nfref{N_Embody}), and development environments (\nfref{N_Env}). Together with 
its requirement to clearly document usage information for user-facing 
(\nfref{N_CodeDoc}), \progname{} already lends itself to a black box as it 
focuses on the relation between inputs and outputs with no claims about the 
underlying process~\citep[p.~601]{wehrle1995potential}. This aligns with the 
description of domain-specific CMEs, where the transformation of inputs into 
affective phenomena is not important, as long as it has the desired effects 
(\citepg{hudlicka2019modeling}{130--131}; 
\citepg{osuna2020seperspective}{4--6}). However, it would be difficult to 
support the Verifiability requirements (\nfref{N_Atomic}, \nfref{N_Trace}) if 
one could not see how \progname{}'s parts passed information to each other. 
Therefore, \progname{}'s components should be black boxes, but not their 
connections so that its overall behaviour can be 
explained~\citep[p.~20]{guimaraes2022fatima}.

This implies that a component-based software 
architecture~\citep[p.~248--261]{qian2010software} is best, where each step in 
emotion generation is a component. This would:
\begin{itemize}
    \item Increase \progname{}'s portability by specifying what each component 
    is guaranteed to provide~\citep[p.~254]{qian2010software} so that designers 
    can use existing, tested and validated systems and enables fast and easy 
    integration of new components~\citep[p.~443]{rodriguez2015computational}, 
    increasing its compatibility with agent architectures/frameworks 
    (\nfref{N_Arch}), game entity embodiment (\nfref{N_Embody}), and 
    development environments (\nfref{N_Env})

    \item Allow designers to call \progname{}'s components as they see fit,     
    supporting customization and modifiability (\nfref{N_Mod}, 
    \nfref{N_ChooseEm}), efficiency (\nfref{N_Complex}, 
    \nfref{N_Efficient})~\citep[p.~466]{carbone2020radically}, enabling users 
    to specify how to use (\nfref{N_Output}) and verify (\nfref{N_Atomic}, 
    \nfref{N_Trace}) \progname{} outputs, and---potentially---demonstrating how 
    \progname{} could improve the player experience (\nfref{N_PX}) and/or 
    create novel game experiences (\nfref{N_Novel})

    \item Have the potential for developing authoring tools that interface with 
    and manage \progname{} components, which could reduce total authorial 
    effort (\nfref{N_AuthorialEffort})
\end{itemize}

Ongoing work on the FAtiMA architecture and its descendants, the FAtiMA Modular 
framework and FAtiMA Toolkit, found this approach 
successful~\citep[p.~8:2, 8:12--8:13]{mascarenhas2022fatima}. After gaining 
feedback from people in the games industry\footnote{As part of the ``Realising 
an Applied Gaming Eco-system'' (RAGE) project 
(\url{https://cordis.europa.eu/project/id/644187}).}, the FAtiMA Toolkit's 
designers realized it as a library of components so that its parts can work 
autonomously. A library-based approach also alleviates some of \progname{}'s 
design requirements, as it no long has to ``...be generic enough to encompass 
all possible forms of a perception-action cycle in an 
agent.''~\citep[p.~8:12]{mascarenhas2022fatima}. This also increased the 
Toolkit's chances of adoption, as it allows game designers to use it in their 
existing systems and/or frameworks and avoiding the complexity and 
accessibility issues of other agent 
architectures~\citep[p.~3]{guimaraes2022fatima}. In the same vein, \progname{} 
would be a \textit{library of components} so it can take advantage of these 
benefits. Users would also be free to focus on what sequence of components for 
emotion processing works for their needs because no one really knows what order 
they truly run in~\citep[p.~142]{moffat1997personality}.

\paragraph{Pre-Built Components}
A library of components creates opportunities for \progname{} to address
additional nonfunctional requirements by pre-building some compound components,
such as a default \textit{``engine''}. 

While each emotion generation component would be a black 
box~\citep[p.~253]{qian2010software}, a user might not know when they should 
use a component---or if it is even necessary, which might require some 
knowledge of affective science and/or emotion research to bridge (violates 
\nfref{N_Knowledge}). A component-based software architecture supports this 
need too by pre-building some compound components. It could come with an 
``engine''---a pre-built component that is itself a ``system'' of 
components~\citep[p.~249]{qian2010software} that minimizes the necessary inputs 
for emotion generation. It would accept data and returns an emotion state, 
without the designer knowing how it 
works~\citep[p.~443]{rodriguez2015computational}. These would also serve as 
usage examples (\nfref{N_CodeDoc}, \nfref{N_Manual}, \nfref{N_Knowledge}) 
and/or as parts for automating tasks (\nfref{N_Auto}).

In this form, \progname{} would be similar to GAMYGDALA which compares itself 
to a physics engine~\citep[p.~32]{popescu2014gamygdala}. Designers have applied 
GAMYGDALA to: an arcade and puzzle game~\citep{broekens2015emotion}; a 
narrative generation framework to drive character 
emotions~\citep{kaptein2015affective}; implement affective decision-making in 
fighting game characters~\citep{yuda2019creating}; and modify it with the fuzzy 
logic, classical conditioning, and learning from another 
CME\footnote{FLAME~\citep{el2000flame}}~\citep[p.~4]{code2015learning}. The 
breadth of games that GAMYGDALA appears in suggests that, even pre-packaged as 
a large component, \progname{} could also find success.

\paragraph{Separating Data Types and Functions}
\progname{}'s module decomposition separates data types (\mref{mIntensity},
\mref{mDecay}, \mref{mDecayState}, \mref{mPADType}, \mref{mEmotionType}) from
functions that use them (\mref{mIntensityFun}, \mref{mDecayFun},
\mref{mPADFun}, \mref{mEmotionFun}). While this increases the total modules, it
also improves an end-user's ability to adopt individual parts of \progname{} as
needed (\nfref{N_Efficient}, \nfref{N_Reuse}, \nfref{N_Mod}) and change the
underlying theories (\nfref{N_Evolve}), and ensures that theory-agnostic
components (e.g. \mref{mStateType}) are separated from theory-specific ones
(e.g. \mref{mGenerate}).

\paragraph{User-Defined APIs}
\progname{} also encapsulates each Application Programming Interface (API) that
end-users must define in their own modules (\mref{mTime}, \mref{mWorld},
\mref{mAttention}, \mref{mSocial}). This allows a team to divide their
implementation between members, reduces concerns regarding compatibility with
othe \progname{} components (\nfref{N_Mod}) and allows end-users to test and
optimize them individually (\nfref{N_Efficient}, \nfref{N_Trace}).