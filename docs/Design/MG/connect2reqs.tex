\section{Connection Between Requirements and Design} \label{SecConnection}

The design of the system is intended to satisfy the requirements developed in
the SRS (Version~\srsVersion). In this stage, the system is decomposed into
modules. Table~\ref{TblRT} lists the connection between requirements and
modules.

\paragraph{Library of Components}
This MG treats \progname{} as a library of components due to its need to be an
``understandable black box'' design. \progname{} must not require users to have
an understanding of affective science and/or emotion research
(\nfref{N_Knowledge}) and have reasonable compatibility with different agent
architectures/frameworks (\nfref{N_Arch}), game entity embodiment
(\nfref{N_Embody}), and development environments (\nfref{N_Env}). Together with
its requirement to clearly document usage information for user-facing
(\nfref{N_CodeDoc}), \progname{} already lends itself to a black box as it
focuses on the relation between inputs and outputs with no claims about the
underlying process~\citep[p.~601]{wehrle1995potential}. This aligns with the
description of domain-specific CMEs, where the transformation of inputs into
affective phenomena is not important, as long as it has the desired effects
(\citepg{hudlicka2019modeling}{130--131};
\citepg{osuna2020seperspective}{4--6}). However, it would be difficult to
support the Verifiability requirements (\nfref{N_Atomic}, \nfref{N_Trace}) if
one could not see how \progname{}'s parts passed information to each other.
Therefore, \progname{}'s components should be black boxes, but not their
connections so that its overall behaviour can be
explained~\citep[p.~20]{guimaraes2022fatima}.

This suggests that a component-based software
architecture~\citep[p.~248--261]{qian2010software} is best, where each of
\progname{}'s ``tasks'' is a discrete component that users can include,
exclude, and change as needed. A component-based design approach is common in
CME development~\citep[p.~141]{osuna2021toward}. This approach would:
\begin{itemize}
    \item Increase \progname{}'s portability by specifying what each component
    is guaranteed to provide so that designers can use existing, tested and
    validated systems and enables fast and easy integration of new
    components~\citep[p.~443]{rodriguez2015computational}, increasing its
    compatibility with agent architectures/frameworks (\nfref{N_Arch}), game
    entity embodiment (\nfref{N_Embody}), and development environments
    (\nfref{N_Env})

    \item Mandates well-defined interfaces to show what each component requires
    and provides, including its configuration parameters, supporting
    customization (\nfref{N_Custom}, \nfref{N_ChooseEm})

    \item Allow designers to call \progname{}'s components as they see fit,
    supporting modifiability (\nfref{N_Mod}) and automation of emotion decay
    \nfref{N_Auto}

    \item Enable users to specify how to use (\nfref{N_Output}) and verify
    (\nfref{N_Atomic}, \nfref{N_Trace}) \progname{} outputs,
    and---potentially---demonstrating how \progname{} could improve the player
    experience (\nfref{N_PX}) and/or create novel game experiences
    (\nfref{N_Novel})

    \item Allow designers to add or swap \progname{}'s components, supporting
    \textit{Allow the Integration of New Components} (\ref{flexNew}),
    \textit{Allowing developers to choose which kinds of emotion \progname{}
        produces} (\ref{flexEm}), and efficiency (\nfref{N_Complex},
        \nfref{N_Efficient})~\citep[p.~466]{carbone2020radically}

    \item Have the potential for developing authoring tools that interface with
    and manage \progname{} components, which could reduce total authorial
    effort (\nfref{N_AuthorialEffort})
\end{itemize}

Ongoing work on the FAtiMA architecture and its descendants, the FAtiMA Modular
framework and FAtiMA Toolkit, found this approach
successful~\citep[p.~8:2, 8:12--8:13]{mascarenhas2022fatima}. After gaining
feedback from people in the games industry\footnote{As part of the ``Realising
an Applied Gaming Eco-system'' (RAGE) project
(\href{https://cordis.europa.eu/project/id/644187}{https://cordis.europa.eu/project/id/644187}).},
 the FAtiMA Toolkit's designers realized it as a library of components so that
its parts can work autonomously. This also increased the Toolkit's chances of
adoption, as it allows game designers to use it in their existing systems
and/or frameworks and avoiding the complexity and accessibility issues of other
agent architectures~\citep[p.~3]{guimaraes2022fatima}. This also lets users
focus on what emotion processing sequence works for their needs rather than
constraining them to ``\progname{}'s process'' because no one really knows what
order emotion processes truly run in~\citep[p.~142]{moffat1997personality}.

\paragraph{Pre-Built Components}
A library of components creates opportunities for \progname{} to address
additional nonfunctional requirements by pre-building some compound components,
such as a default \textit{``engine''}.

While each of \progname{}'s components are
black-boxes~\citep[p.~253]{qian2010software}, users might not know when they
should use a component---or if it is even necessary. This might require some
knowledge of affective science and/or emotion research to bridge (violates
\nfref{N_Knowledge}). A component-based software architecture supports this need
too by allowing the specification of a prebuilt component---an
``engine''---that is itself a ``system'' of
components~\citep[p.~249]{qian2010software} that minimizes the necessary
decisions and inputs needed for emotion processing. It would accept data and
return an emotion state, without the designer knowing how it
works~\citep[p.~443]{rodriguez2015computational}. These would also serve as
usage examples (\nfref{N_CodeDoc}, \nfref{N_Manual}, \nfref{N_Knowledge})
and/or as parts for automating tasks (\nfref{N_Auto}).

This prebuilt component or ``engine'' is similar to GAMYGDALA, which compares
itself to a physics engine~\citep[p.~32]{popescu2014gamygdala}. Due to its
plugin nature, designers have successfully applied GAMYGDALA to: arcade and
puzzle games~\citep{broekens2015emotion}; a narrative generation framework to
drive character emotions~\citep{kaptein2015affective}; and implement affective
decision-making in fighting game characters~\citep{yuda2019creating}. A
developer has also successfully integrated the fuzzy logic, classical
conditioning, and learning from another CME\footnote{FLAME~\citep{el2000flame}}
with GAMYGDALA~\citep[p.~4]{code2015learning}. The breadth of games that
developers have applied GAMYGDALA to and the potential for extensibility
suggests that \progname{}'s inclusion of a large, prefabricated component could
further increase its chances for success.

\paragraph{Separating Data Types and Functions}
Where possible, \progname{}'s module decomposition separates data types
(\mref{mIntensity}, \mref{mPADType}, \mref{mEmotionType}) from functions that
use them (\mref{mIntensityFun}, \mref{mPADFun}, \mref{mEmotionFun}). While this
increases the total modules, it also improves an end-user's ability to adopt
individual parts of \progname{} as needed (\nfref{N_Efficient},
\nfref{N_Reuse}, \nfref{N_Mod}) and change the underlying theories
(\nfref{N_Evolve}), and ensures that theory-agnostic components (e.g.
\mref{mStateType}) are separated from theory-specific ones (e.g.
\mref{mGenerate}).

\paragraph{User-Defined APIs}
\progname{} also encapsulates each Application Programming Interface (API) that
end-users must define in their own modules (\mref{mTime}, \mref{mWorld},
\mref{mAttention}, \mref{mSocial}). This allows a team to divide their
implementation between members, reduces concerns regarding compatibility with
othe \progname{} components (\nfref{N_Mod}) and allows end-users to test and
optimize them individually (\nfref{N_Efficient}, \nfref{N_Trace}).