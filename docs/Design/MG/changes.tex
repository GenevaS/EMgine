\section{Anticipated and Unlikely Changes}\label{SecChange}
Possible changes are classified into two categories based on likeliness:
Anticipated (Section~\ref{SecAchange}) and Unlikely (Section~\ref{SecUchange}).

\subsection{Anticipated Changes}\label{SecAchange}
Anticipated changes are decisions and information that are to be hidden
inside modules. Ideally, changing one of the anticipated changes will only
require changing the one module that hides the associated decision. This is an
adaptation of the \textit{design for change} approach to software design. Where
applicable, associated Likely Changes [SRS-LC] from the SRS (Ver.~\srsVersion)
appear in square brackets.

\begin{description}

    \item[\refstepcounter{acnum} \actheacnum \label{acEmotionIntensityMin}:]
    The minimum value of emotion intensity [\lcref{LC_PositiveIntensity}].

    \item[\refstepcounter{acnum} \actheacnum \label{acIntensityAlgo}:] The
    algorithm/method for calculating emotion intensity
    [\lcref{LC_Goal2Intensity}].

    \item[\refstepcounter{acnum} \actheacnum \label{acUpdateIntensityAlgo}:]
    The algorithm/method for updating emotion intensity with an intensity
    change.

    \item[\refstepcounter{acnum} \actheacnum \label{acIntensityEvalAlgo}:]
    The algorithm/method for calculating emotion intensity
    [\lcref{LC_Interest}]. Although each of \progname{}'s supported emotion
    kinds have their own intensity model, end-users do not need to know that
    they are calculated differently. They also collectively share the same
    secret: their method for evaluating emotion intensity. Therefore, these are
    associated with one AC.

    \item[\refstepcounter{acnum} \actheacnum \label{acElicitAcceptanceAlgo}:]
    The algorithm/method for calculating intensity of \textit{Acceptance}.

    \item[\refstepcounter{acnum} \actheacnum \label{acEmotionKindType}:] The
    implementation of emotion kinds data structure [\lcref{LC_EmotionPairs}].

    \item[\refstepcounter{acnum} \actheacnum \label{acAppraisalAlgo}:] The
    algorithm/method for generating emotion kinds [\lcref{LC_Subgoal},
    \lcref{LC_Goal2Emotion}, \lcref{LC_EmotionTypeIntensity}]. Although each of
    \progname{}'s supported emotion kinds have their own elicitation model,
    end-users do not need to know that they are calculated differently. They
    also collectively share the same secret: the method for emotion
    elicitation. Therefore, these are gathered into one module.

    \item[\refstepcounter{acnum} \actheacnum \label{acEmotionStateImpl}:] The
    implementation of emotion state data structure.

    \item[\refstepcounter{acnum} \actheacnum \label{acEmotionDecayStateImpl}:]
    The implementation of emotion state decay data structure.

    \item[\refstepcounter{acnum} \actheacnum \label{acDecayIntensityAlgo}:] The
    algorithm/method for decaying emotion intensity [\lcref{LC_DecaySpeed},
    \lcref{LC_Equilibrium}, \lcref{LC_DecayRate}].

    \item[\refstepcounter{acnum} \actheacnum \label{acDecayStateAlgo}:] The
    algorithm/method for decaying an emotion state.

    \item[\refstepcounter{acnum} \actheacnum \label{acEmotionState}:] The
    implementation of emotion data structure.

    \item[\refstepcounter{acnum} \actheacnum \label{acPADPointType}:] The
    implementation of PAD point data structure.

    \item[\refstepcounter{acnum} \actheacnum \label{acState2PADAlgo}:] The
    algorithm/method for converting an emotion state to a PAD point
    [\lcref{LC_EmotionTerms}, \lcref{LC_PADStats}].

    \item[\refstepcounter{acnum} \actheacnum \label{acGoalImpl}:] The
    implementation of goal data structure.

    \item[\refstepcounter{acnum} \actheacnum \label{acPlanImpl}:] The
    implementation of plan data structure.

    \item[\refstepcounter{acnum} \actheacnum \label{acAttachmentImpl}:] The
    definition and/or implementation of social attachment data structure.

    \item[\refstepcounter{acnum} \actheacnum \label{acAttentionImpl}:] The
    definition and/or implementation of attention data structure.

    \item[\refstepcounter{acnum} \actheacnum \label{acTimeImpl}:] The
    implementation of time and, consequently, delta time.

    \item[\refstepcounter{acnum} \actheacnum \label{acWorldStateImpl}:] The
    implementation of World State View (WSV) data structure.

    \item[\refstepcounter{acnum} \actheacnum \label{acWorldStateChangeImpl}:]
    The implementation of world event data structure.

    \item[\refstepcounter{acnum} \actheacnum \label{acDistanceImpl}:] The
    implementation of distance between WSVs data structure.

    \item[\refstepcounter{acnum} \actheacnum \label{acDistanceChangeImpl}:] The
    implementation of change in distance between WSVs data structure.

\end{description}

\subsection{Unlikely Changes}\label{SecUchange}
Module design should be as general as possible, but a general system is
typically more complex. Sometimes this complexity is not necessary and fixing
some design decisions at the system architecture stage can simplify the
software design. These decisions are \textit{unlikely changes} because they are
not intended to be changed due to the potentially large number of design
elements that need to be modified to accommodate them.

\progname{} does not have any known unlikely changes, as its goal is for users
to choose, organize, modify, and reuse its modules as they see fit. These needs
also motivate its conception as a library of components (see
Section~\ref{SecConnection} for reasoning).