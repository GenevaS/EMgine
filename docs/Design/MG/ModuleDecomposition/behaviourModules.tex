\subsection{Behaviour-Hiding Module}

\begin{description}[font=\scshape]
    \item[Secrets:]The contents of the required behaviours.

    \item[Services:] Includes programs that provide externally visible behaviour
    of the system as specified in the SRS (Ver.~\srsVersion). The programs in
    this module will need to change if the SRS changes.

    \item[Implemented By:] --
\end{description}

\subsubsection{Emotion Intensity Module}

\begin{description}[font=\scshape]
    \item[Secrets:] ``Virtual'' module containing the emotion intensity
    modules. These are subdivided because several other modules need the
    Emotion Intensity data type and the methods for calculating emotion
    intensity are very likely to change.

    \item[Services:] Using the Emotion Intensity data types and functions.

    \item[Implemented By:] --
\end{description}

\subsubsection{Emotion Intensity Type Module (\mref{mIntensity})}

\begin{description}[font=\scshape]
    \item[Secrets:] The internal representation of the Emotion Intensity and
    Emotion Intensity Change data types.

    \item[Services:] Stores the Emotion Intensity and Emotion Intensity Change
    data types, enforce their constraints, and provides methods for
    manipulating them.

    \item[Implemented By:] \progname{}
\end{description}

\subsubsection{Emotion Intensity Function Module (\mref{mIntensityFun})}

\begin{description}[font=\scshape]
    \item[Secrets:] The method for calculating emotion intensity.

    \item[Services:] Evaluates emotion intensity using the Entity (\mref{mGoal},
    \mref{mPlan}, \mref{mSocial}, \mref{mAttention}) and World (\mref{mTime},
    \mref{mWorld}) modules. Returns the result as an Emotion Intensity Change
    data type from the Emotion Intensity Type Module (\mref{mIntensity}).

    \item[Implemented By:] \progname{}
\end{description}

\subsubsection{Emotion State Module}

\begin{description}[font=\scshape]
    \item[Secrets:] ``Virtual'' module containing the Emotion State data type
    and generation modules. These are subdivided because the data type and
    generation modules rely on different affective theories.

    \item[Services:] Using the Emotion State data type and emotion elicitation
    functions.

    \item[Implemented By:] --
\end{description}

\subsubsection{Emotion State Type Module (\mref{mStateType})}

\begin{description}[font=\scshape]
    \item[Secrets:] The internal representation of the Emotion Kind and Emotion
    State data types.

    \item[Services:] Stores the Emotion Kind and Emotion State data types,
    enforces their constraints, and provides methods for using them.

    \item[Implemented By:] \progname{}
\end{description}

\subsubsection{Emotion Generation Module (\mref{mGenerate})}

\begin{description}[font=\scshape]
    \item[Secrets:] The method for evaluating if the current entity and world
    state elicit emotion(s).

    \item[Services:] Evaluates what emotion an entity is ``experiencing'' using
    the Entity (\mref{mGoal}, \mref{mPlan}, \mref{mSocial}, \mref{mAttention})
    and World (\mref{mTime}, \mref{mWorld}) modules. Returns the result as
    tuples of Entity and World data types from the Entity and World modules.

    \item[Implemented By:] \progname{}
\end{description}

\subsubsection{Emotion Decay Module}

\begin{description}[font=\scshape]
    \item[Secrets:] ``Virtual'' module containing the emotion decay data types
    and function modules. The Emotion Intensity Decay and Emotion State Decay
    data types are separated into different modules because Emotion State Decay
    relies on more complex data types than Emotion Intensity Decay.

    \item[Services:] Using the Emotion Decay data types and functions.

    \item[Implemented By:] --
\end{description}

\subsubsection{Emotion Intensity Decay Module (\mref{mDecay})}

\begin{description}[font=\scshape]
    \item[Secrets:] The internal representation of the Emotion Intensity Decay
    Rate data type and the decay methods. The data type and methods are together
    because they are tightly coupled due to the underlying model.

    \item[Services:] Stores the Emotion Intensity Decay Rate data type,
    enforces its constraints, and provides methods for manipulating it.
    Evaluates a ``decayed'' emotion intensity using that data type and the
    Delta Time data type from the Time module (\mref{mTime}), returned as an
    Emotion Intensity data type from the Emotion Intensity Type Module
    (\mref{mIntensity}).

    \item[Implemented By:] \progname{}
\end{description}

\subsubsection{Emotion State Decay Module (\mref{mDecayState})}

\begin{description}[font=\scshape]
    \item[Secrets:] The internal representation of the Emotion Intensity Decay
    State data type and associated decay functions. These are separated from
    the Emotion Intensity Decay module because of their dependence on the
    Emotion State data type.

    \item[Services:] Stores the Emotion Intensity Decay State data type,
    enforces its constraints, and provides methods for manipulating it.
    Evaluates a ``decayed'' emotion state using that data type and the Delta
    Time data type from the Time module (\mref{mTime}), returned as an Emotion
    State data type from the Emotion State Type Module (\mref{mStateType}).

    \item[Implemented By:] \progname{}
\end{description}

\subsubsection{PAD Module}

\begin{description}[font=\scshape]
    \item[Secrets:] ``Virtual'' module containing the PAD data type and
    function modules. These are separated into different modules because the
    data type definition is highly unlikely to change whereas the conversion
    function is likely to change.

    \item[Services:] Using the PAD data type and functions.

    \item[Implemented By:] --
\end{description}

\subsubsection{PAD Type Module (\mref{mPADType})}

\begin{description}[font=\scshape]
    \item[Secrets:] The internal representation of the PAD data type.

    \item[Services:] Stores the PAD data type, enforces its constraints, and
    provides methods for manipulating it.

    \item[Implemented By:] \progname{}
\end{description}

\subsubsection{PAD Function Module (\mref{mPADFun})}

\begin{description}[font=\scshape]
    \item[Secrets:] The method of converting Emotion State data into a PAD data.

    \item[Services:] Evaluates ``equivalent'' PAD representation of an Emotion
    State data type instance from the Emotion State Type module
    (\mref{mStateType}). Returns the result as a PAD data type from the PAD
    Type Module (\mref{mPADType}).

    \item[Implemented By:] \progname{}
\end{description}

\subsubsection{Emotion Module}

\begin{description}[font=\scshape]
    \item[Secrets:] ``Virtual'' module containing the Emotion data type and
    function modules. These are separated into different modules because the
    functions are for ease-of-use, which a user might not want to use because
    they have defined their own or they are not relevant to the task.

    \item[Services:] Using the Emotion data type and functions.

    \item[Implemented By:] --
\end{description}

\subsubsection{Emotion Type Module (\mref{mEmotionType})}

\begin{description}[font=\scshape]
    \item[Secrets:] The internal representation of the Emotion data type.

    \item[Services:] Stores the Emotion data type, enforces its constraints,
    and provides methods for manipulating it.

    \item[Implemented By:] \progname{}
\end{description}

\subsubsection{Emotion Function Module (\mref{mEmotionFun})}

\begin{description}[font=\scshape]
    \item[Secrets:] The method for evaluating the next Emotion State from an
    Emotion data type.

    \item[Services:] Calculates and adds the next Emotion State data type
    instance from the Emotion State Type Module (\mref{mStateType}) to an
    Emotion data type instance using Emotion Intensity Decay State
    (\mref{mDecayState}) and Time modules (\mref{mTime}). Returns the result as
    an Emotion data type from the Emotion Type Module (\mref{mEmotionType}).

    \item[Implemented By:] \progname{}
\end{description}

\subsubsection{Entity Module}

\begin{description}[font=\scshape]
    \item[Secrets:] ``Virtual'' module containing Entity-focused data types.
    Only the Goal module is mandatory and there are no dependencies between the
    Plan, Social Attachment, and Attention modules. Separating them into
    dedicated modules allows users to add and remove them as needed.

    \item[Services:] Using the Goal, Plan, Social Attachment, and Attention
    data types.

    \item[Implemented By:] --
\end{description}

\subsubsection{Goal Module (\mref{mGoal})}

\begin{description}[font=\scshape]
    \item[Secrets:] Implementation of the Goal data structure.

    \item[Services:] Stores the Goal data type, enforces its constraints, and
    provides methods for manipulating it.

    \item[Implemented By:] \progname{}
\end{description}

\subsubsection{Plan Module (\mref{mPlan})}

\begin{description}[font=\scshape]
    \item[Secrets:] Implementation of the Plan data structure.

    \item[Services:] Stores the Plan data type, enforces its constraints,
    and provides methods for manipulating it.

    \item[Implemented By:] \progname{}
\end{description}

\subsubsection{Social Attachment Module (\mref{mSocial})}

\begin{description}[font=\scshape]
    \item[Secrets:] Implementation of the Social Attachment data structure.

    \item[Services:] Provides an API and default implementation for the Social
    Attachment data type.

    \item[Implemented By:] \progname{}
\end{description}

\subsubsection{Attention Module (\mref{mAttention})}

\begin{description}[font=\scshape]
    \item[Secrets:] Implementation of the Attention data structure.

    \item[Services:] Provides an API and default implementation for the
    Attention data type.

    \item[Implemented By:] \progname{}
\end{description}