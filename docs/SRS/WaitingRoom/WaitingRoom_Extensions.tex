\section{Suggestions for Extending Modelled
Components}\label{appendix_extensions}

\subsection{Modelling Energy}
The type of energy (\waitref{TY_Energy}) currently represents a generic store
that an NPC can have. This can be expanded into different types of energy.
\textit{Emotional Energy} is one possible kind to include, which could affect
the updating (\iref{IM_UpdateEmotionState}) and decaying
(\iref{IM_DecayEmotionState}) of the NPC's internal emotion state. For example,
an NPC with high emotional energy has an easier time maintaining their current
emotion than one with low emotional energy.

\subsubsection{Type Definitions}
\noindent
\begin{minipage}{\textwidth}
    \renewcommand*{\arraystretch}{1.5}
    \begin{tabular}{| p{\colAwidth}  p{\colBwidth}|}
        \hline
        \rowcolor[gray]{0.9}
        \bf WR\refstepcounter{waitnum}\thewaitnum \label{TY_Energy} & \bf Type
        of Energy \\
        \hline
    \end{tabular}

    \renewcommand*{\arraystretch}{1.5}
    \begin{tabular}{ p{\colAwidth}  p{\colBwidth}}
        \bf Symbol & $\energytype$ \\

        \bf Type & -- \\\hline
    \end{tabular}
\end{minipage}

\paragraph{Description} Energy is an abstract type that is linearly ordered.
\\\hrule

\paragraph{Sources} --

\paragraph{Depends On} --

\paragraph{Ref. By} \waitref{T_CopingPotential}, \waitref{T_FearIntensity}
\\\hrule\vspace{0.5mm}\hrule

~\newline

\noindent
\begin{minipage}{\textwidth}
    \renewcommand*{\arraystretch}{1.5}
    \begin{tabular}{| p{\colAwidth}  p{\colBwidth}|}
        \hline
        \rowcolor[gray]{0.9}
        \bf WR\refstepcounter{waitnum}\thewaitnum \label{TY_EnergyChange} & \bf
        Type of Energy Change \\
        \hline
    \end{tabular}

    \renewcommand*{\arraystretch}{1.5}
    \begin{tabular}{ p{\colAwidth}  p{\colBwidth}}
        \bf Symbol & $\energychangetype$ \\

        \bf Type & -- \\\hline
    \end{tabular}
\end{minipage}

\paragraph{Description} Energy change is an abstract type representing a
magnitude.
\\\hrule

\paragraph{Sources} --

\paragraph{Depends On} --

\paragraph{Ref. By} \waitref{TY_Action}, \waitref{T_CopingPotential} (via
\waitref{TY_Action}) \\\hrule\vspace{0.5mm}\hrule

\subsection{Modelling Actors and Actions}

\subsubsection{Type Definitions}

\noindent
\begin{minipage}{\textwidth}
    \renewcommand*{\arraystretch}{1.5}
    \begin{tabular}{| p{\colAwidth}  p{\colBwidth}|}
        \hline
        \rowcolor[gray]{0.9}
        \bf WR\refstepcounter{waitnum}\thewaitnum \label{TY_Actor} & \bf Type
        of Actor \\
        \hline
    \end{tabular}

    \renewcommand*{\arraystretch}{1.5}
    \begin{tabular}{ p{\colAwidth}  p{\colBwidth}}
        \bf Symbol & $\actortype$ \\

        \bf Type & -- \\\hline
    \end{tabular}
\end{minipage}

\paragraph{Description} Actor is an abstract type representing actors in the
game environment. Actors are game entities that are able to act on the game
environment, and include the player, NPCs, and non-sentient environment
elements (e.g. a storm). \\\hrule

\paragraph{Sources} --

\paragraph{Depends On} --

\paragraph{Ref. By} \waitref{T_Accountability}, \waitref{T_AngerIntensity}
\\\hrule\vspace{0.5mm}\hrule

~\newline

\noindent
\begin{minipage}{\textwidth}
    \renewcommand*{\arraystretch}{1.5}
    \begin{tabular}{| p{\colAwidth}  p{\colBwidth}|}
        \hline
        \rowcolor[gray]{0.9}
        \bf WR\refstepcounter{waitnum}\thewaitnum \label{TY_Action} & \bf Type
        of Action \\
        \hline
    \end{tabular}

    \renewcommand*{\arraystretch}{1.5}
    \begin{tabular}{ p{\colAwidth}  p{\colBwidth}}
        \bf Symbol & $ \actiontype $ \\

        \bf Type & $ \left\{ \mathtt{makesChange} : \worldstatetype \rightarrow
        \worldstatechangetype, \mathtt{successLikelihood} : \probabilitytype,
        \mathtt{energyCost} : \energychangetype \right\} $ \\
        \hline
    \end{tabular}
\end{minipage}

\paragraph{Description} An action ($\actiontype$) is a record containing its
effect on the world state ($\mathtt{makesChange} : \worldstatetype \rightarrow
\worldstatechangetype$), its likelihood of being successful
($\mathtt{successLikelihood} : \probabilitytype$), and the amount of energy
that is needed to execute it ($\mathtt{energyCost} : \energychangetype$).
\\\hrule

\paragraph{Sources} --

\paragraph{Depends On} \tyref{TY_WorldState}, \tyref{TY_WorldStateChange},  
\waitref{TY_EnergyChange}

\paragraph{Ref. By} \waitref{T_Accountability}, \waitref{T_CopingPotential},
\waitref{T_FutureExpectancy}, \waitref{T_FearIntensity},
\waitref{T_AngerIntensity}, \waitref{I_CopingPotential}
\\\hrule\vspace{0.5mm}\hrule

\subsection{Coping Potential}
The model for \textit{Coping Potential} (\waitref{T_CopingPotential}) is
intended to evaluate individual actions that an NPC can take, but a game
designer might want to create a plan with these actions. A plan carries
more weight, but the weight might not be equal to the sum of weights for
each action. Two models have been created for this based on a separate
conceptual model, representing empirically collected
data~\citep{copingquestions}, which can be realized as types (TY) or data
definitions (DD).

\subsubsection{Conceptual Models}

\noindent
\begin{minipage}{\textwidth}
    \renewcommand*{\arraystretch}{1.5}
    \begin{tabular}{| p{\colAwidth}  p{\colBwidth}|}
        \hline
        \rowcolor[gray]{0.9}
        \bf WR\refstepcounter{waitnum}\thewaitnum \label{C_Coping}&\bf Coping
        Strategies\\\hline
    \end{tabular}
\end{minipage}

\paragraph{Description} Coping strategies are ill-defined classes of
related
behaviours developed to address a general collection of related stimuli and
scenarios. Using a strategy changes the individual-environment
relationship,
which can change the individual's emotional state by altering the original
appraisal (\waitref{C_Appraisal}). There are two types of coping:
\begin{itemize}
    \item Problem-focused: Acting directly on the environment to manage
    demands and change the individual-environment relationship, creating new
    information for Appraisal. These strategies can be risky and expensive, and
    are closely related to the individual's power and control over the
    environment.

    \item Emotion-focused: Changing the individual's internal representations
    of the individual-environment relationship---changing how the available
    information is perceived---in order to reduce or eliminate perceived harms
    and threats. These strategies include changing or omitting information or
    redirecting attention (\waitref{C_Attend}). In extreme cases, the individual
    alters or changes their goals (\cref{C_Goals}).
\end{itemize} \hrule

\paragraph{Source} \cite{lazarus1991emotion}

\paragraph{Depends On} --

\paragraph{Ref. By} \waitref{WR_CopingTypes}, \waitref{WR_CopingSubtypes}, 
\waitref{C_SA}, \waitref{C_Fear}, \waitref{C_Anger}
\\\hrule\vspace{0.5mm}\hrule

\subsubsection{Type Definitions}

\noindent
\begin{minipage}{\textwidth}
    \renewcommand*{\arraystretch}{1.5}
    \begin{tabular}{| p{\colAwidth}  p{\colBwidth}|}
        \hline
        \rowcolor[gray]{0.9}
        \bf WR\refstepcounter{waitnum}\thewaitnum \label{WR_CopingTypes} & \bf
        Representation of Coping Types \\
        \hline
    \end{tabular}

    \renewcommand*{\arraystretch}{1.5}
    \begin{tabular}{ p{\colAwidth}  p{\colBwidth}}
        \bf Symbol &$\mathbb{CP}$\\

        \bf Type & $\left(name: \stringtype, value: \mathbb{R}\right)$ \\

        \bf Equation&$\mathbb{CP} = \mathtt{CopingType}$,
        where \\
        & $\mathtt{CopingType} \in \left\{
        \begin{array}{l}
            (\mathtt{Confrontive}, 0.70), (\mathtt{Distance}, 0.61),
            (\mathtt{Self-Control}, 0.70), \\
            (\mathtt{SeekSocialSupport}, 0.76),
            (\mathtt{AcceptResponsibility},
            0.66), \\
            (\mathtt{EscapeAvoid}, 0.72), (\mathtt{PlanfulProblemSolving},
            0.68), \\
            (\mathtt{Positive Reappraisal}, 0.79)
        \end{array}  \right\}$ \\
        \hline
    \end{tabular}
\end{minipage}

\paragraph{Description} Coping types are a set of coping action classes
(\waitref{C_Coping}) and their value based on the ``Ways of Coping'' factor
analysis. Each class has a general theme of coping that can be acted on in
different ways. \\\hrule

\paragraph{Sources} \cite{copingquestions}

\paragraph{Depends On} \waitref{C_Coping}

\paragraph{Ref. By} \waitref{WR_CopingSubtypes} \\\hrule\vspace{0.5mm}\hrule

~\newline

\noindent
\begin{minipage}{\textwidth}
    \renewcommand*{\arraystretch}{1.5}
    \begin{tabular}{| p{\colAwidth}  p{\colBwidth}|}
        \hline
        \rowcolor[gray]{0.9}
        \bf WR\refstepcounter{waitnum}\thewaitnum \label{WR_CopingSubtypes} &
        \bf Representation of Coping Subtypes \\
        \hline
    \end{tabular}

    \renewcommand*{\arraystretch}{1.5}
    \begin{tabular}{ p{\colAwidth}  p{\colBwidth}}
        \bf Symbol &$\mathbb{CP_{<:}}$\\

        \bf Type & $\left(name : \stringtype, value : \mathbb{R}\right)$ \\

        \bf Equation&$\mathbb{CP_{<:}} = \mathtt{CS_{>:}}$,
        where \\
        & $\mathtt{CS_{Confrontive}} \in \left\{
        \begin{array}{l}
            (\mathtt{StandGround}, 0.70), (\mathtt{ChangeAccountablesMind},
            0.62),
            \\
            (\mathtt{ExpressAnger}, 0.61), (\mathtt{LetFeelingsOut}, 0.58),
            \\
            (\mathtt{TakeChanceOrRisk}, 0.32), (\mathtt{UnlikelyOutcome},
            0.30)
        \end{array}  \right\}$ \\

        & $\mathtt{CS_{Distance}} \in \left\{
        \begin{array}{l}
            (\mathtt{RefuseSeriousness}, 0.55), (\mathtt{IgnoreSituation},
            0.54),
            \\
            (\mathtt{RefuseToDwell}, 0.50), (\mathtt{AttemptForget}, 0.50),
            \\
            (\mathtt{SeekBrightSide}, 0.34), (\mathtt{AcceptFate}, 0.25)
        \end{array}  \right\}$ \\

        & $\mathtt{CS_{Self-Control}} \in \left\{
        \begin{array}{l}
            (\mathtt{BottleFeelings}, 0.55), (\mathtt{HideFeelings}, 0.46),
            \\
            (\mathtt{SaveRelationship}, 0.40), (\mathtt{ReconsiderPlan},
            0.40), \\
            (\mathtt{StopInterference}, 0.37), (\mathtt{ImagineRoleModel},
            0.37), \\
            (\mathtt{SeeOtherView}, 0.28)
        \end{array}  \right\}$ \\

        & $\mathtt{CS_{SeekSocialSupport}} \in \left\{
        \begin{array}{l}
            (\mathtt{GetInformation}, 0.73), (\mathtt{GetConcreteHelp},
            0.68), \\
            (\mathtt{GetAdvice}, 0.58), (\mathtt{DiscussFeelings}, 0.57),
            \\
            (\mathtt{AcceptSYmpathy}, 0.56), (\mathtt{ProfessionalHelp},
            0.45)
        \end{array}  \right\}$ \\

        & $\mathtt{CS_{AcceptResponsibility}} \in \left\{
        \begin{array}{l}
            (\mathtt{CritisizeSelf}, 0.71), (\mathtt{RealizeOwnRole},
            0.68), \\
            (\mathtt{PromiseToChange}, 0.49), (\mathtt{Apologize}, 0.39),
            \\
        \end{array}  \right\}$ \\

        & $\mathtt{CS_{EscapeAvoid}} \in \left\{
        \begin{array}{l}
            (\mathtt{WishAway}, 0.66), (\mathtt{HopeForMiracle}, 0.55), \\
            (\mathtt{Fantasize}, 0.54), (\mathtt{Overindulge}, 0.49), \\
            (\mathtt{AvoidOthers}, 0.46), (\mathtt{Denial}, 0.42), \\
            (\mathtt{TargetOthers}, 0.40), (\mathtt{Oversleep}, 0.36)
        \end{array}  \right\}$ \\

        & $\mathtt{CS_{PlanfulProblemSolving}} \in \left\{
        \begin{array}{l}
            (\mathtt{DoubleEffort}, 0.71), (\mathtt{MakePlan}, 0.61), \\
            (\mathtt{NextStep}, 0.45), (\mathtt{ImproveSituation}, 0.44),
            \\
            (\mathtt{Experience}, 0.40), (\mathtt{MultipleSolutions}, 0.38)
        \end{array}  \right\}$ \\

        & $\mathtt{CS_{PositiveReappraisal}} \in \left\{
        \begin{array}{l}
            (\mathtt{GoodPersonalChange}, 0.79),
            (\mathtt{GainedExperience}, 0.67),
            \\
            (\mathtt{NewFaith}, 0.64), (\mathtt{RediscoverImportance},
            0.64), \\
            (\mathtt{Pray}, 0.56), (\mathtt{ChangeSelf}, 0.55), \\
            (\mathtt{InspiredCreativity}, 0.43)
        \end{array}  \right\}$ \\
        \hline
    \end{tabular}
\end{minipage}

\paragraph{Description} Coping subtypes are more concrete instances of
coping types (\waitref{WR_CopingTypes}) with their own values that reflect
their relative importance to other subtypes of the same super type. Values
are based on the ``Ways of Coping'' factor analysis. \\\hrule

\paragraph{Sources} \cite{copingquestions}

\paragraph{Depends On} \waitref{C_Coping}

\paragraph{Ref. By} -- \\\hrule\vspace{0.5mm}