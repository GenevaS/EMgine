\section{Suggestions for Other Cognitive Architecture Module
Specifications}\label{appendix_cognitive}
These are preliminary specifications for types and models that are not
within the scope of \progname{}, but are necessary for it to function. They
are to be seen only as suggestions when specifying these supporting modules.

\subsection{Defining Lists of Goals}
This is a simple specification of an indexed list of items with the type of
Goal (\tyref{TY_Goal}).

\subsubsection{Type Definitions}

\noindent
\begin{minipage}{\textwidth}
    \renewcommand*{\arraystretch}{1.5}
    \begin{tabular}{| p{\colAwidth}  p{\colBwidth}|}
        \hline
        \rowcolor[gray]{0.9}
        \bf WR\refstepcounter{waitnum}\thewaitnum \label{TY_GoalIndex} & \bf
        Type of Goal Labels \\
        \hline
    \end{tabular}

    \renewcommand*{\arraystretch}{1.5}
    \begin{tabular}{ p{\colAwidth}  p{\colBwidth}}
        \bf Symbol & $ \goallabeltype $ \\

        \bf Type & -- \\
        \hline
    \end{tabular}
\end{minipage}

\paragraph{Description} An ordered set of labels. These will typically be
the
set $[1, 2, ... , n - 1, n]$. \\\hrule

\paragraph{Sources} --

\paragraph{Depends On} --

\paragraph{Ref. By} \waitref{TY_GoalSet}
\\\hrule\vspace{0.5mm}\hrule

~\newline

\noindent
\begin{minipage}{\textwidth}
    \renewcommand*{\arraystretch}{1.5}
    \begin{tabular}{| p{\colAwidth}  p{\colBwidth}|}
        \hline
        \rowcolor[gray]{0.9}
        \bf WR\refstepcounter{waitnum}\thewaitnum \label{TY_GoalSet} & \bf Type
        of Indexed Goal Set \\
        \hline
    \end{tabular}

    \renewcommand*{\arraystretch}{1.5}
    \begin{tabular}{ p{\colAwidth}  p{\colBwidth}}
        \bf Symbol & $ \indexsettype $ \\

        \bf Type & $ \goallabeltype \rightarrow \goaltype $ \\
        \hline
    \end{tabular}
\end{minipage}

\paragraph{Description} An indexed goal set is an assignment of $\goaltype$
to
the labels $\goallabeltype$. Each element in $\goallabeltype$ is uniquely
assigned a $\goaltype$.
\\\hrule

\paragraph{Sources} --

\paragraph{Depends On} \tyref{TY_Goal}, \waitref{TY_GoalIndex}

\paragraph{Ref. By} -- \\\hrule\vspace{0.5mm}\hrule

\subsection{Evaluating the Static Importance of a Goal}
The preliminary specification for evaluating the importance of a Goal
(\tyref{TY_Goal}) is inspired by \cite{lazarus1991emotion}, which relates
the importance of goals to an individual's ego-identity.

\subsubsection{Assumptions}
\begin{itemize}
    \item[A\refstepcounter{assumpnum}\theassumpnum\label{A_PAEgoOrder}:] Ego
    types have an order of importance [\waitref{TY_EgoIdentity}].
\end{itemize}

\subsubsection{Conceptual Models}

\noindent
\begin{minipage}{\textwidth}
    \renewcommand*{\arraystretch}{1.5}
    \begin{tabular}{| p{\colAwidth}  p{\colBwidth}|}
        \hline
        \rowcolor[gray]{0.9}
        \bf WR\refstepcounter{waitnum}\thewaitnum\label{C_Ego} & \bf Ego
        Identity \\\hline
    \end{tabular}
\end{minipage}

\paragraph{Description} Ego-identity is a representation of the individual
in the world, encompassing roles, relationships, and societal functions. It is
used to categorize, organize, and prioritize personal goals (\cref{C_Goals}).

There are seven types of ego-identity: self esteem, social esteem, moral
values, ego-ideals, meanings and ideas, other persons and their well-being,
and life goals. \\\hrule

\paragraph{Source} \citet[p.~101--102]{lazarus1991emotion}

\paragraph{Depends On} --

\paragraph{Ref. By} \waitref{TY_Ego}, \waitref{TY_EgoIdentity},
\waitref{TY_GoalEgoRelation}, \waitref{T_GoalPriority}
\\\hrule\vspace{0.5mm}\hrule

~\newline
\subsubsection{Type Definitions}

\noindent
\begin{minipage}{\textwidth}
    \renewcommand*{\arraystretch}{1.5}
    \begin{tabular}{| p{\colAwidth}  p{\colBwidth}|}
        \hline
        \rowcolor[gray]{0.9}
        \bf WR\refstepcounter{waitnum}\thewaitnum \label{TY_Ego} & \bf Type of
        Ego \\
        \hline
    \end{tabular}

    \renewcommand*{\arraystretch}{1.5}
    \begin{tabular}{ p{\colAwidth}  p{\colBwidth}}
        \bf Symbol & $\egotype$ \\

        \bf Type & $ \{ \mathtt{SelfEsteem}, \mathtt{SocialEsteem},
        \mathtt{Morals}, \mathtt{Ideals}, \mathtt{Meaning},
        \mathtt{WellBeingOfOthers}, $ \\
        & $ \mathtt{LifeGoals} \} $ \\

        \bf Invariants & $\egotype$ is finite and ordered \\
        \hline
    \end{tabular}
\end{minipage}

\paragraph{Description} Ego is a set of labels representing the ways an
individual can view their relationship with the world itself and the
entities
in it. \\\hrule

\paragraph{Sources} --

\paragraph{Depends On} \waitref{C_Ego}

\paragraph{Ref. By} \waitref{TY_EgoIdentity}, \waitref{TY_GoalEgoRelation},
\waitref{T_GoalPriority}
\\\hrule\vspace{0.5mm}\hrule

~\newline

\noindent
\begin{minipage}{\textwidth}
    \renewcommand*{\arraystretch}{1.5}
    \begin{tabular}{| p{\colAwidth}  p{\colBwidth}|}
        \hline
        \rowcolor[gray]{0.9}
        \bf WR\refstepcounter{waitnum}\thewaitnum \label{TY_EgoIdentity} & \bf
        Type of Ego Identity \\\hline
    \end{tabular}

    \renewcommand*{\arraystretch}{1.5}
    \begin{tabular}{ p{\colAwidth}  p{\colBwidth}}
        \bf Symbol & $\egoidentitytype$ \\

        \bf Type & $ \left\{ \mathtt{importance} : \egotype \rightarrow
        \mathbb{R^+} \right\} $ \\

        \hline
    \end{tabular}
\end{minipage}

\paragraph{Description} Each element in $\egotype$ has a importance
function
($\mathtt{importance}$) that returns a non-negative real value denoting its
relative weight compared to other elements in $\egotype$
(\aref{A_PAEgoOrder}).
\\\hrule

\paragraph{Sources} --

\paragraph{Depends On} \aref{A_PAEgoOrder}, \waitref{C_Ego}, \waitref{TY_Ego}

\paragraph{Ref. By} \waitref{T_GoalPriority} \\\hrule\vspace{0.5mm}\hrule

~\newline

\noindent
\begin{minipage}{\textwidth}
    \renewcommand*{\arraystretch}{1.5}
    \begin{tabular}{| p{\colAwidth}  p{\colBwidth}|}
        \hline
        \rowcolor[gray]{0.9}
        \bf WR\refstepcounter{waitnum}\thewaitnum\label{TY_GoalEgoRelation} &
        \bf Type of the Goal-Ego Importance \\
        \hline
    \end{tabular}

    \renewcommand*{\arraystretch}{1.5}
    \begin{tabular}{ p{\colAwidth}  p{\colBwidth}}
        \bf Symbol & $ \goalegotype $ \\

        \bf Type & $ \goaltype \times \egotype \rightarrow [ 0, 1 ] $ \\

        \hline
    \end{tabular}
\end{minipage}

\paragraph{Description} The goal-ego importance describes the value each
$\goaltype$ has with respect to an ego type $o : \egotype$, described as a
weight in $[0, 1]$  (\waitref{C_Ego}). \\\hrule

\paragraph{Sources} --

\paragraph{Depends On} \tyref{TY_Goal}, \waitref{C_Ego}, \waitref{TY_Ego}

\paragraph{Ref. By} \waitref{T_GoalPriority} \\\hrule\vspace{0.5mm}\hrule

~\newline
\subsubsection{Theoretical Models}

\noindent
\begin{minipage}{\textwidth}
    \renewcommand*{\arraystretch}{1.5}
    \begin{tabular}{| p{\colAwidth}  p{\colBwidth}|}
        \hline
        \rowcolor[gray]{0.9}
        \bf WR\refstepcounter{waitnum}\thewaitnum \label{T_GoalPriority}
        &
        \bf Evaluating Static Goal Importance \\
        \hline
    \end{tabular}

    \renewcommand*{\arraystretch}{1.5}
    \begin{tabular}{ p{\colAwidth}  p{\colBwidth}}
        \bf Input & $g : \goaltype$, $ge : \goalegotype$, $id :
        \egoidentitytype$ \\

        \bf Output & $\mImportance \defEq \sum_{e \in \egotype} ge(g, e)
        \cdot id.\mathtt{importance}(e)$ \\
        \hline
    \end{tabular}
\end{minipage}

\paragraph{Description} The static importance of $g$ given by the sum of
weights for each type of ego proportional to how much the goal impacts an
ego type and the importance of the ego type to the NPC ($\sum_{e \in
    \egotype} ge(g, e) \cdot id.\mathtt{importance}(e)$).

This can be combined with the size of impact that an event has on $g$ to
get an in-the-moment importance of $g$. \\\hrule

\paragraph{Sources} \citet[p.~94--98]{lazarus1991emotion}

\paragraph{Depends On} \tyref{TY_Goal}, \waitref{C_Ego}, \waitref{TY_Ego},
\waitref{TY_EgoIdentity} \waitref{TY_GoalEgoRelation}

\paragraph{Ref. By} -- \\\hrule\vspace{0.5mm}\hrule

\subsection{Modelling Attention}
Attention is an independent component of cognition that an influence and be
influenced by emotion.

\subsubsection{Conceptual Model}
\noindent
\begin{minipage}{\textwidth}
    \renewcommand*{\arraystretch}{1.5}
    \begin{tabular}{| p{\colAwidth}  p{\colBwidth}|}
        \hline
        \rowcolor[gray]{0.9}
        \bf WR\refstepcounter{waitnum}\thewaitnum \label{C_Attend} &\bf
        Attention \\\hline
    \end{tabular}
\end{minipage}

\paragraph{Description} The amount of information gathered by the senses is
usually enormous and the individual is unable to process it all in an efficient
manner. The attention mechanism evolved to filter, classify, and amplify
incoming information to manage this volume and ensure that the most important
information is handled first. Information relevant to an individual's goals
(\cref{C_Goals}) must be reliably identified in order to produce behaviours
that increase the likelihood of achieving them. Novel and unexpected stimuli
prompt a quicker response to maximize the amount of available information for
cognitive processes. Attention can also be influenced by perceived personal
relevance, familiarity, and the current emotion state (\cref{C_Emotion}).
\\\hrule

\paragraph{Source} \cite{robert1980emotion, lazarus1991emotion,
    slavin2012educational}

\paragraph{Depends On} --

\paragraph{Ref. By} \waitref{TY_Attend}, \waitref{C_Interest},
\waitref{C_Surprise}
\\\hrule\vspace{0.5mm}\hrule

\subsubsection{Type Definitions}

\noindent
\begin{minipage}{\textwidth}
    \renewcommand*{\arraystretch}{1.5}
    \begin{tabular}{| p{\colAwidth}  p{\colBwidth}|}
        \hline
        \rowcolor[gray]{0.9}
        \bf WR\refstepcounter{waitnum}\thewaitnum \label{TY_Attend} & \bf
        Type of Attention \\
        \hline
    \end{tabular}

    \renewcommand*{\arraystretch}{1.5}
    \begin{tabular}{ p{\colAwidth}  p{\colBwidth}}
        \bf Symbol & $ \attentiontype $ \\

        \bf Type & $ \Bigg\{ \mathtt{focus} : \{ [0,1]_0 : \mathbb{R}, [0,1]_1
        : \mathbb{R}, ..., [0,1]_n : \mathbb{R} \},
        \mathtt{novelty} : \{ \mathbb{R}, ..., \mathbb{R} \}, $\\
        & $\mathtt{familiarity} : \dfrac{1}{\mathtt{novelty}} \Bigg\} $ \\

        \bf Invariant & The sum of the elements in $\mathtt{focus}$ is less
        than 1 \\
        & There is a matching element in $\mathtt{novelty}$ for every element
        in $\mathtt{focus}$ \\
        \hline
    \end{tabular}
\end{minipage}

\paragraph{Description} Attention ($\attentiontype$) is a record of three,
equal length vectors representing elements in the game environment:
\begin{itemize}
    \item The $\mathtt{focus}$ vector represents how much something in the game
    environment is taking of the NPC's attention as a percentage. The sum of
    values in $\mathtt{focus}$ cannot exceed 1, or more attention is being used
    than is available.

    \item The $\mathtt{novelty}$ vector, which corresponds to the elements in
    $\mathtt{focus}$ represents how unfamiliar the NPC is with that element in
    the game environment.

    \item The $\mathtt{familiarity}$ vector, derived from the
    $\mathtt{novelty}$ vector, representing how familiar an NPC is with the
    element in the game environment
\end{itemize}\hrule

\paragraph{Sources} --

\paragraph{Depends On} \waitref{C_Attend}

\paragraph{Ref. By} -- \\\hrule\vspace{0.5mm}\hrule

\subsection{Modelling Social Relationships}
Social relationships are prominent elements in CTE. The components needed to
model and maintain them could stand as an independent unit.

\subsubsection{Conceptual Models}

\noindent
\begin{minipage}{\textwidth}
    \renewcommand*{\arraystretch}{1.5}
    \begin{tabular}{| p{\colAwidth}  p{\colBwidth}|}
        \hline
        \rowcolor[gray]{0.9}
        \bf WR\refstepcounter{waitnum}\thewaitnum \label{C_Relation} & \bf
        Social Relationship \\\hline
    \end{tabular}
\end{minipage}

\paragraph{Description} A relationship can be defined by three
factors---\textit{volatility}, \textit{dependability}, and
\textit{faith}---which are treated as subjective probability judgements. The
factors are weighted based on the security and confidence that the individual
has in the relationship. A relationship's security and confidence is built over
time via mutually satisfying interactions where the partner demonstrates their
benevolence and honesty to the individual. These are tied directly to its
component factors of volatility, dependability, and faith.
\begin{itemize}
    \item In the beginning, the \textit{volatility} factor is key as the
    partner builds confidence by acting in predictable ways that demonstrate
    care for the individual. At this stage, relationship confidence is
    negatively impacted by volatile actions as it is difficult for the
    individual to make accurate predictions about the partner's actions
    and intentions.

    \item In an established relationship, the partner's \textit{dependability}
    is developed with each accurate prediction about events containing
    increasing levels of personal risk where they influenced the
    outcome. Shared goals, similar beliefs, and attributing part of a
    personal gain to the partner's involvement all increase
    dependability. While self-sacrificing actions quickly increase the
    strength of a relationship, mutual exchanges can also increase a
    relationship's confidence if the individual's self-esteem is
    reinforced by their partner.

    \item As an evaluation of the relationship itself, \textit{faith} directly
    increases the confidence and security of the relationship
    proportional to the size of the risk taken after an accurate
    predictions is made. An inaccurate prediction surely decreases
    confidence in the relationship, but the size and nature of this
    effect is unclear.
\end{itemize}

After an event where another player impacted the outcome, the associated
relationship is updated: positively if a benefit was obtained and negative if a
loss was incurred. The size of the impact is determined by the size of gain or
loss, regardless of the accuracy of the prediction. \\\hrule

\paragraph{Source} \cite{rempel1985trust}

\paragraph{Depends On} --

\paragraph{Ref. By} \waitref{TY_SocialRelation}, \waitref{C_Trust}
\\\hrule\vspace{0.5mm}\hrule

\subsubsection{Type Definitions}

\noindent
\begin{minipage}{\textwidth}
    \renewcommand*{\arraystretch}{1.5}
    \begin{tabular}{| p{\colAwidth}  p{\colBwidth}|}
        \hline
        \rowcolor[gray]{0.9}
        \bf WR\refstepcounter{waitnum}\thewaitnum \label{TY_SocialRelation} &
        \bf Type of a Social Relationship \\
        \hline
    \end{tabular}

    \renewcommand*{\arraystretch}{1.5}
    \begin{tabular}{ p{\colAwidth}  p{\colBwidth}}
        \bf Symbol & $ \socialrelationtype $ \\

        \bf Equation & $ \{ \mathtt{volatility} : \mathbb{R},
        \mathtt{dependability} : \mathbb{R},
        \mathtt{faith} : \mathbb{R^+}, $ \\
        & $\mathtt{security} : \mathtt{volatility} \times
        \mathtt{dependability} \times \mathtt{faith} \rightarrow \mathbb{R^+},$
        \\
        & $\mathtt{confidence} : \mathtt{volatility} \times
        \mathtt{dependability} \times \mathtt{faith} \rightarrow \mathbb{R^+}
        \} $
        \\
        \hline
    \end{tabular}
\end{minipage}

\paragraph{Description} The social relationship type ($\socialrelationtype$)
represents the relationship between two NPCs (\waitref{C_Relation}) defined by:
\begin{itemize}
    \item Values representing the perceived $\mathtt{volatility}$ and
    $\mathtt{dependability}$ of the other NPC's behaviours with respect to this
    NPC's well-being as a real value,

    \item A non-zero, real value for $\mathtt{faith}$ which represents the
    perceived ``goodness'' of the relationship,

    \item A $\mathtt{security}$ value, a non-negative real value that is a
    function of $\mathtt{volatility}$, $\mathtt{dependability}$, and
    $\mathtt{faith}$, where the value of $\mathtt{security}$ is inversely
    proportional to the belief that the other NPC wishes harm on the NPC
    holding this relationship, and

    \item A $\mathtt{confidence}$ value, a non-negative real value that is a
    function of $\mathtt{volatility}$, $\mathtt{dependability}$, and
    $\mathtt{faith}$, where a higher value represents a greater confidence
    in the perceived plausibility of this relationship's representation,
\end{itemize}

The $\mathtt{security}$ and $\mathtt{confidence}$ functions can have different
definitions between NPCs. There outputs must always depend on the
$\mathtt{volatility}$, $\mathtt{dependability}$, and $\mathtt{faith}$
variables. \\\hrule

\paragraph{Sources} --

\paragraph{Depends On} \waitref{C_Relation}

\paragraph{Ref. By} \waitref{T_AngerIntensity} \\\hrule\vspace{0.5mm}\hrule