\section{Suggestions for Modelling Lazarus's Cognitive
Appraisal}\label{appendix_lazarus}

\subsection{Assumptions}

\begin{itemize}
    \item[A\refstepcounter{assumpnum}\theassumpnum \label{A_AchievingGoals}:]
    Changing the completion status of a goal is worth more than a change that
    does not [\waitref{T_GoalCongruence}].

    \item[A\refstepcounter{assumpnum}\theassumpnum \label{A_ActOn}:]
    Only one NPC action can be executed at a time [\waitref{T_CopingPotential}].

    \item[A\refstepcounter{assumpnum}\theassumpnum \label{A_AppraisalTime}:]
    Appraisals operate on the scale of time steps [\waitref{T_CopingPotential}].

    \item[A\refstepcounter{assumpnum}\theassumpnum \label{A_emotionRepr}:]
    Each emotion category can be uniquely identified by a combination of
    appraisal values [\waitref{T_AppraisalCA}].

    \item[A\refstepcounter{assumpnum}\theassumpnum \label{A_EgoIdentity}:]
    If ego involvement is listed as ego-identity, any of the ego types can be
    used as well as references to relationships and knowledge
    [\waitref{T_AppraisalCA}].

    \item[A\refstepcounter{assumpnum}\theassumpnum \label{A_PAEgo}:]
    The ego type used in an appraisal is tied to the affected goal
    [\waitref{T_AppraisalCA}].

    \item[A\refstepcounter{assumpnum}\theassumpnum \label{A_EgoAffirmation}:]
    Confirming the importance of an ego type, goal, relationship, or other type
    of knowledge is a praise-worthy act
\end{itemize}

\subsection{Conceptual Models}
~\newline

\noindent
\begin{minipage}{\textwidth}
    \renewcommand*{\arraystretch}{1.5}
    \begin{tabular}{| p{\colAwidth}  p{\colBwidth}|}
        \hline
        \rowcolor[gray]{0.9}
        \bf WR\refstepcounter{waitnum}\thewaitnum \label{C_Appraisal} &
        \bf Appraisal \\\hline
    \end{tabular}
\end{minipage}

\paragraph{Description} In CA, an appraisal assigns personal
significance to the individual-environment relationship. It integrates
objective, but not always truthful, knowledge with the individual's goals
equally via cognitive processes. Both knowledge and Goals (\cref{C_Goals}) are
necessary in appraisal and make the process faster and more selective.

Appraisal is divided into two units, primary (\waitref{C_PA})
and secondary (\waitref{C_SA}), whose responses represent knowledge that
have been assigned personal significance. These are mapped to a core relational
theme and the associated emotional response pattern.

\begin{table}[H]
    \centering
    \footnotesize
    \begin{threeparttable}
        \begin{tabular}{@{}lccl|lll@{}}
            \toprule
            & \multicolumn{3}{c|}{\textbf{Primary} (\waitref{C_PA})} &
            \multicolumn{3}{c}{\textbf{Secondary} (\waitref{C_SA})} \\
            & \textbf{Rel.} & \textbf{Congr.} & \textbf{Ego} &
            \textbf{Acc.} & \textbf{CP} & \textbf{FE} \\ \midrule

            \textbf{Fear} (\waitref{C_Fear}) & Y* & -*\textpmhg{\Hc} &
            Identity, Meaning* & N/A & Uncertain & Uncertain \\

            \textbf{Anger} (\waitref{C_Anger}) & Y* & -* & Esteem (Self,
            Social)* & Blame* & Favours \textit{Attack}! & Improve with
            \textit{Attack}! \\

            \textbf{Sadness} (\waitref{C_Sadness}) & Y* & -* & Any* & None* &
            Unfavourable* & Potential to improve\textpmhg{\HW} \\

            \textbf{Joy} (\waitref{C_Joy}) & Y* & +* & Any & N/A & N/A &
            Favours continuation*\textpmhg{\Hi} \\

            \textbf{Disgust} (\waitref{C_Disgust}) & Y* & -* & Any* & N/A & N/A
            & N/A \\

            \textbf{Trust} (\waitref{C_Trust}) & Y* & +* &
            Identity*\textpmhg{\Hibp} & N/A & N/A & Favours
            continuation\textpmhg{\Hi} \\
            \midrule\bottomrule
        \end{tabular}

        \begin{tablenotes}
            \item [*] Sufficient and necessary to the appraisal.

            \item [!] Facilitates emotion.

            \item [\textpmhg{\Hc}] Expected congruence.

            \item [\textpmhg{\HW}] Associates it with hope. If unfavourable, it
            is associated with hopelessness and depression. CA sees
            \textit{Sadness} and depression as different states.

            \item [\textpmhg{\Hi}] If unfavourable or uncertain, the emotion is
            muted or undermined.

            \item [\textpmhg{\Hibp}] Affirmed by another individual (``...a
            desire for mutual affection, which is affirming to our
            ego-identity...'' \cite[p.~278]{lazarus1991emotion}.).
        \end{tablenotes}
    \end{threeparttable}
\end{table}

Appraisals are not necessarily sequential in nature, but are ordered by their
relative importance---if the results of primary appraisal determine the event
to be irrelevant, then secondary appraisal has limited, if any, value.
Appraisal questions can be answered in any order and are used to filter through
potential emotions in structure reminiscent of a decision tree, from least to
most specific emotion, starting with its valence.

While not disagreeing with the purpose of \textit{Interest} or
\textit{Surprise} in PES, CA does not consider them full emotions. They are
instead considered to be states of arousal that watches and waits for
additional information for the appraisal process and does not have an
appraisal pattern itself.\\\hrule

\paragraph{Source} \cite{lazarus1991emotion}

\paragraph{Depends On} \cref{C_Goals}, \waitref{C_PA}, \waitref{C_SA}

\paragraph{Ref. By} --
\\\hrule\vspace{0.5mm}\hrule

~\newline

\noindent
\begin{minipage}{\textwidth}
    \renewcommand*{\arraystretch}{1.5}
    \begin{tabular}{| p{\colAwidth}  p{\colBwidth}|}
        \hline
        \rowcolor[gray]{0.9}
        \bf WR\refstepcounter{waitnum}\thewaitnum \label{C_PA} & \bf
        Primary
        Appraisal \\\hline
    \end{tabular}
\end{minipage}

\paragraph{Description} In CA, primary appraisal -- which can be
influenced by needs, desires, values, and beliefs about personal relevance
--
is required for all emotions and can usually determine if the experience is
positive or negative by determining:
\begin{itemize}
    \item \textbf{Goal Relevance (Rel.)}: Establishes if there are personal
    goal (\cref{C_Goals}) or stakes affected by the current
    individual-environment relationship. ``If there is no goal relevance,
    there
    cannot be an emotion; if there is, one or another emotion will occur,
    depending on the outcome of the transaction.''

    \item \textbf{Goal Congruence (Congr.)}: ``Goal congruence or
    incongruence
    refers to the extent to which a transaction is consistent or
    inconsistent
    with what a person wants -- that is, either it thwarts or facilitates
    personal goals'' (\cref{C_Goals}).

    \item \textbf{Type of Ego-involvement (Ego)} (\waitref{C_Ego}): ``...refers
    to the diverse aspects of ego-identity or personal commitments.'' It
    establishes what aspect of the self is affected. Ego-involvement is
    ``...probably involved in all or most emotions, but in different ways
    depending on the type of ego-involvement that is engaged by a
    transaction'', distinguishing between similar emotions, such as
    \textit{Joy} (no involvement) and \textit{Affection} (identity).
\end{itemize}

\hrule

\paragraph{Source} \citet[p.~149--150]{lazarus1991emotion}

\paragraph{Depends On} \cref{C_Goals}

\paragraph{Ref. By} \waitref{C_Fear}, \waitref{C_Anger}, \waitref{C_Sadness},
\waitref{C_Joy}, \waitref{C_Interest}, \waitref{C_Surprise},
\waitref{C_Disgust}, \waitref{C_Trust}, \waitref{C_Appraisal},
\waitref{T_GoalRelevance}, \waitref{T_GoalCongruence}
\\\hrule\vspace{0.5mm}\hrule

~\newline

\noindent
\begin{minipage}{\textwidth}
    \renewcommand*{\arraystretch}{1.5}
    \begin{tabular}{| p{\colAwidth}  p{\colBwidth}|}
        \hline
        \rowcolor[gray]{0.9}
        \bf WR\refstepcounter{waitnum}\thewaitnum \label{C_SA} & \bf
        Secondary Appraisal \\\hline
    \end{tabular}
\end{minipage}

\paragraph{Description} In CA, secondary appraisal asks what resources
and actions are available to the individual for coping with the
individual-environment relationship, which specifies which emotion is felt.
It
can be influenced by expectations and beliefs about potential actions and
the
individual's ability to act on them. It establishes:
\begin{itemize}
    \item \textbf{Accountability (Acc.)}: Assigns \textit{Blame} and
    \textit{Credit} for the individual-environment relationship. It
    ``derives from knowing who or what is accountable or responsible...if this
    knowledge is accompanied by the knowledge that the...act was under the
    accountable person's control, credit or blame is assigned.''

    \item \textbf{Coping Potential (CP)}: ``Coping potential refers to whether
    and how the person can manage the demands of the encounter or actualized
    personal commitments...'' (\waitref{C_Coping}). At this point, nothing is
    acted on as it ``is not actually coping but only an evaluation by a
    person of the prospects for doing or thinking something that will, in turn,
    change or protect the person-environment relationship''. How much energy
    the individual believes they have is likely a factor, as the pursuit of
    goals must require it.

    \item \textbf{Future Expectations (FE)}: ``Future expectancy has to do with
    whether for any reason things are likely to change psychologically for the
    better or worse (i.e., becoming more or less goal congruent).''
\end{itemize} \hrule

\paragraph{Source} \citet[p.~150]{lazarus1991emotion}

\paragraph{Depends On} \cref{C_Goals}, \waitref{C_Coping}

\paragraph{Ref. By} \waitref{C_Appraisal}, \waitref{C_Fear}, \waitref{C_Anger},
\waitref{C_Sadness}, \waitref{C_Joy}, \waitref{C_Interest},
\waitref{C_Surprise}, \waitref{C_Disgust}, \waitref{C_Trust},
\waitref{T_Accountability}, \waitref{T_CopingPotential},
\waitref{T_FutureExpectancy}
\\\hrule\vspace{0.5mm}\hrule

~\newline

\noindent
\begin{minipage}{\textwidth}
    \renewcommand*{\arraystretch}{1.5}
    \begin{tabular}{| p{\colAwidth}  p{\colBwidth}|}
        \hline
        \rowcolor[gray]{0.9}
        \bf WR\refstepcounter{waitnum}\thewaitnum \label{C_Fear} & \bf Fear
        \\\hline
    \end{tabular}
\end{minipage}

\paragraph{Description} \textit{Fear} is characterized by:
\begin{itemize}
    \item Self-preservation or the avoidance of pain---physical, mental, or
    spiritual
    \item Changes in the person-environment relationship that create:
    \begin{itemize}
        \item A threat or increased risk of imminent physical harm, or
        \item Uncertain threats to the individual's mental or spiritual
        well-being
    \end{itemize}
    \item Both \textit{Coping Potential} and the potential \textit{Future
        Impact (Expectancy)} are uncertain (\waitref{C_SA})
\end{itemize}

Stimuli are evaluated on their perceived probability of risk increase. The
most
common triggers are directly associated with self-preservation and pain
avoidance, such as falling and snakes, or with the absence of a stimuli or
event associated with safety, such as a protective individual or object.

The intensity of \textit{Fear} is likely impacted by the severity of the
potential harm (\waitref{C_PA}), if the harm is immediate or pending, and if
there
are coping strategies (\waitref{C_Coping}) that can be used to mitigate the
threat.\\\hrule

\paragraph{Source} \cite{robert1980emotion, lazarus1991emotion,
    ekman2007emotions}

\paragraph{Depends On} \cref{C_Emotion}, \waitref{C_Coping}, \waitref{C_PA}, 
\waitref{C_SA}

\paragraph{Ref. By} \waitref{T_FearIntensity}
\\\hrule\vspace{0.5mm}\hrule

~\newline

\noindent
\begin{minipage}{\textwidth}
    \renewcommand*{\arraystretch}{1.5}
    \begin{tabular}{| p{\colAwidth}  p{\colBwidth}|}
        \hline
        \rowcolor[gray]{0.9}
        \bf WR\refstepcounter{waitnum}\thewaitnum \label{C_Anger} & \bf
        Anger \\\hline
    \end{tabular}
\end{minipage}

\paragraph{Description} A harmful action that halts the progression of goal
achievement impacting the perception of the self is key to \textit{Anger},
where something:
\begin{itemize}
    \item Is identified as being accountable for the harmful action
    \item They are perceived to have had control or agency over the harmful
    action
\end{itemize}

Stimuli or events that are unwanted and predicted to lead to harmful
consequences, such as threats and rejection, often result in \textit{Anger}. A
frustrated goal (\cref{C_Goals}) can result in \textit{Anger} as it might be
perceived as a threat the the individual's identity if they tie its achievement
to their self-worth. \textit{Anger} can also be caused by another's perceived
\textit{Anger} if the cause is known.

The importance of the affected goal, degree of blockage (\waitref{C_PA}), and
the degree of blame (\waitref{C_SA}) assigned impact the intensity of
\textit{Anger}. If blame cannot be assigned, a scapegoat might be used as a
coping strategy (\waitref{C_Coping}) or compound the intensity of the current
scenario with the next one that induces \textit{Anger}. \\\hrule

\paragraph{Source} \cite{robert1980emotion, lazarus1991emotion,
    ekman2007emotions}

\paragraph{Depends On} \cref{C_Emotion}, \waitref{C_Coping} \waitref{C_PA}, 
\waitref{C_SA}

\paragraph{Ref. By} \waitref{T_AngerIntensity}
\\\hrule\vspace{0.5mm}\hrule

~\newline

\noindent
\begin{minipage}{\textwidth}
    \renewcommand*{\arraystretch}{1.5}
    \begin{tabular}{| p{\colAwidth}  p{\colBwidth}|}
        \hline
        \rowcolor[gray]{0.9}
        \bf WR\refstepcounter{waitnum}\thewaitnum \label{C_Sadness} & \bf
        Sadness \\\hline
    \end{tabular}
\end{minipage}

\paragraph{Description} The definition of loss is essential to identifying
triggers of \textit{Sadness}. In this context, a loss is something that the
individual:
\begin{itemize}
    \item Wants but no longer possesses,
    \item Are helpless to regain, and
    \item Have not yet been able to compensate or adjust for.
\end{itemize}

Separation, real or imagined failures, and helplessness are instances of
this
type of loss. \textit{Sadness} can also be triggered by another's
\textit{Sadness} as it is an empathetic emotion and easily mirrored.

The intensity of \textit{Sadness} is directly impacted by the degree
(\waitref{C_PA}) and permanence (\waitref{C_SA}) of the loss. The intensity of
previous experiences of \textit{Joy} (\waitref{C_Joy}) have also been cited,
implying that a priority value is assigned to the retainment of what was
lost and has a role in determining the intensity of the experienced
\textit{Sadness}.\\\hrule

\paragraph{Source} \cite{robert1980emotion, lazarus1991emotion,
    ekman2007emotions, izard1977human}

\paragraph{Depends On} \cref{C_Emotion}, \waitref{C_PA}, \waitref{C_SA}, 
\waitref{C_Joy}

\paragraph{Ref. By} --
\\\hrule\vspace{0.5mm}\hrule

~\newline

\noindent
\begin{minipage}{\textwidth}
    \renewcommand*{\arraystretch}{1.5}
    \begin{tabular}{| p{\colAwidth}  p{\colBwidth}|}
        \hline
        \rowcolor[gray]{0.9}
        \bf WR\refstepcounter{waitnum}\thewaitnum \label{C_Joy} &\bf
        Joy
        \\\hline
    \end{tabular}
\end{minipage}

\paragraph{Description} \textit{Joy} is defined by goal progression
(\waitref{C_PA}) when the individual perceives their life to be generally
good.
Depending on the type of goals (\cref{C_Goals}) affected, this can mean:
\begin{itemize}
    \item Maintaining the person-environment relationship as it is when the
    goal is to maintain the current situation, or
    \item Observing changes in the person-environment relationship that
    makes
    the achievement of a goal more likely to occur.
\end{itemize}

\textit{Joy} is triggered by stimuli and events that cause the individual to
feel safe and secure in their surroundings where little personal effort is
required to maintain it. This can include being with loved ones, relief from
physical pain and discomfort, and praise.

Factors impacting the intensity of \textit{Joy} include a positive evaluation
of future progress (\waitref{C_SA}) and the elapsed time since progression was
made on the affected goal such that a large time gap increases intensity.
Intensity might also be inversely proportional to the likelihood of goal
achievement, where the more unlikely goal achievements elicit stronger
reactions than highly likely goal achievements. \\\hrule

\paragraph{Source} \cite{robert1980emotion, lazarus1991emotion,
    ekman2003unmasking, ekman2007emotions}

\paragraph{Depends On} \cref{C_Emotion}, \waitref{C_PA}, \waitref{C_SA}

\paragraph{Ref. By} \waitref{C_Sadness}
\\\hrule\vspace{0.5mm}\hrule

~\newline

\noindent
\begin{minipage}{\textwidth}
    \renewcommand*{\arraystretch}{1.5}
    \begin{tabular}{| p{\colAwidth}  p{\colBwidth}|}
        \hline
        \rowcolor[gray]{0.9}
        \bf WR\refstepcounter{waitnum}\thewaitnum \label{C_Interest} & \bf
        Interest \\\hline
    \end{tabular}
\end{minipage}

\paragraph{Description} \textit{Interest} is a response to something that:
\begin{itemize}
    \item Will likely impact a goal in the future
    \item Can be inspected to reduce the uncertainty of its future impact
\end{itemize}

\textit{Interest} arises when something catches the individual's attention
(\waitref{C_Attend}) and provides some level of stimulation---a conscious
change in perception, by observing something novel, or by considering
possibilities in relation to goal achievement (\waitref{C_PA})---principally
guided by novelty and complexity. In general, people tend to find the most
interest in one of three types: objects, ideas, or people.

It is likely that the desirability of the potential benefit of the event,
person, or object, coupled with the likelihood of a beneficial pay-off
(\waitref{C_SA}), are factors in the evaluation of \textit{Interest}. As
indicators of when \textit{Interest} might be invoked, novelty and complexity
might also play a role in the emotion's intensity.\\\hrule

\paragraph{Source} \cite{robert1980emotion, lazarus1991emotion, izard1977human,
    tomkins1962affect, occ}

\paragraph{Depends On} \cref{C_Emotion}, \waitref{C_Attend}, \waitref{C_PA},
\waitref{C_SA}

\paragraph{Ref. By} --
\\\hrule\vspace{0.5mm}\hrule

~\newline

\noindent
\begin{minipage}{\textwidth}
    \renewcommand*{\arraystretch}{1.5}
    \begin{tabular}{| p{\colAwidth}  p{\colBwidth}|}
        \hline
        \rowcolor[gray]{0.9}
        \bf WR\refstepcounter{waitnum}\thewaitnum \label{C_Surprise} & \bf
        Surprise \\\hline
    \end{tabular}
\end{minipage}

\paragraph{Description} \textit{Surprise} is a response to events that an
individual is ill-prepared for which are:
\begin{itemize}
    \item Sudden
    \item Unexpected or incorrectly predicted
\end{itemize}

As it is impossible to elicit \textit{Surprise} from a correctly predicted
outcome, the trigger is the size of the difference between what is observed
(\waitref{C_Attend}) and what was expected regardless of the person, event, or
object itself.

\textit{Surprise} is influenced by the level of uncertainty, which reflects how
much information the individual needs to collect in order to appraise the
situation effectively (\waitref{C_PA}, \waitref{C_SA}) and the confidence level
of a prediction about the individual-environment relationship. This is
complemented by familiarity, which can help reduce the amount of information
necessary to gather information. \\\hrule

\paragraph{Source} \cite{robert1980emotion, lazarus1991emotion,
    ekman2007emotions, occ, izard1977human}

\paragraph{Depends On} \cref{C_Emotion}, \waitref{C_Attend}, \waitref{C_PA},
\waitref{C_SA}

\paragraph{Ref. By} --
\\\hrule\vspace{0.5mm}\hrule

~\newline

\noindent
\begin{minipage}{\textwidth}
    \renewcommand*{\arraystretch}{1.5}
    \begin{tabular}{| p{\colAwidth}  p{\colBwidth}|}
        \hline
        \rowcolor[gray]{0.9}
        \bf WR\refstepcounter{waitnum}\thewaitnum \label{C_Disgust} & \bf
        Disgust \\\hline
    \end{tabular}
\end{minipage}

\paragraph{Description} Distaste is the underlying theme of \textit{Disgust},
which is identified by:
\begin{itemize}
    \item Self-preservation (\cref{C_Goals})
    \item The anticipation of harm if contacted (\waitref{C_PA})
    \item Coping potential and future expectancy are known (\waitref{C_SA})
\end{itemize}

Compared to other emotions, \textit{Disgust} is restricted in content and rigid
in the type of factors that can elicit it. The most common universal trigger of
\textit{Disgust} is bodily fluid that has left the body -- blood, mucus,
faeces, vomit, and urine. Other themes -- including strange things, diseases,
misfortune, and moral taint -- are learned and heavily influenced by
personality and culture.

The level of unappealingness of the stimuli has a significant impact on the
intensity of \textit{Disgust}, both from sensory evaluations and in
anticipation of contamination from ``indigestion''. A counter factor, the
degree of familiarity with it and its origins, can temper the reaction.
\\\hrule

\paragraph{Source} \cite{robert1980emotion, lazarus1991emotion,
    rozin1999disgust, occ}

\paragraph{Depends On} \cref{C_Emotion}, \cref{C_Goals}, \waitref{C_PA}, 
\waitref{C_SA}

\paragraph{Ref. By} --
\\\hrule\vspace{0.5mm}\hrule

~\newline

\noindent
\begin{minipage}{\textwidth}
    \renewcommand*{\arraystretch}{1.5}
    \begin{tabular}{| p{\colAwidth}  p{\colBwidth}|}
        \hline
        \rowcolor[gray]{0.9}
        \bf WR\refstepcounter{waitnum}\thewaitnum \label{C_Trust} & \bf
        Trust \\\hline
    \end{tabular}
\end{minipage}

\paragraph{Description} Affective \textit{Trust}, or \textit{Affection} in CA,
is an emotion arising from a long-term relationship with an honest and
benevolent partner. This suggests that \textit{Trust} is identified by:
\begin{itemize}
    \item A desire to care for the partner and a willingness to put oneself at
    risk for them -- physically, emotionally, or socially

    \item An future expectation that the partner will demonstrate their care
    for the individual, often via an affirmation of the individual's value and
    increased security, confidence, and intimacy in the relationship
\end{itemize}

As it is tied to relationships (\waitref{C_Relation}) and expectations of future
benefits, \textit{Trust} is likely triggered when an individual makes a risk
assessment involving the partner where a favourable outcome is expected
(\waitref{C_PA}, \waitref{C_SA}). After the event has passed, the individual
will likely experience another emotion fitting to the outcome.

The intensity of \textit{Trust} is influenced by the individual's relationship
with the partner. The component factors of a relationship
-- volatility, dependability, and faith -- influence the intensity of
\textit{Trust}. Their relative weights are determined by the confidence and
security of the relationship, and also by the individual's gender -- men treat
these factors independently whereas women tend to associate them.
\begin{itemize}
    \item When no personal relationship exists intensity is influenced solely
    on the partner's reputation and past and current actions -- their
    volatility.

    \item Once a relationship is established, the individual's perception of
    their partner's dependability has more influence than their past
    experiences on predictions of future outcomes -- their dependability.

    \item When the relationship has developed to the highest level of
    confidence and security, the individual puts weight in faith -- the
    strength of the relationship and their personal security and self-esteem
    than their partner's history, actions, or dependability when determining
    how much risk they are willing to put themselves in.
\end{itemize}

\hrule

\paragraph{Source} \cite{robert1980emotion, lazarus1991emotion,
    rempel1985trust}

\paragraph{Depends On} \cref{C_Emotion}, \waitref{C_Relation}, \waitref{C_PA},
\waitref{C_SA}

\paragraph{Ref. By} --
\\\hrule\vspace{0.5mm}\hrule

\subsection{Theoretical Models}
~\newline

\noindent
\begin{minipage}{\textwidth}
    \renewcommand*{\arraystretch}{1.5}
    \begin{tabular}{| p{\colAwidth}  p{\colBwidth}|}
        \hline
        \rowcolor[gray]{0.9}
        \bf WR\refstepcounter{waitnum}\thewaitnum \label{T_GoalRelevance} &
        \bf Evaluating Goal Relevance \\
        \hline
    \end{tabular}

    \renewcommand*{\arraystretch}{1.5}
    \begin{tabular}{ p{\colAwidth}  p{\colBwidth}}
        \bf Input & $g : \goaltype$, $s : \worldstatetype$, $s_{\Delta} :
        \worldstatechangetype$ \\

        \bf Output & $r \defEq |g.\mathtt{goal'}(s, s_{\Delta})|$
        \\\hline
    \end{tabular}
\end{minipage}

\paragraph{Description} A change in the game state ($s_{\Delta}$) from the
current state ($s$) is relevant to a goal ($g$) if its function
$\mathtt{goal'}(s, s_{\Delta})$ returns a value whose magnitude is non-zero.
This indicates that there is a change in $s$ by $s_\Delta$ which moves it
closer to the state represented by $\mathtt{goal}$. \\\hrule

\paragraph{Sources} --

\paragraph{Depends On} \tyref{TY_WorldState}, \tyref{TY_WorldStateChange}, 
\tyref{TY_Goal}, \waitref{C_PA}

\paragraph{Ref. By} -- \\\hrule\vspace{0.5mm}\hrule

~\newline

\noindent
\begin{minipage}{\textwidth}
    \renewcommand*{\arraystretch}{1.5}
    \begin{tabular}{| p{\colAwidth}  p{\colBwidth}|}
        \hline
        \rowcolor[gray]{0.9}
        \bf WR\refstepcounter{waitnum}\thewaitnum \label{T_GoalCongruence} &
        \bf Evaluating Goal Congruence \\
        \hline
    \end{tabular}

    \renewcommand*{\arraystretch}{1.5}
    \begin{tabular}{ p{\colAwidth}  p{\colBwidth}}
        \bf Input & $g : \goaltype$, $s : \worldstatetype$ $s_{\Delta} :
        \worldstatechangetype$, $B \in \mathbb{R^+}$ \\

        \bf Output & $c \defEq cg + b$ where \\

        & $cg = g.\mathtt{goal'}(s, s_{\Delta})$, and \\
        & $b = \begin{cases}
            B \cdot cg, & G > 0 \wedge G' = 0 \\

            -B \cdot cg, & G = 0 \wedge G' > 0\\

            0, & \text{Otherwise} \\
        \end{cases}$ \\
        & \hspace{5mm} where $ G = g.\mathtt{goal}(s) $\\
        & \hspace{8mm} and $ G' = g.\mathtt{goal'}(s, s_{\Delta}) $ \\
        \hline
    \end{tabular}
\end{minipage}

\paragraph{Description} Goal congruence ($c$) is a measure of how a change in
the world state ($s_{\Delta}$) affects the completion status of a goal
$g.\mathtt{goal'}(s, s_{\Delta})$.

A value $b$ proportional to $cg$ is added to $c$ to ``boost'' the value
(\aref{A_AchievingGoals}) when a previously unsatisfied goal is now satisfied
($g.\mathtt{goal}(s) > 0 \wedge g.\mathtt{goal'}(s, s_{\Delta}) = 0$) or a
previously completed goal is no longer satisfied ($g.\mathtt{goal}(s) = 0
\wedge g.\mathtt{goal'}(s, s_{\Delta}) > 0$). \\\hrule

\paragraph{Sources} --

\paragraph{Depends On} \tyref{TY_WorldState}, \tyref{TY_WorldStateChange}, 
\tyref{TY_Goal}, \aref{A_AchievingGoals}, \waitref{C_PA}

\paragraph{Ref. By} \waitref{T_Accountability}, \waitref{T_CopingPotential},
\waitref{T_FutureExpectancy}, \waitref{T_AppraisalCA},
\waitref{T_AngerIntensity}, \waitref{I_CopingPotential}
\\\hrule\vspace{0.5mm}\hrule

~\newline

\noindent
\begin{minipage}{\textwidth}
    \renewcommand*{\arraystretch}{1.5}
    \begin{tabular}{| p{\colAwidth}  p{\colBwidth}|}
        \hline
        \rowcolor[gray]{0.9}
        \bf WR\refstepcounter{waitnum}\thewaitnum \label{T_Accountability} &
        \bf Evaluating Accountability \\
        \hline
    \end{tabular}

    \renewcommand*{\arraystretch}{1.5}
    \begin{tabular}{ p{\colAwidth}  p{\colBwidth}}
        \bf Input & $(\mActor : \actortype, a : \actiontype, \mDeliberate :
        \mathbb{B})$, $g : \goaltype$, $s : \worldstatetype$, $B \in
        \mathbb{R^+}$ \\

        \bf Output & $acc \defEq \begin{cases}
            \text{Credit}, & \mDeliberate \wedge C > 0 \\

            \text{Blame}, & \mDeliberate \wedge C < 0 \\

            \text{None}, & \text{Otherwise} \\
        \end{cases}$ \\

        & where $C = c(g, s, a.\mathtt{makesChange}(s), B)$ \\
        \hline
    \end{tabular}
\end{minipage}

\paragraph{Description} Accountability ($acc$) is a categorical assignment of
fault for an action ($a$) that is caused by an actor ($\mActor$). It is derived
from an assigned value of the actor's intention ($\mDeliberate$) and an
evaluation of benefit or harm given by goal congruence ($c(g, s,
a.\mathtt{makesChange}(s), B)$):
\begin{itemize}
    \item Credit is given if the event is deliberately caused and is beneficial
    ($\mDeliberate \wedge c(g, s, a.\mathtt{makesChange}(s), B) > 0$),

    \item Blame is given if the event is deliberate and causes harm
    ($\mDeliberate \wedge c(g, s, a.\mathtt{makesChange}(s), B) < 0$), and

    \item Neither Credit or Blame is given if the event is not deliberately
    caused ($\neg \mDeliberate$) or has no effect on the evaluated goal ($c(g,
    s, a.\mathtt{makesChange}(s), B) = 0$).
\end{itemize}
\hrule

\paragraph{Sources} --

\paragraph{Depends On} \tyref{TY_WorldState}, \tyref{TY_Goal}, 
\waitref{TY_Actor}, \waitref{TY_Action}, \waitref{C_SA}, 
\waitref{T_GoalCongruence}

\paragraph{Ref. By} \waitref{T_AppraisalCA}, \waitref{T_AngerIntensity}
\\\hrule\vspace{0.5mm}\hrule

~\newline

\noindent
\begin{minipage}{\textwidth}
    \renewcommand*{\arraystretch}{1.5}
    \begin{tabular}{| p{\colAwidth}  p{\colBwidth}|}
        \hline
        \rowcolor[gray]{0.9}
        \bf WR\refstepcounter{waitnum}\thewaitnum \label{T_CopingPotential}
        &
        \bf Evaluating Coping Potential \\
        \hline
    \end{tabular}

    \renewcommand*{\arraystretch}{1.5}
    \begin{tabular}{ p{\colAwidth}  p{\colBwidth}}
        \bf Input & $a : \overrightarrow{\actiontype}$, $g : \goaltype$, $s :
        \worldstatetype$, $B \in \mathbb{R^+}$, $E : \energytype$, $ d : [0, 1]
        \in \mathbb{R^+}$ \\

        \bf Output & $cp \defEq \sum_{i = 0}^{n} d^i \cdot q(i)$ \\

        & where $q = p \text{ sorted in descending order}$ \\

        & and $p(i) = \begin{cases}
            C(i) \cdot L(i) \cdot J(i), & J(i) < 1 \wedge C(i) > 0 \\

            0, & \text{Otherwise} \\
        \end{cases} $ \\

        & \hspace{5mm} where $C(i) = c(g, s, a(i).\mathtt{makesChange}(s), B)$,
        \\
        & \hspace{15mm} $L(i) = a(i).\mathtt{successLikelihood} $, \\
        & \hspace{9mm} and $J(i) = 1 - \frac{a(i).\mathtt{energyCost}}{E}$ \\
        \hline
    \end{tabular}
\end{minipage}

\paragraph{Description} Coping potential ($cp$) is a sum of the effectiveness
of a vector of actions ($a : \overrightarrow{\actiontype}$). Actions are
assumed to be independent because only one action can be executed at a time
(\aref{A_ActOn}) and appraisals are evaluated on a moment-to-moment basis
(\aref{A_AppraisalTime}).

The vector of action effectiveness values is sorted in descending order ($q$),
and each value is decayed by a factor of $d^i$. This maintains the full value
of the most effective action's value the total coping potential, but the value
of additional actions has increasingly diminished returns. This ensures that the
additional value associated with having multiple effective actions is
preserved, but recognizes that one might not see a difference when adding
another effective action to an already large set.

The effectiveness of an action ($p(i)$) is measured by how useful it is, its
likelihood of success, and how easy it is to act on. This is given by the
product of its expected goal congruence ($c(g, s, a(i).\mathtt{makesChange},
B)$), likelihood of success ($a(i).\mathtt{successLikelihood}$), and
anticipated energy usage ($a(i).\mathtt{energyCost}$) as it relates to the
available energy ($E$).

An action is excluded from the vector of potential actions if it does not
affect or negatively affects the given goal ($C \not > 0$), or it requires more
than the available energy ($J \not < 1$). \\\hrule

\paragraph{Sources} --

\paragraph{Depends On} \tyref{TY_WorldState}, \tyref{TY_Goal}, \aref{A_ActOn}, 
\aref{A_AppraisalTime},  \waitref{TY_Energy}, \waitref{TY_EnergyChange} (via 
\waitref{TY_Action}), \waitref{TY_Action}, \waitref{C_SA}, 
\waitref{T_GoalCongruence}

\paragraph{Ref. By} \waitref{T_AppraisalCA}, \waitref{T_FearIntensity},
\waitref{I_CopingPotential} \\\hrule\vspace{0.5mm}\hrule

~\newline

\noindent
\begin{minipage}{\textwidth}
    \renewcommand*{\arraystretch}{1.5}
    \begin{tabular}{| p{\colAwidth}  p{\colBwidth}|}
        \hline
        \rowcolor[gray]{0.9}
        \bf WR\refstepcounter{waitnum}\thewaitnum \label{T_FutureExpectancy}
        &
        \bf Evaluating Future Expectancy \\
        \hline
    \end{tabular}

    \renewcommand*{\arraystretch}{1.5}
    \begin{tabular}{ p{\colAwidth}  p{\colBwidth}}
        \bf Input & $g : \goaltype$, $s : \worldstatetype$, $a :
        \overrightarrow{\actiontype}$, $\mOtherChange : \worldstatechangetype$,
        $B : \mathbb{R^+}$ \\

        \bf Output & $fe \defEq c(g, s, \Delta S, B) - c(g, s, s_{\Delta}, B)$
        \\
        & where $\Delta S = a(i).\mathtt{makesChange}(s) + \mOtherChange$ \\
        & and $i = \mathtt{arg max}_{a} \text{ }
        a.\mathtt{successLikelihood}(s)$ \\
        \hline
    \end{tabular}
\end{minipage}

\paragraph{Description} Future expectancy ($fe$) is an evaluation of the
difference in goal congruence between this world state ($c(g, s, s_{\Delta},
B)$) and the next predicted world state ($c(g, s, \Delta S, B)$). If the
difference is:
\begin{itemize}
    \item Positive ($fe > 0$), then the next world state is predicted to be
    more goal congruent -- an improvement over the current world state

    \item Negative ($fe < 0$), then the next world state is predicted to be
    less goal congruent -- worse than the current than the current world state

    \item Zero ($fe = 0$), then the next world state is predicted to have the
    same goal congruence -- there is no improvement or loss over the current
    world state
\end{itemize}

The next predicted world state is the sum of two world state changes. One is
the action that is most likely to succeed from a vector of actions ($a(i) \in
\overrightarrow{\actiontype}$), representing the part of the future that the
NPC can control. The other is a world state change that is not caused by the
NPC's actions -- the parts of the future that the NPC cannot control -- such as
the actions of others and naturally occurring events ($\mOtherChange$). \\\hrule

\paragraph{Sources} --

\paragraph{Depends On} \tyref{TY_WorldState}, \tyref{TY_WorldStateChange}, 
\tyref{TY_Goal}, \waitref{TY_Action}, \waitref{C_SA}, \waitref{T_GoalCongruence}

\paragraph{Ref. By} \waitref{T_AppraisalCA}, \waitref{I_FutureExpectancy}
\\\hrule\vspace{0.5mm}\hrule

~\newline

\noindent
\begin{minipage}{\textwidth}
    \renewcommand*{\arraystretch}{1.5}
    \begin{tabular}{| p{\colAwidth}  p{\colBwidth}|}
        \hline
        \rowcolor[gray]{0.9}
        \bf WR\refstepcounter{waitnum}\thewaitnum \label{T_AppraisalCA} &
        \bf Evaluating Emotion Type from an Appraisal (\textit{Fear},
        \textit{Anger}, \textit{Sadness}, \textit{Joy}, \textit{Disgust}, and
        \textit{Trust}) \\
        \hline
    \end{tabular}

    \renewcommand*{\arraystretch}{1.5}
    \begin{tabular}{ p{\colAwidth}  p{\colBwidth}}
        \bf Input & $g : \goaltype$, $ge(g) : \goalegotype$, $s :
        \worldstatetype$, $s_{\Delta} : \worldstatechangetype$, $B :
        \mathbb{R^+}$, $(\mActor : \actortype, a : \actiontype, \mDeliberate :
        \mathbb{B})$, $a : \overrightarrow{\actiontype}$, $E : \energytype$, $d
        : [0,1] \in \mathbb{R^+}$, $T_{cp} : \mathbb{R^+}$, $T_{fe} :
        \mathbb{R^+}$ \\

        \bf Output & $e \defEq \text{match } (C, GE, A, CP, FE) \text{ to}$ \\

        & $\mid (C < 0, O, \text{None}, CP < T_{cp}, \_) \rightarrow
        \text{Sadness}$ \\

        & $\mid (C < 0, ge(g, \mathtt{SelfEsteem}) > 0 \vee ge(g,
        \mathtt{SocialEsteem}) > 0, \text{Blame}, \_, \_) \rightarrow
        \text{Anger}$ \\

        & $\mid (C > 0, \_, \_, \_, FE > T_{fe}) \rightarrow
        \text{Joy}$ \\

        & $\mid (C < 0, O, \_, \_, \_) \rightarrow \text{Fear}$ \\

        & $\mid (C < 0, O, \_, \_, \_) \rightarrow \text{Disgust}$ \\

        & $\mid (C > 0, O, \_, \_, \_) \rightarrow \text{Trust}$ \\

        & $\text{fi}$ \\

        & where $R = r(g, s, s_{\Delta}, \varepsilon)$, \\
        & $C = c(g, s, s_{\Delta}, B)$, \\
        & $A = acc((\mActor, a, \mDeliberate), g, s, B)$, \\
        & $CP = cp(\{a\}, g, s, B, E, d)$, $FE = fe()$, \\
        & and $O = \{\exists o \in ge(g) \mid ge(g).o > 0\}$ \\\hline
    \end{tabular}
\end{minipage}

\paragraph{Description} An appraisal (\waitref{C_Appraisal}) is the output of
pattern matching where each emotion kind has a unique pattern
(\aref{A_emotionRepr}). Since some patterns are more detailed than others, the
patterns are read sequentially to avoid unintentional matches.

A pattern is comprised of values for goal congruence ($C$), the ego type
associated with $g$ ($GE$, \aref{A_EgoIdentity}, \aref{A_PAEgo}),
accountability ($A$), coping potential ($CP$), and future expectancy
($FE$).\\\hrule

\paragraph{Sources} --

\paragraph{Depends On} \tyref{TY_WorldState}, \tyref{TY_WorldStateChange}, 
\tyref{TY_Goal}, \aref{A_emotionRepr}, \aref{A_EgoIdentity}, \aref{A_PAEgo}, 
\waitref{TY_Energy}, \waitref{TY_Actor}, \waitref{TY_Action}, 
\waitref{TY_GoalEgoRelation}, \waitref{C_Appraisal}, 
\waitref{T_GoalCongruence}, \waitref{T_Accountability}, 
\waitref{T_CopingPotential}, \waitref{T_FutureExpectancy}

\paragraph{Ref. By} -- \\\hrule\vspace{0.5mm}\hrule

~\newline

\noindent
\begin{minipage}{\textwidth}
    \renewcommand*{\arraystretch}{1.5}
    \begin{tabular}{| p{\colAwidth}  p{\colBwidth}|}
        \hline
        \rowcolor[gray]{0.9}
        \bf WR\refstepcounter{waitnum}\thewaitnum
        \label{T_FearIntensity} &
        \bf Determining the Intensity of \textit{Fear} \\
        \hline
    \end{tabular}

    \renewcommand*{\arraystretch}{1.5}
    \begin{tabular}{ p{\colAwidth}  p{\colBwidth}}
        \bf Input & $g : \goaltype$, $s : \worldstatetype$, $B : \mathbb{R^+}$,
        $a : \overrightarrow{\actiontype}$, $E : \energytype$, $d : [0,1] \in
        \mathbb{R^+}$, $t_{\Delta} : \timetype$ \\

        \vspace*{-1.5mm} \bf Output & \vspace*{-1.5mm}
        $I_{\mFear} : \responsestrength \defEq \begin{cases}
            \dfrac{g.\mathtt{importance}}{t_{\Delta} \cdot
                cp(a, g, s, B, E, d)}, & cp > 0 \wedge
            t_{\Delta} > 0 $\vspace*{1mm}$\\
            g.\mathtt{importance}, & \text{Otherwise} \\
        \end{cases}
        $
        \vspace*{1mm}\\\hline
    \end{tabular}
\end{minipage}

\paragraph{Description} A change in the intensity of \textit{Fear} ($I_{\mFear}
: \responsestrength$) is:
\begin{itemize}
    \item Proportional to the importance of the affected goal
    ($g.\mathtt{importance}$)---a higher importance increases intensity

    \item Inversely proportional to the time that the event is expected
    ($t_{\Delta}$), such that a shorter expected time increases intensity

    \item Inversely proportional to the NPC's ability to mitigate risk as
    measured by their coping potential ($cp(a, g, s, B, E,$ $d)$), such that a
    higher coping potential decreases intensity
\end{itemize}

If at least one of $cp$ or $t_{\Delta}$ are 0, then the intensity of
\textit{Fear} is equivalent to the importance of the goal. This represents
situations where the NPC has no available actions that can affect the current
world state ($cp = 0$), or when there is no time to prepare for the harmful
event ($t_{\Delta} = 0$). \\\hrule

\paragraph{Sources} --

\paragraph{Depends On} \tyref{TY_Time}, \tyref{TY_DeltaIntensity}, 
\tyref{TY_WorldState}, \tyref{TY_Goal}, \waitref{TY_Energy}, 
\waitref{TY_Action}, \waitref{C_Fear}, \waitref{T_CopingPotential}

\paragraph{Ref. By} \waitref{I_FearIntensityE}, \waitref{I_FearIntensityT},
\waitref{I_FearIntensityTE} \\\hrule\vspace{0.5mm}\hrule

~\newline

\noindent
\begin{minipage}{\textwidth}
    \renewcommand*{\arraystretch}{1.5}
    \begin{tabular}{| p{\colAwidth}  p{\colBwidth}|}
        \hline
        \rowcolor[gray]{0.9}
        \bf WR\refstepcounter{waitnum}\thewaitnum
        \label{T_AngerIntensity} &
        \bf Determining the Intensity of \textit{Anger} \\
        \hline
    \end{tabular}

    \renewcommand*{\arraystretch}{1.5}
    \begin{tabular}{ p{\colAwidth}  p{\colBwidth}}
        \bf Input & $g : \goaltype$, $s : \worldstatetype$, $B : \mathbb{R^+}$,
        $(\mActor : \actortype, a : \actiontype, \mDeliberate : \mathbb{B})$,
        $rl : \socialrelationtype$\\

        \bf Output & $I_{\mAnger} : \responsestrength \defEq \begin{cases}
            A, & acc((\mActor, a, \mDeliberate), g, s, B) = \text{Blame} \\
            0, & \text{Otherwise} \\
        \end{cases}$
        \\
        \vspace*{-2mm} & \vspace*{-2mm}
        where $A = g.\mathtt{importance} \cdot \mControl + I_{\mStore}$, \\
        & $\mControl = \dfrac{rl.\mathtt{confidence}}{rl.\mathtt{security}} +
        c(g, s, a.\mathtt{makesChange}(s), B)$,
        \vspace*{1mm}\\
        & and $I_{\mStore} = \begin{cases}
            A, & acc((\mActor, a, \mDeliberate), g, s, B) \neq \text{Blame} \\
            0, & \text{Otherwise} \\
        \end{cases}$ \vspace*{1.5mm}\\\hline
    \end{tabular}
\end{minipage}

\paragraph{Description} A change in the intensity of \textit{Anger}
($I_{\mAnger} : \responsestrength$) is:
\begin{itemize}
    \item Proportional to the importance of the affected goal
    ($g.\mathtt{importance}$)---a higher importance increases intensity

    \item Proportional to the degree of \textit{Blame} that the offending
    $\mActor$ is perceived to have, represented as a perceived amount of
    control over their actions ($\mControl$), such that a higher degree of
    control increases intensity

    \item Linearly increased by the total intensity of all previous encounters
    of \textit{Anger} since the last \textit{Blame} assignment where it could
    not be assigned ($I_{\mStore}$), representing a suppression of
    \textit{Anger} since no suitable target is available
\end{itemize}

If Blame has been successfully assigned to $\mActor$ in the evaluation of
accountability ($acc((\mActor, a,\allowbreak \mDeliberate), g, s,$ $B) = 
\text{Blame}$), the intensity is equivalent to $A$ and $I_{\mStore}$ is reset 
to $0$; otherwise, the emotion is suppressed and the intensity is $0$ and the 
stored intensity of \textit{Anger} is $I_{\mStore}$.

The degree of $\mControl$ that $\mActor$ has is a linear combination of two
parts. The first is an evaluation of the relationship between this NPC and
$\mActor$. The relationship is evaluated by: the confidence in the NPC's
knowledge of the relationship ($rl.\mathtt{confidence}$), such that a higher
confidence increases perceived control; and an evaluation of how much harm the
NPC believes $\mActor$ intends ($rl.\mathtt{security}$), such that a higher
value decreases perceived control. The second part is an evaluation of how much
harm the action of $\mActor$ has caused, given as a congruence evaluation on
their action ($c(g, s, a.\mathtt{makesChange}(s)$). \\\hrule

\paragraph{Sources} --

\paragraph{Depends On} \tyref{TY_DeltaIntensity}, \tyref{TY_WorldState}, 
\tyref{TY_Goal}, \waitref{TY_Actor}, \waitref{TY_Action}, 
\waitref{TY_SocialRelation}, \waitref{C_Anger}, \waitref{T_GoalCongruence}, 
\waitref{T_Accountability}

\paragraph{Ref. By} \waitref{I_AngerIntensityS}, \waitref{I_AngerIntensityR},
\waitref{I_AngerIntensitySR} \\\hrule\vspace{0.5mm}\hrule

\subsection{Instance Models}
~\newline

\noindent
\begin{minipage}{\textwidth}
    \renewcommand*{\arraystretch}{1.5}
    \begin{tabular}{| p{\colAwidth}  p{\colBwidth}|}
        \hline
        \rowcolor[gray]{0.9}
        \bf WR\refstepcounter{waitnum}\thewaitnum \label{I_CopingPotential}
        &
        \bf (Instance Model) Evaluating Coping Potential with No Energy
        Cost \\
        \hline
    \end{tabular}

    \renewcommand*{\arraystretch}{1.5}
    \begin{tabular}{ p{\colAwidth}  p{\colBwidth}}
        \bf Input& $a : \overrightarrow{\actiontype}$, $g : \goaltype$, $s
        :
        \worldstatetype$, $B \in \mathbb{R^+}$, $ d : [0, 1] \in
        \mathbb{R^+}$
        \\

        \bf Output & $cp \defEq \sum_{i = 0}^{n} d^i \cdot q(i)$ \\

        & where $q = p \text{ sorted in descending order}$ \\

        & and $p(i) = \begin{cases}
            C(i) \cdot L(i), & C(i) > 0 \\

            0, & \text{Otherwise} \\
        \end{cases} $ \\

        & \hspace{5mm} where $C(i) = c(g, s, a(i).\mathtt{makesChange}(s),
        B)$,
        \\
        & \hspace{9mm} and $L(i) = a(i).\mathtt{successLikelihood} $ \\
        \hline
    \end{tabular}
\end{minipage}

\paragraph{Description} If energy is assumed to be infinite, then the
effectiveness of an action is only measured by its goal congruence ($c(g,
s,
a(i).\mathtt{makesChange}, B)$) and likelihood of success
($a(i).\mathtt{successLikelihood}$). Actions with no or negative effects on
the
goal ($g$) are excluded from the vector of potential actions.

Coping potential ($cp$) remains the sum of the effectiveness ($p(i)$) of
the
vector of these actions ($a : \overrightarrow{\actiontype}$). The vector is
sorted in descending order ($q$) and each action is decayed by a factor of
$d^i$. This ensures that the full value of the most effective action is
maintained in the sum, but each subsequent effectiveness is reduced such
that
there are diminishing returns on large sets of actions. \\\hrule

\paragraph{Sources} --

\paragraph{Depends On} \tyref{TY_WorldState}, \tyref{TY_Goal}, 
\waitref{TY_Action}, \waitref{T_GoalCongruence}, \waitref{T_CopingPotential}

\paragraph{Ref. By} -- \\\hrule\vspace{0.5mm}\hrule

~\newline

\noindent
\begin{minipage}{\textwidth}
    \renewcommand*{\arraystretch}{1.5}
    \begin{tabular}{| p{\colAwidth}  p{\colBwidth}|}
        \hline
        \rowcolor[gray]{0.9}
        \bf WR\refstepcounter{waitnum}\thewaitnum \label{I_FutureExpectancy} &
        \bf Evaluating Future Expectancy with NPC Actions Only \\
        \hline
    \end{tabular}

    \renewcommand*{\arraystretch}{1.5}
    \begin{tabular}{ p{\colAwidth}  p{\colBwidth}}
        \bf Input & $g : \goaltype$, $s : \worldstatetype$, $a :
        \overrightarrow{\actiontype}$, $B : \mathbb{R^+}$ \\

        \bf Output & $fe \defEq c(g, s, \Delta S, B) - c(g, s, s_{\Delta}, B)$
        \\
        & where $\Delta S = a(i).\mathtt{makesChange}(s)$ \\
        & and $i = \mathtt{arg max}_{a} \text{ }
        a.\mathtt{successLikelihood}(s)$ \\
        \hline
    \end{tabular}
\end{minipage}

\paragraph{Description} Future expectancy ($fe$) is an evaluation of the
difference in goal congruence between this world state ($c(g, s, s_{\Delta},
B)$) and the next predicted world state ($c(g, s, \Delta S, B)$). If the
difference is:
\begin{itemize}
    \item Positive ($fe > 0$), then the next world state is predicted to be
    more goal congruent -- an improvement over the current world state

    \item Negative ($fe < 0$), then the next world state is predicted to be
    less goal congruent -- worse than the current than the current world state

    \item Zero ($fe = 0$), then the next world state is predicted to have the
    same goal congruence -- there is no improvement or loss over the current
    world state
\end{itemize}

The next predicted world state is determined by an action that is most likely
to succeed from a vector of available actions ($a(i) \in a :
\overrightarrow{\actiontype}$). \\\hrule

\paragraph{Sources} --

\paragraph{Depends On} \tyref{TY_WorldState}, \tyref{TY_WorldStateChange}, 
\tyref{TY_Goal}, \waitref{TY_Action}, \waitref{T_GoalCongruence}, 
\waitref{T_FutureExpectancy}

\paragraph{Ref. By} -- \\\hrule\vspace{0.5mm}\hrule

~\newline

\noindent
\begin{minipage}{\textwidth}
    \renewcommand*{\arraystretch}{1.5}
    \begin{tabular}{| p{\colAwidth}  p{\colBwidth}|}
        \hline
        \rowcolor[gray]{0.9}
        \bf WR\refstepcounter{waitnum}\thewaitnum \label{I_FearIntensityE} &
        \bf Calculating the Intensity of \textit{Fear} with Energy and without
        Time \\
        \hline
    \end{tabular}

    \renewcommand*{\arraystretch}{1.5}
    \begin{tabular}{ p{\colAwidth}  p{\colBwidth}}
        \bf Input & $g : \goaltype$, $s : \worldstatetype$, $B : \mathbb{R^+}$,
        $\{a : \actiontype\}$, $E : \energytype$, $d : [0,1] \in \mathbb{R^+}$
        \\

        \vspace*{-1.5mm} \bf Output & \vspace*{-1.5mm}
        $I_{\mFear} : \responsestrength \defEq
        \dfrac{g.\mathtt{importance}}{cp(\{a
            : \actiontype\}, g, s, B, E, d)} $
        \vspace*{1mm}\\\hline
    \end{tabular}
\end{minipage}

\paragraph{Description} A change in the intensity of \textit{Fear} ($I_{\mFear}
: \responsestrength$) without time considerations is:
\begin{itemize}
    \item Proportional to the importance of the affected goal
    ($g.\mathtt{importance}$)---a higher importance increases intensity

    \item Inversely proportional to the NPC's ability to mitigate risk as
    measured by their coping potential ($cp(\{a : \actiontype\}, g, s, B, E,
    d)$), such that a higher coping potential decreases intensity
\end{itemize}
\hrule

\paragraph{Sources} --

\paragraph{Depends On} \tyref{TY_DeltaIntensity}, \tyref{TY_WorldState}, 
\tyref{TY_Goal}, \waitref{TY_Energy}, \waitref{TY_Action}, 
\waitref{T_CopingPotential}, \waitref{T_FearIntensity}

\paragraph{Ref. By} -- \\\hrule\vspace{0.5mm}\hrule

~\newline

\noindent
\begin{minipage}{\textwidth}
    \renewcommand*{\arraystretch}{1.5}
    \begin{tabular}{| p{\colAwidth}  p{\colBwidth}|}
        \hline
        \rowcolor[gray]{0.9}
        \bf WR\refstepcounter{waitnum}\thewaitnum \label{I_FearIntensityT} &
        \bf Calculating the Intensity of \textit{Fear} without Energy and with
        Time \\
        \hline
    \end{tabular}

    \renewcommand*{\arraystretch}{1.5}
    \begin{tabular}{ p{\colAwidth}  p{\colBwidth}}
        \bf Input & $g : \goaltype$, $s : \worldstatetype$, $B : \mathbb{R^+}$,
        $\{a : \actiontype\}$, $d : [0,1] \in \mathbb{R^+}$, $t_{\Delta} :
        \timetype$ \\

        \vspace*{-1.5mm} \bf Output & \vspace*{-1.5mm}
        $I_{\mFear} : \responsestrength \defEq
        \dfrac{g.\mathtt{importance}}{t_{\Delta} \cdot cp(\{a : \actiontype\},
            g, s, B, d)} $
        \vspace*{1mm}\\\hline
    \end{tabular}
\end{minipage}

\paragraph{Description} A change in the intensity of \textit{Fear} ($I_{\mFear}
: \responsestrength$) without energy considerations is:
\begin{itemize}
    \item Proportional to the importance of the affected goal
    ($g.\mathtt{importance}$)---a higher importance increases intensity

    \item Inversely proportional to the time that the event is expected
    ($t_{\Delta}$), such that a shorter expected time increases intensity

    \item Inversely proportional to the NPC's ability to mitigate risk as
    measured by their coping potential ($cp(\{a : \actiontype\}, g, s, B, d)$),
    such that a higher coping potential decreases intensity
\end{itemize}
\hrule

\paragraph{Sources} --

\paragraph{Depends On} \tyref{TY_Time}, \tyref{TY_DeltaIntensity}, 
\tyref{TY_WorldState}, \tyref{TY_Goal}, \waitref{TY_Action}, 
\waitref{T_FearIntensity}, \waitref{I_CopingPotential}

\paragraph{Ref. By} -- \\\hrule\vspace{0.5mm}\hrule

~\newline

\noindent
\begin{minipage}{\textwidth}
    \renewcommand*{\arraystretch}{1.5}
    \begin{tabular}{| p{\colAwidth}  p{\colBwidth}|}
        \hline
        \rowcolor[gray]{0.9}
        \bf WR\refstepcounter{waitnum}\thewaitnum \label{I_FearIntensityTE} &
        \bf Calculating the Intensity of \textit{Fear} without Energy or Time \\
        \hline
    \end{tabular}

    \renewcommand*{\arraystretch}{1.5}
    \begin{tabular}{ p{\colAwidth}  p{\colBwidth}}
        \bf Input & $g : \goaltype$, $s : \worldstatetype$, $B : \mathbb{R^+}$,
        $\{a : \actiontype\}$, $d : [0,1] \in \mathbb{R^+}$
        \\

        \vspace*{-1.5mm} \bf Output & \vspace*{-1.5mm}
        $I_{\mFear} : \responsestrength \defEq
        \dfrac{g.\mathtt{importance}}{cp(\{a : \actiontype\}, g, s, B, d)} $
        \vspace*{1mm}\\\hline
    \end{tabular}
\end{minipage}

\paragraph{Description} A change in the intensity of \textit{Fear} ($I_{\mFear}
: \responsestrength$) without energy or time considerations is:
\begin{itemize}
    \item Proportional to the importance of the affected goal
    ($g.\mathtt{importance}$)---a higher importance increases intensity

    \item Inversely proportional to the NPC's ability to mitigate risk as
    measured by their coping potential ($cp(\{a : \actiontype\}, g, s, B, d)$),
    such that a higher coping potential decreases intensity
\end{itemize}
\hrule

\paragraph{Sources} --

\paragraph{Depends On} \tyref{TY_DeltaIntensity}, \tyref{TY_WorldState}, 
\tyref{TY_Goal}, \waitref{TY_Action}, \waitref{T_FearIntensity}, 
\waitref{I_CopingPotential}

\paragraph{Ref. By} -- \\\hrule\vspace{0.5mm}\hrule

~\newline

\noindent
\begin{minipage}{\textwidth}
    \renewcommand*{\arraystretch}{1.5}
    \begin{tabular}{| p{\colAwidth}  p{\colBwidth}|}
        \hline
        \rowcolor[gray]{0.9}
        \bf WR\refstepcounter{waitnum}\thewaitnum \label{I_AngerIntensityS} &
        \bf Calculating the Intensity of \textit{Anger} with Stored Intensity
        and without a Social Relationship \\
        \hline
    \end{tabular}

    \renewcommand*{\arraystretch}{1.5}
    \begin{tabular}{ p{\colAwidth}  p{\colBwidth}}
        \bf Input & $g : \goaltype$, $s : \worldstatetype$, $B : \mathbb{R^+}$,
        $(\mActor : \actortype, a : \actiontype, \mDeliberate : \mathbb{B})$\\

        \bf Output & $I_{\mAnger} : \responsestrength \defEq \begin{cases}
            A, & acc((\mActor, a, \mDeliberate), g, s, B) = \text{Blame} \\
            0, & \text{Otherwise} \\
        \end{cases}$
        \\
        \vspace*{-2mm} & \vspace*{-2mm}
        where $A = g.\mathtt{importance} \cdot c(g, s,
        a.\mathtt{makesChange}(s),
        B) + I_{\mStore}$,
        \vspace*{1mm}\\
        & and $I_{\mStore} = \begin{cases}
            A, & acc((\mActor, a, \mDeliberate), g, s, B) \neq \text{Blame} \\
            0, & \text{Otherwise} \\
        \end{cases}$ \vspace*{1.5mm}\\\hline
    \end{tabular}
\end{minipage}

\paragraph{Description} A change in the intensity of \textit{Anger}
($I_{\mAnger} : \responsestrength$) is:
\begin{itemize}
    \item Proportional to the importance of the affected goal
    ($g.\mathtt{importance}$)---a higher importance increases intensity

    \item Proportional to the degree of \textit{Blame} that the offending
    $\mActor$ is perceived to have, such that a higher degree of control
    increases intensity

    \item Linearly increased by the total intensity of all previous encounters
    of \textit{Anger} since the last \textit{Blame} assignment where it could
    not be assigned ($I_{\mStore}$), representing a suppression of
    \textit{Anger} since no suitable target is available
\end{itemize}

If Blame has been successfully assigned to $\mActor$ in the evaluation of
accountability ($acc((\mActor, a,\allowbreak \mDeliberate), g, s,$ $B) = 
\text{Blame}$), the intensity is equivalent to $A$ and $I_{\mStore}$ is reset 
to $0$; otherwise, the emotion is suppressed and the intensity is $0$ and the 
stored intensity of \textit{Anger} is $I_{\mStore}$.

The degree of control that $\mActor$ has is an evaluation of how much harm the
action of $\mActor$ has caused, given as a congruence evaluation on their
action ($c(g, s, a.\mathtt{makesChange}(s)$). \\\hrule

\paragraph{Sources} --

\paragraph{Depends On} \tyref{TY_DeltaIntensity}, \tyref{TY_WorldState}, 
\tyref{TY_Goal}, \waitref{TY_Actor}, \waitref{TY_Action}, 
\waitref{T_GoalCongruence}, \waitref{T_Accountability}, 
\waitref{T_AngerIntensity}

\paragraph{Ref. By} -- \\\hrule\vspace{0.5mm}\hrule

~\newline

\noindent
\begin{minipage}{\textwidth}
    \renewcommand*{\arraystretch}{1.5}
    \begin{tabular}{| p{\colAwidth}  p{\colBwidth}|}
        \hline
        \rowcolor[gray]{0.9}
        \bf WR\refstepcounter{waitnum}\thewaitnum \label{I_AngerIntensityR} &
        \bf Calculating the Intensity of \textit{Anger} without Stored
        Intensity and with a Social Relationship \\
        \hline
    \end{tabular}

    \renewcommand*{\arraystretch}{1.5}
    \begin{tabular}{ p{\colAwidth}  p{\colBwidth}}
        \bf Input & $g : \goaltype$, $s : \worldstatetype$, $B : \mathbb{R^+}$,
        $(\mActor : \actortype, a : \actiontype, \mDeliberate : \mathbb{B})$,
        $rl : \socialattachmenttype$\\

        \bf Output & $I_{\mAnger} : \responsestrength \defEq \begin{cases}
            A, & acc((\mActor, a, \mDeliberate), g, s, B) = \text{Blame} \\
            0, & \text{Otherwise} \\
        \end{cases}$
        \\
        \vspace*{-2mm} & \vspace*{-2mm}
        where $A = g.\mathtt{importance} \cdot \mControl$
        \\
        & and $\mControl = \dfrac{rl.\mathtt{confidence}}{rl.\mathtt{security}}
        + c(g, s, a.\mathtt{makesChange}(s), B)$ \vspace*{1.5mm}\\\hline
    \end{tabular}
\end{minipage}

\paragraph{Description} A change in the intensity of \textit{Anger}
($I_{\mAnger} : \responsestrength$) is:
\begin{itemize}
    \item Proportional to the importance of the affected goal
    ($g.\mathtt{importance}$)---a higher importance increases intensity

    \item Proportional to the degree of \textit{Blame} that the offending
    $\mActor$ is perceived to have, represented as a perceived amount of
    control over their actions ($\mControl$), such that a higher degree of
    control increases intensity
\end{itemize}

If Blame has been successfully assigned to $\mActor$ in the evaluation of
accountability ($acc((\mActor, a,\allowbreak \mDeliberate), g, s,$ $B) = 
\text{Blame}$), the intensity is equivalent to the calculated intensity ($A$); 
otherwise, the emotion is suppressed and the intensity is $0$.

The degree of $\mControl$ that $\mActor$ has is a linear combination of two
parts. The first is an evaluation of the relationship between this NPC and
$\mActor$. The relationship is evaluated by: the confidence in the NPC's
knowledge of the relationship ($rl.\mathtt{confidence}$), such that a higher
confidence increases perceived control; and an evaluation of how much harm the
NPC believes $\mActor$ intends ($rl.\mathtt{security}$), such that a higher
value decreases perceived control. The second part is an evaluation of how much
harm the action of $\mActor$ has caused, given as a congruence evaluation on
their action ($c(g, s, a.\mathtt{makesChange}(s)$). \\\hrule

\paragraph{Sources} --

\paragraph{Depends On} \tyref{TY_DeltaIntensity}, \tyref{TY_WorldState}, 
\tyref{TY_Goal}, \waitref{TY_Actor}, \waitref{TY_Action}, 
\waitref{TY_SocialRelation}, \waitref{C_Anger}, \waitref{T_GoalCongruence}, 
\waitref{T_Accountability}, \waitref{T_AngerIntensity}

\paragraph{Ref. By} -- \\\hrule\vspace{0.5mm}\hrule

~\newline

\noindent
\begin{minipage}{\textwidth}
    \renewcommand*{\arraystretch}{1.5}
    \begin{tabular}{| p{\colAwidth}  p{\colBwidth}|}
        \hline
        \rowcolor[gray]{0.9}
        \bf WR\refstepcounter{waitnum}\thewaitnum \label{I_AngerIntensitySR} &
        \bf Calculating the Intensity of \textit{Anger} without Stored
        Intensity and without a Social Relationship \\
        \hline
    \end{tabular}

    \renewcommand*{\arraystretch}{1.5}
    \begin{tabular}{ p{\colAwidth}  p{\colBwidth}}
        \bf Input & $g : \goaltype$, $s : \worldstatetype$\, $B :
        \mathbb{R^+}$, $(\mActor : \actortype, a : \actiontype, \mDeliberate :
        \mathbb{B})$ \\

        \bf Output & $I_{\mAnger} : \responsestrength \defEq \begin{cases}
            A, & acc((\mActor, a, \mDeliberate), g, s, B) = \text{Blame} \\
            0, & \text{Otherwise} \\
        \end{cases}$
        \\
        \vspace*{-2mm} & \vspace*{-2mm}
        where $A = g.\mathtt{importance} \cdot c(g, s,
        a.\mathtt{makesChange}(s), B) $, \\\hline
    \end{tabular}
\end{minipage}

\paragraph{Description} A change in the intensity of \textit{Anger}
($I_{\mAnger} : \responsestrength$) is:
\begin{itemize}
    \item Proportional to the importance of the affected goal
    ($g.\mathtt{importance}$)---a higher importance increases intensity

    \item Proportional to the degree of \textit{Blame} that the offending
    $\mActor$ is perceived to have, represented as a perceived amount of
    control over their actions, such that a higher degree of control increases
    intensity
\end{itemize}

If Blame has been successfully assigned to $\mActor$ in the evaluation of
accountability ($acc((\mActor, a,\allowbreak \mDeliberate), g, s,$ $B) = 
\text{Blame}$), the intensity is equivalent to the calculated intensity ($A$); 
otherwise, the emotion is suppressed and the intensity is $0$.

The degree of control that $\mActor$ has is an evaluation of how much harm the
action of $\mActor$ has caused, given as a congruence evaluation on their
action ($c(g, s, a.\mathtt{makesChange}(s)$). \\\hrule

\paragraph{Sources} --

\paragraph{Depends On} \tyref{TY_DeltaIntensity}, \tyref{TY_WorldState}, 
\tyref{TY_Goal}, \waitref{TY_Actor}, \waitref{TY_Action}, \waitref{C_Anger}, 
\waitref{T_GoalCongruence}, \waitref{T_Accountability}, 
\waitref{T_AngerIntensity}

\paragraph{Ref. By} -- \\\hrule\vspace{0.5mm}\hrule