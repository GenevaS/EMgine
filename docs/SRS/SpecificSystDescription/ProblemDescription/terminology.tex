\subsubsection{Terminology and  Definitions}\label{sec_terms}
The Conceptual Models (Section~\ref{sec_conceptual}) contain many definitions 
necessary for reducing ambiguity and making it easier to understand models and 
requirements. These are additional terms that do not appear in the Conceptual 
Models, but are still useful for the same reasons. \\

\noindent\textbf{Affect}: Any type of affective experience that has a
changeable, short-term state originating from the body and impacting the
mind in some way~\citep{barrett2009affect, oxfordAffect}. A general
affective state is generally identified as weaker than an emotion state. \\

\noindent\textbf{Affective Science}: The interdisciplinary study of affective 
phenomena, related processes, and its influencing 
factors~\citep[p.~xiii]{davidson2003handbook}. \\

\noindent\textbf{Appraisal Theories (Perspective)}: These theories shift the 
evaluation of the environment from its objective qualities to its relation to 
an individual's well-being~\citep[p.~86]{smith2000consequences}. They mainly 
focus on the link between cognitive processing and emotion 
elicitation~\citep[p.~354]{broekens2021emotion}. \\

\noindent\textbf{Arousal}: A short-term state of excitement or energy
expenditure~\citep{oxfordArousal}. \\

\noindent\textbf{Cognition}: ``All forms of knowing and awareness, such as
perceiving, conceiving, remembering, reasoning, judging, imagining, and problem
solving''~\citep{cognitiondef}. \\

\noindent\textbf{Core Affect}: Based on affect, ``...a state of pleasure or
displeasure with some degree of arousal''~\citep[p.~170]{barrett2009affect}.
See also \textit{Affect}. \\

\noindent\textbf{Dimensional Theories (Perspective)}: These theories can more 
easily distinguish between different emotions (\citepg{scherer2010emotion}{12}; 
\citepg{smith1985patterns}{813}) using a small number of continuous dimensions. 
They view emotion as an individual's interpretation of their current ``core 
affect''~\citep[p.~353]{broekens2021emotion} (i.e. elementary affective 
feelings). While any point in dimensional space is part of ``core 
affect''~\citep[p.~97]{lisetti2015and}, it is possible for individuals to 
verbally label points representing their subjective feeling of an 
emotion~\citep[p.~12]{scherer2010emotion}. This creates an effect of 
``plotting'' emotion categories (e.g. discrete theories) as points in the 
space. \\

\noindent\textbf{Discrete Theories (Perspective)}: These theories propose that 
innate, hard-wired circuits or programs elicit emotions 
(\citepg{ortony2021all}{41}; \citepg{scherer2021towards}{280}). One of the core 
features of these theories is the definition of distinct
emotion categories or types, like \textit{Joy} and \textit{Anger}, that are
recognizable by a set of observable features (e.g. facial expression, typical
behaviours). This ``...fits with the way we talk about emotion every
day...people automatically and effortlessly perceive emotion in themselves and
others...''~\citep[p.~47--48]{barrett2006emotions}. \\

\noindent\textbf{Emotion Expression}: An affective computing task where a CME
changes an entity's ``observable'' behaviour (e.g. facial expressions,
gestures, or movements) given an emotion state (\citepg{scherer2010emotion}{4};
\citepg{fathalla2020emotional}{2}). \\

\noindent\textbf{Emotion Generation}: An affective computing task where a CME
produces an emotion state given the current program and environment state
(\citepg{scherer2010emotion}{4}; \citepg{fathalla2020emotional}{2}). \\

\noindent\textbf{Emotion Kind}: Names given to discrete categories/types of
emotion (e.g. \textit{Joy}, \textit{Sadness}). \\

\noindent\textbf{Emotion State}: The set of values for each possible emotion
kind. \\

\noindent\textbf{Entity}: Any discrete, identifiable, and separate object that
is significant in and of themselves. \\

\noindent\textbf{``Fast, Primary Emotions''}: Hard-wired and potentially 
inaccurate responses to innate knowledge elicited by fundamental mechanisms 
(e.g. instinctual fear of pain)~\citep[p.~60--70]{picard1997affective}. \\

\noindent\textbf{Feeling}: Conscious mental representations and interpretations
of an emotional response, and follow emotions evolutionarily and experientially
(\citepg{oxfordFeelings}{184}; \citepg{scherer2000psychological}{139}).
Feelings are ill-defined from a modelling perspective, and rarely, if ever,
appear computationally. They are also, allegedly, unnecessary for understanding
emotional behaviours~\citep{fellous2004human}. \\

\noindent\textbf{Game World}: An imaginary universe in which game events  take
place~\citep{adams2014fundamentals}. They are often two or three dimensional
spaces containing characters and objects. \\

\noindent\textbf{Global Adaptational Problem}: In PES, a survival-related issue
that emotion-based behaviours evolved to address and are found in some form at
all evolutionary levels~\citep{robert1980emotion}. \\

\noindent\textbf{Gustatory}: Concerned with tasting or the sense of taste. In
CTE, this is associated with the mental process of withdrawing from toxins and
potentially infectious agents~\citep[p.~57]{oatley1992best}. In this way, it
could also be a prototype for other withdrawal-based emotions (e.g.
\textit{Hatred}, \textit{Contempt}). \\

\noindent\textbf{``Model of Self''}: In CTE, a cognitive representation of the
individual's own goals, abilities, habits, and bodies, chiefly in relation with
others~\citep[p.~195]{oatley1992best}. \\

\noindent\textbf{Mood}: Enduring, less intense, and more diffuse states than
emotions~\citep{oxfordMood}. Their presence is typically unclear to the
experiencing individual and often have a more prolonged influence on an
individual’s cognition and behaviours. \\

\noindent\textbf{Non-propositional Meaning}: In relation to communication,
signals with no symbolic structure or literal meaning of significance within a
system~\citep[p.~32]{oatley1987towards}. \\

\noindent\textbf{Personality}: Permanent or difficult to change variables that
impact affective processes~\citep{oxfordPersonality}. \\

\noindent\textbf{Propositional Meaning}: In relation to communication, signals
having a symbolic structure with a literal meaning of significance to a
system~\citep[p.~32]{oatley1987towards}. \\

\noindent\textbf{Self-Preservational}: In relation to self-preservation,
especially regarded as a basic instinct in human beings and animals, the
protection of oneself from harm or death. \\

\noindent\textbf{``Slow, Secondary Emotions''}: Sometimes called 
``cognitively-generated emotions'', these require some level of reasoning to 
elicit (e.g. learned fear of public 
speaking)~\citep[p.~60--70]{picard1997affective}. \\

\noindent\textbf{Valence}: Describes the positive or negative character of
emotions, their response components, and emotion-eliciting
stimuli~\cite{oxfordValence}. In affective science, it often appears as a
dimensional variable with positive and negative poles. Assigning valence to
an emotion depends on its context (e.g. behavioural tendencies vs. social
interactions).