\subsubsection{Data Types}\label{sec_typedefs}
This section defines types for realizing \progname{}'s Instance Models
(Section~\ref{sec_instance}). Data types also allow for type checking to verify
and enforce constraints (Section~\ref{sec_DataConstraints}), and defining model
interfaces.

Data types \tyref{TY_Time}, \tyref{TY_WorldState}, \tyref{TY_WorldStateChange},
\tyref{TY_DistanceBetweenWorldStates},
\tyref{TY_DistanceBetweenWorldStatesChange}, \tyref{TY_Goal}, \tyref{TY_Plan},
\tyref{TY_Relation-CTE}, and \tyref{TY_Attention} are APIs for users to
implement for their specific needs.
\begin{itemize}
    \item \tyref{TY_Time}, \tyref{TY_WorldState}, \tyref{TY_WorldStateChange},
    \tyref{TY_DistanceBetweenWorldStates},
    \tyref{TY_DistanceBetweenWorldStatesChange}, and \tyref{TY_Goal} are
    mandatory

    \item \tyref{TY_Plan} is optional, but necessary for modelling
    \textit{Anger} and using the complete \textit{Sadness} model

    \item \tyref{TY_Relation-CTE} is optional, but necessary for modelling
    \textit{Acceptance}

    \item \tyref{TY_Attention} is optional, but necessary for modelling
    \textit{Interest}

\end{itemize}

~\newline\noindent
\begin{minipage}{\textwidth}
    \renewcommand*{\arraystretch}{1.5}
    \begin{tabular}{| p{\colAwidth}  p{\colBwidth}|}
        \hline
        \rowcolor[gray]{0.9}
        \bf TY\refstepcounter{typenum}\thetypenum
        \label{TY_EmotionIntensity} & \bf Type of Emotion Intensity \\
        \hline
    \end{tabular}

    \renewcommand*{\arraystretch}{1.5}
    \begin{tabular}{ p{\colAwidth}  p{\colBwidth}}
        \bf Symbol & $\emotionintensitytype$ \\

        \bf Type & $ \mathbb{R}_{\geq0} $
        \\\hline
    \end{tabular}
\end{minipage}

\paragraph{Description} Emotion intensity (\tref{T_CalculateEmotionIntensity},
\tref{T_CalculateEmotionAcceptance}, \tref{T_CalculateEmotionInterest},
\tref{T_CalculateEmotionSurprise}) is a non-negative real value
(\aref{A_ContinuousIntensity}, \aref{A_PositiveIntensity}). \\\hrule

\paragraph{Sources} --

\paragraph{Depends On} \aref{A_ContinuousIntensity}, \aref{A_PositiveIntensity},
\tref{T_CalculateEmotionIntensity}, \tref{T_CalculateEmotionAcceptance},
\tref{T_CalculateEmotionInterest}, \tref{T_CalculateEmotionSurprise}

\paragraph{Ref. By} \tyref{TY_DeltaIntensity}, \tyref{TY_EmotionState},
\tyref{TY_EmotionDecayState}, \iref{IM_DecayEmotionIntensity},
\iref{IM_UpdateEmotionState}, \rref{R_IntensityTypeUse},
\rref{R_MixingEmotionsPES}, \rref{R_PartitionEmotions}
\\\hrule\vspace{0.5mm}\hrule

~\newline

\noindent
\begin{minipage}{\textwidth}
    \renewcommand*{\arraystretch}{1.5}
    \begin{tabular}{| p{\colAwidth}  p{\colBwidth}|}
        \hline
        \rowcolor[gray]{0.9}
        \bf TY\refstepcounter{typenum}\thetypenum
        \label{TY_DeltaIntensity} & \bf Type of Emotion Intensity Change \\
        \hline
    \end{tabular}

    \renewcommand*{\arraystretch}{1.5}
    \begin{tabular}{ p{\colAwidth}  p{\colBwidth}}
        \bf Symbol & $ \responsestrength $ \\

        \bf Type & $ \mathbb{R} $ \\\hline
    \end{tabular}
\end{minipage}

\paragraph{Description} A change in emotion intensity represents the magnitude
of a difference to be applied to an emotion intensity
(\tyref{TY_EmotionIntensity}). It is a real value because:
\begin{itemize}
    \item Emotion intensity is assumed to both increase and decrease
    (\aref{A_RealIntensityChanges}), and

    \item The magnitude of a change should not be explicitly coupled to fit the
    type of $\emotionintensitytype$
\end{itemize} \hrule

\paragraph{Sources} --

\paragraph{Depends On} \aref{A_RealIntensityChanges},
\tyref{TY_EmotionIntensity}

\paragraph{Ref. By} \iref{IM_JoyIntensity}, \iref{IM_SadnessIntensity},
\iref{IM_FearIntensity}, \iref{IM_AngerIntensity}, \iref{IM_DisgustIntensity},
\iref{IM_CalculateEmotionAcceptance}, \iref{IM_CalculateEmotionInterest},
\iref{IM_CalculateEmotionSurprise}, \iref{IM_UpdateEmotionState},
\iref{IM_UpdateEmotionState2}, \rref{R_IntensityChangeType}
\\\hrule\vspace{0.5mm}\hrule

~\newline

\noindent
\begin{minipage}{\textwidth}
    \renewcommand*{\arraystretch}{1.5}
    \begin{tabular}{| p{\colAwidth}  p{\colBwidth}|}
        \hline
        \rowcolor[gray]{0.9}
        \bf TY\refstepcounter{typenum}\thetypenum
        \label{TY_EmotionDecay} & \bf Type of Emotion Intensity Decay Rate \\
        \hline
    \end{tabular}

    \renewcommand*{\arraystretch}{1.5}
    \begin{tabular}{ p{\colAwidth}  p{\colBwidth}}
        \bf Symbol & $ \emotiondecaytype $ \\

        \bf Type & $ \mathbb{R}_{>0} $ \\\hline
    \end{tabular}
\end{minipage}

\paragraph{Description} Emotion decay rate (\tref{T_DecayEmotionState}) is a
real-valued, strictly positive constant. It is equivalent to the spring
constant $k_s$ in a damped harmonic oscillator, which is always strictly
positive.

Note that this tightly couples the emotion decay model
(\iref{IM_DecayEmotionState}) with the data type such that changing the
underlying model of one likely means changing the other as well. \\\hrule

\paragraph{Sources}
{\small\url{https://en.wikipedia.org/wiki/Harmonic_oscillator#Damped_harmonic_oscillator}}

\paragraph{Depends On} \tref{T_DecayEmotionState},
\iref{IM_DecayEmotionIntensity}

\paragraph{Ref. By} \tyref{TY_EmotionDecayState},
\iref{IM_DecayEmotionIntensity}, \rref{R_IntensityDecayType}
\\\hrule\vspace{0.5mm}\hrule

~\newline

\noindent
\begin{minipage}{\textwidth}
    \renewcommand*{\arraystretch}{1.5}
    \begin{tabular}{| p{\colAwidth}  p{\colBwidth}|}
        \hline
        \rowcolor[gray]{0.9}
        \bf TY\refstepcounter{typenum}\thetypenum
        \label{TY_EmotionKind} & \bf Type of Emotion Kinds \\
        \hline
    \end{tabular}

    \renewcommand*{\arraystretch}{1.5}
    \begin{tabular}{ p{\colAwidth}  p{\colBwidth}}
        \bf Symbol & $\emotionkindstype$ \\

        \bf Type & $ \left< \mFear, \mAnger, \mSadness, \mJoy, \mInterest,
        \mSurprise, \mDisgust, \mTrust \right> $ \\

        \bf Invariants & $\bullet$ $\emotionkindstype$ is finite and ordered

        $\bullet$ Each element of $\emotionkindstype$ is uniquely paired with
        exactly one other element that is not itself \\
        \hline
    \end{tabular}
\end{minipage}

\paragraph{Description} Emotion kinds is an enumeration representing the PES
emotion kinds (\cref{C_EmotionStruct}). ``Opposing'' emotions as defined in PES
are enforced via the invariant regarding unique pairings. They are not
functionally paired such that the experience of one reduces the experience of
the other because it would make a state of ``deep sleep'' impossible
(\aref{A_EmotionPairs}).

Defining Emotion Kinds independently of Emotion Intensity
(\tyref{TY_EmotionIntensity}) allows them to be theory-agnostic (i.e. not
strictly tied to PES). This means that \progname{} can still be functional if
there is \textit{some} definition for Emotion Kinds, which could have differing
types and number labels than PES. \\\hrule

\paragraph{Sources} --

\paragraph{Depends On} \aref{A_EmotionPairs}, \cref{C_EmotionStruct}

\paragraph{Ref. By} \tyref{TY_EmotionState}, \tyref{TY_EmotionDecayState},
\iref{IM_UpdateEmotionState2}, \rref{R_EmotionKindsType},
\rref{R_MixingEmotionsPES}, \rref{R_PartitionEmotions},
\rref{R_MixingEmotionsCTE} \\\hrule\vspace{0.5mm}\hrule

~\newline

\noindent
\begin{minipage}{\textwidth}
    \renewcommand*{\arraystretch}{1.5}
    \begin{tabular}{| p{\colAwidth}  p{\colBwidth}|}
        \hline
        \rowcolor[gray]{0.9}
        \bf TY\refstepcounter{typenum}\thetypenum
        \label{TY_EmotionState} & \bf Type of Emotion State \\
        \hline
    \end{tabular}

    \renewcommand*{\arraystretch}{1.5}
    \begin{tabular}{ p{\colAwidth}  p{\colBwidth}}
        \bf Symbol & $\emotionstatetype$ \\

        \bf Type & $ \left\{ \mathtt{intensities} : \emotionkindstype
        \rightarrow \emotionintensitytype, \mathtt{max} : \emotionkindstype
        \rightarrow \emotionintensitytype \right\}  $ \\

        \bf Invariants & $\bullet$ $\forall k : \emotionkindstype \rightarrow 0
        \leq \mathtt{intensities}(k) \leq \mathtt{max}(k) $

        $\bullet$ $\exists k : \emotionkindstype \rightarrow \mathtt{max}(k) >
        0 $ \\

        \hline
    \end{tabular}
\end{minipage}

\paragraph{Description} An emotion state is a record with:
\begin{itemize}

    \item A function $\mathtt{intensities}$ mapping Emotion Kinds
    (\tyref{TY_EmotionKind}) to Emotion Intensities
    (\tyref{TY_EmotionIntensity}), representing the current intensity of each
    emotion kind in the state. This is similar to a vector of intensity values,
    which Computational Models of Emotion commonly use to represent affective
    states.

    \item A function $\mathtt{max}$ mapping Emotion Kinds
    (\tyref{TY_EmotionKind}) to a maximum intensity value
    (\tyref{TY_EmotionIntensity}). This encodes a maximum intensity for each
    emotion kind individually, allowing users to vary this value between
    emotion kinds in a state. Storing maximum intensities in the Emotion State
    Data Type localizes these constraints to a specific state, allowing the
    maximum intensities of other states to vary. This also makes it easy to
    ensure that its constraints are satisfied when updating
    $\mathtt{intensities}$. It must satisfy the invariant $\exists k :
    \emotionkindstype \rightarrow \mathtt{max}(k) > 0 $ to prevent situations
    where every value in $\mathtt{max}$ is zero (i.e. constantly zero,
    \aref{A_PositiveIntensity}). This is assumed to be a state of ``deep
    sleep'', so at least one emotion type must have a non-zero maximum for the
    entity to be ``awake''.

\end{itemize}

\noindent The invariant ensures that emotion intensity is finite
(\aref{A_LimitIntensity}). Each element in $\emotionstatetype$ is independent
(\aref{A_EmotionPairs}, \cref{C_Emotion}).\\\hrule

\paragraph{Sources} \citet[p.~358]{broekens2021emotion}

\paragraph{Depends On} \aref{A_PositiveIntensity}, \aref{A_EmotionPairs},
\aref{A_LimitIntensity}, \cref{C_Emotion}, \tyref{TY_EmotionIntensity},
\tyref{TY_EmotionKind}

\paragraph{Ref. By} \tyref{TY_Emotion}, \iref{IM_UpdateEmotionState2},
\iref{IM_UpdateEmotion}, \iref{IM_GetEmotionState}, \iref{IM_DecayEmotionState},
\iref{IM_GetEmotionStatePAD}, \rref{R_EmotionStateType}
\\\hrule\vspace{0.5mm}\hrule

%~\newline
\clearpage

\noindent
\begin{minipage}{\textwidth}
    \renewcommand*{\arraystretch}{1.5}
    \begin{tabular}{| p{\colAwidth}  p{\colBwidth}|}
        \hline
        \rowcolor[gray]{0.9}
        \bf TY\refstepcounter{typenum}\thetypenum
        \label{TY_EmotionDecayState} & \bf Type of Emotion Decay State \\
        \hline
    \end{tabular}

    \renewcommand*{\arraystretch}{1.5}
    \begin{tabular}{ p{\colAwidth}  p{\colBwidth}}
        \bf Symbol & $ \emotionstatedecaytype $ \\

        \bf Type & $ \left\{ \mathtt{equilibrium} : \emotionkindstype
        \rightarrow \emotionintensitytype, \mathtt{decayRates} :
        \emotionkindstype \rightarrow \emotiondecaytype \right\} $ \\

        \bf Invariant & $\exists k \in \emotionkindstype \rightarrow
        \mathtt{equilibrium}\left(k\right) > 0$ \vspace*{2mm}\\\hline
    \end{tabular}
\end{minipage}

\paragraph{Description} Emotion state decay (\tref{T_DecayEmotionState}) is a
record tied to Emotion State via $\emotionkindstype$ (\tyref{TY_EmotionKind})
that has:
\begin{itemize}

    \item The function $\mathtt{equilibrium}$ maps Emotion Kinds to Emotion
    Intensities ($\emotionkindstype \rightarrow \emotionintensitytype$,
    \tyref{TY_EmotionIntensity}), encoding the ``equilibrium'' intensity for
    each $k : \emotionkindstype$. It must satisfy the invariant $\exists k \in
    \emotionkindstype \rightarrow \mathtt{equilibrium}\left(k\right) > 0$ to
    prevent situations where every value in $\mathtt{equilibrium}$ is zero (i.e.
    constantly zero, \aref{A_PositiveIntensity}), assumed to be a state of
    ``deep sleep''. Therefore, at least one emotion type in the equilibrium
    state must be non-zero for the entity to be ``awake''.

    \item The function $\mathtt{decayRates}$ maps Emotion Kinds to Emotion
    Intensity Decay Rates ($\emotionkindstype \rightarrow \emotiondecaytype$,
    \tyref{TY_EmotionDecay}), encoding the decay rate for each emotion kind.
    This allows users to vary decay rates between kinds
    (\tref{T_DecayEmotionState}).

\end{itemize}

Collecting this information into Emotion Decay State makes it easier to
maintain and access for automating a decay process (\ref{easeAuto}). \\\hrule

\paragraph{Sources} --

\paragraph{Depends On} \aref{A_PositiveIntensity}, \tref{T_DecayEmotionState},
\tyref{TY_EmotionIntensity}, \tyref{TY_EmotionDecay}, \tyref{TY_EmotionKind}

\paragraph{Ref. By} \iref{IM_DecayEmotionState},
\iref{IM_GetNextEmotionByDecay}, \rref{R_EmotionDecayStateType}
\\\hrule\vspace{0.5mm}\hrule

~\newline

\noindent
\begin{minipage}{\textwidth}
    \renewcommand*{\arraystretch}{1.5}
    \begin{tabular}{| p{\colAwidth}  p{\colBwidth}|}
        \hline
        \rowcolor[gray]{0.9}
        \bf TY\refstepcounter{typenum}\thetypenum
        \label{TY_Emotion} & \bf Type of Emotion \\
        \hline
    \end{tabular}

    \renewcommand*{\arraystretch}{1.5}
    \begin{tabular}{ p{\colAwidth}  p{\colBwidth}}
        \bf Symbol & $\emotiontype$\\

        \bf Type & $ \timetype \rightarrow \emotionstatetype $ \\

        \hline
    \end{tabular}
\end{minipage}

\paragraph{Description} Emotion $\emotiontype$ (\cref{C_Emotion}) is the
assignment of an emotion state $\emotionstatetype$ (\tyref{TY_EmotionState}) to
an instant in time $t : \timetype$ (\tyref{TY_Time}).
\\\hrule

\paragraph{Sources} --

\paragraph{Depends On} \cref{C_Emotion}, \tyref{TY_EmotionState},
\tyref{TY_Time}

\paragraph{Ref. By} \iref{IM_UpdateEmotion}, \iref{IM_GetEmotionState},
\iref{IM_GetNextEmotionByDecay}, \rref{R_EmotionType}
\\\hrule\vspace{0.5mm}\hrule

~\newline

\noindent
\begin{minipage}{\textwidth}
    \renewcommand*{\arraystretch}{1.5}
    \begin{tabular}{| p{\colAwidth}  p{\colBwidth}|}
        \hline
        \rowcolor[gray]{0.9}
        \bf TY\refstepcounter{typenum}\thetypenum
        \label{TY_PAD} & \bf Type of PAD Point \\
        \hline
    \end{tabular}

    \renewcommand*{\arraystretch}{1.5}
    \begin{tabular}{ p{\colAwidth}  p{\colBwidth}}
        \bf Symbol & $\padpoint$ \\

        \bf Type & $ \left( \mathtt{pleasure} : \left[-1,1\right] \subset
        \mathbb{R}, \mathtt{arousal} : \left[-1,1\right] \subset \mathbb{R},
        \mathtt{dominance} : \left[-1,1\right] \subset \mathbb{R} \right) $ \\

        \hline
    \end{tabular}
\end{minipage}

\paragraph{Description} A PAD point (\cref{C_PAD}) is a 3-tuple representing a
point in the space defined by the \textit{pleasure} ($P$), \textit{arousal}
($A$), and \textit{dominance} ($D$) dimensions (\tref{T_GetEmotionStatePAD}).
\\\hrule

\paragraph{Sources} \cite{mehrabian1980basic}

\paragraph{Depends On} \cref{C_PAD}

\paragraph{Ref. By} \iref{IM_GetEmotionStatePAD}, \rref{R_PADPointType}
\\\hrule\vspace{0.5mm}\hrule

~\newline

\noindent
\begin{minipage}{\textwidth}
    \renewcommand*{\arraystretch}{1.5}
    \begin{tabular}{| p{\colAwidth}  p{\colBwidth}|}
        \hline
        \rowcolor[gray]{0.9}
        \bf TY\refstepcounter{typenum}\thetypenum
        \label{TY_Time} & \bf Type of Time and Delta Time \\
        \hline
    \end{tabular}

    \renewcommand*{\arraystretch}{1.5}
    \begin{tabular}{ p{\colAwidth}  p{\colBwidth}}
        \bf Symbol & $\timetype, \deltatimetype$ \\

        \bf Type & -- \\\hline
    \end{tabular}
\end{minipage}

\paragraph{Description} Time is an abstract type that is linearly ordered. A
Delta Time is the difference between two Time values. \\\hrule

\paragraph{Sources} --

\paragraph{Depends On} --

\paragraph{Ref. By} \tyref{TY_Emotion}, \tyref{TY_Attention},
\iref{IM_DecayEmotionIntensity}, \iref{IM_UpdateEmotion},
\iref{IM_GetEmotionState}, \iref{IM_DecayEmotionState},
\iref{IM_GetNextEmotionByDecay}, \rref{R_TimeType} \\\hrule\vspace{0.5mm}\hrule

~\newline

\noindent
\begin{minipage}{\textwidth}
    \renewcommand*{\arraystretch}{1.5}
    \begin{tabular}{| p{\colAwidth}  p{\colBwidth}|}
        \hline
        \rowcolor[gray]{0.9}
        \bf TY\refstepcounter{typenum}\thetypenum
        \label{TY_WorldState} & \bf Type of Game World State View (WSV) \\
        \hline
    \end{tabular}

    \renewcommand*{\arraystretch}{1.5}
    \begin{tabular}{ p{\colAwidth}  p{\colBwidth}}
        \bf Symbol & $\worldstatetype$ \\

        \bf Type & -- \\

        \hline
    \end{tabular}
\end{minipage}

\paragraph{Description} A world state ``view'' is a subset of the game world
state $\mathbb{W}$, an abstract type describing a configuration of game world
characters, objects, and variables (\aref{A_Events}). A subset is sufficient
because entities only need to know aspects that are relevant to them, which is
frequently only a portion of $\mathbb{W}$. \\\hrule

\paragraph{Sources} --

\paragraph{Depends On} \aref{A_Events}

\paragraph{Ref. By} \tyref{TY_DistanceBetweenWorldStates},
\tyref{TY_DistanceBetweenWorldStatesChange}, \tyref{TY_Goal}, \tyref{TY_Plan},
\iref{IM_ElicitJoy}, \iref{IM_ElicitSadness}, \iref{IM_ElicitFear},
\iref{IM_ElicitAnger}, \iref{IM_ElicitDisgust},
\iref{IM_CalculateEmotionAcceptanceElicit},
\iref{IM_CalculateEmotionSurpriseElicit}, \iref{IM_SadnessIntensity},
\rref{R_WorldType} \\\hrule\vspace{0.5mm}\hrule

~\newline

\noindent
\begin{minipage}{\textwidth}
    \renewcommand*{\arraystretch}{1.5}
    \begin{tabular}{| p{\colAwidth}  p{\colBwidth}|}
        \hline
        \rowcolor[gray]{0.9}
        \bf TY\refstepcounter{typenum}\thetypenum
        \label{TY_WorldStateChange} & \bf Type of Game World Event \\
        \hline
    \end{tabular}

    \renewcommand*{\arraystretch}{1.5}
    \begin{tabular}{ p{\colAwidth}  p{\colBwidth}}
        \bf Symbol & $\worldstatechangetype$ \\

        \bf Type & -- \\

        \hline
    \end{tabular}
\end{minipage}

\paragraph{Description} A game world event is an abstract type describing a
game action that changes a game world's configuration of characters, objects,
and/or variables. An event is a subset of elements in a WSV
(\tyref{TY_WorldState}) because evaluations only need to know about
``changing'' aspects relevant to the entity.

\progname{} evaluates the next WSV caused by an event by applying the event to
the current WSV $\worldstatetype \times \worldstatechangetype$. \progname{}
evaluate this as $s \oplus s_{\Delta}$, shorthand for $\mathtt{apply}(x, y)$,
which is a function that changes $x$ by $y$ (\aref{A_Events}). \\\hrule

\paragraph{Sources} --

\paragraph{Depends On} \aref{A_Events}

\paragraph{Ref. By} \tyref{TY_DistanceBetweenWorldStatesChange},
\tyref{TY_Goal}, \tyref{TY_Plan}, \iref{IM_ElicitJoy}, \iref{IM_ElicitSadness},
\iref{IM_ElicitFear}, \iref{IM_ElicitAnger}, \iref{IM_ElicitDisgust},
\iref{IM_CalculateEmotionAcceptanceElicit},
\iref{IM_CalculateEmotionSurpriseElicit}, \rref{R_WorldChangeType}
\\\hrule\vspace{0.5mm}\hrule

~\newline

\noindent
\begin{minipage}{\textwidth}
    \renewcommand*{\arraystretch}{1.5}
    \begin{tabular}{| p{\colAwidth}  p{\colBwidth}|}
        \hline
        \rowcolor[gray]{0.9}
        \bf TY\refstepcounter{typenum}\thetypenum
        \label{TY_DistanceBetweenWorldStates} & \bf Type of Distance
        Between Two Game World States \\
        \hline
    \end{tabular}

    \renewcommand*{\arraystretch}{1.5}
    \begin{tabular}{ p{\colAwidth}  p{\colBwidth}}
        \bf Symbol & $\statedistancetype$ \\

        \bf Type & -- \\

        \hline
    \end{tabular}
\end{minipage}

\paragraph{Description} The distance between two WSVs $s_x : \worldstatetype$
and $s_y : \worldstatetype$ (\tyref{TY_WorldState}) is an abstract type
describing the difference between each configuration element in $s_x$ and
$s_y$. \\\hrule

\paragraph{Sources} --

\paragraph{Depends On} \tyref{TY_WorldState}

\paragraph{Ref. By} \tyref{TY_Goal}, \iref{IM_ElicitJoy},
\iref{IM_ElicitSadness}, \iref{IM_ElicitFear}, \iref{IM_DisgustIntensity},
\rref{R_DistanceType} \\\hrule\vspace{0.5mm}\hrule

~\newline

\noindent
\begin{minipage}{\textwidth}
    \renewcommand*{\arraystretch}{1.5}
    \begin{tabular}{| p{\colAwidth}  p{\colBwidth}|}
        \hline
        \rowcolor[gray]{0.9}
        \bf TY\refstepcounter{typenum}\thetypenum
        \label{TY_DistanceBetweenWorldStatesChange} & \bf Type of Change in
        Distance of a Game World State \\
        \hline
    \end{tabular}

    \renewcommand*{\arraystretch}{1.5}
    \begin{tabular}{ p{\colAwidth}  p{\colBwidth}}
        \bf Symbol & $\statedistancechangetype$ \\

        \bf Type & -- \\
        \hline
    \end{tabular}
\end{minipage}

\paragraph{Description} A change in distance of a WSV $s : \worldstatetype$
(\tyref{TY_WorldState}) caused by a game world event $s_\Delta :
\worldstatechangetype$ (\tyref{TY_WorldStateChange}) is an abstract type
describing the magnitude of change that $s_\Delta$ causes in $s$. \\\hrule

\paragraph{Sources} --

\paragraph{Depends On} \tyref{TY_WorldState}, \tyref{TY_WorldStateChange}

\paragraph{Ref. By} \tyref{TY_Goal}, \iref{IM_ElicitJoy},
\iref{IM_ElicitSadness}, \iref{IM_ElicitFear}, \iref{IM_ElicitDisgust},
\iref{IM_CalculateEmotionAcceptanceElicit}, \iref{IM_JoyIntensity},
\iref{IM_FearIntensity}, \iref{IM_CalculateEmotionAcceptance},
\rref{R_DistanceChangeType} \\\hrule\vspace{0.5mm}\hrule

~\newline

\noindent
\begin{minipage}{\textwidth}
    \renewcommand*{\arraystretch}{1.5}
    \begin{tabular}{| p{\colAwidth}  p{\colBwidth}|}
        \hline
        \rowcolor[gray]{0.9}
        \bf TY\refstepcounter{typenum}\thetypenum
        \label{TY_Goal} & \bf Type of a Goal \\
        \hline
    \end{tabular}

    \renewcommand*{\arraystretch}{1.5}
    \begin{tabular}{ p{\colAwidth}  p{\colBwidth}}
        \bf Symbol & $ \goaltype $ \\

        \bf Type & $ \{ \mathtt{goalState} : \worldstatetype \rightarrow
            \mathbb{B}, \mathtt{goal} : \worldstatetype \rightarrow
            \statedistancetype, \mathtt{goal'} : \worldstatetype \times
            \worldstatechangetype \rightarrow \statedistancechangetype,
            \mathtt{importance} : \mathbb{R}_{\geq0}, $
            $ \mathtt{type} \subseteq \{ \mathtt{SelfPreservation},
            \mathtt{Gustatory} \} \} $ \\
        \hline
    \end{tabular}
\end{minipage}

\paragraph{Description} A goal (\cref{C_Goals}) is represented by:
\begin{itemize}

    \item The $\mathtt{goalState}$ predicate represents the entity's desired
    WSV $\worldstatetype$ (\tyref{TY_WorldState}). If they are not already in
    $\mathtt{goalState}$ and striving to maintain it, they are in another state
    and want to move towards or away from $\mathtt{goalState}$.

    \item The function $\mathtt{goal}$ maps a WSV $\worldstatetype$ to a
    distance $\statedistancetype$ (\tyref{TY_DistanceBetweenWorldStates})
    between it and $\mathtt{goalState}$ to measure the difference between the
    current WSV and desired one.

    \item The function $\mathtt{goal'}$ is the derivative of $\mathtt{goal}$,
    measuring a change in the distance $\statedistancechangetype$
    (\tyref{TY_DistanceBetweenWorldStatesChange}) to $\mathtt{goalState}$ when
    a WSV $s: \worldstatetype$ is changed by a game event $s_{\Delta} :
    \worldstatechangetype$ (\tyref{TY_WorldStateChange}).

    \item The goal's perceived relative $\mathtt{importance}$ to the entity
    such that higher values reflect a higher importance, mimicking an
    individual's tendency to be motivated to achieve higher importance goals
    over lower importance ones. If this is set to zero, \progname{} assumes
    that the goal has no importance to the entity and does not trigger emotion
    processes when affected by world events.

    \item CTE's descriptions of emotion-elicitation conditions imply that goals
    can be assigned the types of \textit{Self-Preservation} and/or
    \textit{Gustatory} (\tref{T_CalculateEmotionGP}). Goal $\mathtt{type}$
    stores this information, allowing a goal to have none, one, or both of
    these types.

\end{itemize}
\hrule

\paragraph{Sources} \citet[p.~361]{broekens2021emotion},
\citet[p.~204]{izard1977human}

\paragraph{Depends On} \cref{C_Goals}, \tref{T_CalculateEmotionGP},
\tyref{TY_WorldState}, \tyref{TY_WorldStateChange},
\tyref{TY_DistanceBetweenWorldStates},
\tyref{TY_DistanceBetweenWorldStatesChange}

\paragraph{Ref. By} \iref{IM_ElicitJoy}, \iref{IM_ElicitSadness},
\iref{IM_ElicitFear}, \iref{IM_ElicitDisgust},
\iref{IM_CalculateEmotionAcceptanceElicit}, \iref{IM_JoyIntensity},
\iref{IM_SadnessIntensity}, \iref{IM_FearIntensity},
\iref{IM_DisgustIntensity}, \rref{R_GoalType} \\\hrule\vspace{0.5mm}\hrule

~\newline

\noindent
\begin{minipage}{\textwidth}
    \renewcommand*{\arraystretch}{1.5}
    \begin{tabular}{| p{\colAwidth}  p{\colBwidth}|}
        \hline
        \rowcolor[gray]{0.9}
        \bf TY\refstepcounter{typenum}\thetypenum
        \label{TY_Plan} & \bf Type of Plan \\
        \hline
    \end{tabular}

    \renewcommand*{\arraystretch}{1.5}
    \begin{tabular}{ p{\colAwidth}  p{\colBwidth}}
        \bf Symbol & $\plantype$ \\

        \bf Type & $\{ \mathtt{actions} : ({\worldstatechangetype}_1, ...,
        {\worldstatechangetype}_n), \mathtt{toProgress} : ((\worldstatetype
        \rightarrow \mathbb{B})_0, ..., (\worldstatetype \rightarrow
        \mathbb{B})_n),$ \newline
        $ \mathtt{nextStep} : \worldstatetype \times \mathbb{N}
        \rightarrow \worldstatetype, \mathtt{isFeasible} : \worldstatetype
        \rightarrow \mathbb{B} \} $ \newline
        $\text{where } n : \mathbb{N}_{>0}, $ \newline
        $\mathtt{nextStep}(s : \worldstatetype, i : \mathbb{N}) = \begin{cases}
            s, & i = 0 \\
            \mathtt{nextStep}(s, i) \oplus \mathtt{planActions}(i), &
            \mathit{Otherwise}
        \end{cases}, $ \newline
        $\text{and } \mathtt{isFeasible}(s : \worldstatetype) =
        \bigwedge_{i=0}^{n} \mathtt{toProgress}(i, \mathtt{nextStep}(s, i))$ \\
        \hline
    \end{tabular}
\end{minipage}

\paragraph{Description} A plan (\cref{C_Plans}) is represented by:
\begin{itemize}

    \item A sequence of $\mathtt{actions}$ as world events
    (\tyref{TY_WorldStateChange}) such that applying them to an initial WSV
    (\tyref{TY_WorldState}) generates a series of ``good'' WSVs that satisfy a
    sequence of predicates on them representing plan progression
    ($\mathtt{toProgress}$), where each element in $\mathtt{actions}$ is
    something the entity can do (i.e. there are no elements in
    $\mathtt{actions}$ that the entity believes are impossible)

    \item At some step $i : \mathbb{N}$, the function $\mathtt{nextStep}$
    evaluates the next WSV in the plan by applying the $ith$ plan action to the
    $ith$ $\mathtt{nextStep}$ where $\mathtt{nextStep}(s, 0)$ is the initial
    state $s : \worldstatetype$

    \item A constant $\mathtt{isFeasible}$ generated by checking that, for each
    step $i : \mathbb{N}$ starting from $i = 0$ and a WSV $s :
    \worldstatetype$, each evaluation of $\mathtt{nextStep}(s, i)$ satisfies
    the $ith$ condition in $\mathtt{toProgress}$

\end{itemize}

The Plan Data Type is not explicitly connected to the Goal type so that users
can apply it to entity plans that are not oriented towards an
\progname{}-specific goal $\goaltype$ (\tyref{TY_Goal}). \\\hrule

\paragraph{Sources} --

\paragraph{Depends On} \cref{C_Plans}, \tyref{TY_WorldState},
\tyref{TY_WorldStateChange}

\paragraph{Ref. By} \iref{IM_ElicitSadness}, \iref{IM_ElicitAnger},
\iref{IM_SadnessIntensity}, \iref{IM_AngerIntensity}, \rref{R_PlanType}
\\\hrule\vspace{0.5mm}\hrule

~\newline

\noindent
\begin{minipage}{\textwidth}
    \renewcommand*{\arraystretch}{1.5}
    \begin{tabular}{| p{\colAwidth}  p{\colBwidth}|}
        \hline
        \rowcolor[gray]{0.9}
        \bf TY\refstepcounter{typenum}\thetypenum
        \label{TY_Relation-CTE} & \bf Type of Social Attachment \\
        \hline
    \end{tabular}

    \renewcommand*{\arraystretch}{1.5}
    \begin{tabular}{ p{\colAwidth}  p{\colBwidth}}
        \bf Symbol & $\socialattachmenttype$ \\

        \bf Type & $\mathbb{Z}$ \\\hline
    \end{tabular}
\end{minipage}

\paragraph{Description} A social attachment is an extensible type containing a
``degree'' or ``level'' of attachment to some other entity $A$. A social
attachment is linearly ordered such that higher ``degrees'' or ``levels''
represents a closer attachment to $A$ which can reflect a ``history'' with $A$
(\cref{C_Relation-CTE}).

Since Social Attachment is just an association between two entities, what users
take as the ``other'' entity need not be limited to other characters---it could
refer to objects, actions, or other game elements. \\\hrule

\paragraph{Sources} \citet[p.~359--360]{broekens2021emotion}

\paragraph{Depends On} \cref{C_Relation-CTE}

\paragraph{Ref. By} \iref{IM_CalculateEmotionAcceptanceElicit},
\iref{IM_CalculateEmotionAcceptance}, \rref{R_SocialAttachment}
\\\hrule\vspace{0.5mm}\hrule

~\newline

\noindent
\begin{minipage}{\textwidth}
    \renewcommand*{\arraystretch}{1.5}
    \begin{tabular}{| p{\colAwidth}  p{\colBwidth}|}
        \hline
        \rowcolor[gray]{0.9}
        \bf TY\refstepcounter{typenum}\thetypenum
        \label{TY_Attention} & \bf Type of Attention \\
        \hline
    \end{tabular}

    \renewcommand*{\arraystretch}{1.5}
    \begin{tabular}{ p{\colAwidth}  p{\colBwidth}}
        \bf Symbol & $\attentiontype$ \\

        \bf Type & $\deltatimetype$ \\\hline
    \end{tabular}
\end{minipage}

\paragraph{Description} Attention (\cref{C_Attention}) is an extensible type
containing the number of consecutive time steps $t : \deltatimetype$
(\tyref{TY_Time}) that have elapsed while focusing on some $x$. \\\hrule

\paragraph{Sources} --

\paragraph{Depends On} \cref{C_Attention}, \tyref{TY_Time}

\paragraph{Ref. By} \iref{IM_CalculateEmotionInterestElicit},
\iref{IM_CalculateEmotionInterest}, \rref{R_Attention}
\\\hrule\vspace{0.5mm}\hrule