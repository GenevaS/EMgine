\subsubsection{Theoretical Models}\label{sec_theoretical}

These models refine the Conceptual Models (Section~\ref{sec_conceptual}) using
natural language to improve their precision. This is a necessary step for
reducing ambiguity by explicitly stating how the primary sources are understood
and showing how they relate to the mathematically-defined Data Types
(Section~\ref{sec_typedefs}) and Instance Models (Section~\ref{sec_instance})
using Assumptions (Section~\ref{sec_assumptions}).

~\newline\noindent
\begin{minipage}{\textwidth}
    \renewcommand*{\arraystretch}{1.5}
    \begin{tabular}{| p{\colAwidth}  p{\colBwidth}|}
        \hline
        \rowcolor[gray]{0.9}
        \bf T\refstepcounter{theorynum}\thetheorynum
        \label{T_CalculateEmotionGP} &
        \bf Evaluate Emotion Kind from Goals and Plans \\
        \hline
    \end{tabular}
\end{minipage}

\paragraph{Description} The emotion kinds (``modes'') that CTE defines
(\cref{C_Appraisal-CTE})---\textit{Happiness}, \textit{Sadness}, \textit{Fear},
\textit{Anger}, \textit{Disgust}---in terms of goals (\cref{C_Goals}) and plans
(\cref{C_Plans}) can be re-conceptualized as follows:
\begin{itemize}
    \item \textit{Joy} (\textit{Happiness}) occurs when an action moves an
    entity closer to achieving a goal, assuming that each such action is a goal
    itself (i.e. a ``sub-goal'' of the goal, \aref{A_Subgoal})

    \item \textit{Sadness} occurs when an entity has a plan where a next step
    is impossible to achieve (i.e. ``failure of a plan''), or a goal becomes
    impossible to achieve (i.e. ``loss of a goal'')

    \item \textit{Fear} (\textit{Anxiety}) occurs when there is a threat to
    self-preservation (i.e. ``self-preservation goal threatened''), which
    requires a prediction about a future world state based on the current world
    state and an action that could change it

    \item \textit{Anger} occurs when an entity has a plan where the next step
    cannot be reached after an intentional action to achieve it, but there are
    one or more other actions that can (i.e. the plan is ``frustrated'' because
    it is still feasible, but it had to change to remain so)

    \item \textit{Disgust} occurs when an entity has been ``contaminated'' or
    encounters ``contaminated'' substances that it wants to avoid (i.e.
    ``gustatory goal violated'')
\end{itemize}

It is assumed that a single goal can trigger more than one emotion
(\aref{A_Goal2Emotion}) and they all contribute to a single emotion state
(\aref{A_OneState}).

Emotion intensity is evaluated separately (\aref{A_EmotionTypeIntensity}, see
\tref{T_CalculateEmotionIntensity}). The emotions \textit{Surprise},
\textit{Interest}, and \textit{Acceptance} are evaluated differently because
they are not part of CTE's primary emotions (\aref{A_AppraisalProcess}, see
\tref{T_CalculateEmotionSurprise}, \tref{T_CalculateEmotionInterest}, and
\tref{T_CalculateEmotionAcceptance} respectively). \\\hrule

\paragraph{Sources} --

\paragraph{Depends On} \aref{A_Subgoal}, \aref{A_Goal2Emotion},
\aref{A_OneState}, \aref{A_EmotionTypeIntensity},
\aref{A_AppraisalProcess}, \cref{C_Appraisal-CTE}, \cref{C_Goals},
\cref{C_Plans}

\paragraph{Ref. By} \iref{IM_CalculateEmotionGP} \\\hrule\vspace{0.5mm}\hrule

~\newpage\noindent
\begin{minipage}{\textwidth}
    \renewcommand*{\arraystretch}{1.5}
    \begin{tabular}{| p{\colAwidth}  p{\colBwidth}|}
        \hline
        \rowcolor[gray]{0.9}
        \bf T\refstepcounter{theorynum}\thetheorynum
        \label{T_CalculateEmotionIntensity} &
        \bf Evaluate Emotion Intensity \\
        \hline
    \end{tabular}
\end{minipage}

\paragraph{Description} Emotion intensity (\cref{C_EmIntensity-CTE}) is
proportional to the force causing an emotion state (``entrained'') and how
fixed or non-adjustable that force is (``locked in''). This could be related to
the degree that something impacts a goal (\aref{A_Goal2Intensity})---either a
stand-alone goal or as part of a plan (i.e. a ``sub-goal'')---such that an
entity would want to maintain the momentum caused by an emotion ``mode'' as
long as that goal is affected. \\\hrule

\paragraph{Sources} --

\paragraph{Depends On} \aref{A_Goal2Intensity}, \cref{C_EmIntensity-CTE}

\paragraph{Ref. By} \tyref{TY_EmotionIntensity},
\iref{IM_CalculateEmotionIntensity} \\\hrule\vspace{0.5mm}\hrule

~\newline

\noindent
\begin{minipage}{\textwidth}
    \renewcommand*{\arraystretch}{1.5}
    \begin{tabular}{| p{\colAwidth}  p{\colBwidth}|}
        \hline
        \rowcolor[gray]{0.9}
        \bf T\refstepcounter{theorynum}\thetheorynum
        \label{T_CalculateEmotionSurprise} &
        \bf Evaluate \textit{Surprise} Emotion and Intensity \\
        \hline
    \end{tabular}
\end{minipage}

\paragraph{Description} An ``interruption and abrupt transition'' to a
different emotion state elicits \textit{Surprise} (\cref{C_EmOther}). This
suggests that it occurs when there is a large, rapid change in emotion
intensity (\aref{A_Surprise}), and/or there the emotion state recently changed
and it is changing again (``sudden, unexpected event'', \aref{A_Surprise2}).

For intensity, this implies that it is inversely proportional to time such that
less time it takes for the emotion intensity change to occur and/or between the
recently changed emotion state and the current change to a ``new'' emotion
state creates a higher intensity response. Intensity rapidly decreases as time
elapses. \\\hrule

\paragraph{Sources} --

\paragraph{Depends On} \aref{A_Surprise}, \aref{A_Surprise2}, \cref{C_EmOther}

\paragraph{Ref. By} \tyref{TY_EmotionIntensity},
\iref{IM_CalculateEmotionSurpriseElicit}, \iref{IM_CalculateEmotionSurprise}
\\\hrule\vspace{0.5mm}\hrule

~\newline

\noindent
\begin{minipage}{\textwidth}
    \renewcommand*{\arraystretch}{1.5}
    \begin{tabular}{| p{\colAwidth}  p{\colBwidth}|}
        \hline
        \rowcolor[gray]{0.9}
        \bf T\refstepcounter{theorynum}\thetheorynum
        \label{T_CalculateEmotionInterest} &
        \bf Evaluate \textit{Interest} Emotion and Intensity \\
        \hline
    \end{tabular}
\end{minipage}

\paragraph{Description} ``Sustained attention to certain external events''
elicits \textit{Interest} (\cref{C_EmOther}), implying that it occurs when an
entity is focused on the same thing (e.g. task, another entity) for an extended
period of time (\aref{A_Interest}). This suggests that there is a ``baseline''
amount of attention/time paid to something before it triggers
\textit{Interest}, which can differ between foci.

For intensity, this implies that it is proportional to the amount of time spent
focused on the same thing and relative to the ``baseline'' time spent. \\\hrule

\paragraph{Sources} --

\paragraph{Depends On} \aref{A_Interest}, \cref{C_EmOther}

\paragraph{Ref. By} \tyref{TY_EmotionIntensity},
\iref{IM_CalculateEmotionInterestElicit}, \iref{IM_CalculateEmotionInterest}
\\\hrule\vspace{0.5mm}\hrule

~\newline

\noindent
\begin{minipage}{\textwidth}
    \renewcommand*{\arraystretch}{1.5}
    \begin{tabular}{| p{\colAwidth}  p{\colBwidth}|}
        \hline
        \rowcolor[gray]{0.9}
        \bf T\refstepcounter{theorynum}\thetheorynum
        \label{T_CalculateEmotionAcceptance} &
        \bf Evaluate \textit{Acceptance} Emotion and Intensity \\
        \hline
    \end{tabular}
\end{minipage}

\paragraph{Description} Taking \textit{Acceptance} as a ``complex'' emotion
(\cref{C_ComplexEmotions-CTE}), it can be defined as \textit{Joy}
(\textit{Happiness}) elaborated with information about social relationships
(\cref{C_Relation-CTE}) such that an entity experiences \textit{Acceptance}
when it attributes a state of \textit{Joy} to another entity that they have an
established relationship with (\aref{A_Acceptance}). An entity might also
establish a relationship with another entity that is attributed with causing
the state, which would also elicit \textit{Acceptance}.

For intensity, this implies that it is proportional to the intensity of
\textit{Joy} and the strength of the relationship with the entity that the
state is attributed to. \\\hrule

\paragraph{Sources} \citet[p.~178--179]{oatley1992best}

\paragraph{Depends On} \aref{A_Acceptance}, \cref{C_ComplexEmotions-CTE},
\cref{C_Relation-CTE}

\paragraph{Ref. By} \tyref{TY_EmotionIntensity},
\iref{IM_CalculateEmotionAcceptanceElicit},
\iref{IM_CalculateEmotionAcceptance} \\\hrule\vspace{0.5mm}\hrule

~\newline

\noindent
\begin{minipage}{\textwidth}
    \renewcommand*{\arraystretch}{1.5}
    \begin{tabular}{| p{\colAwidth}  p{\colBwidth}|}
        \hline
        \rowcolor[gray]{0.9}
        \bf T\refstepcounter{theorynum}\thetheorynum
        \label{T_DecayEmotionState} &
        \bf Decaying Emotion State \\
        \hline
    \end{tabular}
\end{minipage}

\paragraph{Description} Emotion decay (\cref{C_EmDecay}) is a function of time
such that the emotion state returns to its ``equilibrium'' intensities as time
progresses. It is assumed that:
\begin{itemize}
    \item The speed that intensities return to ``equilibrium'' are assumed to
    be functions of distance such that larger differences between an intensity
    and its ``equilibrium'' cause larger changes in intensity
    (\aref{A_DecaySpeed})

    \item Each emotion kind has its own ``equilibrium'' value
    (\aref{A_Equilibrium}) and can decay at different rates (\aref{A_DecayRate})

    \item Decay rates and equilibrium values can differ between entities
    (\aref{A_DecayUnique})
\end{itemize}
\hrule

\paragraph{Sources} --

\paragraph{Depends On} \aref{A_DecaySpeed}, \aref{A_Equilibrium},
\aref{A_DecayRate}, \aref{A_DecayUnique}, \cref{C_EmDecay}

\paragraph{Ref. By} \tyref{TY_EmotionDecay}, \tyref{TY_EmotionDecayState},
\iref{IM_DecayEmotionState} \\\hrule\vspace{0.5mm}\hrule

~\clearpage

\noindent
\begin{minipage}{\textwidth}
    \renewcommand*{\arraystretch}{1.5}
    \begin{tabular}{| p{\colAwidth}  p{\colBwidth}|}
        \hline
        \rowcolor[gray]{0.9}
        \bf T\refstepcounter{theorynum}\thetheorynum
        \label{T_GetEmotionStatePAD} &
        \bf Getting an Emotion State as a PAD Point \\
        \hline
    \end{tabular}
\end{minipage}

\paragraph{Description} Assuming that an entity can only occupy one point in
PAD Space at any given time (\aref{A_OnePADPoint}), the intensities for each
discrete emotion in an emotion state must be combined into a single value for
each dimension in the space. This requires reference points for each emotion
type in PES because it defines the emotion state structure
(\cref{C_EmotionStruct}) that emotion intensities are mapped to.

In both theories, laypeople evaluated emotion terms for their contents. In both
experiments, the laypeople were likely from the same age group (university
undergraduates, college and graduate students), likely from the United States
(in the case of PAD, Californian), and likely to have participated in the
experiments around the same time given the publishing year of the work (1980).
Therefore, it is reasonable to assume that the meaning of the terms in PES and
PAD Space would be judged to be identical or nearly identical to each other by
the same laypeople (\aref{A_EmotionTerms}).

The circumplex representing the relative similarity of emotion term
meanings~\citep[p.~170]{robert1980emotion} was divided into ranges,
representing what language laypeople typical describe PES emotion kinds with.
Gaps between areas are small (0.3\textdegree{} at least, 7.3\textdegree{} at
most), so they are ignored. The list of 145 emotion terms from PES and 151 from
PAD~\citep[p.~42--45]{mehrabian1980basic} were compared for common terms, which
were listed by PES emotion kind. An emotion term was selected from each list
based on its similarity to the emotion kind label. There was no such term for
\textit{Acceptance}, so \progname{} uses the term closest to the mean range on
the circumplex (i.e. ``Affectionate''). Where possible, emotion terms that had
statistical significance ($p < 0.01$) for all dimensions define reference
points (e.g. \textit{Astonished} is statistically significant on all three
dimensions, whereas \textit{Surprised} is only significant for two). All terms
achieved significance for \textit{pleasure} and \textit{arousal}.
\textit{Interested} and \textit{Disgust} are not statistically significant for
\textit{dominance}.

An emotion term's \textit{pleasure}, \textit{arousal}, and \textit{dominance}
mean values (\cref{C_PAD}) form a reference point. Number of ratings and
standard deviation do not affect reference points (\aref{A_PADStats}).

\begin{table}[H]
    \centering
    \renewcommand{\arraystretch}{1.2}
    \begin{tabular}{ccccc}
        \toprule
        \multicolumn{2}{c}{\textbf{PES}} &  &  & \\
        $k \in \emotionkindstype$ & Range on Circumplex (\textdegree) & Term  &
        Circumplex Location (\textdegree) & PAD Ref.~\# \\
        \midrule

        \rowcolor[gray]{0.9}\textit{Fear} & $[65.0, 86.0]$ & Terrified & 75.5 &
        102 \\

        \textit{Anger} & $[200.6, 249.0]$ & Angry & 212.0 & 82 \\

        \rowcolor[gray]{0.9}\textit{Sadness} & $[88.3, 138.0]$ & Sad & 108.5 &
        151 \\

        \textit{Joy} & $[323.4, 338.3]$ & Joyful & 323.4 & 20 \\

        \rowcolor[gray]{0.9}\textit{Interest} & $[249.7, 322.4]$ & Interested &
        315.7 & 8 \\

        \textit{Surprise} & $[138.3, 156.7]$ & Astonished & 148.0 & 74 \\

        \rowcolor[gray]{0.9}\textit{Disgust} & $[160.3, 193.7]$ & Disgusted &
        161.3 & 75 \\

        \textit{Acceptance} & $[340.7, 57.7]$ & Affectionate & 52.3 & 34 \\

        \bottomrule
    \end{tabular}
\end{table}

\hrule

\paragraph{Sources} \citet[p.~40, 42--45]{mehrabian1980basic},
\citet[p.~159, 170]{robert1980emotion}, \cite{conte1976circumplex}

\paragraph{Depends On} \aref{A_OnePADPoint}, \aref{A_EmotionTerms},
\aref{A_PADStats}, \cref{C_EmotionStruct}, \cref{C_PAD}

\paragraph{Ref. By} \iref{IM_GetEmotionStatePAD} \\\hrule\vspace{0.5mm}\hrule