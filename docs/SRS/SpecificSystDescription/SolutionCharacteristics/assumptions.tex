\subsubsection{Assumptions}\label{sec_assumptions}
This section helps formalize the original problem to aid the development of
types, models, and definitions by filling in the ambiguous and missing
information. The numbers given in the square brackets refer to where the
assumption appears: Theoretical Model [T], General Definition [GD], Data
Definition [DD], Data Type [TY], Instance Model [IM], or Likely Change [LC]. If
there are no square brackets next to an assumption, it is generally applicable.

The Conceptual Models (Section~\ref{sec_conceptual}) have no related
assumptions because they are taken from primary sources, with no attempts at
elaboration or disambiguation.

\begin{itemize}

    \item[A\refstepcounter{assumpnum}\theassumpnum \label{A_TotalFunctions}:]
    All functions are \textit{total} unless otherwise stated.

    \item[A\refstepcounter{assumpnum}\theassumpnum \label{A_formal}:]
    Words describing concepts in conceptual models do not necessarily reflect
    how they are represented in a formal context.

    \item[A\refstepcounter{assumpnum}\theassumpnum \label{A_Cognition}:]
    \textit{Cognition} refers to the representation and transformation of
    knowledge, which might not be concious~\citep[p.~30]{oatley1987towards}.

    \item[A\refstepcounter{assumpnum}\theassumpnum \label{A_Modular}:]
    The human cognitive system is modular and
    asynchronous~\citep[p.~31]{oatley1987towards}.

    \item[A\refstepcounter{assumpnum}\theassumpnum
    \label{A_Subgoal}:] Each action and/or event that moves an entity towards
    its goal is a ``sub-goal'' [\tref{T_CalculateEmotionGP},
    \lcref{LC_Subgoal}].

    \item[A\refstepcounter{assumpnum}\theassumpnum
    \label{A_Goal2Emotion}:] A goal can elicit multiple emotions simultaneously
    [\tref{T_CalculateEmotionGP}, \lcref{LC_Goal2Emotion}].

    \item[A\refstepcounter{assumpnum}\theassumpnum
    \label{A_OneState}:] An entity is in exactly one emotion state at any given
    time [\tref{T_CalculateEmotionGP}].

    \item[A\refstepcounter{assumpnum}\theassumpnum
    \label{A_EmotionTypeIntensity}:] Emotion type and intensity do not have to
    be calculated together [\tref{T_CalculateEmotionGP},
    \lcref{LC_EmotionTypeIntensity}].

    \item[A\refstepcounter{assumpnum}\theassumpnum \label{A_AppraisalProcess}:]
    Different appraisal processes can elicit different emotions
    [\tref{T_CalculateEmotionGP}].

    \item[A\refstepcounter{assumpnum}\theassumpnum \label{A_Goal2Intensity}:]
    For CTE-based emotion kinds, emotion intensity is related to the degree
    that something impacts a goal and/or plan
    [\tref{T_CalculateEmotionIntensity}, \lcref{LC_Goal2Intensity}].

    \item[A\refstepcounter{assumpnum}\theassumpnum \label{A_CTE2PES}:]
    Emotions with the same or synonymous names in CTE and PES represent the
    same concept [\tref{T_CalculateEmotionGP}].

    \item[A\refstepcounter{assumpnum}\theassumpnum \label{A_Acceptance}:]
    \textit{Acceptance} is a ``complex'' emotion based on \textit{Joy}
    [\tref{T_CalculateEmotionAcceptance}].

    \item[A\refstepcounter{assumpnum}\theassumpnum \label{A_Interest}:]
    \textit{Interest} is related to the amount of time an entity spends
    focusing on the same thing [\tref{T_CalculateEmotionInterest},
    \lcref{LC_Interest}].

    \item[A\refstepcounter{assumpnum}\theassumpnum \label{A_Surprise}:]
    \textit{Surprise} is related to event probability
    [\tref{T_CalculateEmotionSurprise}].

     \item[A\refstepcounter{assumpnum}\theassumpnum \label{A_DecaySpeed}:]
     Emotion decays more quickly the farther it is from the ``equilibrium''
     state [\tref{T_DecayEmotionState}, \lcref{LC_DecaySpeed}].

     \item[A\refstepcounter{assumpnum}\theassumpnum \label{A_Equilibrium}:]
     Each emotion kind can have its own ``equilibrium'' value
     [\tref{T_DecayEmotionState}, \lcref{LC_Equilibrium}].

     \item[A\refstepcounter{assumpnum}\theassumpnum \label{A_DecayRate}:]
     Each emotion kind in a state can decay at its own rate
     [\tref{T_DecayEmotionState}, \lcref{LC_DecayRate}].

    \item[A\refstepcounter{assumpnum}\theassumpnum \label{A_DecayUnique}:]
    Decay rates and equilibrium values can vary between entities
    [\tref{T_DecayEmotionState}].

    \item[A\refstepcounter{assumpnum}\theassumpnum \label{A_OnePADPoint}:]
    An emotion state represents a single point in PAD Space
    [\tref{T_GetEmotionStatePAD}].

    \item[A\refstepcounter{assumpnum}\theassumpnum \label{A_EmotionTerms}:]
    Laypeople that judged emotion terms in one of PES and PAD would find that
    terms used in the other theory would have identical or nearly identical
    meanings [\tref{T_GetEmotionStatePAD}, \lcref{LC_EmotionTerms}].

    \textbf{Reasoning} This is derived from published, independently run
    empirical studies by Plutchik (PES) and Mehrabian (PAD) where they were
    evaluating their own affective models. For this study on emotion language,
    Plutchik created a list of 145 emotion terms and asked participants judged
    how ``similar'' the terms are~\citep[p.~159, 168--170]{robert1980emotion}.
    Terms were then assigned angular placements on a circumplex based on their
    relative ``similarity'' to each other. Mehrabian asked participants to
    judge the contribution of each dimension---\textit{pleasure},
    \textit{arousal}, and \textit{dominance}---in the experiences described by
    151 terms from which it derived statistical mean and standard deviation
    values~\citep[p.~39--45]{mehrabian1980basic}. Reports about the
    participants in these studies suggest that they were:
    \begin{itemize}

        \item Likely in the same age group (university undergraduates, college
        and graduate students), and

        \item Likely had an North American cultural perspective (studies done
        in the United States).

    \end{itemize}

    The publication dates (1980) further suggest that Plutchik and Mehrabian
    likely conducted their studies around the same time frame. Taken together,
    this implies that the laypeople in these studies shared common temporal and
    cultural experiences that would have influenced their interpretation of
    natural language terms.

    \item[A\refstepcounter{assumpnum}\theassumpnum \label{A_PADStats}:] The
    number of ratings and standard deviation do not affect an emotion term's
    mean values for \textit{pleasure}, \textit{arousal}, and \textit{dominance}
    [\tref{T_GetEmotionStatePAD}, \lcref{LC_PADStats}].

    \item[A\refstepcounter{assumpnum}\theassumpnum
    \label{A_ContinuousIntensity}:] Emotion intensities can be continuous
    values [\tyref{TY_EmotionIntensity}].

    \item[A\refstepcounter{assumpnum}\theassumpnum
    \label{A_RealIntensityChanges}:] Emotion intensities changes can be
    positive or negative [\tyref{TY_DeltaIntensity}].

    \item[A\refstepcounter{assumpnum}\theassumpnum \label{A_PositiveIntensity}:]
    In PES, all emotion state intensities are zero (0) in the state of ``deep
    sleep'' because ``deep sleep'' implies that the entity lost
    consciousness~\citep[p.~1--2]{mondino2021definitions} and is not
    experiencing emotion at all. [\tyref{TY_EmotionIntensity},
    \tyref{TY_EmotionState}, \tyref{TY_EmotionDecayState},
    \lcref{LC_PositiveIntensity}].

    \item[A\refstepcounter{assumpnum}\theassumpnum \label{A_EmotionPairs}:] In
    PES, the state of ``deep sleep'' implies that emotion type pairs are not
    coupled [\tyref{TY_EmotionKind}, \tyref{TY_EmotionState},
    \lcref{LC_EmotionPairs}].

    \item[A\refstepcounter{assumpnum}\theassumpnum \label{A_LimitIntensity}:]
    Emotion intensities are finite [\tyref{TY_EmotionState}].

    \item[A\refstepcounter{assumpnum}\theassumpnum \label{A_Events}:]
    The environment can be represented by variables and events are changes in
    those variables [\tyref{TY_WorldState}, \tyref{TY_WorldStateChange}].

    \item[A\refstepcounter{assumpnum}\theassumpnum \label{A_CostFunction}:]
    There is an external function $\mathtt{Cost} : \plantype \rightarrow
    \mathbb{R}$ that evaluates the ``cost'' of the plan such that low costs are
    desirable [\iref{IM_ElicitAnger}, \iref{IM_AngerIntensity}].

    \item[A\refstepcounter{assumpnum}\theassumpnum \label{A_CausedByFunction}:]
    There is an external function $\mathtt{CausedBy} : \worldstatechangetype
    \times A \rightarrow \mathbb{B}$ evaluates event causality, returning
    $\True$ if the entity believes that $A$ is responsible for causing an event
    [\iref{IM_CalculateEmotionAcceptanceElicit}].

    \item[A\refstepcounter{assumpnum}\theassumpnum
    \label{A_EventProbabilityFunction}:]
    There is an external function of that evaluates an event's probability,
    which \textit{depends on} the previous WSV because what is ``expected'' and
    ``unexpected'' depends on defined preconditions (e.g. water falling on
    someone is unexpected on a sunny day bt not on a rainy one). This model
    assumes that there are exactly two outcomes for any given event---either it
    happens or it does not. This is for
    simplicity~\citep[p.~56]{reisenzein2019cognitive} and an assumption that
    users will want more control over entity reactions in complex scenarios.
    [\iref{IM_CalculateEmotionSurpriseElicit}].

    \item[A\refstepcounter{assumpnum}\theassumpnum \label{A_DistFunction}:]
    There is an external function $\mathtt{Dist} : \worldstatetype \times
    \worldstatetype \rightarrow \statedistancetype$ that evaluates the distance
    between two world states [\iref{IM_SadnessIntensity}].

    \item[A\refstepcounter{assumpnum}\theassumpnum
    \label{A_MaxGoalImportanceFunction}:] There is an external user-defined
    value $m_G$ representing the maximum value that goal importance can be
    [\iref{IM_SadnessIntensity}].

\end{itemize}