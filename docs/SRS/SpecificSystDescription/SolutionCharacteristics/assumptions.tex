\subsubsection{Assumptions}\label{sec_assumptions}
This section helps formalize the original problem to aid the development of 
types, models, and definitions by filling in the ambiguous and missing 
information. The numbers given in the square brackets refer to where the 
assumption appears: Theoretical Model [T], General Definition [GD], Data 
Definition [DD], Data Type [TY], Instance Model [IM], or Likely Change [LC]. If 
there are no square brackets next to an assumption, it is generally applicable.

The Conceptual Models (Section~\ref{sec_conceptual}) have no related 
assumptions because they are taken from primary sources, with no attempts at 
elaboration or disambiguation. 

\begin{itemize}

    \item[A\refstepcounter{assumpnum}\theassumpnum \label{A_TotalFunctions}:]
    All functions are \textit{total} unless otherwise stated.

    \item[A\refstepcounter{assumpnum}\theassumpnum \label{A_formal}:]
    Words describing concepts in conceptual models do not necessarily reflect
    how they are represented in a formal context.

    \item[A\refstepcounter{assumpnum}\theassumpnum \label{A_Cognition}:]
    \textit{Cognition} refers to the representation and transformation of
    knowledge, which might not be concious~\citep[p.~30]{oatley1987towards}.

    \item[A\refstepcounter{assumpnum}\theassumpnum \label{A_Modular}:]
    The human cognitive system is modular and
    asynchronous~\citep[p.~31]{oatley1987towards}.

    \item[A\refstepcounter{assumpnum}\theassumpnum 
    \label{A_Subgoal}:] Each action and/or event that moves an entity towards 
    its goal is a ``sub-goal'' [\tref{T_CalculateEmotionGP}, 
    \lcref{LC_Subgoal}].

    \item[A\refstepcounter{assumpnum}\theassumpnum 
    \label{A_Goal2Emotion}:] A goal can elicit multiple emotions simultaneously 
    [\tref{T_CalculateEmotionGP}, \lcref{LC_Goal2Emotion}].

    \item[A\refstepcounter{assumpnum}\theassumpnum 
    \label{A_OneState}:] An entity is in exactly one emotion state at any given 
    time [\tref{T_CalculateEmotionGP}].

    \item[A\refstepcounter{assumpnum}\theassumpnum 
    \label{A_EmotionTypeIntensity}:] Emotion type and intensity do not have to 
    be calculated together [\tref{T_CalculateEmotionGP}, 
    \lcref{LC_EmotionTypeIntensity}].

    \item[A\refstepcounter{assumpnum}\theassumpnum \label{A_AppraisalProcess}:]
    Not all emotions come from the same appraisal process 
    [\tref{T_CalculateEmotionGP}].

    \item[A\refstepcounter{assumpnum}\theassumpnum \label{A_Goal2Intensity}:]
    Emotion intensity is related to the degree that something impacts a goal 
    [\tref{T_CalculateEmotionIntensity}, \lcref{LC_Goal2Intensity}].

    \item[A\refstepcounter{assumpnum}\theassumpnum \label{A_Surprise}:]
    \textit{Surprise} is related to large, rapid changes in an emotion state 
    [\tref{T_CalculateEmotionSurprise}, \lcref{LC_Surprise}].

    \item[A\refstepcounter{assumpnum}\theassumpnum \label{A_Surprise2}:]
    \textit{Surprise} is related to changes in an emotion state that occur soon 
    after a previous change in emotion state 
    [\tref{T_CalculateEmotionSurprise}, \lcref{LC_Surprise}].

     \item[A\refstepcounter{assumpnum}\theassumpnum \label{A_Interest}:] 
     \textit{Interest} is related to the amount of time an entity spends 
     focusing on the same thing [\tref{T_CalculateEmotionInterest}, 
     \lcref{LC_Interest}].

     \item[A\refstepcounter{assumpnum}\theassumpnum \label{A_Acceptance}:] 
     \textit{Acceptance} is a ``complex'' emotion based on \textit{Joy}
     [\tref{T_CalculateEmotionAcceptance}].

     \item[A\refstepcounter{assumpnum}\theassumpnum \label{A_DecaySpeed}:] 
     Emotion decays more quickly the farther it is from the ``equilibrium'' 
     state [\tref{T_DecayEmotionState}, \lcref{LC_DecaySpeed}].

     \item[A\refstepcounter{assumpnum}\theassumpnum \label{A_Equilibrium}:] 
     Each emotion kind can have its own ``equilibrium'' value 
     [\tref{T_DecayEmotionState}, \lcref{LC_Equilibrium}].

     \item[A\refstepcounter{assumpnum}\theassumpnum \label{A_DecayRate}:] 
     Each emotion kind in a state can decay at its own rate 
     [\tref{T_DecayEmotionState}, \lcref{LC_DecayRate}].

    \item[A\refstepcounter{assumpnum}\theassumpnum \label{A_DecayUnique}:] 
    Decay rates and equilibrium values can vary between entities 
    [\tref{T_DecayEmotionState}].

    \item[A\refstepcounter{assumpnum}\theassumpnum \label{A_OnePADPoint}:] 
    An emotion state represents a single point in PAD Space 
    [\tref{T_GetEmotionStatePAD}].

    \item[A\refstepcounter{assumpnum}\theassumpnum \label{A_EmotionTerms}:] 
    Laypeople that judged emotion terms in one of PES and PAD would find that 
    terms used in the other theory would have comparable meanings 
    [\tref{T_GetEmotionStatePAD}, \lcref{LC_EmotionTerms}].

    \item[A\refstepcounter{assumpnum}\theassumpnum \label{A_PADStats}:] The 
    number of ratings and standard deviation do not affect an emotion term's 
    mean values for \textit{pleasure}, \textit{arousal}, and \textit{dominance}
    [\tref{T_GetEmotionStatePAD}, \lcref{LC_PADStats}].

    \item[A\refstepcounter{assumpnum}\theassumpnum \label{A_LimitIntensity}:]
    Emotion intensities are finite [\tyref{TY_EmotionState}].

    \item[A\refstepcounter{assumpnum}\theassumpnum \label{A_PositiveIntensity}:]
    In PES, the state of ``deep sleep'' implies that all emotion state
    intensities are zero (0) [\tyref{TY_EmotionIntensity}, 
    \tyref{TY_EmotionDecayState}, \lcref{LC_PositiveIntensity}].

    \item[A\refstepcounter{assumpnum}\theassumpnum \label{A_EmotionPairs}:] In 
    PES, the state of ``deep sleep'' implies that emotion type pairs are not 
    coupled [\tyref{TY_EmotionState}, \lcref{LC_EmotionPairs}].

    \item[A\refstepcounter{assumpnum}\theassumpnum \label{A_Events}:]
    The environment can be represented by variables and events are changes in
    those variables  [\tyref{TY_Goal}].

    \item[A\refstepcounter{assumpnum}\theassumpnum \label{A_GustatoryGoal}:]
    Gustatory and self-preservation goals are satisfied by default 
    [\iref{IM_CalculateEmotionGP}].

    \item[A\refstepcounter{assumpnum}\theassumpnum 
    \label{A_UpdateEmotionState}:] Emotion state updates are linear 
    [\iref{IM_UpdateEmotionState}, \lcref{LC_UpdateEmotionState}].

\end{itemize}