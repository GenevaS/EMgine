\subsubsection{Emotion Theories and Models}\label{sec_theories}
All potential \progname{} solutions must be based on emotion theories and data
to improve its \textit{psychologically validity}, which is necessary for its
behaviours to be \textit{plausible}~\citep[p.~216--217]{broekens2016emotional}.
The theories that \progname{} uses must align with its overall design goals.

\progname{}'s theories are Oatley \& Johnson-Laird's Communicative Theory of
Emotion (CTE), Plutchik's Psycho-Evolutionary Synthesis (PES), and Mehrabian's
Pleasure-Arousal-Dominance Space (PAD). These were chosen from a pool of
potential candidates (Table~\ref{tab:theories}) based on their ability to
satisfy the \textit{flexibility} and \textit{ease-of-use} high-level
requirements (Section~\ref{sec_systemConstraints}). Note that these
requirements are theory-agnostic, so any emotion theory that supports them is a
reasonable choice.

The selection process follows these general steps:
\begin{enumerate}
    \item Scoping the high-level requirements as needed
    \item Scoring each candidate theory
    \item Choosing theories based on their scores
\end{enumerate}

\paragraph{Scoping}
The analysis separates \textit{Providing a clear and understandable API}
(\ref{easeAPI}) into two---Input and Output---to get a better feel for each
theory's usefulness. This is an acknowledgement that some theories are better
for emotion expression, such as Ekman \& Friesen.

The requirement for \textit{Allowing the Integration of New Components}
(\ref{flexNew}) is broad and should be scoped for the initial design of
\progname{}. New \progname{} components could be non-affective (e.g. attention)
and affective (e.g. personality) in nature. Integrating non-affective
components should be theory-agnostic, as they are part of separate components
of mind~\citep{cognitiondef}. From a software engineering perspective, one
could view the mind as a system with distinct, interacting subsystems. Modular
interfaces that control interactions between \progname{}---as the affective
subsystem---and components from other subsystems would support this concept,
while also maintaining \progname{}'s requirements for \textit{Independence from
an Agent Architecture} (\ref{flexArch}) and \textit{Allowing Developers to
Specify How to Use Outputs} (\ref{flexOut}). Integrating other affective
components would depend on \progname{} and its foundational theories. One would
be limited to only those components that can be represented in terms of the
emotions or other types of affect that have been connected to them.

This requirements analysis focuses on three other types of affect: core affect,
mood, and personality. Of these, it prioritizes personality because it is
necessary for creating the consistent and coherent agent behaviours that
influence believability (\citepg{reilly1996believable}{26};
\citepg{loyall1997believable}{19}; \citepg{ortony2002making}{203}).

Finally, the \textit{Minimizing authorial burden} requirement
(\ref{easeAuthor}) is excluded from the analysis. This is due to its focus on
helping game developers manage the creation of an increasing NPC population,
rather than the functionality of \progname{} itself.

\begin{table}[!tbh]
    \renewcommand{\arraystretch}{1.2}
    \centering
    \caption{Theories Analysed}
    \label{tab:theories}
    \begin{tabular}{P{0.2\columnwidth}P{0.65\columnwidth}}
        \toprule
        \textbf{Perspective} & \textbf{Theories} \\
        \midrule

        \rowcolor[gray]{0.9}Discrete & Ekman \& Friesen (Ek.), Izard (Iz.),
        Plutchik (Plu.) \\

        Dimensional & Valence-Arousal (V-A), PAD Space (PAD) \\

        \rowcolor[gray]{0.9}Appraisal & Frijda (Frj.), Lazarus (Laz.), Scherer
        (Sch.), Roseman (Ros.), Ortony, Clore, and Collins (OCC), Smith \&
        Kirby (S \& K), Oatley \& Johnson-Laird (O \& JL) \\

        \bottomrule
    \end{tabular}
\end{table}

\paragraph{Scoring} Each theory has a set of notes made during their
examination, guided by individual high-level requirements. After reviewing the
notes, each theory is assigned a \textit{score} describing its relative
``suitability'' for that requirement (Table~\ref{tab:scoring}). This step is
somewhat subjective because a judgement is made without true objective measures
or methods. The notes are in Appendix~\ref{chapter:reqsTheoryNotes}.

\progname{}'s high-level requirements are divided into two sets:
\textit{system-level}, which applies to \progname{} as a whole
(Tables~\ref{tab:theory-req-sys-summary-flexibility} and
\ref{tab:theory-req-sys-summary-easeofuse}), and \textit{component-level} for
requirements that only apply to specific pieces of \progname{}
(Tables~\ref{tab:theory-req-comp-summary-flexibility} and
\ref{tab:theory-req-comp-summary-easeofuse}). Only appraisal theories are
assigned categories for component-level requirements because they concern
process-related elements that discrete and dimensional theories do not address.
Each theory has unique elements, which might make it better or worse suited for
a particular requirement. These are noted in the analysis. However, is not
unusual for theories from the same perspective to satisfy a requirement equally
well. In these cases, the theories are treated as a collective unit when
examining the requirement.
Tables~\ref{tab:theory-req-sys-summary-flexibility},
\ref{tab:theory-req-sys-summary-easeofuse},
\ref{tab:theory-req-comp-summary-flexibility}, and
\ref{tab:theory-req-comp-summary-easeofuse} summarize the scores for each
theory and requirement.

\begin{table}[!h]
    \renewcommand{\arraystretch}{1.2}
    \centering
    \caption{Summary of Scoring Categories}
    \label{tab:scoring}
    \begin{tabular}{ccP{0.65\columnwidth}}
        \toprule
        \textbf{Score Category} & \textbf{Symbol} & \textbf{Definition} \\

        \midrule

        \rowcolor[gray]{0.9}\textit{Strong} & \strong & The theory
        appears to satisfy the requirement in a clear, understandable way and
        is likely to aid in \progname{}'s usability
        \\

        \textit{Good} & \good & The theory appears to satisfy the requirement
        and is somewhat defined \\

        \rowcolor[gray]{0.9}\textit{Weak} & \weak & The theory
        describes ways that \textit{could} satisfy the requirement, but it is
        not fully defined or could make \progname{} harder to use \\

        \textit{Disqualified} & \disqualified & The theory does not seem likely
        to be able to satisfy this requirement, or it violates other
        requirements when it can (including psychological validity) \\

        \bottomrule

    \end{tabular}
\end{table}

\begin{table}[!tbh]
    \renewcommand{\arraystretch}{1.2}
    \centering
    \caption[Support for System-Level Flexibility High-Level
    Requirements]{Support for System-Level \textbf{Flexibility} High-Level
        Requirements}
    \label{tab:theory-req-sys-summary-flexibility}
    \footnotesize
    \begin{threeparttable}
        \begin{tabular}{@{}cP{0.13\linewidth}cccccccccccc@{}}

            \toprule
            \multicolumn{2}{c}{} & \hd{\textbf{Ek.}} &
            \hd{\textbf{Iz.}} & \hd{\textbf{Plu.}} & \hd{\textbf{V-A}}
            & \hd{\textbf{PAD}} & \hd{\textbf{Frj.}} &
            \hd{\textbf{Laz.}\textsuperscript{\footnotesize\textpmhg{\Hi}}} &
            \hd{\textbf{Sch.}} & \hd{\textbf{Ros.}} &
            \hd{\textbf{OCC}} & \hd{\textbf{S \& K}} & \hd{\textbf{O \&
                    JL}} \\
            \midrule

            \rowcolor[gray]{0.9}\ref{flexArch} & \textit{Independence
                from an Agent Architecture} & {\normalsize\good} &
            {\normalsize\good} & {\normalsize\good} & {\normalsize\good} &
            {\normalsize\good} & {\normalsize\strong} & {\normalsize\good} &
            \disqualified & {\normalsize\good} & {\normalsize\strong} &
            {\normalsize\strong} & {\normalsize\strong} \\

            \ref{flexNew} & \textit{Allowing the Integration of
                Components}{\large\textpmhg{\Hl}} & {\normalsize\weak} &
            {\normalsize\weak}\textsuperscript{\normalsize\Moon} &
            {\normalsize\good}\textsuperscript{\normalsize\Moon} &
            {\normalsize\strong}\textsuperscript{\large\Jupiter} &
            {\normalsize\strong}\textsuperscript{\large\Jupiter} &
            {\normalsize\strong} & {\normalsize\weak} & {\normalsize\weak} &
            {\normalsize\weak} & {\normalsize\strong} & {\normalsize\weak} &
            {\normalsize\strong} \\

            \rowcolor[gray]{0.9}\ref{flexEm} & \textit{Choosing NPC
                Emotions} & {\normalsize\weak} & \disqualified &
            {\normalsize\strong} & {\normalsize\weak} & {\normalsize\good} &
            {\normalsize\weak} & {\normalsize\weak} & {\normalsize\good} &
            {\normalsize\good} & {\normalsize\good} & {\normalsize\weak} &
            {\normalsize\strong} \\

            \ref{flexOut} & \textit{Allowing Developers to Specify How to Use
                CME Outputs} & {\normalsize\good} & {\normalsize\good} &
            {\normalsize\good} & {\normalsize\strong} & {\normalsize\strong} &
            {\normalsize\strong} & {\normalsize\strong} & {\normalsize\strong} &
            {\normalsize\strong} & {\normalsize\strong} & {\normalsize\strong}
            & {\normalsize\strong} \\

            \rowcolor[gray]{0.9}\ref{flexComplex} & \textit{Ability to
                Operate on Different Levels of NPC Complexity} &
                {\normalsize\good}
            & {\normalsize\good} & {\normalsize\good} & {\normalsize\weak} &
            {\normalsize\weak} & {\normalsize\weak} & {\normalsize\strong} &
            {\normalsize\strong} & {\normalsize\good} & {\normalsize\good} &
            {\normalsize\strong} & {\normalsize\good} \\

            \ref{flexScale} & \textit{Be Efficient and Scalable} &
            {\normalsize\good} & {\normalsize\good} & {\normalsize\good} &
            {\normalsize\weak} & {\normalsize\weak} &
            {\normalsize\good}\textsuperscript{\large\Pluto} &
            {\normalsize\strong} & {\normalsize\good} & {\normalsize\weak} &
            {\normalsize\good} & {\normalsize\strong} & {\normalsize\strong} \\

            \hline\bottomrule
        \end{tabular}
        \begin{tablenotes}

            \footnotesize
            \vspace*{2mm}

            \item \textit{See Table~\ref{tab:scoring} for score category
                descriptions.}

            \item {\small\textpmhg{\Hi}} \textit{Excludes the
                \textit{Coping Process} because it is part of action
                generation.}

            \item {\Large\textpmhg{\Hl}} \textit{Strictly focusing on
                core affect, mood, and personality.}

            \item {\normalsize\Moon} \textit{Natively supports integration
                with Personality.}

            \item {\Large\Jupiter} \textit{Personality integration
                based on the Five Factor Model (OCEAN).}

            \item {\Large\Pluto} \textit{Might improve if some factors
                are not necessary for implementation scope.}

        \end{tablenotes}
    \end{threeparttable}%
\end{table}

\begin{table}[!tbh]
    \renewcommand{\arraystretch}{1.2}
    \centering
    \caption[Support for System-Level Ease-of-Use High-Level
    Requirements]{Support for System-Level \textbf{Ease-of-Use}
        High-Level Requirements}
    \label{tab:theory-req-sys-summary-easeofuse}
    \footnotesize
    \begin{threeparttable}
        \begin{tabular}{@{}cP{0.21\linewidth}cccccccccccc@{}}

            \toprule
            \multicolumn{2}{c}{} & \hd{\textbf{Ek.}} &
            \hd{\textbf{Iz.}} & \hd{\textbf{Plu.}} &
            \hd{\textbf{V-A}} & \hd{\textbf{PAD}} &
            \hd{\textbf{Frj.}} &
            \hd{\textbf{Laz.}\textsuperscript{\footnotesize\textpmhg{\Hi}}} &
            \hd{\textbf{Sch.}} & \hd{\textbf{Ros.}} &
            \hd{\textbf{OCC}} & \hd{\textbf{S \& K}} &
            \hd{\textbf{O \& JL}} \\
            \midrule

            \rowcolor[gray]{0.9}\ref{easeAPI} & \textit{Having a Clear
                API (Output)} & {\normalsize\strong} & {\normalsize\weak} &
            {\normalsize\good} & {\normalsize\weak} & {\normalsize\weak} &
            {\normalsize\good} & {\normalsize\strong} & {\normalsize\weak} &
            {\normalsize\strong} & {\normalsize\good} & {\normalsize\strong} &
            {\normalsize\strong} \\

            \ref{easePX} & \textit{Showing that Emotions Improve the Player
                Experience} & {\normalsize\good} & {\normalsize\good} &
            {\normalsize\good} & {\normalsize\good} & {\normalsize\good} &
            {\normalsize\good} & {\normalsize\good} & {\normalsize\strong} &
            {\normalsize\good} & {\normalsize\weak} & {\normalsize\good} &
            {\normalsize\good} \\

            \rowcolor[gray]{0.9}\ref{easeNovel} & \textit{Providing
                Examples of Novel Game Experiences} & {\normalsize\weak} &
            {\normalsize\weak} & {\normalsize\weak} & {\normalsize\good} &
            {\normalsize\good} & {\normalsize\good} & {\normalsize\good} &
            {\normalsize\good} & {\normalsize\good} & {\normalsize\good} &
            {\normalsize\good} & {\normalsize\good} \\

            \hline\bottomrule
        \end{tabular}
        \begin{tablenotes}

            \footnotesize
            \vspace*{2mm}

            \item \textit{See Table~\ref{tab:scoring} for score category
                descriptions.}

            \item {\small\textpmhg{\Hi}} \textit{Excludes the
                \textit{Coping Process} because it is part of action
                generation.}

        \end{tablenotes}
    \end{threeparttable}%
\end{table}

\begin{table}[!tbh]
    \renewcommand{\arraystretch}{1.2}
    \centering
    \caption[Support for Component-Level Flexibility High-Level
    Requirements]{Support for Component-Level \textbf{Flexibility}
        High-Level Requirements}
    \label{tab:theory-req-comp-summary-flexibility}
    \footnotesize
    \begin{threeparttable}
        \begin{tabular}{@{}cP{0.25\linewidth}ccccccc@{}}

            \toprule
            \multicolumn{2}{c}{} & \hd{\textbf{Frj.}} &
            \hd{\textbf{Laz.}\textsuperscript{\footnotesize\textpmhg{\Hi}}} &
            \hd{\textbf{Sch.}} & \hd{\textbf{Ros.}} &
            \hd{\textbf{OCC}} & \hd{\textbf{S \& K}} & \hd{\textbf{O \&
                    JL}} \\
            \midrule

            \rowcolor[gray]{0.9}\ref{flexTasks} & \textit{Choosing
                Which Tasks to Use} & {\normalsize\weak} & {\normalsize\weak} &
            {\normalsize\strong} & \disqualified & {\normalsize\good} &
            {\normalsize\good} & {\normalsize\good} \\

            \ref{flexCustom} & \textit{Customization of Existing Task
                Parameters} & {\normalsize\strong} & \disqualified &
            {\normalsize\good} & \disqualified & {\normalsize\good} &
            {\normalsize\good} & {\normalsize\good} \\

            \hline\bottomrule
        \end{tabular}
        \begin{tablenotes}

            \footnotesize
            \vspace*{2mm}

            \item \textit{See Table~\ref{tab:scoring} for score category
                descriptions.}

            \item {\small\textpmhg{\Hi}} \textit{Excludes the
                \textit{Coping Process} because it is part of action
                generation.}

        \end{tablenotes}
    \end{threeparttable}%
\end{table}

\begin{table}[!tbh]
    \renewcommand{\arraystretch}{1.2}
    \centering
    \caption[Support for Component-Level Ease-of-Use High-Level
    Requirements]{Support for Component-Level \textbf{Ease-of-Use}
        High-Level Requirements}
    \label{tab:theory-req-comp-summary-easeofuse}
    \footnotesize
    \begin{threeparttable}
        \begin{tabular}{@{}cP{0.25\linewidth}ccccccc@{}}

            \toprule
            \multicolumn{2}{c}{} & \hd{\textbf{Frj.}} &
            \hd{\textbf{Laz.}\textsuperscript{\footnotesize\textpmhg{\Hi}}}
            & \hd{\textbf{Sch.}} & \hd{\textbf{Ros.}} &
            \hd{\textbf{OCC}} & \hd{\textbf{S \& K}} &
            \hd{\textbf{O \& JL}} \\
            \midrule

            \rowcolor[gray]{0.9}\ref{easeHide} & \textit{Hiding the
                Complexity of Emotion Generation} & {\normalsize\strong} &
            {\normalsize\strong} & {\normalsize\strong} & {\normalsize\good} &
            {\normalsize\good} & {\normalsize\good} & {\normalsize\good} \\

            \ref{easeAPI} & \textit{Having a Clear API (Input)} &
            {\normalsize\good} & {\normalsize\disqualified} &
            {\normalsize\good} & {\normalsize\good} & {\normalsize\weak} &
            {\normalsize\good} & {\normalsize\strong} \\

            \rowcolor[gray]{0.9}\ref{easeTrace} & \textit{Traceable
                CME Outputs} & {\normalsize\strong} & {\normalsize\strong} &
            {\normalsize\weak} & {\normalsize\strong} & {\normalsize\strong} &
            {\normalsize\strong} & {\normalsize\strong} \\

            \ref{easeAuto} & \textit{Allowing the Automatic Storage and Decay
                of the Emotion State} & {\normalsize\good} & {\normalsize\good}
                &
            {\normalsize\good} & {\normalsize\disqualified} &
            {\normalsize\good} & {\normalsize\good} & {\normalsize\good} \\

            \hline\bottomrule
        \end{tabular}
        \begin{tablenotes}

            \footnotesize
            \vspace*{2mm}

            \item \textit{See Table~\ref{tab:scoring} for score category
                descriptions.}

            \item {\small\textpmhg{\Hi}} \textit{Excludes the
                \textit{Coping Process} because it is part of action
                generation.}

        \end{tablenotes}
    \end{threeparttable}%
\end{table}

\paragraph{Choosing Theories} \progname{} prioritizes \ref{flexArch},
\ref{flexOut}, \ref{flexComplex}, \ref{flexScale}, \ref{easeHide},
\ref{easeAPI}, and \ref{easeTrace} because it is unlikely that developers will
adopt \progname{} without them. The remaining requirements offer more options
to tailor it to different game designs (\ref{flexTasks}, \ref{flexCustom},
\ref{easeAuto}) and/or could be satisfied as an extension later on
(\ref{flexNew}, \ref{flexEm}, \ref{easePX}, \ref{easeNovel}).

Theories are \textit{not} immediately discounted if they cannot satisfy a
requirement (i.e. given score is \textit{Disqualified}). They might be
extremely strong in other areas. Instead, the coverage achieved by the
\textit{set} of chosen theories must satisfy all requirements. This allows
\progname{} to take advantage of the strengths of different theories while also
compensating for their weaknesses. \progname{} uses one theory from each of the
discrete, dimensional, and appraisal perspectives to take advantage of each
perspective's strengths and mitigate its weaknesses.

The discrete and dimensional theories cannot satisfy the need for
\textit{cognitively generated/slow, secondary emotions}
(Section~\ref{sec_systemConstraints}). This means that \progname{} must use at
least one appraisal theory. It is also known that the discrete theories can
best satisfy the need for \textit{fast, primary emotions}. There is no reason
to limit the number of theories in \progname{}'s design, as the added
complexity would be internal to \progname{} (otherwise it would conflict with
\ref{easeHide}). Therefore, \progname{} must use at least one appraisal and one
discrete theory.

\progname{} also uses a dimensional theory because they are especially suitable
for representing different types of affect and their interactions in a common
space (\ref{flexNew}) and afford more control over what emotions could ``do''
in an NPC (\ref{flexOut}). The additional design freedom afforded to game
developers, including a dimensional theory could increase \progname{}'s overall
applicability.
\begin{itemize}

    \item \textbf{Appraisal Theory: Oatley \& Johnson-Laird} \\
    Three theories have \textit{Disqualified} scores: Lazarus, Scherer, and
    Roseman. Roseman has an ill-defined emotion elicitation process compared to
    the others, so \progname{} should not use it. Both Lazarus and Scherer are
    \textit{Disqualified} for a priority requirement (\ref{easeAPI} (Input) and
    \ref{flexArch} respectively), so \progname{} should not use them either. Of
    the remaining theories, Oatley \& Johnson-Laird seems the most promising.
    It has only \textit{Good} or \textit{Strong} scores for all requirements,
    and all but one priority requirement (\ref{easeHide}) has a \textit{Strong}
    score.

    \item \textbf{Discrete Theory: Plutchik} \\
    The discrete theories vary in score for only three requirements:
    \ref{flexNew}, \ref{flexEm}, and \ref{easeAPI} (Output). \progname{} should
    not use Izard because it is \textit{Disqualified} for one requirement
    (\ref{flexEm}) and has no \textit{Strong} scores. Plutchik has a better
    overall score distribution compared to Ekman \& Friesen (7 \textit{Good}
    scores to 5, and the same number of \textit{Disqualified} and
    \textit{Strong} scores) and to better support \ref{flexEm}. Oatley \&
    Johnson-Laird already has a \textit{Strong} score for \ref{easeAPI}
    (Output), so the overall design benefits more from Plutchik than Ekman \&
    Friesen.

    \item \textbf{Dimensional Theory: PAD Space} \\
    V-A and PAD Space have identical scores, except for
    one---\ref{flexEm}---which PAD Space scores higher on. Therefore, I chose
    to use PAD Space for \progname{}. If needed, V-A space can be constructed
    in PAD space due to their overlapping dimensions
    (Table~\ref{tab:affectiveDimensions}).

\end{itemize}

\begin{table}[!tbh]
    \renewcommand{\arraystretch}{1.2}
    \centering
    \caption{Comparison of Dimensions in Dimensional Theories}
    \label{tab:affectiveDimensions}
    \begin{tabular}{lcc}
        \toprule
        \textbf{Dimension} &
        {\begin{tabular}[c]{@{}c@{}}\textbf{Valence-Arousal} \\
                (\textbf{V-A})\end{tabular}} &
        {\begin{tabular}[c]{@{}c@{}}\textbf{PAD} \\
                \textbf{\citep{mehrabian1996pleasure}}\end{tabular}} \\ \hline

        \rowcolor[gray]{0.9}Pleasure/Valence & \checkmark &
        \checkmark \\

        Arousal & \checkmark & \checkmark \\

        \rowcolor[gray]{0.9}Dominance &  & \checkmark \\
        \hline\bottomrule
    \end{tabular}
\end{table}