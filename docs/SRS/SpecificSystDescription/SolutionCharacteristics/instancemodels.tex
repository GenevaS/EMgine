\subsubsection{Instance Models} \label{sec_instance}
Instance Models refine the Theoretical Models (Section~\ref{sec_theoretical})
into mathematical representations using Assumptions
(Section~\ref{sec_assumptions}) and Data Types (Section~\ref{sec_typedefs}).
These models address \progname{}'s Goals (Section~\ref{sec_goals}):
\begin{itemize}
    \item \gsref{G_EmotionGeneration} is solved by
    \iref{IM_CalculateEmotionGP}, \iref{IM_CalculateEmotionSurpriseElicit},
    \iref{IM_CalculateEmotionInterestElicit},
    \iref{IM_CalculateEmotionAcceptanceElicit},
    \iref{IM_CalculateEmotionIntensity}, \iref{IM_CalculateEmotionSurprise},
    \iref{IM_CalculateEmotionInterest}, and \iref{IM_CalculateEmotionAcceptance}

    \item \gsref{G_EmotionDecay} is solved by \iref{IM_DecayEmotionState} and
    \iref{IM_DecayEmotion}

    \item \gsref{G_UpdateEmotionState} is solved by
    \iref{IM_UpdateEmotionState}, \iref{IM_UpdateEmotionState2}, and
    \iref{IM_UpdateEmotion}

    \item \gsref{G_QueryEmotionState} is solved by \iref{IM_GetEmotionState} and
    \iref{IM_GetEmotionStatePAD}
\end{itemize}

~\newline\noindent
\begin{minipage}{\textwidth}
    \renewcommand*{\arraystretch}{1.5}
    \begin{tabular}{| p{\colAwidth}  p{\colBwidth}|}
        \hline
        \rowcolor[gray]{0.9}
        \bf IM\refstepcounter{instnum}\theinstnum
        \label{IM_CalculateEmotionGP} &
        \bf Calculate Emotion from Goals and Plans (\textit{Joy},
        \textit{Sadness}, \textit{Anger}, \textit{Fear}, \textit{Disgust}) \\
        \hline
    \end{tabular}

    \renewcommand*{\arraystretch}{1.5}
    \begin{tabular}{ p{\colAwidth}  p{\colBwidth}}
        \bf Input & $s : \worldstatetype$, $s_\Delta : \worldstatechangetype$,
        $g : \goaltype$, $p : \plantype$, $o \in \left\{ \mathtt{Gustatory},
        \mathtt{Preserve} \right\}$ \vspace*{2mm}\\

        \bf Output & $\begin{aligned}[t]
            \mathit{em} : \{ \mJoy &= (\statedistancetype, \statedistancetype,
            \statedistancechangetype)^?, \mSadness = ( \worldstatetype,
            \statedistancetype )^?, \mFear = (\statedistancetype,
            \statedistancetype, \statedistancechangetype)^?, \\
            \mAnger &= (\worldstatetype, ({\worldstatechangetype}^m))^?,
            \mDisgust = (\statedistancetype, \statedistancetype,
            \statedistancechangetype)^? \} \\
            &\defEq \{ \mJoy = J, \mSadness = S, \mFear = F, \mAnger = A,
            \mDisgust = D \}
        \end{aligned}$ \vspace*{2mm}\\

        & where $J : (\statedistancetype, \statedistancetype,
        \statedistancechangetype)^? = \begin{cases}
            \parbox{0.15\columnwidth}{$(g.\mathtt{goal}\left(s\right), \\
            g.\mathtt{goal}\left(s \oplus s_\Delta\right), \\
            g.\mathtt{goal'}\left(s, s_\Delta\right))$,} &
            \parbox{0.38\columnwidth}{$g.\mathtt{goal'}\left(s, s_\Delta\right)
            > 0 \\ \land \, g.\mathtt{goal}\left(s \oplus
            s_\Delta\right) < g.\mathtt{goal}\left(s\right)$} \\
            \emptyset, & Otherwise \\
        \end{cases}$ \vspace*{1mm}\\

        & $S : ( \worldstatetype, \statedistancetype )^? = \begin{cases}
            (s \, {\oplus} \, s_\Delta, g.\mathtt{goal}\left(s\right)), &
            \parbox{0.6\columnwidth}{$\left(s \, {=} \, p_i \, {\land} \,
            \nexists j :\mathbb{N} \, {|} \, s \, {\oplus} \, s_\Delta \,
            {\neq} \, p_{i+j} \, {\land} \\ \nexists \, d \, {:} \,
            \worldstatechangetype \, {|} \, s \, {\oplus} \, d \, {=} \,
            p_{i+1}\right) \\ \lor \; g.\mathtt{goal}\left(s\right) =
            \pm\infty$} \\
            \emptyset, & Otherwise \\
        \end{cases}$ \vspace*{1mm}\\

        & $\mathit{F} : ( \statedistancetype, \statedistancetype,
        \statedistancechangetype )^? = \begin{cases}
           \parbox{0.15\columnwidth}{$(g.\mathtt{goal}\left(s\right), \\
           g.\mathtt{goal}\left(s \oplus s_\Delta\right), \\
           g.\mathtt{goal'}\left(s, s_\Delta\right))$,} &
           \parbox{0.6\columnwidth}{$o = \mathtt{Preserve}
           \land g.\mathtt{goal'}\left(s, s_\Delta\right) > 0 \\ \land
           g.\mathtt{goal}\left(s \oplus s_\Delta\right) >
           g.\mathtt{goal}\left(s\right)$} \\
          \emptyset, & Otherwise \\
        \end{cases}$ \vspace*{1mm}\\

        & $\mathit{A} : ( \worldstatetype, ({\worldstatechangetype}^m) )^? =
        \begin{cases}
            \parbox{0.15\columnwidth}{$(p_i \, {\oplus} \, s_\Delta, \\
                \left(d_0, .., d_m\right))$,} &
            \parbox{0.6\columnwidth}{$s \, {=} \, p_i \, {\land} \,
                p_i \, {\oplus} \, s_\Delta \, {\neq} \, p_{i+1} \\ \land
                \exists \, \left(d_0, .., d_m\right), m > 0 \; | \; s \oplus
                d_0 \oplus ... \oplus d_m = p_{i+1}$} \\
            \emptyset, & Otherwise \\
        \end{cases}$ \vspace*{1mm}\\

        & $\mathit{D} : ( \statedistancetype, \statedistancetype,
        \statedistancechangetype )^? = \begin{cases}
            \parbox{0.15\columnwidth}{$(g.\mathtt{goal}\left(s\right), \\
            g.\mathtt{goal}\left(s \oplus s_\Delta\right), \\
            g.\mathtt{goal'}\left(s, s_\Delta\right)$,} &
            \parbox{0.6\columnwidth}{$o = \mathtt{Gustatory} \land
            g.\mathtt{goal'}\left(s, s_\Delta\right) > 0 \\ \land
            g.\mathtt{goal}\left(s \oplus s_\Delta\right) >
            g.\mathtt{goal}\left(s\right)$} \\
            \emptyset, & Otherwise \\
        \end{cases}$ \vspace*{1mm}\\
        \hline
    \end{tabular}
\end{minipage}

\paragraph{Description} Given a world state $s$ (\tyref{TY_WorldState}), a
game world event $s_\Delta$ (\tyref{TY_WorldStateChange}), a goal $g$
(\tyref{TY_Goal}), and a plan to achieve that goal $p$ (\tyref{TY_Plan}),
return a record of world state, event, and distance information for each of
\textit{Joy}, \textit{Sadness}, \textit{Anger}, \textit{Fear}, \textit{Disgust}
if it is elicited and nothing if it is not. These evaluations are based on CTE
(\tref{T_CalculateEmotionGP}):
\begin{itemize}
    \item The game world event causes a change in distance
    ($g.\mathtt{goal'}\left(s, s_\Delta\right) : \statedistancechangetype > 0$,
    \tyref{TY_DistanceBetweenWorldStatesChange}) that \textit{decreases} the
    distance to $g$ ($g.\mathtt{goal}\left(s \oplus s_\Delta\right) :
    \statedistancetype < g.\mathtt{goal}\left(s\right) : \statedistancetype$,
    \tyref{TY_DistanceBetweenWorldStates}), then the entity is achieving its
    ``sub-goals'' and it experiences $\mJoy$

    \item The game world state is in the entity's plan ($s = p_i$), and the
    game world event does not move the entity to the next state in its plan
    ($p_i \oplus s_\Delta : \worldstatetype \neq p_{i+1} : \worldstatetype$)
    and there is no game world event that will ($\nexists d :
    \worldstatechangetype \; | \; p_i \oplus d = p_{i+1}$), or $g$ is no longer
    achievable, i.e. ``lost'' ($g.\mathtt{goal}\left(s\right) :
    \statedistancetype = \pm\infty$), then the entity experiences $\mSadness$

    \item The goal is self-preservational ($o = \mathtt{Preserve}$) and there
    is a game world event ($g.\mathtt{goal'}\left(s, s_\Delta\right) :
    \statedistancechangetype > 0$) that \textit{increases} the distance to $g$
    ($g.\mathtt{goal}\left(s \oplus s_\Delta\right) : \statedistancetype >
    g.\mathtt{goal}\left(s\right) : \statedistancetype$,
    \aref{A_GustatoryGoal}), then the entity experiences $\mFear$

    \item The game world state is in the entity's plan ($s = p_i$) and the game
    world event does not move the entity to the next state in its plan ($p_i
    \oplus s_\Delta : \worldstatetype \neq p_{i+1} : \worldstatetype$) but
    there are at least two game world events that transform $s$ into $p_i$
    ($\exists \, (d_0, .., d_m), m > 0 \; | \; s \oplus d_0 \oplus ... \oplus
    d_m = p_{i+1}$), then the entity's plan is ``frustrated'' but still
    feasible and it experiences $\mAnger$

    \item The goal is gustatory ($o = \mathtt{Gustatory}$) and there is a
    game world event ($g.\mathtt{goal'}\left(s, s_\Delta\right) :
    \statedistancechangetype > 0$) that \textit{increases} the distance to $g$
    ($g.\mathtt{goal}\left(s \oplus s_\Delta\right) : \statedistancetype >
    g.\mathtt{goal}\left(s\right) : \statedistancetype$,
    \aref{A_GustatoryGoal}), then the entity experiences $\mDisgust$

    \item None of these situations apply then the game world event does not
    impact the entity and it does not experience an emotion
\end{itemize}

\paragraph{Comments} The difference between \textit{Fear} and \textit{Disgust}
appears minimal because they are both triggered by \textit{increases} in goal
distance ($g.\mathtt{goal}\left(s \oplus s_\Delta\right) : \statedistancetype >
g.\mathtt{goal}\left(s\right) : \statedistancetype$). This is due to an
assumption that gustatory and self-preservation goals are satisfied by default
(\aref{A_GustatoryGoal}) such that changes increasing distance trigger the
emotion (e.g. there is no reason to be fearful \textit{until} there is a change
in the entity's personal safety, or to be disgusted until something disgusting
appears). However, the choice of $s$ and $s_\Delta$ changes \textit{when} the
world state changes i.e. $s$ is the previous state and $s_\Delta$ changed it
into the current game world state, or $s$ is the current world state and
$s_\Delta$ predicts what the next world state will be relative to $s$. \\\hrule

\paragraph{Sources} --

\paragraph{Depends On} \aref{A_GustatoryGoal}, \tref{T_CalculateEmotionGP},
\tyref{TY_EmotionKind}, \tyref{TY_WorldState}, \tyref{TY_WorldStateChange},
\tyref{TY_DistanceBetweenWorldStates},
\tyref{TY_DistanceBetweenWorldStatesChange}, \tyref{TY_Goal}, \tyref{TY_Plan}

\paragraph{Ref. By} \rref{R_GenerateEmotionCTE} \\\hrule\vspace{0.5mm}\hrule

~\newline

\noindent
\begin{minipage}{\textwidth}
    \renewcommand*{\arraystretch}{1.5}
    \begin{tabular}{| p{\colAwidth}  p{\colBwidth}|}
        \hline
        \rowcolor[gray]{0.9}
        \bf IM\refstepcounter{instnum}\theinstnum
        \label{IM_CalculateEmotionSurpriseElicit} &
        \bf Calculate Emotion of \textit{Surprise} \\
        \hline
    \end{tabular}

    \renewcommand*{\arraystretch}{1.5}
    \begin{tabular}{ p{\colAwidth}  p{\colBwidth}}
        \bf Input &  $\{ t : \timetype, t' : \timetype \rightarrow t < t' \}$,
        $\mathit{tolerance} : \deltatimetype $ \\

        \bf Output & $ s : \deltatimetype^? \defEq \begin{cases}

            t' - t, & t' - t < \mathit{tolerance} \\

            \emptyset, & Otherwise \\

        \end{cases} $
        \vspace*{2mm}\\ \hline
    \end{tabular}
\end{minipage}

\paragraph{Description} Given the current time $t' : \timetype$ and the time
that the system last changed its emotion ``mode'' $t : \timetype$
(\tyref{TY_Time}), as well as a tolerance threshold $\mathit{tolerance} :
\deltatimetype $, output the difference between $t'$ and $t$ if
\textit{Surprise} is elicited and nothing if it is not.

\textit{Surprise} is elicited if the difference between $t'$ and $t$ is less
than $\mathit{tolerance}$. The closer that $t'$ and $t$ are, the smaller the
difference between them (more ``sudden''). The value of $\mathit{tolerance}$ is
a ``tolerance'' such that higher values makes an entity more susceptible to
\textit{Surprise} (i.e. easier to elicit because it allows for larger
differences between $t'$ and $t$). \\\hrule

\paragraph{Sources} --

\paragraph{Depends On} \tref{T_CalculateEmotionSurprise}, \tyref{TY_Time}

\paragraph{Ref. By} \rref{R_GenerateEmotionCTE} \\\hrule\vspace{0.5mm}\hrule

~\newline

\noindent
\begin{minipage}{\textwidth}
    \renewcommand*{\arraystretch}{1.5}
    \begin{tabular}{| p{\colAwidth}  p{\colBwidth}|}
        \hline
        \rowcolor[gray]{0.9}
        \bf IM\refstepcounter{instnum}\theinstnum
        \label{IM_CalculateEmotionInterestElicit} &
        \bf Calculate Emotion of \textit{Interest} \\
        \hline
    \end{tabular}

    \renewcommand*{\arraystretch}{1.5}
    \begin{tabular}{ p{\colAwidth}  p{\colBwidth}}
        \bf Input & $ at : \attentiontype $, $\mathit{tolerance} :
        \attentiontype $ \\

        \bf Output & $ i : \attentiontype^? \defEq \begin{cases}

            at - \mathit{tolerance}, & at > \mathit{tolerance} \\

            \emptyset, & Otherwise \\

        \end{cases} $ \vspace*{1mm}\\
        \hline
    \end{tabular}
\end{minipage}

\paragraph{Description} Given the attention $at : \attentiontype$ paid to an
entity $x$ (\tyref{TY_Attention}) and a tolerance threshold $\mathit{tolerance}
: \attentiontype $, output the difference between $at$ and $\mathit{tolerance}$
(i.e. the ``clearance'' between $at$ and the threshold value) if
\textit{Interest} is elicited and nothing if it is not.

\textit{Interest} is elicited if the amount of attention $at$ an entity has
paid to $x$ is greater than $\mathit{tolerance}$. The value of
$\mathit{tolerance}$ is a ``tolerance'' such that higher values makes an entity
less susceptible to \textit{Interest} (i.e. harder to elicit because it
requires a higher $at$ value).

\paragraph{Comments} The elicitation of \textit{Interest} is necessarily tied
to an entity $x$ so that its behaviour can always be linked to a specific game
event. This makes it possible to reliably test and debug. Modelling a
``curious'' entity would then require an explicit goal that can be monitored,
which users can tie to $at$ such that it is checked and/or updated when that
goal is affected. \\\hrule

\paragraph{Sources} --

\paragraph{Depends On} \tref{T_CalculateEmotionInterest}, \tyref{TY_Attention}

\paragraph{Ref. By} \rref{R_GenerateEmotionCTE} \\\hrule\vspace{0.5mm}\hrule

~\newline

\noindent
\begin{minipage}{\textwidth}
    \renewcommand*{\arraystretch}{1.5}
    \begin{tabular}{| p{\colAwidth}  p{\colBwidth}|}
        \hline
        \rowcolor[gray]{0.9}
        \bf IM\refstepcounter{instnum}\theinstnum
        \label{IM_CalculateEmotionAcceptanceElicit} &
        \bf Calculate Emotion of \textit{Acceptance} \\
        \hline
    \end{tabular}

    \renewcommand*{\arraystretch}{1.5}
    \begin{tabular}{ p{\colAwidth}  p{\colBwidth}}
        \bf Input & $s : (\socialattachmenttype)^?$, $\mathit{joyIntensity} :
        \emotionintensitytype$, $\mathit{tolerance} : \emotionintensitytype$ \\

        \bf Output & $ r : (\socialattachmenttype, \responsestrength)^?
        \defEq \begin{cases}

            (s, \mathit{joyIntensity} - \mathit{tolerance}), & s \neq \emptyset
            \, \wedge \, \mathit{joyIntensity} > \mathit{tolerance} \\

            \emptyset, & Otherwise \\

        \end{cases} $ \vspace*{1mm}\\
        \hline
    \end{tabular}
\end{minipage}

\paragraph{Description} Given an Option type that contains a social attachment
that the entity has to $A$ if it exists and is empty otherwise $s :
(\socialattachmenttype)^?$, the entity's \textit{Joy} intensity
$\mathit{joyIntensity} : \emotionintensitytype$ (\tyref{TY_EmotionIntensity}),
and a tolerance threshold $\mathit{tolerance} : \emotionintensitytype $, output
the social attachment $s : \socialattachmenttype$ (\tyref{TY_Relation-CTE})
that this entity has to $A$ and the difference between $\mathit{joyIntensity}$
and $\mathit{tolerance}$ as a change in intensity (\tyref{TY_DeltaIntensity},
i.e. the intensity change from baseline intensity of zero, where the zero point
is shifted by $\mathit{tolerance}$) if \textit{Acceptance} is elicited and
nothing if it is not.

\textit{Acceptance} is elicited if there is a social attachment between the
entity and $A$ and the entity's experienced \textit{Joy} intensity is greater
than $\mathit{tolerance}$. The value of $\mathit{tolerance}$ is a ``tolerance''
such that higher values makes an entity less susceptible to \textit{Acceptance}
(i.e. harder to elicit because it requires a higher $\mathit{joyIntensity}$).
If $\mathit{joyIntensity} = 0$, the entity is not experiencing \textit{Joy} and
it is not possible to experience \textit{Acceptance} towards $A$---even if
$\mathit{tolerance} = 0$.

If the entity has an existing social attachment to $A$, then a value for $s$
already exists and returning it does not offer new information. However, if the
entity could establish a social attachment with $A$ if \textit{Acceptance} is
elicited (\tref{T_CalculateEmotionAcceptance}), a new one must be created and
assigned to $s$. In this way, if \textit{Acceptance} is elicited the value of
$s$ is a valid social attachment between the entity and $A$. Therefore, it can
be added to the entity's collection of social attachments. \\\hrule

\paragraph{Sources} --

\paragraph{Depends On} \tref{T_CalculateEmotionAcceptance},
\tyref{TY_EmotionIntensity}, \tyref{TY_DeltaIntensity}, \tyref{TY_Relation-CTE}

\paragraph{Ref. By} \rref{R_GenerateEmotionCTE} \\\hrule\vspace{0.5mm}\hrule

~\newline

\noindent
\begin{minipage}{\textwidth}
    \renewcommand*{\arraystretch}{1.5}
    \begin{tabular}{| p{\colAwidth}  p{\colBwidth}|}
        \hline
        \rowcolor[gray]{0.9}
        \bf IM\refstepcounter{instnum}\theinstnum
        \label{IM_CalculateEmotionIntensity} &
        \bf Calculate Emotion Intensity (\textit{Joy}, \textit{Sadness},
        \textit{Anger}, \textit{Fear}, \textit{Disgust})\\
        \hline
    \end{tabular}

    \renewcommand*{\arraystretch}{1.5}
    \begin{tabular}{ p{\colAwidth}  p{\colBwidth}}
        \bf Input & $ s : \worldstatetype $, $ s_\Delta : \worldstatechangetype
        $, $ g : \goaltype $ \\

        \bf Output & $ i_\Delta : \responsestrength \defEq
        g.\mathtt{goal'}\left(s, s_\Delta\right) \cdot g.\mathtt{importance} $
        \\ \hline
    \end{tabular}
\end{minipage}

\paragraph{Description} Given a world state $s : \worldstatetype$
(\tyref{TY_WorldState}), a game event that changes that state $s_\Delta :
\worldstatechangetype$ (\tyref{TY_WorldStateChange}), and a goal $g :
\goaltype$ (\tyref{TY_Goal}), output a change in intensity $i_\Delta :
\responsestrength$ (\tyref{TY_DeltaIntensity}) proportional to the change in
game state towards goal achievement $g.\mathtt{goal'}\left(s, s_\Delta\right)$,
scaled by the goal's importance to the entity $g.\mathtt{importance}$.

This assumes that the faster an entity is achieving a goal (i.e. the larger the
change in distance\linebreak $g.\mathtt{goal'}\left(s, s_\Delta\right)$), the
larger the momentum behind it. This causes a higher intensity because the
entity wants to maintain the conditions contributing to goal achievement
(\tref{T_CalculateEmotionIntensity}). \\\hrule

\paragraph{Sources} --

\paragraph{Depends On} \tref{T_CalculateEmotionIntensity},
\tyref{TY_DeltaIntensity}, \tyref{TY_WorldState}, \tyref{TY_WorldStateChange},
\tyref{TY_Goal}

\paragraph{Ref. By} \rref{R_CalculateIntensity} \\\hrule\vspace{0.5mm}\hrule

~\newline

\noindent
\begin{minipage}{\textwidth}
    \renewcommand*{\arraystretch}{1.5}
    \begin{tabular}{| p{\colAwidth}  p{\colBwidth}|}
        \hline
        \rowcolor[gray]{0.9}
        \bf IM\refstepcounter{instnum}\theinstnum
        \label{IM_CalculateEmotionSurprise} &
        \bf Calculate Emotion Intensity (\textit{Surprise}) \\
        \hline
    \end{tabular}

    \renewcommand*{\arraystretch}{1.5}
    \begin{tabular}{ p{\colAwidth}  p{\colBwidth}}
        \bf Input &  $ \Delta t : \deltatimetype$, $\delta : \mathbb{R}_{>0}$ \\

        \bf Output & $ i_\Delta : \responsestrength \defEq \dfrac{1}{\delta
            \cdot \Delta t} $
        \vspace*{2mm}\\ \hline
    \end{tabular}
\end{minipage}

\paragraph{Description} Given the difference between two times $\Delta t :
\deltatimetype$ (\tyref{TY_Time}), as well as a strictly positive and
real-valued $\delta$, output a change in intensity $i_\Delta :
\responsestrength$ equivalent to:
$$y = \dfrac{1}{\delta \cdot x}$$

where $y = i$ and $x = \Delta t$. This creates a larger response when $\Delta
t$ is small and quickly decreases as $\Delta t$ grows. The value of $\delta$ is
a ``tolerance'' such that higher values decreases the time window and strength
of an entity's \textit{Surprise}. \\\hrule

\paragraph{Sources} --

\paragraph{Depends On} \tref{T_CalculateEmotionSurprise}, \tyref{TY_Time},
\tyref{TY_DeltaIntensity}

\paragraph{Ref. By} \rref{R_CalculateIntensity} \\\hrule\vspace{0.5mm}\hrule

~\newline

\noindent
\begin{minipage}{\textwidth}
    \renewcommand*{\arraystretch}{1.5}
    \begin{tabular}{| p{\colAwidth}  p{\colBwidth}|}
        \hline
        \rowcolor[gray]{0.9}
        \bf IM\refstepcounter{instnum}\theinstnum
        \label{IM_CalculateEmotionInterest} &
        \bf Calculate Emotion Intensity (\textit{Interest}) \\
        \hline
    \end{tabular}

    \renewcommand*{\arraystretch}{1.5}
    \begin{tabular}{ p{\colAwidth}  p{\colBwidth}}
        \bf Input & $ at : \attentiontype $, $ i_\delta : \responsestrength $,
        $ d : [0, 1] \in \mathbb{R}_{\geq 0}$ \vspace*{1mm}\\

        \bf Output & $ i_\Delta : \responsestrength \defEq {\mathlarger\sum_{i
        = 0}^{at}} d^i \cdot i_\delta $ \vspace*{2mm}\\
        \hline
    \end{tabular}
\end{minipage}

\paragraph{Description} Given the attention $at : \attentiontype$ paid to an
entity $x$ (\tyref{TY_Attention}) and how much each time step in $at$ should
change the intensity by as given by $ i_\delta : \responsestrength $
(\tyref{TY_DeltaIntensity}) such that a higher $i_\delta$ implies that it is
easier for the entity to experience \textit{Interest} towards $x$, output a
change in intensity $i_\Delta : \responsestrength$ equivalent to the product of
$at$ and $i_\delta$. This represents a constant change in emotion intensity of
\textit{Interest} for each time step in $at$.

Although constraining $i_\delta$ to strictly positive values is recommended,
this is not enforced because users might find a use for negative values here.
\\\hrule

\paragraph{Sources} --

\paragraph{Depends On} \tref{T_CalculateEmotionInterest},
\tyref{TY_DeltaIntensity},
\tyref{TY_Attention}

\paragraph{Ref. By} \rref{R_CalculateIntensity} \\\hrule\vspace{0.5mm}\hrule

~\newline

\noindent
\begin{minipage}{\textwidth}
    \renewcommand*{\arraystretch}{1.5}
    \begin{tabular}{| p{\colAwidth}  p{\colBwidth}|}
        \hline
        \rowcolor[gray]{0.9}
        \bf IM\refstepcounter{instnum}\theinstnum
        \label{IM_CalculateEmotionAcceptance} &
        \bf Calculate Emotion Intensity (\textit{Acceptance}) \\
        \hline
    \end{tabular}

    \renewcommand*{\arraystretch}{1.5}
    \begin{tabular}{ p{\colAwidth}  p{\colBwidth}}
        \bf Input & $ r : \socialattachmenttype $, $ i_\delta :
        \responsestrength $, $ d : [0, 1] \in \mathbb{R}_{\geq 0}$
        \vspace*{1mm}\\

        \bf Output & $ i_\Delta : \responsestrength \defEq {\mathlarger\sum_{i
        = 0}^{r}} d^i \cdot i_\delta $ \vspace*{2mm}\\
        \hline
    \end{tabular}
\end{minipage}

\paragraph{Description} Given a social attachment to entity $A$
(\tyref{TY_Relation-CTE}) $r$ and how much each ``level'' of attachment should
change the intensity as given by $ i_\delta : \responsestrength $
(\tyref{TY_DeltaIntensity}), output a change in intensity $i_\Delta :
\responsestrength$ equivalent to the sum of $i_\delta$ values for each
``level'' in $r$ such that each value is decayed by a factor of $d^i$. This
causes increasing ``levels'' of $r$ to have diminishing returns, preserving the
value of having large $r$ values while recognizes that one might not see a
difference when adding another ``level'' to an already high-level social
attachment.

The possible values for $i_\delta$ are not restricted to positive or negative
values because it is possible to have less or no trust in entity $A$. This
allows this function to be called if an entity experiences emotions that should
decrease their trust in $A$, which \progname{} does not model. \\\hrule

\paragraph{Sources} --

\paragraph{Depends On} \tref{T_CalculateEmotionAcceptance},
\tyref{TY_DeltaIntensity}, \tyref{TY_Relation-CTE}

\paragraph{Ref. By} \rref{R_CalculateIntensity} \\\hrule\vspace{0.5mm}\hrule

~\newline

\noindent
\begin{minipage}{\textwidth}
    \renewcommand*{\arraystretch}{1.5}
    \begin{tabular}{| p{\colAwidth}  p{\colBwidth}|}
        \hline
        \rowcolor[gray]{0.9}
        \bf IM\refstepcounter{instnum}\theinstnum
        \label{IM_DecayEmotionState} &
        \bf Decaying Emotion State \\
        \hline
    \end{tabular}

    \renewcommand*{\arraystretch}{1.5}
    \begin{tabular}{ p{\colAwidth}  p{\colBwidth}}
        \bf Input & $i_0 : \emotionintensitytype$, $\Delta t : \deltatimetype$,
        $c : \mathbb{R}_{>0}$, $\left\{ \, i_\lambda : \emotiondecaytype,
        i_{eq} : \emotionintensitytype \, \right\}$
        \vspace*{2mm}\\

        \bf Output & $ i : \emotionintensitytype \defEq \left(i_0 - i_{eq}
        \right) \cdot e^{-\mbox{\scriptsize $\dfrac{c}{2}$} \cdot \Delta t}
        \cdot \cos \left(\sqrt{i_\lambda - \left(\dfrac{c}{2} \right)^2} \cdot
        \Delta t \right) $
        \vspace*{2mm}\\\hline
    \end{tabular}
\end{minipage}

\paragraph{Description} Given an the emotion intensity before it has been
decayed $i_0 : \emotionintensitytype$ (\tyref{TY_EmotionIntensity}), the
elapsed time since emotion decay began $\Delta t : \deltatimetype$
(\tyref{TY_Time}), a strictly positive and real-valued damping coefficient $c$,
an emotion decay rate $i_\lambda : \emotiondecaytype$
(\tyref{TY_EmotionDecay}), and ``equilibrium'' intensity $i_{eq} :
\emotionintensitytype$, output a ``decayed'' intensity $i :
\emotionintensitytype$. Emotion decay (\tref{T_DecayEmotionState}) is modelled
by a damped harmonic oscillator mass-spring system of the form:
$$x''\left(t\right) + c \cdot x'\left(t\right) + k_s \cdot x\left(t\right) = 0
$$

where the spring constant $k_s = i_\lambda$. The general model for position at
time $t = t_0 + \Delta t$ in this system is given by:
$$x\left(t\right) = A_0 \cdot e^{-\sigma \cdot t} \cdot \cos\left(\omega \cdot
t - \phi\right)$$

where $A_0 = x\left(t_0\right)$, $\sigma = \dfrac{c}{2 \cdot m}$, and $\omega =
\sqrt{\dfrac{k_s}{m} - \left(\dfrac{c}{2 \cdot m}\right)^2}$. Assuming that
$k_s = i_\lambda$, $m = 1$, $A_0 = x\left(t_0\right) = i - i_{eq}$, phase shift
$\phi = 0$, and $t_0 = 0$ such that $t = \Delta t$ gives the function for
emotion decay.

The damping coefficient $c$ determines the amount of oscillation that occurs as
an emotion intensity returns to equilibrium. Oscillation behaviour is given by:
$$\zeta = \frac{c}{2 \cdot \sqrt{m \cdot k_s}}$$

The system's ``mass'' $m$ is assumed to be $1$ and $k_s = i_\lambda$, so this
simplifies to:
$$\zeta = \dfrac{c}{2 \cdot \sqrt{i_\lambda}}$$

When:
\begin{itemize}
    \item $\zeta > 1$, emotion decay is \textit{overdamped} such that it does
    not oscillate as it returns to $i_{eq}$. It reaches equilibrium more slowly
    as $\zeta$ increases.

    \item $\zeta = 1$, emotion decay is \textit{critically damped} such that it
    returns to $i_{eq}$ as quickly as possible without overshooting it.

    \item $0 < \zeta < 1$, emotion decay is \textit{underdamped} such that it
    oscillates as it returns to $i_{eq}$. Oscillations decrease more quickly as
    $\zeta$ approaches 1.
\end{itemize}
\hrule

\paragraph{Sources}
{\small\url{https://en.wikipedia.org/wiki/Harmonic_oscillator#Damped_harmonic_oscillator},

    \url{https://www.lehman.edu/faculty/anchordoqui/chapter23.pdf}}

\paragraph{Depends On} \tref{T_DecayEmotionState}, \tyref{TY_Time},
\tyref{TY_EmotionIntensity}, \tyref{TY_EmotionDecay}

\paragraph{Ref. By} \iref{IM_DecayEmotion}, \rref{R_DecayIntensity}
\\\hrule\vspace{0.5mm}\hrule

~\newline

\noindent
\begin{minipage}{\textwidth}
    \renewcommand*{\arraystretch}{1.5}
    \begin{tabular}{| p{\colAwidth}  p{\colBwidth}|}
        \hline
        \rowcolor[gray]{0.9}
        \bf IM\refstepcounter{instnum}\theinstnum
        \label{IM_UpdateEmotionState} &
        \bf Updating the Intensity of an Emotion Kind \\
        \hline
    \end{tabular}

    \renewcommand*{\arraystretch}{1.5}
    \begin{tabular}{ p{\colAwidth}  p{\colBwidth}}
        \bf Input & $k : \emotionkindstype$, $ i : \emotionkindstype
        \rightarrow \emotionintensitytype $, $ I : \emotionkindstype
        \rightarrow \responsestrength, m : \emotionkindstype \rightarrow
        \emotionintensitytype $ \\

        \bf Output & $ i'\left(k\right): \emotionintensitytype \defEq
        \mathtt{clamp}\left(i\left(k\right) + I\left(k\right), 0,
        m\left(k\right) \right) $ \\
        \hline
    \end{tabular}
\end{minipage}

\paragraph{Description} Given emotion kinds $k : \emotionkindstype$
(\tyref{TY_EmotionKind}), a function from emotion kinds to intensities $i :
\emotionkindstype \rightarrow \emotionintensitytype$
(\tyref{TY_EmotionIntensity}), and to changes in intensity $I :
\emotionkindstype \rightarrow \responsestrength$ (\tyref{TY_DeltaIntensity}),
output a new intensity for each emotion kind, $i'\left(k\right) :
\emotionintensitytype$, by adding $i\left(k\right) : \emotionintensitytype$ and
$I\left(k\right) : \responsestrength$ (\aref{A_UpdateEmotionState}). The sum is
clamped to the range $\left[0, m(k)\right]$, where m is a function from emotion
kinds to maximum intensities. \\\hrule

\paragraph{Sources} --

\paragraph{Depends On} \aref{A_UpdateEmotionState},
\tyref{TY_EmotionIntensity}, \tyref{TY_DeltaIntensity}, \tyref{TY_EmotionKind}

\paragraph{Ref. By} \rref{R_UpdateAnIntensity} \\\hrule\vspace{0.5mm}\hrule

~\newline

\noindent
\begin{minipage}{\textwidth}
    \renewcommand*{\arraystretch}{1.5}
    \begin{tabular}{| p{\colAwidth}  p{\colBwidth}|}
        \hline
        \rowcolor[gray]{0.9}
        \bf IM\refstepcounter{instnum}\theinstnum
        \label{IM_UpdateEmotionState2} &
        \bf Updating the Intensity Component of an Emotion State \\
        \hline
    \end{tabular}

    \renewcommand*{\arraystretch}{1.5}
    \begin{tabular}{ p{\colAwidth}  p{\colBwidth}}
        \bf Input & $ es : \emotionstatetype $, $ i : \emotionkindstype
        \rightarrow \emotionintensitytype $ \\

        \bf Output & $ es' : \emotionstatetype \defEq \left\{ \, es \text{ with
        } \mathtt{intensities} = i \, \right\} $ \\
        \hline
    \end{tabular}
\end{minipage}

\paragraph{Description} Given an emotion state $es : \emotionstatetype$
(\tyref{TY_EmotionState}) and a function from emotion kinds to intensity $i :
\emotionkindstype \rightarrow \emotionintensitytype$ (\tyref{TY_EmotionKind},
\tyref{TY_EmotionIntensity}), output a new emotion state $es'$ by replacing
$es.\mathtt{intensities}$ with $i$. \\\hrule

\paragraph{Sources} --

\paragraph{Depends On} \tyref{TY_EmotionIntensity}, \tyref{TY_EmotionKind},
\tyref{TY_EmotionState}

\paragraph{Ref. By} \iref{IM_DecayEmotion}, \rref{R_UpdateEmotionState}
\\\hrule\vspace{0.5mm}\hrule

~\newline

\noindent
\begin{minipage}{\textwidth}
    \renewcommand*{\arraystretch}{1.5}
    \begin{tabular}{| p{\colAwidth}  p{\colBwidth}|}
        \hline
        \rowcolor[gray]{0.9}
        \bf IM\refstepcounter{instnum}\theinstnum
        \label{IM_UpdateEmotion} &
        \bf Updating Emotion \\
        \hline
    \end{tabular}

    \renewcommand*{\arraystretch}{1.5}
    \begin{tabular}{ p{\colAwidth}  p{\colBwidth}}
        \bf Input & $e : \emotiontype$, $t : \timetype$, $es :
        \emotionstatetype$ \\

        \bf Output & $ e' : \emotiontype \defEq \left\{ \, e \text{ with }
        e\left(t\right) = es \, \right\} $ \\
        \hline
    \end{tabular}
\end{minipage}

\paragraph{Description} Given an emotion $e : \emotiontype$
(\tyref{TY_Emotion}), emotion state $es : \emotionstatetype$
(\tyref{TY_EmotionState}), and a time $t : \timetype$ (\tyref{TY_Time}), output
a new emotion $e'$ by adding $es$ to $e$ at time $t$. \\\hrule

\paragraph{Sources} --

\paragraph{Depends On} \tyref{TY_Time}, \tyref{TY_EmotionState},
\tyref{TY_Emotion}

\paragraph{Ref. By} \rref{R_UpdateEmotion} \\\hrule\vspace{0.5mm}\hrule

~\newline

\noindent
\begin{minipage}{\textwidth}
    \renewcommand*{\arraystretch}{1.5}
    \begin{tabular}{| p{\colAwidth}  p{\colBwidth}|}
        \hline
        \rowcolor[gray]{0.9}
        \bf IM\refstepcounter{instnum}\theinstnum
        \label{IM_GetEmotionState} &
        \bf Getting an Emotion State \\
        \hline
    \end{tabular}

    \renewcommand*{\arraystretch}{1.5}
    \begin{tabular}{ p{\colAwidth}  p{\colBwidth}}
        \bf Input & $e : \emotiontype$, $t : \timetype$ \\

        \bf Output & $ es : \emotionstatetype \defEq e\left(t\right) $ \\
        \hline
    \end{tabular}
\end{minipage}

\paragraph{Description} Given an emotion $e : \emotiontype$
(\tyref{TY_Emotion}) and a time $t : \timetype$ (\tyref{TY_Time}), output the
emotion state $es : \emotionstatetype$ (\tyref{TY_EmotionState}) at
$e\left(t\right)$. \\\hrule

\paragraph{Sources} --

\paragraph{Depends On} \tyref{TY_Time}, \tyref{TY_EmotionState},
\tyref{TY_Emotion}

\paragraph{Ref. By} \iref{IM_DecayEmotion}, \rref{R_GetEmotionState}
\\\hrule\vspace{0.5mm}\hrule

~\newline

\noindent
\begin{minipage}{\textwidth}
    \renewcommand*{\arraystretch}{1.5}
    \begin{tabular}{| p{\colAwidth}  p{\colBwidth}|}
        \hline
        \rowcolor[gray]{0.9}
        \bf IM\refstepcounter{instnum}\theinstnum
        \label{IM_DecayEmotion} &
        \bf Decaying Emotion \\
        \hline
    \end{tabular}

    \renewcommand*{\arraystretch}{1.5}
    \begin{tabular}{ p{\colAwidth}  p{\colBwidth}}
        \bf Input & $e : \emotiontype$, $\left\{ t : \timetype, t' : \timetype
        \rightarrow t < t' \right\}$, $c : \mathbb{R}_{>0}$, , $es_\lambda :
        \emotionstatedecaytype$ \\

        \bf Output & $ es : \emotionstatetype \defEq \left\{ \, \forall k
        \rightarrow es \text{ with } es.\mathtt{intensities}\left(k\right) = D
        \, \right\} $ from \iref{IM_UpdateEmotionState2} \\

        & where $D = \mathtt{decay}\left(\,
        es_0.\mathtt{intensities}\left(k\right), t' - t, c, E \, \right) $
        from \iref{IM_DecayEmotionState} \\
        & \hspace*{3mm} where $es_0 = e\left(t'\right)$ from
        \iref{IM_GetEmotionState} \\
        & and $E = \left\{es_\lambda.\mathtt{decayRates}\left(k\right),
        es_\lambda.\mathtt{equilibrium}\left(k\right) \right\}$ \\
        \hline
    \end{tabular}
\end{minipage}

\paragraph{Description} Given an emotion $e : \emotiontype$
(\tyref{TY_Emotion}), two times $t', t : \timetype$ (\tyref{TY_Time}), a
damping coefficient $c : \mathbb{R}_{>0}$, and emotion state decay values
$es_\lambda : \emotionstatedecaytype$ (\tyref{TY_EmotionDecayState}), create a
new emotion state $es : \emotionstatetype$ (\tyref{TY_EmotionState}) by
decaying intensity for every $k : \emotionkindstype$ (\tyref{TY_EmotionKind})
in $e\left(t'\right).\mathtt{intensities}$. \\\hrule

\paragraph{Sources} --

\paragraph{Depends On} \tyref{TY_Time}, \tyref{TY_EmotionKind},
\tyref{TY_EmotionState}, \tyref{TY_EmotionDecayState}, \tyref{TY_Emotion},
\iref{IM_DecayEmotionState}, \iref{IM_UpdateEmotionState2},
\iref{IM_GetEmotionState}

\paragraph{Ref. By} \rref{R_DecayEmotion} \\\hrule\vspace{0.5mm}\hrule

~\newline

\noindent
\begin{minipage}{\textwidth}
    \renewcommand*{\arraystretch}{1.5}
    \begin{tabular}{| p{\colAwidth}  p{\colBwidth}|}
        \hline
        \rowcolor[gray]{0.9}
        \bf IM\refstepcounter{instnum}\theinstnum
        \label{IM_GetEmotionStatePAD} &
        \bf Getting an Emotion State as a PAD Point \\
        \hline
    \end{tabular}

    \renewcommand*{\arraystretch}{1.5}
    \begin{tabular}{ p{\colAwidth}  p{\colBwidth}}
        \bf Input & $es : \emotionstatetype$ \\

        \bf Output & $ p : \padpoint \defEq \mathtt{clamp} \left(
        {\mathlarger\sum_{k \in \emotionkindstype}} v\left(k\right) \cdot
        \dfrac{es.\mathtt{intensities}\left(k\right)}{es.\mathtt{max}\left(k\right)},
         -1, 1 \right) $
        \vspace*{2mm}\\

        & where $v\left(k\right) : \padpoint = \begin{cases}
            \left\{ -0.62, +0.82, -0.43 \right\}, & k = \mFear \\
            \left\{ -0.51, +0.59, +0.25 \right\}, & k = \mAnger \\
            \left\{ -0.63, -0.27, -0.33 \right\}, & k = \mSadness \\
            \left\{ +0.76, +0.48, +0.35 \right\}, & k = \mJoy \\
            \left\{ +0.64, +0.51, +0.17 \right\}, & k = \mInterest \\
            \left\{ +0.16, +0.88, -0.15 \right\}, & k = \mSurprise \\
            \left\{ -0.60, +0.35, +0.11 \right\}, & k = \mDisgust \\
            \left\{ +0.64, +0.35, +0.24 \right\}, & k = \mTrust \\
        \end{cases}$
        \vspace*{2mm}\\
        \hline
    \end{tabular}
\end{minipage}

\paragraph{Description} Given an emotion state $es : \emotionstatetype$
(\tyref{TY_EmotionState}), output a point in PAD space $p : \padpoint$
(\tyref{TY_PAD}) by summing the PAD point value $v\left(k\right) : \padpoint$
multiplied by the intensity relative to the maximum possible intensity
$es.\mathtt{max}$ for each emotion $k : \emotionkindstype$
(\tyref{TY_EmotionKind}). The purpose of making
$es.\mathtt{intensities}\left(k\right)$ relative to $es.\mathtt{max}$ is so
that higher intensity emotions weigh more than lower intensity emotions,
contributing more to the overall PAD value.

Values for $v\left(k\right)$ correspond to the statistical means of PAD terms
for \textit{pleasure}, \textit{arousal}, and \textit{dominance}
(\tref{T_GetEmotionStatePAD}). Each element in $p$ is clamped to the range
$\left[-1, 1\right]$ due to \tyref{TY_PAD}. \\\hrule

\paragraph{Sources} \citet[p.~42--45]{mehrabian1980basic}

\paragraph{Depends On} \tref{T_GetEmotionStatePAD}, \tyref{TY_EmotionKind},
\tyref{TY_EmotionState}, \tyref{TY_PAD}

\paragraph{Ref. By} \rref{R_Convert2PAD} \\\hrule\vspace{0.5mm}\hrule