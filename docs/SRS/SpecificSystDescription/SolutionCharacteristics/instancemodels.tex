\subsubsection{Instance Models} \label{sec_instance}
Instance Models refine the Theoretical Models (Section~\ref{sec_theoretical})
into mathematical representations using Assumptions
(Section~\ref{sec_assumptions}) and Data Types (Section~\ref{sec_typedefs}).
These models address \progname{}'s Goals (Section~\ref{sec_goals}):
\begin{itemize}
    \item \gsref{G_EmotionElicitation} is addressed by \iref{IM_ElicitJoy},
    \iref{IM_ElicitSadness}, \iref{IM_ElicitFear}, \iref{IM_ElicitAnger},
    \iref{IM_ElicitDisgust}, \iref{IM_CalculateEmotionAcceptanceElicit},
    \iref{IM_CalculateEmotionInterestElicit}, and
    \iref{IM_CalculateEmotionSurpriseElicit}

    \item \gsref{G_EmotionIntensity} is addressed by \iref{IM_JoyIntensity},
    \iref{IM_SadnessIntensity}, \iref{IM_FearIntensity},
    \iref{IM_AngerIntensity}, \iref{IM_DisgustIntensity},
    \iref{IM_CalculateEmotionAcceptance}, \iref{IM_CalculateEmotionInterest},
    and \iref{IM_CalculateEmotionSurprise}

    \item \gsref{G_EmotionDecay} is addressed by
    \iref{IM_DecayEmotionIntensity}, \iref{IM_DecayEmotionState}, and
    \iref{IM_GetNextEmotionByDecay}

    \item \gsref{G_UpdateEmotionState} is addressed by
    \iref{IM_UpdateEmotionState}, \iref{IM_UpdateEmotionState2}, and
    \iref{IM_UpdateEmotion}

    \item \gsref{G_QueryEmotionState} is addressed by \iref{IM_GetEmotionState}
    and \iref{IM_GetEmotionStatePAD}
\end{itemize}

~\newline\noindent
\begin{minipage}{\textwidth}
    \renewcommand*{\arraystretch}{1.5}
    \begin{tabular}{| p{\colAwidth}  p{\colBwidth}|}
        \hline
        \rowcolor[gray]{0.9}
        \bf IM\refstepcounter{instnum}\theinstnum
        \label{IM_ElicitJoy} &
        \bf Evaluate \textit{Joy} Elicitation \\
        \hline
    \end{tabular}

    \renewcommand*{\arraystretch}{1.5}
    \begin{tabular}{ p{\colAwidth}  p{\colBwidth}}
        \bf Input & $g : \goaltype, s_{prev} : \worldstatetype, s_\Delta :
        \worldstatechangetype, \epsilon_{J} : \statedistancechangetype$
        \vspace*{2mm}\\

        \bf Output & $ (\mathit{dist}_{prev} : \statedistancetype,
        \mathit{dist}_{now} : \statedistancetype, \mathit{dist}_\Delta :
        \statedistancechangetype)^? \defEq J$ \vspace*{2mm}\\

        & where $J = \begin{cases}
            \parbox{0.25\linewidth}{$(g.\mathtt{goal}(s_{prev}), \\
                g.\mathtt{goal}(s_{prev} \oplus s_\Delta), \\
                g.\mathtt{goal'}(s_{prev}, s_\Delta))$, } &
            \parbox{0.38\columnwidth}{$g.\mathtt{goal}(s_{prev}) >
            g.\mathtt{goal}(s_{prev} \oplus s_\Delta) \\
              \land | \, g.\mathtt{goal'}(s_{prev}, s_\Delta) \, | >
              \epsilon_{J} $} \\[20pt]
            \text{None}, & Otherwise \\
        \end{cases}$ \vspace*{1mm}\\
        \hline
    \end{tabular}
\end{minipage}

\paragraph{Description} Given an entity goal (\tyref{TY_Goal}), the
\textit{previous WSV} (\tyref{TY_WorldState}), an event that transformed the
\textit{previous WSV} into the current one (\tyref{TY_WorldStateChange}), and a
``tolerance'' threshold for distance changes between WSVs
(\tyref{TY_DistanceBetweenWorldStatesChange}), evaluate the elicitation of
\textit{Joy} (\tref{T_CalculateEmotionGP}) by determining if the event
progresses the entity \textit{towards} goal achievement by evaluating if the
distance to the goal state (\tyref{TY_DistanceBetweenWorldStates}) is larger in
the \textit{previous WSV} unchanged by the event compared to the WSV changed by
the event ($g.\mathtt{goal}(s_{prev}) > g.\mathtt{goal}(s_{prev} \oplus
s_\Delta)$) such that it causes a noticeable change in distance to a goal from
a WSV by evaluating if its magnitude exceeds a minimum ``threshold''
($| \, g.\mathtt{goal'}(s_{prev}, s_\Delta) \, | > \epsilon_{J}$).

If the condition fails, the function returns None because the event did not
elicit \textit{Joy} for this goal.

The threshold $\epsilon_{J}$ controls the entity's ``sensitivity'' to changes
such it experiences \textit{Joy} more easily with lower threshold values
compared to high ones. \\\hrule

\paragraph{Sources} --

\paragraph{Depends On} \tref{T_CalculateEmotionGP}, \tyref{TY_WorldState},
\tyref{TY_WorldStateChange}, \tyref{TY_DistanceBetweenWorldStates},
\tyref{TY_DistanceBetweenWorldStatesChange}, \tyref{TY_Goal}

\paragraph{Ref. By} \iref{IM_CalculateEmotionAcceptanceElicit},
\rref{R_GenerateEmotionCTE} \\\hrule\vspace{0.5mm}\hrule

~\newline

\noindent
\begin{minipage}{\textwidth}
    \renewcommand*{\arraystretch}{1.5}
    \begin{tabular}{| p{\colAwidth}  p{\colBwidth}|}
        \hline
        \rowcolor[gray]{0.9}
        \bf IM\refstepcounter{instnum}\theinstnum
        \label{IM_ElicitSadness} &
        \bf Evaluate \textit{Sadness} Elicitation \\
        \hline
    \end{tabular}

    \renewcommand*{\arraystretch}{1.5}
    \begin{tabular}{ p{\colAwidth}  p{\colBwidth}}
        \bf Input & $g : \goaltype^?, p : \plantype^?, s_{prev} :
        \worldstatetype, s_\Delta : \worldstatechangetype$ \vspace*{2mm}\\

        \bf Output & $( g_{sadness} : \goaltype^?,
        p_{sadness} : \plantype^?, s_{now} : \worldstatetype,
        \mathit{dist}_{now} : \statedistancetype^? )^? \defEq S$ \vspace*{2mm}\\

        & $S = \begin{cases}
            (\text{None}, p, s_{prev} \oplus s_\Delta, \text{None}), &
            \parbox{0.54\linewidth}{$p \neq \text{None} \land
                p.\mathtt{isFeasible}(s_{prev}) \\
                \land \neg p.\mathtt{isFeasible}(s_{prev} \oplus s_\Delta)$}
            \\[10pt]

            \parbox{0.22\linewidth}{$(g, \text{None}, s_{prev} \oplus s_\Delta,
                \\
                g.\mathtt{goal}(s_{prev} \oplus s_\Delta)),$} &
            g \neq \text{None} \land | \, g.\mathtt{goal}(s_{prev} \oplus
            s_\Delta) \, | \, = +\infty \\[10pt]

            \text{None}, & Otherwise \\
        \end{cases}$ \vspace*{1mm}\\
        \hline
    \end{tabular}
\end{minipage}

\paragraph{Description} Given either an entity goal (\tyref{TY_Goal}) or plan
(\tyref{TY_Plan}), the \textit{previous WSV} (\tyref{TY_WorldState}), and an
event that transformed the \textit{previous WSV} into the current one
(\tyref{TY_WorldStateChange}), evaluate the elicitation of \textit{Sadness}
(\tref{T_CalculateEmotionGP}) by:
\begin{itemize}

    \item If given a plan ($p \neq \text{None}$), determine if the plan was
    feasible in the \textit{previous WSV} and is not feasible in the current
    WSV created by applying the event to the previous WSV
    ($p.\mathtt{isFeasible}(s_{prev}) \land \neg p.\mathtt{isFeasible}(s_{prev}
    \oplus s_\Delta)$).

    \item If given a goal ($g \neq \text{None}$), determine if the event
    created a WSV from the \textit{previous WSV} where the distance to the goal
    state (\tyref{TY_DistanceBetweenWorldStates}) is infinitely large in WSV
    changed by the event (i.e. is unachievable, $| \, g.\mathtt{goal}(s_{prev}
    \oplus s_\Delta) \, | \, = +\infty$).

\end{itemize}

If both conditions fail, the function returns None because the event did not
elicit \textit{Sadness} for this goal. \\\hrule

\paragraph{Sources} --

\paragraph{Depends On} \tref{T_CalculateEmotionGP}, \tyref{TY_WorldState},
\tyref{TY_WorldStateChange}, \tyref{TY_DistanceBetweenWorldStates},
\tyref{TY_Goal}, \tyref{TY_Plan}

\paragraph{Ref. By} \rref{R_GenerateEmotionCTE} \\\hrule\vspace{0.5mm}\hrule

~\newline

\noindent
\begin{minipage}{\textwidth}
    \renewcommand*{\arraystretch}{1.5}
    \begin{tabular}{| p{\colAwidth}  p{\colBwidth}|}
        \hline
        \rowcolor[gray]{0.9}
        \bf IM\refstepcounter{instnum}\theinstnum
        \label{IM_ElicitFear} &
        \bf Evaluate \textit{Fear} Elicitation \\
        \hline
    \end{tabular}

    \renewcommand*{\arraystretch}{1.5}
    \begin{tabular}{ p{\colAwidth}  p{\colBwidth}}
        \bf Input & $g : \goaltype, g' : \goaltype^?, s_{now} : \worldstatetype,
        s_\Delta : \worldstatechangetype, \epsilon_{F} :
        \statedistancechangetype$ \vspace*{2mm}\\

        \bf Output & $( g_{fear} : \goaltype, \mathit{dist}_{now} :
        \statedistancetype, \mathit{dist}_{next} : \statedistancetype,
        \mathit{dist}_\Delta : \statedistancechangetype, g_{lost} : \goaltype^?
        )^? \defEq F$ \vspace*{2mm}\\

        & where $\mathit{F} = \begin{cases}
            \parbox{0.25\linewidth}{$(g, g.\mathtt{goal}(s_{now}), \\
                g.\mathtt{goal}(s_{now} \oplus s_\Delta), \\
                g.\mathtt{goal'}(s_{now}, s_\Delta), \\
                \text{None})$,} &
            \parbox{0.46\linewidth}{$\mathtt{SelfPreservation} \in
                g.\mathtt{type} \\
                \land \neg(g.\mathtt{goal}(s_{now}) > g.\mathtt{goal}(s_{now}
                \oplus s_\Delta)) \\
                \land | \, g.\mathtt{goal'}(s_{now}, s_\Delta) \, | >
                \epsilon_{F}$} \\[25pt]

            \parbox{0.25\linewidth}{$(g, g.\mathtt{goal}(s_{now}), \\
                g.\mathtt{goal}(s_{now} \oplus s_\Delta), g')$,} &
            \parbox{0.46\linewidth}{$g' \neq \text{None} \\
                \land \mathtt{WillConflict}(g, g', s_{now}, s_\Delta)$} \\[15pt]

            \parbox{0.25\linewidth}{$(g', g'.\mathtt{goal}(s_{now}), \\
                g'.\mathtt{goal}(s_{now} \oplus s_\Delta), g)$,} &
            \parbox{0.46\linewidth}{$g' \neq \text{None} \\
                \land \mathtt{WillConflict}(g', g, s_{now}, s_\Delta)$} \\[10pt]

            \text{None}, & Otherwise \\

        \end{cases}$ \vspace*{1em}\newline
        $\text{where } \mathtt{WillConflict}(g_1, g_2, s_{now}, s_\Delta)$
        \newline
        $= (g_1.\mathtt{goal}(s_{now}) > g_1.\mathtt{goal}(s_{now} \oplus
        s_\Delta)) \land | \, g_2.\mathtt{goal}(s_{now} \oplus s_\Delta) \, |
        \, = +\infty$ \vspace*{1mm}\\
        \hline
    \end{tabular}
\end{minipage}

\paragraph{Description} Given at least one of two entity goals
(\tyref{TY_Goal}), the \textit{current WSV} (\tyref{TY_WorldState}), an event
that will transform the \textit{current WSV} into a future one
(\tyref{TY_WorldStateChange}), and a ``tolerance'' threshold for distance
changes between WSVs (\tyref{TY_DistanceBetweenWorldStatesChange}), evaluate
the elicitation of \textit{Fear} (\tref{T_CalculateEmotionGP}) by:
\begin{itemize}

    \item Determining if the entity perceives a threat to its goal $g$ by
    evaluating if it concerns self-preservation ($\mathtt{SelfPreservation} \in
    g.\mathtt{type}$) and there is a potential event that progresses an entity
    \textit{away} from goal achievement by increasing the distance to the goal
    state compared to the \textit{current WSV} ($\neg(g.\mathtt{goal}(s_{now})$
    $> g.\mathtt{goal}(s_{now} \oplus s_\Delta))$) such that it causes a
    noticeable change from an evaluation of its minimum ``threshold''
    ($| \, g.\mathtt{goal'}(s_{now}, s_\Delta) \, | > \epsilon_{F}$).

    \item If given two goals ($g' \neq \text{None}$), determining if they
    conflict by evaluating if the potential event progresses one while
    simultaneously making the distance to the other
    (\tyref{TY_DistanceBetweenWorldStates}) infinitely large
    ($\mathtt{WillCon-}$ $\mathtt{flict}$ $(g_1, g_2, s_{now}, s_\Delta)$).

\end{itemize}

If all conditions fail, the function returns None because the event did not
elicit \textit{Fear} for these goals.

The threshold $\epsilon_{F}$ controls the entity's ``sensitivity'' to changes
such it experiences \textit{Fear} more easily with lower threshold values
compared to high ones. \\\hrule

\paragraph{Sources} --

\paragraph{Depends On} \tref{T_CalculateEmotionGP}, \tyref{TY_WorldState},
\tyref{TY_WorldStateChange}, \tyref{TY_DistanceBetweenWorldStates},
\tyref{TY_DistanceBetweenWorldStatesChange}, \tyref{TY_Goal}

\paragraph{Ref. By} \rref{R_GenerateEmotionCTE} \\\hrule\vspace{0.5mm}\hrule

~\newline

\noindent
\begin{minipage}{\textwidth}
    \renewcommand*{\arraystretch}{1.5}
    \begin{tabular}{| p{\colAwidth}  p{\colBwidth}|}
        \hline
        \rowcolor[gray]{0.9}
        \bf IM\refstepcounter{instnum}\theinstnum
        \label{IM_ElicitAnger} &
        \bf Evaluate \textit{Anger} Elicitation \\
        \hline
    \end{tabular}

    \renewcommand*{\arraystretch}{1.5}
    \begin{tabular}{ p{\colAwidth}  p{\colBwidth}}
        \bf Input & $s_{prev} : \worldstatetype, s_\Delta :
        \worldstatechangetype, ps : \{\plantype\}$ \vspace*{2mm}\\

        \bf Output & $( s_{now} :
        \worldstatetype, p_{fail} : \plantype, ps_{alt} : \{\plantype\} )^?
        \defEq A$ \vspace*{2mm}\\

        & where $\mathit{A} = \begin{cases}
            \parbox{0.35\linewidth}{$(s_{prev} \oplus s_\Delta, p_\alpha,
                \forall p \in \{\plantype\} \\ \rightarrow
                p.\mathtt{isFeasible}(s_{prev} \oplus s_\Delta) $,} &
            \parbox{0.45\linewidth}{$\exists p_\alpha \in ps \rightarrow
                (\forall p \in ps \\
                \text{   }\rightarrow p \neq p_\alpha \land
                \mathtt{Cost}(p_\alpha)
                \leq \mathtt{Cost}(p)) \\
                \land \neg p_\alpha.\mathtt{isFeasible}(s_{prev} \oplus
                s_\Delta) \\
                \land \exists p \in ps \rightarrow
                p.\mathtt{isFeasible}(s_{prev}
                \oplus s_\Delta)$} \\[25pt]

            \text{None}, & Otherwise \\
        \end{cases}$ \vspace*{1mm}\\
        \hline
    \end{tabular}
\end{minipage}

\paragraph{Description} Given the previous WSV (\tyref{TY_WorldState}), and
event that transformed the \textit{previous WSV} into the current one
(\tyref{TY_WorldStateChange}), and a set of entity plans (\tyref{TY_Plan}),
evaluate the elicitation of \textit{Anger} (\tref{T_CalculateEmotionGP}) by
determining if the transition from the \textit{previous WSV} into the current
one makes the entity's lowest effort plan ($\exists p_\alpha \in ps
\rightarrow (\forall p \in ps \rightarrow p \neq p_\alpha \land
\mathtt{Cost}(p_\alpha) \leq \mathtt{Cost}(p)$, \aref{A_CostFunction})
impossible to progress ($\neg p_\alpha.\mathtt{isFeasible}(s_{prev} \oplus
s_\Delta)$), but there is at least one other plan for achieving the same
end-state ($\exists p \in ps \rightarrow p.\mathtt{isFeasible}(s_{prev} \oplus
s_\Delta)$). Therefore, the entity can continue working towards a desired
end-state but must use a plan that requires more effort (``frustrated'') and
the entity experiences \textit{Anger}.

If the condition fails, the function returns None because the event did not
elicit \textit{Anger} for this goal.

Note that the set of plans that the model returns is a strict subset of the
provided set of plans $ps_f \subset ps$ because it has at least one plan fewer
due to the infeasibility of $p_\alpha$. \\\hrule

\paragraph{Sources} --

\paragraph{Depends On} \aref{A_CostFunction}, \tref{T_CalculateEmotionGP},
\tyref{TY_WorldState}, \tyref{TY_WorldStateChange}, \tyref{TY_Plan}

\paragraph{Ref. By} \rref{R_GenerateEmotionCTE} \\\hrule\vspace{0.5mm}\hrule

~\newline

\noindent
\begin{minipage}{\textwidth}
    \renewcommand*{\arraystretch}{1.5}
    \begin{tabular}{| p{\colAwidth}  p{\colBwidth}|}
        \hline
        \rowcolor[gray]{0.9}
        \bf IM\refstepcounter{instnum}\theinstnum
        \label{IM_ElicitDisgust} &
        \bf Evaluate \textit{Disgust} Elicitation \\
        \hline
    \end{tabular}

    \renewcommand*{\arraystretch}{1.5}
    \begin{tabular}{ p{\colAwidth}  p{\colBwidth}}
        \bf Input & $g : \goaltype, s_{prev} : \worldstatetype, s_\Delta :
        \worldstatechangetype, \epsilon_{DS} : \statedistancechangetype,
        \epsilon_{DN} : \statedistancechangetype$ \vspace*{2mm}\\

        \bf Output & $(\mathit{dist}_{prev} : \statedistancetype,
        \mathit{dist}_{now} : \statedistancetype, \mathit{dist}_\Delta :
        \statedistancechangetype)^? \defEq D$ \vspace*{2mm}\\

        & where $\mathit{D} = \begin{cases}
            \parbox{0.25\linewidth}{$(g.\mathtt{goal}(s_{prev}), \\
                g.\mathtt{goal}(s_{prev} \oplus s_\Delta), \\
                g.\mathtt{goal'}(s_{prev}, s_\Delta))$,} &
            \parbox{0.4\linewidth}{$\mathtt{Gustatory} \in
                g.\mathtt{type} \\
                \land g.\mathtt{goal}(s_{prev}) \leq \epsilon_{DS} \\
                \land g.\mathtt{goal}(s_{prev} \oplus s_\Delta) > \epsilon_{DS}
                \\
                \land | \, g.\mathtt{goal'}(s_{prev}, s_\Delta) \, | >
                \epsilon_{DN}) $} \\[25pt]

            \text{None}, & Otherwise \\
        \end{cases}$ \vspace*{1mm}\\
        \hline
    \end{tabular}
\end{minipage}

\paragraph{Description} Given an entity goal (\tyref{TY_Goal}), the
\textit{previous WSV} (\tyref{TY_WorldState}), an event that will transform the
\textit{previous WSV} into the current one (\tyref{TY_WorldStateChange}), and
two ``tolerance'' thresholds for distance changes between WSVs
(\tyref{TY_DistanceBetweenWorldStatesChange}), evaluate the elicitation of
\textit{Disgust} (\tref{T_CalculateEmotionGP}) by determining if the goal is
gustatory-related ($\mathtt{Gustatory} \in g.\mathtt{type}$), \textit{the
previous WSV} satisfied that goal within some ``satisfaction threshold''
($g.\mathtt{goal}$ $(s_{prev}) \leq \epsilon_{DS}$), and the event transitioned
into the current WSV where the goal is unsatisfied ($g.\mathtt{goal}(s_{prev}
\oplus s_\Delta) > \epsilon_{DS}$) such that the difference is noticeable
($| \, g.\mathtt{goal'}(s_{prev}, s_\Delta) \, | > \epsilon_{DN}$).

The threshold $\epsilon_{DS}$ defines an entity's ``tolerance'' for goal
dissatisfaction such that higher values means that the entity allows larger
distances between the current WSV and its goal state before experiencing
\textit{Disgust}. The threshold $\epsilon_{DN}$ controls the entity's
``sensitivity'' to WSV changes such it experiences \textit{Disgust} more easily
with lower threshold values compared to high ones. \\\hrule

\paragraph{Sources} --

\paragraph{Depends On} \tref{T_CalculateEmotionGP}, \tyref{TY_WorldState},
\tyref{TY_WorldStateChange}, \tyref{TY_DistanceBetweenWorldStatesChange},
\tyref{TY_Goal}

\paragraph{Ref. By} \rref{R_GenerateEmotionCTE} \\\hrule\vspace{0.5mm}\hrule

~\newline

\noindent
\begin{minipage}{\textwidth}
    \renewcommand*{\arraystretch}{1.5}
    \begin{tabular}{| p{\colAwidth}  p{\colBwidth}|}
        \hline
        \rowcolor[gray]{0.9}
        \bf IM\refstepcounter{instnum}\theinstnum
        \label{IM_CalculateEmotionAcceptanceElicit} &
        \bf Evaluate \textit{Acceptance} Elicitation \\
        \hline
    \end{tabular}

    \renewcommand*{\arraystretch}{1.5}
    \begin{tabular}{ p{\colAwidth}  p{\colBwidth}}
        \bf Input & $r_A : {\socialattachmenttype}^?, g : \goaltype, s_{prev} :
        \worldstatetype, s_\Delta : \worldstatechangetype, \epsilon_{A1} :
        \worldstatechangetype, \epsilon_{A2} : \worldstatechangetype$ \\

        \bf Output & $ (r_A : \socialattachmenttype,
        \mathit{distAttribToA}_\Delta : \statedistancechangetype)^? \defEq
        \mathit{Acc}$ \vspace*{2mm}\\

        & where $\mathit{Acc} = \begin{cases}

            (r_A, dist_\Delta - \epsilon_{A2}), &
            \parbox{0.53\linewidth}{$r_A \neq \text{None} \, \land \, | \, J(g,
                s_{prev}, s_\Delta, \epsilon_{A1}).dist_\Delta \, | >
                \epsilon_{A2} \\
                \land \mathtt{CausedBy}(s_\Delta, A)$} \\[10pt]

            \text{None}, & Otherwise \\

        \end{cases} $ \vspace*{1mm}\\
        \hline
    \end{tabular}
\end{minipage}

\paragraph{Description} Given an Option type that might contain the entity's
social attachment to another (\tyref{TY_Relation-CTE}), one of the entity's
goals (\tyref{TY_Goal}), the \textit{previous WSV} (\tyref{TY_WorldState}), an
event that will transform the \textit{previous WSV} into the current one
(\tyref{TY_WorldStateChange}), and two ``tolerance'' thresholds for distance
changes between WSVs (\tyref{TY_DistanceBetweenWorldStatesChange}), evaluate
the elicitation of \textit{Acceptance} (\tref{T_CalculateEmotionAcceptance})
towards some entity $A$ by determining if: the entity has a social attachment
to $A$ ($r_A \neq \text{None}$); the goal, WSV, event, and a tolerance
``threshold'' elicit \textit{Joy} ($J(g, s_{prev}, s_\Delta, \epsilon_{A1})$
from \iref{IM_ElicitJoy}) and the change in distance returned from that
evaluation exceeds a minimum threshold ($| \, dist_\Delta \, | >
\epsilon_{A2}$); and the entity attributes the occurrence of the event to $A$
($\mathtt{CausedBy}(s_\Delta, A)$, \aref{A_CausedByFunction}).

If these conditions fail, the function returns None because the event did not
elicit \textit{Acceptance} for this goal and entity $A$.

The threshold $\epsilon_{A2}$ controls the entity's ``sensitivity'' to changes
such it experiences \textit{Acceptance} more easily with lower threshold values
compared to high ones. This threshold also moderates the elicitation
``magnitude'' by returning the change in distance between WSVs that exceeds it
($dist_\Delta - \epsilon_{A2}$) so that the ``magnitude'' is relative to how
easily ``impressed'' the entity is.

If an entity has no social attachment yet, users could use this as a mechanism
for establishing one:
\begin{enumerate}[noitemsep, nosep]
    \item Create a new social attachment $r_A' : \socialattachmenttype$
    \item Use it to evaluate the presence of \textit{Acceptance}:
    \begin{enumerate}[noitemsep]
        \item If \textit{Acceptance} is present, store $r_A'$
        \item Otherwise discard it
    \end{enumerate}
\end{enumerate} \hrule

\paragraph{Sources} --

\paragraph{Depends On} \aref{A_CausedByFunction},
\tref{T_CalculateEmotionAcceptance}, \tyref{TY_WorldState},
\tyref{TY_WorldStateChange}, \tyref{TY_DistanceBetweenWorldStatesChange},
\tyref{TY_Goal}, \tyref{TY_Relation-CTE}, \iref{IM_ElicitJoy}

\paragraph{Ref. By} \rref{R_GenerateEmotionCTE} \\\hrule\vspace{0.5mm}\hrule

~\newline

\noindent
\begin{minipage}{\textwidth}
    \renewcommand*{\arraystretch}{1.5}
    \begin{tabular}{| p{\colAwidth}  p{\colBwidth}|}
        \hline
        \rowcolor[gray]{0.9}
        \bf IM\refstepcounter{instnum}\theinstnum
        \label{IM_CalculateEmotionInterestElicit} &
        \bf Evaluate \textit{Interest} Elicitation \\
        \hline
    \end{tabular}

    \renewcommand*{\arraystretch}{1.5}
    \begin{tabular}{ p{\colAwidth}  p{\colBwidth}}
        \bf Input & $ \mathit{at}_x : \attentiontype, \epsilon_{Inr} :
        \attentiontype $ \\

        \bf Output & $ \attentiontype^? \defEq \mathit{Inr}$ \\

        & where $ \mathit{Inr} = \begin{cases}

            \mathit{at}_x - \epsilon_{Inr}, & at > \epsilon_{Inr} \\

            \text{None}, & Otherwise \\

        \end{cases} $ \vspace*{1mm}\\
        \hline
    \end{tabular}
\end{minipage}

\paragraph{Description} Given the attention paid to an entity $x$
(\tyref{TY_Attention}) and a ``tolerance'' threshold, evaluate the elicitation
of \textit{Interest} (\tref{T_CalculateEmotionInterest}) by determining if the
amount of attention paid to $x$ exceeds the threshold ($at > \epsilon_{Inr}$).

If this condition fails, the function returns None because the entity is not
experiencing \textit{Interest} towards $x$.

The threshold also moderates the elicitation ``magnitude'' by returning the
amount of attention that exceeds it ($\mathit{at}_x - \epsilon_{Inr}$) so that
the ``magnitude'' is relative to how much $x$ ``fascinates'' the entity.
\\\hrule

\paragraph{Sources} --

\paragraph{Depends On} \tref{T_CalculateEmotionInterest}, \tyref{TY_Attention}

\paragraph{Ref. By} \rref{R_GenerateEmotionCTE} \\\hrule\vspace{0.5mm}\hrule

~\newline

\noindent
\begin{minipage}{\textwidth}
    \renewcommand*{\arraystretch}{1.5}
    \begin{tabular}{| p{\colAwidth}  p{\colBwidth}|}
        \hline
        \rowcolor[gray]{0.9}
        \bf IM\refstepcounter{instnum}\theinstnum
        \label{IM_CalculateEmotionSurpriseElicit} &
        \bf Evaluate \textit{Surprise} Elicitation \\
        \hline
    \end{tabular}

    \renewcommand*{\arraystretch}{1.5}
    \begin{tabular}{ p{\colAwidth}  p{\colBwidth}}
        \bf Input & $ s_{prev} : \worldstatetype, s_\Delta :
        \worldstatechangetype, \epsilon_P : [0,1] $ \\

        \bf Output & $ [0,1]^? \defEq \mathit{Sur}$ \\

        & where $\mathit{Sur} = \begin{cases}

            \epsilon_P - P(s_\Delta | s_{prev}), & P(s_\Delta | s_{prev}) <
            \epsilon_P \\

            \text{None}, & Otherwise \\

        \end{cases} $
        \vspace*{2mm}\\ \hline
    \end{tabular}
\end{minipage}

\paragraph{Description} Given the \textit{previous WSV}
(\tyref{TY_WorldState}), an event that transformed the \textit{previous WSV}
into the current one (\tyref{TY_WorldStateChange}), and a ``tolerance''
threshold, evaluate the elicitation of \textit{Surprise}
(\tref{T_CalculateEmotionSurprise}) by determining if the improbability of the
event in \textit{the previous WSV} is below the threshold ($P(s_\Delta |
s_{prev}) < \epsilon_P$, \aref{A_EventProbabilityFunction}).

If this condition fails, the function returns None because the event did not
elicit \textit{Surprise} in the entity.

The threshold also moderates the elicitation ``magnitude'' by returning how
much the event's improbability falls below it ($\epsilon_P - P(s_\Delta |
s_{prev})$) so that the ``magnitude'' is relative to how ``impossible'' the
entity believes the event is. \\\hrule

\paragraph{Sources} --

\paragraph{Depends On} \aref{A_EventProbabilityFunction},
\tref{T_CalculateEmotionSurprise}, \tyref{TY_WorldState},
\tyref{TY_WorldStateChange}

\paragraph{Ref. By} \rref{R_GenerateEmotionCTE} \\\hrule\vspace{0.5mm}\hrule

~\newline

\noindent
\begin{minipage}{\textwidth}
    \renewcommand*{\arraystretch}{1.5}
    \begin{tabular}{| p{\colAwidth}  p{\colBwidth}|}
        \hline
        \rowcolor[gray]{0.9}
        \bf IM\refstepcounter{instnum}\theinstnum
        \label{IM_JoyIntensity} &
        \bf Evaluate \textit{Joy} Intensity \\
        \hline
    \end{tabular}

    \renewcommand*{\arraystretch}{1.5}
    \begin{tabular}{ p{\colAwidth}  p{\colBwidth}}
        \bf Input & $ g : \goaltype, d_\Delta : \statedistancechangetype $ \\

        \bf Output & $ i_\Delta : \responsestrength \defEq | \, d_\Delta \, |
        \cdot g.\mathtt{importance} $ \\ \hline
    \end{tabular}
\end{minipage}

\paragraph{Description} Given an entity goal (\tyref{TY_Goal}) and a change in
distance between two WSVs (\tyref{TY_DistanceBetweenWorldStatesChange}),
evaluate the intensity change of \textit{Joy}
(\tref{T_CalculateEmotionIntensity}, \tyref{TY_DeltaIntensity}) by treating the
entity goal as the affected ``internal condition'' whose ``value'' is its
$\mathtt{importance}$ and the magnitude of change in distance caused by the
event is its ``seriousness'' or ``value'' because it measures how much the
event moved the entity towards the desired goal state
(\tref{T_CalculateEmotionGP}).

Using goal $\mathtt{importance}$ as a scaling factor moderates intensity
changes such that its magnitude varies for entities observing the same
$d_\Delta$ whose goals only differ in their $\mathtt{importance}$. Note that
the \textit{Joy} elicitation model outputs a tuple with element
$\mathit{dist}_\Delta$ (\iref{IM_ElicitJoy}), which users can provide as the
input $d_\Delta$. \\\hrule

\paragraph{Sources} --

\paragraph{Depends On} \tref{T_CalculateEmotionGP},
\tref{T_CalculateEmotionIntensity}, \tyref{TY_DeltaIntensity},
\tyref{TY_DistanceBetweenWorldStatesChange}, \tyref{TY_Goal}

\paragraph{Ref. By} \rref{R_CalculateIntensity} \\\hrule\vspace{0.5mm}\hrule

~\newline

\noindent
\begin{minipage}{\textwidth}
    \renewcommand*{\arraystretch}{1.5}
    \begin{tabular}{| p{\colAwidth}  p{\colBwidth}|}
        \hline
        \rowcolor[gray]{0.9}
        \bf IM\refstepcounter{instnum}\theinstnum
        \label{IM_SadnessIntensity} &
        \bf Evaluate \textit{Sadness} Intensity \\
        \hline
    \end{tabular}

    \renewcommand*{\arraystretch}{1.5}
    \begin{tabular}{ p{\colAwidth}  p{\colBwidth}}
        \bf Input & $ g : \goaltype^?, p : \plantype^?, s_{prev} :
        \worldstatetype, i_{\mathit{max}\Delta} : \responsestrength $ \\

        \bf Output & $ i_\Delta : \responsestrength \defEq
        \begin{cases}

            \dfrac{1}{|\mathit{dist}_p|}, & p \neq \text{None} \\[15pt]

            \dfrac{g.\mathtt{importance}}{m_G} \cdot i_{\mathit{max}\Delta}, & g
            \neq \text{None} \\

        \end{cases} $ \vspace*{1em}\newline

        $\text{where } \text{where } \mathit{dist}_p : \statedistancetype =
        \mathtt{Dist}(s_{prev}, p.\mathtt{nextStep}(s_{prev},
        |p.\mathtt{actions}|)) $
        \\ \hline
    \end{tabular}
\end{minipage}

\paragraph{Description} Given either an entity goal (\tyref{TY_Goal}) or plan
(\tyref{TY_Plan}), the \textit{previous WSV} (\tyref{TY_WorldState}), and a
maximum intensity change (\tyref{TY_DeltaIntensity}), evaluate the intensity
change of \textit{Sadness} (\tref{T_CalculateEmotionIntensity}) from two
possible ``internal conditions'' affected by the event that determines the
eliciting event's ``value'' or ``seriousness'' (\tref{T_CalculateEmotionGP}):
\begin{itemize}

    \item An entity plan with a ``value'' equal to the distance to the desired
    end-state before it became infeasible, such that the event's ``value'' or
    ``seriousness'' is inversely proportional to the distance between the
    plan's end-state and the previous WSV where the plan was feasible. This
    means that plans the entity was close to completing elicit more intense
    \textit{Sadness} compared to ones that were farther from completion.

    For evaluating the ``seriousness'' of a plan becoming infeasible, the
    function generates the plan's end-state from the previous WSV by applying
    every plan action to it ($p.\mathtt{nextSteps}(s_{prev},
    |p.\mathtt{actions}|)$), then calculates the distance between the generated
    plan end-state and the previous WSV $s_{prev}$ (\aref{A_DistFunction}).
    This emulates an evaluation of ``how close'' the entity was to plan
    completion.

    \item An entity goal with a ``value'' equal to its $\mathtt{importance}$,
    but the event's ``value'' or ``seriousness'' is not necessarily tied to the
    event---an entity can experience intense \textit{Sadness} if they were
    significantly far from the goal state (e.g. if there is a goal to see a
    loved one before they pass, losing them feels equally painful if one just
    began saving money for a plane ticket or if they have already spent a week
    with them). This means that goals with higher $\mathtt{importance}$ elicit
    more intense \textit{Sadness} compared to less important ones relative to
    some maximum \textit{Sadness} an entity can experience.

    For evaluating the ``seriousness'' of a goal being ``lost'', the goal's
    $\mathtt{importance}$ relative to $m_G$
    (\aref{A_MaxGoalImportanceFunction}) is a scaling factor that moderates a
    maximum intensity change $i_{\mathit{max}\Delta}$ such that its magnitude
    varies for entities with different $\mathtt{importance}$ valuations in
    otherwise identical goals. This effectively normalizing it to $[0,1]$. The
    function can access $m_G$ itself so that the user does not have to provide
    it. Users do provide the value of $i_{\mathit{max}\Delta}$ so that they
    have more control over how much an entity experiences emotion changes (i.e.
    the model does not have to be relative to the maximum possible
    \textit{Sadness} intensity).

\end{itemize}

Note that the \textit{Sadness} elicitation model outputs a tuple with elements
$g_{sadness} : \goaltype^?$ and $p_{sadness} : \plantype^?$
(\iref{IM_ElicitSadness}), which users can supply as the inputs $g$ and $p$.
\\\hrule

\paragraph{Sources} --

\paragraph{Depends On} \aref{A_DistFunction},
\aref{A_MaxGoalImportanceFunction}, \tref{T_CalculateEmotionGP},
\tref{T_CalculateEmotionIntensity}, \tyref{TY_DeltaIntensity},
\tyref{TY_WorldState}, \tyref{TY_Goal}, \tyref{TY_Plan}

\paragraph{Ref. By} \rref{R_CalculateIntensity} \\\hrule\vspace{0.5mm}\hrule

~\newline

\noindent
\begin{minipage}{\textwidth}
    \renewcommand*{\arraystretch}{1.5}
    \begin{tabular}{| p{\colAwidth}  p{\colBwidth}|}
        \hline
        \rowcolor[gray]{0.9}
        \bf IM\refstepcounter{instnum}\theinstnum
        \label{IM_FearIntensity} &
        \bf Evaluate \textit{Fear} Intensity \\
        \hline
    \end{tabular}

    \renewcommand*{\arraystretch}{1.5}
    \begin{tabular}{ p{\colAwidth}  p{\colBwidth}}
        \bf Input & $ g : \goaltype, g_{lost} : \goaltype^?, d_\Delta :
        \statedistancechangetype $ \\

        \bf Output & $ i_\Delta : \responsestrength \defEq \begin{cases}

            d_\Delta \cdot g.\mathtt{importance}, & g_{lost} = \text{None}
            \\[10pt]

            d_\Delta \cdot
            \dfrac{g_{lost}.\mathtt{importance}}{g.\mathtt{importance}}, &
            g_{lost}
            \neq \text{None} \\

        \end{cases} $
        \vspace*{2mm}\\ \hline
    \end{tabular}
\end{minipage}

\paragraph{Description} Given at least one entity goal (\tyref{TY_Goal}) and a
change in distance between two WSVs
(\tyref{TY_DistanceBetweenWorldStatesChange}), evaluate the intensity change of
\textit{Fear} (\tref{T_CalculateEmotionIntensity}, \tyref{TY_DeltaIntensity})
by treating the entity goal as the affected ``internal condition'' whose
``value'' is its $\mathtt{importance}$ and the magnitude of change in distance
caused by the event is its ``seriousness'' or ``value'' because it measures how
much the event will move the entity away the desired goal state or, when there
are two goals, move the entity towards one while making the other unachievable
(\tref{T_CalculateEmotionGP}).

Using goal $\mathtt{importance}$ as a scaling factor moderates intensity
changes such that its magnitude varies for entities with different
$\mathtt{importance}$ valuations in otherwise identical goals that observe the
same $d_\Delta$. In the case where conflicting goals elicited \textit{Fear},
the scaling factor is a ratio between their $\mathtt{importance}$ values such
that the ``value'' of the progressed goal tempers that of the ``lost''
goal---the intensity of \textit{Fear} is higher when the $\mathtt{importance}$
of the ``lost'' goal is larger than the $\mathtt{importance}$ of the other goal.

Note that the \textit{Fear} elicitation model (\iref{IM_ElicitFear}) outputs a
tuple with elements $g_{fear} : \goaltype$, $g_{lost} : \goaltype^?$, and
$\mathit{dist}_\Delta$, which users can use as the inputs $g$, $g_{lost}$, and
$d_\Delta$. \\\hrule

\paragraph{Sources} --

\paragraph{Depends On} \tref{T_CalculateEmotionGP},
\tref{T_CalculateEmotionIntensity}, \tyref{TY_DeltaIntensity},
\tyref{TY_DistanceBetweenWorldStatesChange}, \tyref{TY_Goal}

\paragraph{Ref. By} \rref{R_CalculateIntensity} \\\hrule\vspace{0.5mm}\hrule

~\newline

\noindent
\begin{minipage}{\textwidth}
    \renewcommand*{\arraystretch}{1.5}
    \begin{tabular}{| p{\colAwidth}  p{\colBwidth}|}
        \hline
        \rowcolor[gray]{0.9}
        \bf IM\refstepcounter{instnum}\theinstnum
        \label{IM_AngerIntensity} &
        \bf Evaluate \textit{Anger} Intensity \\
        \hline
    \end{tabular}

    \renewcommand*{\arraystretch}{1.5}
    \begin{tabular}{ p{\colAwidth}  p{\colBwidth}}
        \bf Input & $ p : \plantype, ps : \{\plantype\} $ \\

        \bf Output & $ i_\Delta : \responsestrength \defEq
        \exists p_\beta \in ps \rightarrow $ \newline
        $(\forall p \in ps \rightarrow p \neq p_\beta \land
        \mathtt{Cost}(p_\beta) \leq \mathtt{Cost}(p)) \rightarrow
        \mathtt{Cost}(p_\beta) - \mathtt{Cost}(p)$ \\ \hline
    \end{tabular}
\end{minipage}

\paragraph{Description} Given an infeasible entity plan (\tyref{TY_Plan}) and a
set of feasible plans that achieve the same end-state, evaluate the intensity
change of \textit{Anger} (\tref{T_CalculateEmotionIntensity},
\tyref{TY_DeltaIntensity}) by treating the change in entity plan availability
as the affected ``internal condition'' and their ``value'' is the amount of
effort the entity needs to execute them (i.e. plan ``cost'',
\aref{A_CostFunction}). The difference in ``cost'' between the infeasible plan
and the next lowest ``cost'' plan is its ``seriousness'' or ``value'' because
it measures how much additional effort the entity needs to achieve the same
result. Both of these values are subjective because the entity assigns them.
Therefore, evaluating \textit{Anger} intensity is the difference between the
subjective ``cost'' of the ``frustrated'' plan and the next lowest ``cost''
plan (\tref{T_CalculateEmotionGP}).

Taking the difference between plan ``costs'' ensures that \textit{Anger} is
more intense if the ``cost'' of the next most desirable plan increases compared
to the original one. If these plans are for achieving a goal, users can choose
to scale the resulting \textit{Anger} intensity with the goal's importance
manually.

Note that the \textit{Anger} elicitation model (\iref{IM_ElicitAnger}) outputs
a tuple with elements $p_{fail} : \plantype$ and $ps_{alt} : \{\plantype\}$,
which users can use as the inputs $p$ and $ps$. \\\hrule

\paragraph{Sources} --

\paragraph{Depends On} \aref{A_CostFunction}, \tref{T_CalculateEmotionGP},
\tref{T_CalculateEmotionIntensity}, \tyref{TY_DeltaIntensity}, \tyref{TY_Plan}

\paragraph{Ref. By} \rref{R_CalculateIntensity} \\\hrule\vspace{0.5mm}\hrule

~\newline

\noindent
\begin{minipage}{\textwidth}
    \renewcommand*{\arraystretch}{1.5}
    \begin{tabular}{| p{\colAwidth}  p{\colBwidth}|}
        \hline
        \rowcolor[gray]{0.9}
        \bf IM\refstepcounter{instnum}\theinstnum
        \label{IM_DisgustIntensity} &
        \bf Evaluate \textit{Disgust} Intensity \\
        \hline
    \end{tabular}

    \renewcommand*{\arraystretch}{1.5}
    \begin{tabular}{ p{\colAwidth}  p{\colBwidth}}
        \bf Input & $ g : \goaltype, d : \statedistancetype $ \\

        \bf Output & $ i_\Delta : \responsestrength \defEq d \cdot
        g.\mathtt{importance} $ \\ \hline
    \end{tabular}
\end{minipage}

\paragraph{Description} Given an entity goal (\tyref{TY_Goal}) and the distance
between the \textit{current WSV} and the goal state
(\tyref{TY_DistanceBetweenWorldStates}), evaluate the intensity change of
\textit{Disgust} (\tref{T_CalculateEmotionIntensity},
\tyref{TY_DeltaIntensity}) by treating the the
entity goal as the affected ``internal condition'' and its ``value'' is its
$\mathtt{importance}$. The distance between the current state and the desired
goal state is the event's ``seriousness'' or ``value'' because it measures how
much the event moved the entity out of it (\tref{T_CalculateEmotionGP}).

Using goal $\mathtt{importance}$ as a scaling factor moderates intensity
changes such that its magnitude varies for entities observing the same
$d_\Delta$ whose goals only differ in their $\mathtt{importance}$.

Note that the \textit{Disgust} elicitation model (\iref{IM_ElicitDisgust})
outputs a tuple with element $\mathit{dist}_{now}$, which users can supply as
the input $d$. \\\hrule

\paragraph{Sources} --

\paragraph{Depends On} \tref{T_CalculateEmotionGP},
\tref{T_CalculateEmotionIntensity}, \tyref{TY_DeltaIntensity},
\tyref{TY_DistanceBetweenWorldStates}, \tyref{TY_Goal}

\paragraph{Ref. By} \rref{R_CalculateIntensity} \\\hrule\vspace{0.5mm}\hrule

~\newline

\noindent
\begin{minipage}{\textwidth}
    \renewcommand*{\arraystretch}{1.5}
    \begin{tabular}{| p{\colAwidth}  p{\colBwidth}|}
        \hline
        \rowcolor[gray]{0.9}
        \bf IM\refstepcounter{instnum}\theinstnum
        \label{IM_CalculateEmotionAcceptance} &
        \bf Evaluate \textit{Acceptance} Intensity \\
        \hline
    \end{tabular}

    \renewcommand*{\arraystretch}{1.5}
    \begin{tabular}{ p{\colAwidth}  p{\colBwidth}}
        \bf Input & $ r_A : \socialattachmenttype, r_\mathit{min} :
        \socialattachmenttype, d_\Delta : \statedistancechangetype $
        \vspace*{1mm}\\

        \bf Output & $ i_\Delta : \responsestrength \defEq \begin{cases}
            |d_\Delta| \cdot \dfrac{r_A}{r_\mathit{min}}, & r_A <
            r_\mathit{min} \\
            |d_\Delta|, & \mathit{Otherwise}
        \end{cases} $ \vspace*{2mm}\\
        \hline
    \end{tabular}
\end{minipage}

\paragraph{Description} Given the entity's social attachment to another entity
$A$ (\tyref{TY_Relation-CTE}), the minimum social attachment that an entity
must have with $A$ to ``fully'' experience \textit{Acceptance} towards it, and
a change in distance between two WSVs
(\tyref{TY_DistanceBetweenWorldStatesChange}) evaluate the intensity change of
\textit{Acceptance} (\tref{T_CalculateEmotionAcceptance},
\tyref{TY_DeltaIntensity}) by treating the entity's social attachment to $A$ as
the affected ``internal condition'' and the magnitude of change in distance
caused by the event \textit{attributed} to $A$ is its ``seriousness'' or
``value'' because it measures how much $A$ helped moved the entity towards the
desired goal state.

Using $r_\mathit{min}$ to ``normalize'' $r_A$ creates a scaling factor that
moderates intensity changes such its magnitude varies with social attachment
level. Tuning the minimum ``level'' changes the entity's resistance to the
experience of \textit{Acceptance} so that they appear more trustful or
distrustful of other entities. Consequently, the entity's Emotion Intensity
Change value depends on the entity's relationship to the other relative to
their minimum ``trust level''.

Note that the \textit{Acceptance} elicitation model
(\iref{IM_CalculateEmotionAcceptanceElicit}) outputs a tuple with element
$\mathit{distAttribToA}_\Delta$, which users can use as the input $d_\Delta$.
\\\hrule

\paragraph{Sources} --

\paragraph{Depends On} \tref{T_CalculateEmotionAcceptance},
\tyref{TY_DeltaIntensity}, \tyref{TY_DistanceBetweenWorldStatesChange},
\tyref{TY_Relation-CTE}

\paragraph{Ref. By} \rref{R_CalculateIntensity} \\\hrule\vspace{0.5mm}\hrule

~\newline

\noindent
\begin{minipage}{\textwidth}
    \renewcommand*{\arraystretch}{1.5}
    \begin{tabular}{| p{\colAwidth}  p{\colBwidth}|}
        \hline
        \rowcolor[gray]{0.9}
        \bf IM\refstepcounter{instnum}\theinstnum
        \label{IM_CalculateEmotionInterest} &
        \bf Evaluate \textit{Interest} Intensity \\
        \hline
    \end{tabular}

    \renewcommand*{\arraystretch}{1.5}
    \begin{tabular}{ p{\colAwidth}  p{\colBwidth}}
        \bf Input & $ at : \attentiontype, at_{min} : \attentiontype,
        i_{\delta_x} : \responsestrength$ \vspace*{1mm}\\

        \bf Output & $ i_\Delta : \responsestrength \defEq \begin{cases}
            i_{\delta_x} \cdot \dfrac{at}{at_{min}} & at < at_{min} \\
            i_{\delta_x}, & \mathit{Otherwise}
        \end{cases} $ \vspace*{2mm}\\
        \hline
    \end{tabular}
\end{minipage}

\paragraph{Description} Given the entity's elapsed attention invested in $x$
(\tyref{TY_Attention}), the minimum attention that an entity must spend on
$x$ to ``fully'' experience \textit{Interest} towards it, and a subjective
attention ``value'' $i_{\delta_x}$ (\tyref{TY_DeltaIntensity}) representing the
entity's ``fascination'' with $x$ such higher values elicit more intense
\textit{Interest} with smaller changes in attention, evaluate the intensity
change of \textit{Interest} (\tref{T_CalculateEmotionInterest}) by treating the
entity's entity's attention as the affected ``internal condition'' and the
entity-specific minimum attention as the event's ``seriousness'' or ``value''
because the ``event'' driving \textit{Interest} elicitation is not necessarily
the same as ``world events''.

Users can specify different attention ``valuations'' for $i_{\delta_x}$ so that
entities are more ``intrigued'' by some $x$ than others.

Using the minimum attention to ``normalize'' $at$ creates a scaling factor that
moderates intensity changes such its magnitude varies with uninterrupted,
invested attention. Tuning the minimum ``level'' changes the entity's
resistance to the experience of \textit{Interest} so that they appear to be
more or less ``captivated'' by $x$.

Note that the \textit{Interest} elicitation model
(\iref{IM_CalculateEmotionInterestElicit}) outputs a value with type
$\attentiontype$, which users can supply as the input $at$. \\\hrule

\paragraph{Sources} --

\paragraph{Depends On} \tref{T_CalculateEmotionInterest},
\tyref{TY_DeltaIntensity}, \tyref{TY_Attention}

\paragraph{Ref. By} \rref{R_CalculateIntensity} \\\hrule\vspace{0.5mm}\hrule

~\newline

\noindent
\begin{minipage}{\textwidth}
    \renewcommand*{\arraystretch}{1.5}
    \begin{tabular}{| p{\colAwidth}  p{\colBwidth}|}
        \hline
        \rowcolor[gray]{0.9}
        \bf IM\refstepcounter{instnum}\theinstnum
        \label{IM_CalculateEmotionSurprise} &
        \bf Evaluate \textit{Surprise} Intensity \\
        \hline
    \end{tabular}

    \renewcommand*{\arraystretch}{1.5}
    \begin{tabular}{ p{\colAwidth}  p{\colBwidth}}
        \bf Input &  $ \mathit{discr}_{s_\Delta} : [0,1],
        i_{\mathit{max}\Delta} : \responsestrength$ \\

        \bf Output & $ i_\Delta : \responsestrength \defEq i_{\mathit{max}}
        \cdot \mathit{discr}_{s_\Delta} $
        \vspace*{2mm}\\ \hline
    \end{tabular}
\end{minipage}

\paragraph{Description} Given the ``discrepancy'' between the event and its
improbability and a subjective ``unexpectedness value''
$i_{\mathit{max}\Delta}$ that measures how easily an entity is ``startled''
such higher values elicit more intense \textit{Surprise} with smaller event
probability discrepancies, evaluate the intensity change of \textit{Surprise}
(\tref{T_CalculateEmotionSurprise}, \tyref{TY_DeltaIntensity}) by treating the
entity's prediction about an event's probability as the affected ``internal
condition''. Although the ``event'' driving \textit{Surprise} is a ``world
event'', its elicitation is driven by an entity's internal prediction about it
rather than some event ``valuation'' or ``seriousness''. Therefore, an
entity-specific ``unexpectedness value'' is the event's ``seriousness'' or
``value'' because the ``event'' driving \textit{Interest} elicitation is not
necessarily the same as ``world events''. This value scales with the based on a
common-sense hypotheses about the monotonically increasing relation between
\textit{Surprise} intensity and event unexpectedness.

Users can specify different values for $i_{\mathit{max}\Delta}$ so that
entities are more ``startled'' by some events than others.

Note that the \textit{Surprise} elicitation model
(\iref{IM_CalculateEmotionSurpriseElicit}) outputs a value with type $[0,1]^?$,
which users can supply as the input $\mathit{discr}_{s_\Delta}$ if it is not
None. \\\hrule

\paragraph{Sources} \citet[p.~54, 56]{reisenzein2019cognitive}

\paragraph{Depends On} \tref{T_CalculateEmotionSurprise},
\tyref{TY_DeltaIntensity}

\paragraph{Ref. By} \rref{R_CalculateIntensity} \\\hrule\vspace{0.5mm}\hrule

~\newline

\noindent
\begin{minipage}{\textwidth}
    \renewcommand*{\arraystretch}{1.5}
    \begin{tabular}{| p{\colAwidth}  p{\colBwidth}|}
        \hline
        \rowcolor[gray]{0.9}
        \bf IM\refstepcounter{instnum}\theinstnum
        \label{IM_DecayEmotionIntensity} &
        \bf Decaying an Emotion Intensity \\
        \hline
    \end{tabular}

    \renewcommand*{\arraystretch}{1.5}
    \begin{tabular}{ p{\colAwidth}  p{\colBwidth}}
        \bf Input & $i_0 : \emotionintensitytype, i_{Eq} :
        \emotionintensitytype, i_\lambda : \emotiondecaytype, \Delta t :
        \deltatimetype, \zeta : \mathbb{R}_{>0}$ \vspace*{2mm}\\

        \bf Output & $\begin{gathered}
         i : \emotionintensitytype \defEq I_\lambda + i_{Eq} \\
         \text{where } I_\lambda = \begin{cases}
             \begin{gathered}
                 e^{\text{\mbox{\footnotesize $-\sqrt{i_\lambda} \cdot \zeta
                             \cdot \Delta t$}}} \cdot (i_0 - i_{Eq}) \cdot
                             \Bigg(\cos(\omega
                 \cdot \Delta t) \\
                 + \left(\dfrac{\sqrt{i_\lambda} \cdot \zeta}{\omega}\right)
                 \cdot \sin(\omega \cdot \Delta t) \Bigg) \\
                 \text{where } \omega = \sqrt{i_\lambda} \cdot \sqrt{1 -
                 \zeta^2}
             \end{gathered}, & 0 < \zeta < 1 \\
             & \\
             \begin{gathered}
                 e^{\text{\mbox{\footnotesize $-\sqrt{i_\lambda} \cdot \Delta
                             t$}}} \cdot (i_0 - i_{Eq}) \cdot \left( 1 +
                             \sqrt{i_\lambda}
                 \cdot \Delta t\right)
             \end{gathered}, & \zeta = 1 \\
             & \\
             \begin{gathered}
                 \dfrac{i_0 - i_{Eq}}{2} \cdot \Bigg( \left(1 +
                 \dfrac{\zeta}{Q}\right) \cdot e^{\text{\mbox{\footnotesize
                             $-\sqrt{i_\lambda} \cdot \left(\zeta - Q\right)
                             \cdot \Delta
                             t$}}} \\
                 + \left(1 - \dfrac{\zeta}{Q}\right) \cdot
                 e^{\text{\mbox{\footnotesize $-\sqrt{i_\lambda} \cdot
                             \left(\zeta + Q\right) \cdot \Delta t$}}}\Bigg) \\
                 \text{where } Q = \sqrt{\zeta^2 - 1}
             \end{gathered} , & \zeta > 1
         \end{cases}
        \end{gathered}$
        \vspace*{2mm}\\\hline
    \end{tabular}
\end{minipage}

\paragraph{Description} Given an initial emotion intensity $i_0 :
\emotionintensitytype$ (\tyref{TY_EmotionIntensity}) and equilibrium intensity
$i_{Eq} : \emotionintensitytype$, an emotion decay rate $i_\lambda :
\emotiondecaytype$ (\tyref{TY_EmotionDecay}), the elapsed time since emotion
decay began $\Delta t : \deltatimetype$ (\tyref{TY_Time}), and a strictly
positive and real-valued damping ratio $\zeta$, evaluate the ``decayed''
intensity $i : \emotionintensitytype$ based on the case of $\zeta$.

Adding $i_{Eq}$ to $I_\lambda$ shifts the position from 0 to the equilibrium
point. The damping ratio $\zeta : \mathbb{R}_{>0}$ determines how much emotion
intensity oscillates as an returns to equilibrium.

\paragraph{Model Derivation} Emotion decay (\tref{T_DecayEmotionState}) is
modelled by a damped harmonic oscillator mass-spring system of the form:
$$x''\left(t\right) + c \cdot x'\left(t\right) + k_s \cdot x\left(t\right) = 0$$

where $x''\left(t\right)$, $x'\left(t\right)$, and $x\left(t\right)$ are the
acceleration, speed, and position of a mass $m$ at time $t$, $c$ is a strictly
positive and real-valued damping coefficient, and $k_s$ is a spring constant.
Emotion intensity is then equivalent to the ``position'' of the mass in the
system and the system's oscillation behaviour is a way for users to define an
entity's ``emotional stability''.

The damping ratio $\zeta$ and natural angular frequency of the system
$\omega_n$ govern its behaviour:
$$\zeta = \dfrac{c}{2 \cdot \sqrt{m \cdot k_s}}, \; \omega_n =
\sqrt{\dfrac{k_s}{m}}$$

Closed-forms of each case for $\zeta$ are necessary because the general
solution allows for imaginary numbers, which are not computationally tractable:
\begin{itemize}
    \item If $0 < \zeta < 1$, the system is \textit{underdamped} such that it
    oscillates as it returns to equilibrium. As $\zeta$ approaches 1, the
    oscillations decrease more quickly. This could represent an entity that is
    ``emotionally unstable'', alternating between high and low emotion
    intensities before returning to their ``normal'' state. Position $x$ at
    time $t$ is given by:
    $$\begin{gathered}
        x(t) = e^{-r \cdot t} \cdot \left( A \cdot \cos(\omega \cdot t) + B
        \cdot \sin(\omega \cdot t) \right) \\
        \text{where } A = x_0, \; B = \dfrac{v_0 + x_0 \cdot \omega_n \cdot
            \zeta}{\omega}, \; r = \omega_n \cdot \zeta \\
        \text{and } \omega = \omega_n \cdot \sqrt{1 - \zeta^2}
    \end{gathered}$$

    \item If $\zeta = 1$, the system is \textit{critically damped} such that
    it returns to equilibrium as quickly as possible without overshooting it.
    This could represent an entity that is the most ``emotionally stable'',
    recovering more quickly and directly than entities with other $\zeta$
    values. Position $x$ at time $t$ is given by:
    $$\begin{gathered}
        x(t) = e^{-\omega_n \cdot t} \cdot \left(x_0 + \left(v_0 + x_0
        \cdot \omega_n\right) \cdot t \right)
    \end{gathered}$$

    \item If $\zeta > 1$, the system is \textit{overdamped} such that it does
    not oscillate as it returns to equilibrium. As $\zeta$ increases, the
    system reaches equilibrium more slowly. This could represent an entity that
    is ``emotionally unstable'', experiencing their emotions longer than
    others. Position $x$ at time $t$ is given by:
    $$\begin{gathered}
        x(t) = C \cdot e^{-r_1 \cdot t} + D \cdot e^{-r_2 \cdot t} \\
        \text{where } C = \dfrac{1}{2} \cdot \left( x_0 + \dfrac{v_0 + x_0
            \cdot \omega_n \cdot \zeta}{\omega}\right), \; D = \dfrac{1}{2}
        \cdot \left( x_0 - \dfrac{v_0 + x_0 \cdot \omega_n \cdot
            \zeta}{\omega}\right), \\
        r_1 = \omega_n(\zeta - \sqrt{\zeta^2 - 1}), \; r_2 = \omega_n(\zeta
        + \sqrt{\zeta^2 - 1}) \\
        \text{and } \omega = \omega_n \cdot \sqrt{\zeta^2 - 1}
    \end{gathered}$$
\end{itemize}

Substitutes $k_s = i_\lambda$, $m = 1$ such that $\omega_n = \sqrt{i_\lambda}$,
$x_0 = x\left(t_0\right) = i_0 - i_{eq}$, $v_0 = 0$, and $t_0 = 0$ such that $t
= 0 + \Delta t = \Delta t$ into each model of position, where $i_0 :
\emotionintensitytype$ is the ``initial'' intensity, $i_{eq} :
\emotionintensitytype$ is the ``equilibrium'' intensity, $\Delta t :
\deltatimetype$ is the elapsed time since $t : \timetype = 0$, and $i_\lambda :
\emotiondecaytype$ is the intensity's decay rate and simplifying produces the
described output model. \\\hrule

\paragraph{Sources}
{\small\url{https://en.wikipedia.org/wiki/Harmonic_oscillator#Damped_harmonic_oscillator},

    \url{https://www.lehman.edu/faculty/anchordoqui/chapter23.pdf},
    \cite{alexiou2013solution}}

\paragraph{Depends On} \tref{T_DecayEmotionState}, \tyref{TY_EmotionIntensity},
\tyref{TY_EmotionDecay}, \tyref{TY_Time}

\paragraph{Ref. By} \tyref{TY_EmotionDecay}, \iref{IM_DecayEmotionState},
\rref{R_DecayIntensity} \\\hrule\vspace{0.5mm}\hrule

~\newline

\noindent
\begin{minipage}{\textwidth}
    \renewcommand*{\arraystretch}{1.5}
    \begin{tabular}{| p{\colAwidth}  p{\colBwidth}|}
        \hline
        \rowcolor[gray]{0.9}
        \bf IM\refstepcounter{instnum}\theinstnum
        \label{IM_UpdateEmotionState} &
        \bf Updating an Emotion Intensity with an Emotion Intensity Change \\
        \hline
    \end{tabular}

    \renewcommand*{\arraystretch}{1.5}
    \begin{tabular}{ p{\colAwidth}  p{\colBwidth}}
        \bf Input & $i : \emotionintensitytype, i_{\Delta} : \responsestrength
        $ \\

        \bf Output & $ i' : \emotionintensitytype \defEq \begin{cases}
            0.1 \cdot \log_2 \left(2^{10 \cdot i} + 2^{10 \cdot
                i_{\Delta}}\right), & i_{\Delta} > 0 \\
            0.1 \cdot \log_2 \left(2^{10 \cdot i} - 2^{10 \cdot | i_{\Delta}
                |}\right), & i_{\Delta} < 0 \\
            i, & \mathit{Otherwise} \\
        \end{cases} $ \\
        \hline
    \end{tabular}
\end{minipage}

\paragraph{Description} Given an emotion intensity $i : \emotionintensitytype$
(\tyref{TY_EmotionIntensity}) and emotion intensity change $i_\Delta :
\responsestrength$ (\tyref{TY_DeltaIntensity}), output a new emotion intensity
$i' : \emotionintensitytype$ using the logarithmic function so that it not
strictly additive and both values contribute to the output such that its
magnitude is at least as much as the highest input. \\\hrule

\paragraph{Sources} \cite{reilly2006modelling},
\citet[p.~370]{broekens2021emotion}

\paragraph{Depends On} \tyref{TY_EmotionIntensity}, \tyref{TY_DeltaIntensity}

\paragraph{Ref. By} \iref{IM_UpdateEmotionState2}, \rref{R_UpdateAnIntensity}
\\\hrule\vspace{0.5mm}\hrule

~\newline

\noindent
\begin{minipage}{\textwidth}
    \renewcommand*{\arraystretch}{1.5}
    \begin{tabular}{| p{\colAwidth}  p{\colBwidth}|}
        \hline
        \rowcolor[gray]{0.9}
        \bf IM\refstepcounter{instnum}\theinstnum
        \label{IM_UpdateEmotionState2} &
        \bf Create a new Emotion State by Updating Emotion Intensities in an
        Existing Emotion State \\
        \hline
    \end{tabular}

    \renewcommand*{\arraystretch}{1.5}
    \begin{tabular}{ p{\colAwidth}  p{\colBwidth}}
        \bf Input & $ es : \emotionstatetype, i_\Delta : \emotionkindstype
        \rightarrow \responsestrength $ \\

        \bf Output & $ es' : \emotionstatetype \defEq es' \text{ with }
        (\forall k : \emotionkindstype $ \newline
        $\text{ } \rightarrow es'.\mathtt{intensities}\left(k\right) =
        \mathtt{clamp}\left(I_k, 0, es.\mathtt{max}(k) \right),
        es'.\mathtt{max}(k) = es.\mathtt{max}(k))$ \newline
        $\text{where } I_k =
        \mathtt{UpdateIntensity}(es.\mathtt{intensities}\left(k\right),
        i_{\Delta}\left(k\right)) $ from \iref{IM_UpdateEmotionState} \\
        \hline
    \end{tabular}
\end{minipage}

\paragraph{Description} Given an emotion state $es : \emotionstatetype$
(\tyref{TY_EmotionState}) and a function from emotion kinds to intensity
changes $i_\Delta : \emotionkindstype \rightarrow \emotionintensitytype$
(\tyref{TY_EmotionKind}, \tyref{TY_DeltaIntensity}), output a new emotion state
$es'$ by replacing updating the intensities in $es.\mathtt{intensities}$ with
$i_\Delta$ using \iref{IM_UpdateEmotionState}.

A logarithm is an unbounded function and emotion intensities are assumed to be
finite (i.e. have a maximum value), so it is necessary to clamp the updated
intensity. \\\hrule

\paragraph{Sources} --

\paragraph{Depends On} \tyref{TY_DeltaIntensity}, \tyref{TY_EmotionKind},
\tyref{TY_EmotionState}, \iref{IM_UpdateEmotionState}

\paragraph{Ref. By} \rref{R_UpdateEmotionState} \\\hrule\vspace{0.5mm}\hrule

~\newline

\noindent
\begin{minipage}{\textwidth}
    \renewcommand*{\arraystretch}{1.5}
    \begin{tabular}{| p{\colAwidth}  p{\colBwidth}|}
        \hline
        \rowcolor[gray]{0.9}
        \bf IM\refstepcounter{instnum}\theinstnum
        \label{IM_UpdateEmotion} &
        \bf Updating Emotion \\
        \hline
    \end{tabular}

    \renewcommand*{\arraystretch}{1.5}
    \begin{tabular}{ p{\colAwidth}  p{\colBwidth}}
        \bf Input & $e : \emotiontype$, $t : \timetype$, $es :
        \emotionstatetype$ \\

        \bf Output & $ e' : \emotiontype \defEq \left\{ \, e \text{ with }
        e\left(t\right) = es \, \right\} $ \\
        \hline
    \end{tabular}
\end{minipage}

\paragraph{Description} Given an emotion $e : \emotiontype$
(\tyref{TY_Emotion}), emotion state $es : \emotionstatetype$
(\tyref{TY_EmotionState}), and a time $t : \timetype$ (\tyref{TY_Time}), output
a new emotion $e'$ by adding $es$ to $e$ at time $t$. \\\hrule

\paragraph{Sources} --

\paragraph{Depends On} \tyref{TY_EmotionState}, \tyref{TY_Emotion},
\tyref{TY_Time}

\paragraph{Ref. By} \iref{IM_GetNextEmotionByDecay}, \rref{R_UpdateEmotion}
\\\hrule\vspace{0.5mm}\hrule

~\newline

\noindent
\begin{minipage}{\textwidth}
    \renewcommand*{\arraystretch}{1.5}
    \begin{tabular}{| p{\colAwidth}  p{\colBwidth}|}
        \hline
        \rowcolor[gray]{0.9}
        \bf IM\refstepcounter{instnum}\theinstnum
        \label{IM_GetEmotionState} &
        \bf Getting an Emotion State \\
        \hline
    \end{tabular}

    \renewcommand*{\arraystretch}{1.5}
    \begin{tabular}{ p{\colAwidth}  p{\colBwidth}}
        \bf Input & $e : \emotiontype$, $t : \timetype$ \\

        \bf Output & $ es : \emotionstatetype \defEq e\left(t\right) $ \\
        \hline
    \end{tabular}
\end{minipage}

\paragraph{Description} Given an emotion $e : \emotiontype$
(\tyref{TY_Emotion}) and a time $t : \timetype$ (\tyref{TY_Time}), output the
emotion state $es : \emotionstatetype$ (\tyref{TY_EmotionState}) at
$e\left(t\right)$. \\\hrule

\paragraph{Sources} --

\paragraph{Depends On} \tyref{TY_EmotionState}, \tyref{TY_Emotion},
\tyref{TY_Time}

\paragraph{Ref. By} \iref{IM_GetNextEmotionByDecay}, \rref{R_GetEmotionState}
\\\hrule\vspace{0.5mm}\hrule

~\newline

\noindent
\begin{minipage}{\textwidth}
    \renewcommand*{\arraystretch}{1.5}
    \begin{tabular}{| p{\colAwidth}  p{\colBwidth}|}
        \hline
        \rowcolor[gray]{0.9}
        \bf IM\refstepcounter{instnum}\theinstnum
        \label{IM_DecayEmotionState} &
        \bf Decay an Emotion State \\
        \hline
    \end{tabular}

    \renewcommand*{\arraystretch}{1.5}
    \begin{tabular}{ p{\colAwidth}  p{\colBwidth}}
        \bf Input & $es_0 : \emotionstatetype, es_\lambda :
        \emotionstatedecaytype, \Delta t : \deltatimetype, \zeta :
        \mathbb{R}_{>0}$ \\

        \bf Output & $ \begin{gathered} es : \emotionstatetype \defEq \left\{
        \, \forall k \rightarrow es \text{ with }
            es.\mathtt{intensities}\left(k\right) = D \, \right\} \\
            \text{where } D = \mathtt{Decay}(\,
            es_0.\mathtt{intensities}\left(k\right), \Delta t, \zeta,
            es_\lambda.\mathtt{decayRates}\left(k\right), \\
            es_\lambda.\mathtt{equilibrium}\left(k\right) \, ) \text{ from
            \iref{IM_DecayEmotionIntensity}}
        \end{gathered}$ \\
        \hline
    \end{tabular}
\end{minipage}

\paragraph{Description} Given an initial emotion state $es_0 :
\emotionstatetype$ (\tyref{TY_EmotionState}), an emotion decay state
$es_\lambda : \emotionstatedecaytype$ (\tyref{TY_EmotionDecayState}), the
difference between two times $\Delta t : \deltatimetype$ (\tyref{TY_Time}), and
a damping ratio $\zeta : \mathbb{R}_{>0}$, generate a new emotion state $es :
\emotionstatetype$ by decaying intensity for every emotion intensity in $es_0$
using \iref{IM_DecayEmotionIntensity}. \\\hrule

\paragraph{Sources} --

\paragraph{Depends On} \tyref{TY_EmotionState}, \tyref{TY_EmotionDecayState},
\tyref{TY_Time}, \iref{IM_DecayEmotionIntensity}

\paragraph{Ref. By} \iref{IM_GetNextEmotionByDecay}, \rref{R_DecayEmotion}
\\\hrule\vspace{0.5mm}\hrule

~\newline

\noindent
\begin{minipage}{\textwidth}
    \renewcommand*{\arraystretch}{1.5}
    \begin{tabular}{| p{\colAwidth}  p{\colBwidth}|}
        \hline
        \rowcolor[gray]{0.9}
        \bf IM\refstepcounter{instnum}\theinstnum
        \label{IM_GetNextEmotionByDecay} &
        \bf Generating the Next State in Emotion By Decay \\
        \hline
    \end{tabular}

    \renewcommand*{\arraystretch}{1.5}
    \begin{tabular}{ p{\colAwidth}  p{\colBwidth}}
        \bf Input & $e : \emotiontype, es_\lambda : \emotionstatedecaytype,
        \left\{ t : \timetype, t' : \timetype \rightarrow t < t' \right\},
        \zeta : \mathbb{R}_{>0}$ \\

        \bf Output & $ e \defEq \left\{ \, e \text{ with } e\left(t'\right) =
        \mathtt{DecayState}\left(\, e\left(t\right), t' - t, \zeta, es_\lambda
        \, \right) \, \right\}$ from \iref{IM_UpdateEmotion},
        \iref{IM_GetEmotionState}, and \iref{IM_DecayEmotionState} \\
        \hline
    \end{tabular}
\end{minipage}

\paragraph{Description} Given an emotion $e : \emotiontype$
(\tyref{TY_Emotion}), an emotion decay state $es_\lambda :
\emotionstatedecaytype$ (\tyref{TY_EmotionDecayState}), two times $t, t' :
\timetype$ (\tyref{TY_Time}) such that $t$ comes before $t'$, and a damping
ratio $\zeta : \mathbb{R}_{>0}$, update emotion $e$ by extracting the emotion
state at time $t$ using \iref{IM_GetEmotionState} and generating a new emotion
state from it using \iref{IM_DecayEmotionState} and adding it to $e$ at time
$t'$ using \iref{IM_UpdateEmotion}. \\\hrule

\paragraph{Sources} --

\paragraph{Depends On} \tyref{TY_EmotionDecayState}, \tyref{TY_Emotion},
\tyref{TY_Time}, \iref{IM_UpdateEmotion}, \iref{IM_GetEmotionState},
\iref{IM_DecayEmotionState}

\paragraph{Ref. By} \rref{R_NewESFromDecay} \\\hrule\vspace{0.5mm}\hrule

~\newline

\noindent
\begin{minipage}{\textwidth}
    \renewcommand*{\arraystretch}{1.5}
    \begin{tabular}{| p{\colAwidth}  p{\colBwidth}|}
        \hline
        \rowcolor[gray]{0.9}
        \bf IM\refstepcounter{instnum}\theinstnum
        \label{IM_GetEmotionStatePAD} &
        \bf Getting an Emotion State as a PAD Point \\
        \hline
    \end{tabular}

    \renewcommand*{\arraystretch}{1.5}
    \begin{tabular}{ p{\colAwidth}  p{\colBwidth}}
        \bf Input & $es : \emotionstatetype$ \\

        \bf Output & $ \begin{gathered} p : \padpoint \defEq
        \mathtt{clamp}\left(0.1 \cdot \log_2 \left( {\mathlarger\sum_{k
                    \in \emotionkindstype}} 2^{10 \cdot v\left(k\right) \cdot
                    I_k }
            \right), -1, 1 \right) \\
            \text{where } I_k =
            \dfrac{es.\mathtt{intensities}(k)}{es.\mathtt{max}(k)} \\
            \text{and } v\left(k : \emotionkindstype\right) : \padpoint =
            \begin{cases}
                \left( -0.62, +0.82, -0.43 \right), & k = \mFear \\
                \left( -0.51, +0.59, +0.25 \right), & k = \mAnger \\
                \left( -0.63, -0.27, -0.33 \right), & k = \mSadness \\
                \left( +0.76, +0.48, +0.35 \right), & k = \mJoy \\
                \left( +0.64, +0.51, +0.17 \right), & k = \mInterest \\
                \left( +0.16, +0.88, -0.15 \right), & k = \mSurprise \\
                \left( -0.60, +0.35, +0.11 \right), & k = \mDisgust \\
                \left( +0.64, +0.35, +0.24 \right), & k = \mTrust \\
            \end{cases}
    \end{gathered}$
        \vspace*{2mm}\\
        \hline
    \end{tabular}
\end{minipage}

\paragraph{Description} Given an emotion state $es : \emotionstatetype$
(\tyref{TY_EmotionState}), output a point in PAD space $p : \padpoint$
(\tyref{TY_PAD}) by converting each intensity in $es$ to a PAD point by scaling
it to the associated maximum intensity such that multiplying it with the chosen
reference points returns its relative PAD position to that point
(\tref{T_GetEmotionStatePAD}). This ensures that higher intensity emotions
weigh more than lower intensity emotions, contributing more to the overall PAD
value. The model evaluates these points such that an emotion kind with zero
intensity has the coordinates $(0, 0, 0)$---the neutral PAD value---and one at
maximum intensity has the same value as the corresponding reference point. It
then sums the individual points into an overall PAD point, clamping it to $[-1,
1]$ to adhere to the constraints on \tyref{TY_PAD}.

The function inside $\mathtt{clamp}$ is based on the one from
Em/Oz~\citep{reilly2006modelling} and
GAMYGDALA~\citep[p.~38]{popescu2014gamygdala}. It relies on a logarithm so that
it not strictly additive and all values contribute to the output such that its
magnitude is at least as much as the highest input. Although not experimentally
verified, it emulates these desired behaviours and reportedly works well.

This model does lose information about the converted emotion state because it
is combining information from eight discrete categories into one point in a
three-dimensional space (\citepg{schaap2008towards}{172};
\citepg{broekens2021emotion}{353}). After conversion, it is nearly impossible
to determine which emotion kind-intensity combinations contributed to the
point's generation. Should a user need this information, they must associate
the state with the point manually. \\\hrule

\paragraph{Sources} \citet[p.~40, 42--45]{mehrabian1980basic}

\paragraph{Depends On} \tref{T_GetEmotionStatePAD}, \tyref{TY_EmotionState},
\tyref{TY_PAD}

\paragraph{Ref. By} \rref{R_Convert2PAD} \\\hrule\vspace{0.5mm}\hrule