\subsection{Nonfunctional Requirements}\label{sec_ReqsNF}
The nonfunctional requirement (NF) categories reflect desirable software
qualities~\citep{ghezzifundamentals2003} and Volere nonfunctional requirements
categories~\citep{robertson2010mastering}. \progname{} does not have any known
Security or Political nonfunctional requirements, so the SRS omits these
categories. There are no known requirements for specifying Correctness or
Reliability requirements that do not directly correlate with Verifiability and
are independent of the target application, so the SRS omits these categories.

The SRS adapts relevant nonfunctional requirements from
GLaDOS~\citep[p.~144--164]{glados}, \progname{}'s precursor.

\paragraph{Robustness}
\noindent \begin{itemize}[wide=0pt, leftmargin=*]

    \item[NF\refstepcounter{nfnum}\thenfnum \label{N_RecoverNeutral}:]
    \progname{}'s operations shall return a valid result and inform game
    developers if it is unable to produce one from the given inputs and its
    constraints are known. \vspace*{1mm}\\
    \textit{Rationale}: There could be situations where the given inputs do not
    produce a recognized output, but producing one does not change the
    program's functionality (e.g. a default return value) or is recoverable
    (e.g. number out of its known range). \progname{} must be able to
    communicate this to game developers so that they can create recovery
    strategies. \vspace*{1mm}\\
    \textit{Fit Criterion}: \progname{} always returns a result doing so is
    possible and the game developer is made aware of the issue.

    \item[NF\refstepcounter{nfnum}\thenfnum \label{N_RecoverNoChange}:]
    \progname{} shall make no changes and inform the game developer if it is
    not possible to produce an output from the given inputs and a valid result
    is not available. \vspace*{1mm}\\
    \textit{Rationale}: There could be situations where the given inputs do not
    produce a recognized output, and producing one is not possible (e.g.
    changing a maximum bound). \progname{} must be able to
    communicate this to game developers so that they can create recovery
    strategies, and without completing the operation. \vspace*{1mm}\linebreak
    \textit{Fit Criterion}: \progname{} does make any changes any components if
    it is not possible to return a valid result and the game developer is made
    aware of the issue.

\end{itemize}

\paragraph{Performance}
\noindent \begin{itemize}[wide=0pt, leftmargin=*]

    \item[NF\refstepcounter{nfnum}\thenfnum \label{N_Speed}:] \progname{} shall
    evaluate and create/update an emotion state in a time frame consistent with
    human cognitive processing during a reaction time task. \vspace*{1mm}\\
    \textit{Rationale}: \progname{} imitates emotional responses in digital
    entities. This promotes an expectation that such responses are the same
    speed as human responses, which is independent of its computational
    efficiency (see \nfref{N_Efficient}). \vspace*{1mm}\\
    \textit{Fit Criterion}: The time taken between accepting system inputs and
    producing a response must be within 70ms and
    300ms~\citep[p.~47]{mackenzie2012human}.

    \item[NF\refstepcounter{nfnum}\thenfnum \label{N_Complex}:] \progname{}
    shall allow game developers to configure parameters to change its
    complexity for different types of entities at design time
    (\ref{flexComplex}). \vspace*{1mm}\\
    \textit{Rationale}: In addition to game entities themselves having variable
    levels of complexity (e.g. \textit{Pac-man} ghost~\citep{pacman} vs.
    \textit{Skyrim} citizen~\citep{skyrim}), a game developer might want to
    change an entity's complexity to reflect its relative importance in the
    environment (e.g. distance to player), when there is a high concentration
    of game entities in an area, or other performance-enhancing techniques.
    \vspace*{1mm}\linebreak
    \textit{Fit Criterion}: A game developer is able to configure \progname{}
    for a game entity that has the same complexity as a \textit{Pac-man} ghost
    to their satisfaction.

    \item[NF\refstepcounter{nfnum}\thenfnum \label{N_Efficient}:] \progname{}
    shall be resource efficient (\ref{flexScale}). \vspace*{1mm}\\
    \textit{Rationale}: Video games are known for taxing hardware and software
    resources to the limit. \progname{} must not impede on other game
    components' needs to be viable (related to \nfref{N_Complex}).
    \vspace*{1mm}\\
    \textit{Fit Criterion}: The efficiency of \progname{} components is no
    worse than comparable components in a library for one-dimensional physics.

\end{itemize}

\paragraph{Verifiability}
\noindent \begin{itemize}[wide=0pt, leftmargin=*]

    \item[NF\refstepcounter{nfnum}\thenfnum \label{N_Atomic}:]  \progname{}
    shall allow each of its ``atomic'' functions to be verified individually.
    \vspace*{1mm}\\
    \textit{Rationale}: Game developers are able to pick-and-choose which of
    \progname{}'s components to use. They must be confident that components
    that do not depend on others are verifiable on their own. This also
    supports component maintainability (see \nfref{N_Evolve}) and reusability
    (see \nfref{N_Reuse}). \vspace*{1mm}\\
    \textit{Fit Criterion}: Any \progname{} component that is identified as a
    stand-alone entity can be verified independently of other components.

    \item[NF\refstepcounter{nfnum}\thenfnum \label{N_Trace}:]  \progname{}
    shall allow game developers to trace components' outputs to their inputs
    (\ref{easeTrace}). \vspace*{1mm}\\
    \textit{Rationale}: Game developers will encounter unwanted during
    development. They must be able to see which of \progname{}'s components is
    causing the behaviour, verify that the module itself is working as
    expected, and update their use of it accordingly. \vspace*{1mm}\\
    \textit{Fit Criterion}: When given an expected output and actual output of
    a program that uses \progname{}, a game developer is able to identify where
    the output mismatch starts and correct it.

\end{itemize}

\paragraph{Maintainability}
\noindent \begin{itemize}[wide=0pt, leftmargin=*]

    \item[NF\refstepcounter{nfnum}\thenfnum \label{N_Support}:] \progname{}
    shall be self-supporting. \vspace*{1mm}\\
    \textit{Rationale}: There will be no dedicated team to help \progname{}'s
    game developers. \vspace*{1mm}\\
    \textit{Fit Criterion}: Game developers are able to answer questions with
    only \progname{}'s documentation and examples (related to
    \nfref{N_CodeDoc}, \nfref{N_Manual}).

    \item[NF\refstepcounter{nfnum}\thenfnum \label{N_Evolve}:] \progname{}
    shall isolate theory-specific elements to their own modules. \vspace*{1mm}\\
    \textit{Rationale}: Research evolves, and so does one's understanding of
    established theories. If there is a new development that affects
    \progname{}'s underlying emotion theories, it should be easy to find and
    update those parts. \progname{} can do this by isolating theory-specific
    elements in their own units. This is strongly related to, but different
    from, \nfref{N_Reuse}. \vspace*{1mm}\\
    \textit{Fit Criterion}: Changes to theory-specific components do not affect
    theory-agnostic components.

    \item[NF\refstepcounter{nfnum}\thenfnum \label{N_Mod}:]  \progname{}
    components shall be easy to modify and combine with user-defined ones for
    game developers with an Object-Oriented design/development background
    (\ref{flexTasks}, \ref{flexNew}). \vspace*{1mm}\\
    \textit{Rationale}: \progname{} adheres to standard Object-Oriented design
    practices (see \nfref{N_OOStandard}). \vspace*{1mm}\\
    \textit{Fit Criterion}: After their fifth modification to \progname{}, game
    developers should be able to make modifications within one working day.

    \clearpage
    \item[NF\refstepcounter{nfnum}\thenfnum \label{N_Languages}:] \progname{}
    shall be translatable to non-English languages. \vspace*{1mm}\\
    \textit{Rationale}: \progname{} must not assume that all game developers
    are proficient in English. \vspace*{1mm}\\
    \textit{Fit Criterion}: Natural language components of \progname{} and its
    documentation must be localized to three or fewer files.

\end{itemize}

\paragraph{Reusability}
\noindent \begin{itemize}[wide=0pt, leftmargin=*]

    \item[NF\refstepcounter{nfnum}\thenfnum \label{N_Reuse}:] \progname{} must
    allow a game developer to change the underlying emotion theories and/or
    models. \vspace*{1mm}\\
    \textit{Rationale}: Some CME elements do not necessarily have to be
    theory-specific (e.g. definition of emotion intensity versus how it is
    evaluated). Someone should be able to reuse these parts in completely new
    CME designs that might use different underlying emotion theories and/or
    models. This is strongly related to, but different from,
    \nfref{N_Evolve}.\vspace*{1mm}\\
    \textit{Fit Criterion}: \progname{} components that are not specified from
    a particular emotion theory and/or model can be isolated and retooled for a
    CME that uses different theories.

\end{itemize}

\paragraph{Portability}
\noindent \begin{itemize}[wide=0pt, leftmargin=*]

    \item[NF\refstepcounter{nfnum}\thenfnum \label{N_Env}:] \progname{} shall
    be usable with different development environments and tools that it shares
    an implementation language with directly or could share via an intermediary
    language. \vspace*{1mm}\\
    \textit{Rationale}: Game developers will have preferences about development
    environments, practices, and tools. \progname{} needs to limit its
    dependence on specific environments and language proprietary elements in
    \progname{}'s implementation language for longevity and long-term impact.
    \vspace*{1mm}\\
    \textit{Fit Criterion}: Game developers are able to use \progname{} in
    conjunction with their preferred development environment and tools that
    uses the same implementation language as \progname{} by modifying 5
    interfaces at most. It follows that if it is possible to interface
    \progname{} with an environment or tool that has the same implementation
    language as itself, and it is possible to translate another language into
    the shared implementation language, then it is possible for game developers
    to use \progname{} with environments and/or tools in that other language as
    well.

\end{itemize}

\paragraph{Operational \& Interoperability}
\noindent \begin{itemize}[wide=0pt, leftmargin=*]

    \item[NF\refstepcounter{nfnum}\thenfnum \label{N_Arch}:]  \progname{} shall
    be usable with different agent architectures and/or frameworks
    (\ref{flexArch}) that can be realized in the same implementation language
    as \progname{} or in a compatible intermediary language. \vspace*{1mm}\\
    \textit{Rationale}: Game developers will have preferences about agent
    architectures and frameworks. \progname{} must use generic references to
    agent components that are also not specific to its implementation language
    to avoid association with a specific one, which could discourage game
    developers from using it. \vspace*{1mm}\\
    \textit{Fit Criterion}: Game developers are able to use \progname{} in
    conjunction with their preferred agent architecture or framework that can
    be realized in the same implementation language as \progname{}. It follows
    that if it is possible to interface \progname{} with an architectures or
    framework that has the same implementation language as itself, and it is
    possible to translate another language into the shared implementation
    language, then it is possible for game developers to use \progname{} with
    architectures and/or frameworks in that other language as well.

    \clearpage
    \item[NF\refstepcounter{nfnum}\thenfnum \label{N_Embody}:]  \progname{}'s
    functionality shall not rely on an entity's embodiment or implementation.
    \vspace*{1mm}\\
    \textit{Rationale}: Since \progname{} should be agent architecture and
    framework independent (see \nfref{N_Arch}), it cannot assume that an entity
    will have embodiment-specific features that it can use. \vspace*{1mm}\\
    \textit{Fit Criterion}: \progname{} is usable with game entities with no
    functional embodiment.

\end{itemize}

\paragraph{Understandability}
\noindent \begin{itemize}[wide=0pt, leftmargin=*]

    \item[NF\refstepcounter{nfnum}\thenfnum \label{N_Conventions}:]
    \progname{}'s APIs shall follow the coding conventions of \progname{}'s
    implementation language. \vspace*{1mm}\\
    \textit{Rationale}: Following the code and naming conventions of
    \progname{}'s implementation language will make it easier to understand
    \progname{}'s content and functionality. \vspace*{1mm}\\
    \textit{Fit Criterion}: A game developer familiar with the \progname{}'s
    implementation language understands its structure without referring to
    \progname{}'s documentation.

\end{itemize}

\paragraph{Usability}
\noindent \begin{itemize}[wide=0pt, leftmargin=*]

    \item[NF\refstepcounter{nfnum}\thenfnum \label{N_Knowledge}:] \progname{}
    shall not require game developers to have knowledge of affective science
    and/or emotion research, nor should it use jargon in any of its
    descriptions (\ref{easeHide}). \vspace*{1mm}\\
    \textit{Rationale}: Assuming that most video game designers/developers
    (Section~\ref{sec:doc_stakeholder}) have little/no academic knowledge of
    affective science and/or emotion research, it is unrealistic to expect them
    to know the psychological theories behind \progname{}. \vspace*{1mm}\\
    \textit{Fit Criterion}: Game developers can implement \progname{}
    components as-is in game entities without any knowledge of PES, CTE, or PAD
    Space and without being confused by \progname{}'s terminology and
    descriptions.

    \item[NF\refstepcounter{nfnum}\thenfnum \label{N_CodeDoc}:] \progname{}
    shall document essential usage information in the user-facing software
    components (\ref{easeAPI}) such that game developers can learn it quickly
    and confidently. \vspace*{1mm}\\
    \textit{Rationale}: Game development is a rapid and stressful process, so
    game developers will not adopt any tool or system that cannot be picked up
    easily. New-to-\progname{} game developers are likely to try using it as-is
    to see what it can do and become acquainted with it. Essential information,
    such as a high-level description of a component's purpose, should be
    embedded in the user-facing files as a ``quick start'' guide.
    \vspace*{1mm}\linebreak
    \textit{Fit Criterion}: \begin{itemize}[noitemsep, nosep]
        \item A game developer who has not used \progname{} before should not
        need external product documentation to use its components as-is

        \item An empirical game developer study of game developers
        (Section~\ref{sec:doc_stakeholder}) with Object-Oriented design
        experience and whom are given \progname{}'s user manual shall have
        statistically significant results regarding their confidence in the use
        of \progname{} after one week of practice
    \end{itemize}

    \item[NF\refstepcounter{nfnum}\thenfnum \label{N_Manual}:] \progname{}
    shall document usage examples for each component, and some ways to combine
    them. \vspace*{1mm}\\
    \textit{Rationale}: Game developers that are advanced \progname{} users
    likely want to customize \progname{} and combine its components in
    different ways. Additional information about each component, and examples
    for addressing design problems such as handling multiple simultaneously
    generated emotion types, must be available to support this.
    \vspace*{1mm}\\
    \textit{Fit Criterion}: A game developer should be able to understand and
    apply concepts from \progname{} documentation.

    \item[NF\refstepcounter{nfnum}\thenfnum \label{N_Custom}:]  \progname{}
    shall allow game developers to customize and/or redefine its configuration
    parameters (\ref{flexCustom}). \vspace*{1mm}\\
    \textit{Rationale}: It is highly unlikely that vanilla \progname{} will
    meet a game developer's needs perfectly. game developers must be able to
    tailor \progname{} for their purposes to increase its longevity and
    long-term impact. \linebreak
    \textit{Fit Criterion}: A game developer should be able to modify and/or
    add a \progname{} component without affecting other parts of the system.

    \item[NF\refstepcounter{nfnum}\thenfnum \label{N_Output}:]  \progname{}
    shall allow game developers to specify how to use its outputs and/or states
    (\ref{flexOut}). \vspace*{1mm}\\
    \textit{Rationale}: \progname{} is not supposed to tell game developers
    \textit{what} to do with its components and outputs, only \textit{how} it
    can help them accomplish their design goals. Therefore, its documentation
    should include examples demonstrating what it can do so they can use it as
    they require. \vspace*{1mm}\\
    \textit{Fit Criterion}: A game developer can use \progname{} values with
    game entities that are not NPCs.

    \item[NF\refstepcounter{nfnum}\thenfnum \label{N_AuthorialEffort}:]
    \progname{} shall not require a game developer to add ``emotional''
    functionality on a per-entity basis (\ref{easeAuthor}). \vspace*{1mm}\\
    \textit{Rationale}: Using \progname{} will increase the development and
    testing time, but this must be minimized so that its perceived benefits
    outweigh its costs. \vspace*{1mm}\\
    \textit{Fit Criterion}: As the number of game entities using \progname{}
    increases, game developer effort towards implementing \progname{}
    functionality in entities increases at a rate that is better than linear.

    \item[NF\refstepcounter{nfnum}\thenfnum \label{N_Auto}:]  \progname{} shall
    allow game developers to automate the storing and decaying of an emotion
    state if they wish (\ref{easeAuto}). \vspace*{1mm}\\
    \textit{Rationale}: These are general maintenance tasks that most game
    developers will want to happen in the absence of emotion generation.
    Allowing game developers to automate these tasks as they see fit will also
    reduce authorial burden (see \nfref{N_AuthorialEffort}) \vspace*{1mm}\\
    \textit{Fit Criterion}: Game developers can set \progname{} to
    automatically update an emotion state when it generates emotions, and set
    conditions for initiating emotion decay.

\end{itemize}

\paragraph{Look \& Feel}
\noindent \begin{itemize}[wide=0pt, leftmargin=*]

    \item[NF\refstepcounter{nfnum}\thenfnum \label{N_Professional}:]
    \progname{} shall appear to be professional and approachable.
    \vspace*{1mm}\\
    \textit{Rationale}: For \progname{} to have an impact outside of the
    academic sphere, it must appear to be a professionally developed product so
    that industry people do not hesitate to try it. \vspace*{1mm}\\
    \textit{Fit Criterion}: \progname{} has a user guide, documented design,
    and well-organized source code.

    \item[NF\refstepcounter{nfnum}\thenfnum \label{N_OOStandard}:]  \progname{}
    shall adhere to standard Object-Oriented design practices. \vspace*{1mm}\\
    \textit{Rationale}: Game developers typically use Object-Oriented languages
    such as C/C++, C\#, and Java/ JavaScript \citep{sweeney2006next} and many
    game engines are based on the Object-Oriented
    paradigm\footnote{\url{https://en.wikipedia.org/wiki/List_of_game_engines},
        Accessed July 27, 2022}. \vspace*{1mm}\\
    \textit{Fit Criterion}: A game developer should be able to use \progname{}'s
    components as data and/or objects both within game engines and not.

\end{itemize}

\clearpage
\paragraph{Cultural (World)}
\noindent \begin{itemize}[wide=0pt, leftmargin=*]

    \item[NF\refstepcounter{nfnum}\thenfnum \label{N_ChooseEm}:] \progname{}
    shall allow game developers to choose which kinds of emotion it produces
    (\ref{flexEm}). \vspace*{1mm}\\
    \textit{Rationale}: While the biological basis of emotion categories might
    be universal, the language people describe them with are
    not~\citep[p.~115--122]{oatley1992best}. \progname{} should not enforce the
    Euro-American affective lexicon because it might be difficult to understand
    and/or meaningless to a different culture. This would hinder \progname{}'s
    overall usability. \vspace*{1mm}\\
    \textit{Fit Criterion}: Game developers can create their own emotion labels
    for built-in categories and/or create their own categories (related to
    \rref{R_MixingEmotionsPES}, \rref{R_PartitionEmotions},
    \rref{R_MixingEmotionsCTE}).
\end{itemize}

\paragraph{Cultural (Game Development Work)}
\noindent \begin{itemize}[wide=0pt, leftmargin=*]
    \item[NF\refstepcounter{nfnum}\thenfnum \label{N_PX}:] \progname{} shall
    demonstrate that it improves PX (\ref{easePX}). \vspace*{1mm}\\
    \textit{Rationale}: A common goal of game design is to create an
    interesting PX~\citep[p.~11]{mcallister2015video}. \progname{} must show
    that it contributes to this, which will encourage industry people to try it
    (related to \nfref{N_Professional}). \vspace*{1mm}\\
    \textit{Fit Criterion}: An empirical user study of a game with \progname{}
    functionality and participants that are video game players
    (Section~\ref{sec:doc_stakeholder}) shall have statistically significant
    results implying that the game has a better PX than a game without
    \progname{} functionality.

    \item[NF\refstepcounter{nfnum}\thenfnum \label{N_Novel}:] \progname{} shall
    demonstrate ways that it can create novel game experiences
    (\ref{easeNovel}). \vspace*{1mm}\\
    \textit{Rationale}: Even if \progname{} improves PX (see \nfref{N_PX}),
    game developers might hesitate to try it if there does not appear to be a
    sufficient benefit from learning a new tool and integrating it into a game
    design. \vspace*{1mm}\\
    \textit{Fit Criterion}: Game demos that use \progname{} will be freely
    available.

    \item[NF\refstepcounter{nfnum}\thenfnum \label{N_Examples}:] \progname{}'s
    game demos shall adhere to common expectations about the connection between
    goals, events, and emotions. \vspace*{1mm}\\
    \textit{Rationale}: Even though they are intended as examples, the database
    elements that are included with the basic architecture should still adhere
    to commonly held beliefs about how people react to different events.
    \vspace*{1mm}\linebreak
    \textit{Fit Criterion}: After playing a demo, 75\% of players should be
    able to articulate why a game entity ``experienced'' its emotions in
    relation to its goals and game events.

\end{itemize}

\paragraph{Legal}
\noindent \begin{itemize}[wide=0pt, leftmargin=*]

    \item[NF\refstepcounter{nfnum}\thenfnum \label{N_OSS}:]  \progname{} shall
    be released as Open-Source Software. \vspace*{1mm}\\
    \textit{Rationale}: Allowing people to use, study, and change \progname{}'s
    source code can increase its longevity and long-term impact, as it does not
    rely on a single developer to update and maintain it. It also encourages
    trust in \progname{} because game developers can examine its source code
    (related to \nfref{N_Custom}). \vspace*{1mm}\\
    \textit{Fit Criterion}: \progname{} is published with an open-source
    license.

    \item[NF\refstepcounter{nfnum}\thenfnum \label{N_AcademicIntegrity}:]
    \progname{} shall adhere to all terms and conditions of McMaster's
    Academic Integrity Policy. \vspace*{1mm}\\
    \textit{Rationale}: Intellectual honestly and ethical practices are
    prerequisites for \progname{}'s acceptance, as failure to do so will make
    stakeholders reject it on principle. As \progname{} design and development
    is happening at McMaster University, this is the policy to follow.
    \vspace*{1mm}\\
    \textit{Fit Criterion}: \progname{} does not violate McMaster's Academic
    Integrity Policy.

\end{itemize}