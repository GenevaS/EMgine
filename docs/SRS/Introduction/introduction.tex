\section{Introduction}
This is the Software Requirements Specification (SRS) for \progname{}, a 
Computational Model of Emotion (CME) for Non-Player Characters (NPCs) to 
enhance their believability, with the goal of improving long-term player 
engagement. \progname{} is for \textit{emotion generation}, accepting 
user-defined information from a game environment to determines what emotion and 
intensity a NPC is ``experiencing''. How the emotion is expressed and what 
other effects it could have on game entities is left for game 
designers/developers to decide.

\progname{} aims to provide a feasible and easy-to-use method for game 
designers/developers to include emotion in their NPCs, they perceive to be 
challenging with the current tools and 
restrictions~\citep{broekens2016emotional}. \progname{} should be modular and 
portable such that game designers/developers can use it in their regular 
development environment, and should not require knowledge of affective science, 
psychology, and/or emotion theories.

A \textit{believable character} ``...allows the audience to suspend their 
disbelief and...provides a convincing portrayal of the personality they expect
or come to expect [from the character]''~\cite[p.~1]{loyall1997believable}. 
Believability is not limited to ``smart'' or ``normal'' characters because it
depends on the situational context and the character's
personality~\citep{lisetti2015and, loyall1997believable, reilly1996believable}.
In short: an NPC must behave reasonably within the context of the game world.
Generally, NPCs are believable when they~\citep{lankoski2007gameplay,
    loyall1997believable, warpefelt2013analyzing}:
\begin{itemize}
    \item Appear to be self-motivated,
    
    \item Are aware of what is happening around them, and
    
    \item React in ways appropriate for their surrounding context while
    adhering to their personality.
\end{itemize}

\textit{Emotion}, transient responses to changes in the 
environment~\citep{lazarus1991emotion}, is one element of believable character 
design~\citep{de2015beyond, emmerich2018m, gard2000building, 
lankoski2007gameplay, lisetti2015and, loyall1997believable, paiva2005learning, 
warpefelt2013analyzing}. It is an established aspect of believability in 
animation~\citep{thomas1981illusion}, and game designers have acknowledged its 
importance in NPC design~\citep{hudlicka2009foundations, yannakakis2015emotion}.

Characters with emotion address the core features of believability because they 
convey a character's goals and desires (\textit{self-motivated}) by showing 
their \textit{awareness} of, \textit{responsiveness} to, and care 
(\textit{personality}-driven) for their surroundings~\citep{bates1994, 
broekens2021emotion, reilly1996believable}. It follows that one way to improve 
an NPC's believability is to have them react emotionally to their 
surroundings~\citep{togelius2013assessing, yannakakis2015emotion}.

The template presenting this SRS's information is based on 
\citet{SmithAndLai2005} and \citet{SmithEtAl2007}. The introductory sections 
describe the project stakeholders (Section~\ref{sec:doc_stakeholder}), purpose 
of this document (Section~\ref{sec:doc_purpose}), what background and knowledge 
a reader must have to understand this document (Section~\ref{sec:doc_reader}), 
the requirement scope (Section~\ref{sec:doc_reqscope}), and the organization of 
document sections (Section~\ref{sec:doc_org}).

\subsection{Stakeholders}\label{sec:doc_stakeholder}
\begin{itemize}
    \item \textbf{Primary}
    \begin{enumerate}

        \item \textbf{Video Game Designers/Developers} \\
        Game designers/developers are concerned with making enjoyable
        experiences for their players, so \progname{} must aid them in this.
        Designers/developers interact with \progname{} directly to give their
        NPCs ``emotion''.

        A common issue raised by designers/developers is that adding emotions
        to their game is an involved and complicated
        process~\citep{broekens2016emotional}, implying that \progname{} must
        be simple to use in order for it to have the potential to be effective
        as a development tool. It must also help designers/developers see how
        ``emotional'' NPCs contribute to the player experience (PX), and to
        novel game design.

        Video game designers/developers are not expected to have a background
        in Affective Science or Psychology, or an understanding of Emotion
        Theories.

        \item \textbf{Video Game Players} \\
        Players do not interact with \progname{} directly, instead seeing its
        effects through a game's NPCs. As a key stakeholder, any feature of
        \progname{} that takes away from their entertainment must be seriously
        evaluated for redevelopment or removal. Video Game
        Designers/Developers, too, view their players as primary stakeholders
        of their games for similar reasons. This enforces the need to view
        players as primary stakeholders.

        Video game players are not expected to have a background in Affective
        Science or Psychology, or an understanding of Emotion Theories.

    \end{enumerate}

    \item \textbf{Secondary}
    \begin{enumerate}
        \item \textbf{Geneva M. Smith} \\
        The main developer of \progname{}, Ms. Smith sets time constraints on
        its development. The initial prioritization of \progname{} components
        to develop are set by the time and skills available to her.

        Ms. Smith's background is in Software Engineering (SE) and
        Human-Computer Interaction (HCI).

        \item \textbf{Dr. Jacques Carette} \\
        An employee of McMaster University, Dr. Carette is the direct
        supervisor of Geneva M. Smith and part of her supervisory committee. He
        contributes to \progname{}'s design and the prioritization of
        development tasks.

        Dr. Carette requires \progname{} to be easily explainable to
        non-experts of affective science and psychology, and for it to follow
        software engineering best practices. Dr. Carette also requires that
        \progname{}'s development be serialized to enable the communication of
        progress and creation of academic papers.

        Dr. Carette's background is in Mathematics and Computer Science (CS).

        \item \textbf{Drs Spencer Smith, Alan Wassyng, and Denise Geiskkovitch}
        \\
        Members of Geneva M. Smith's supervisory committee and employees of
        McMaster University. Drs Smith, Wassyng, and Geiskkovitch evaluate her
        development of \progname{}, and provide advice and feedback.

        Drs Smith, Wassyng, and Geiskkovitch require \progname{} to be easily
        explainable to non-experts of affective science and psychology, and for
        it to follow software engineering best practices.

        Dr. Smith's background is SE and Civil Engineering, Dr. Wassyng's is
        SE, and Dr. Geiskkovi-tch's is HCI.

        \item \textbf{McMaster University} \\
        \progname{}'s development is being conducted at McMaster and must
        adhere to its Academic Integrity and Research Ethics rules and
        regulations.

        \item \textbf{Conference and Journal Selection Committees} \\
        As a research project, it is expected that any findings resulting from
        \progname{}'s development and testing is shared with the broader
        academic community. This is commonly done via field-specific conference
        and journal papers that researchers submit for evaluation by a
        committee of experts in the field. Therefore, \progname{} must be
        documented in a traceable and easily explainable manner to accelerate
        its translation into suitable formats for submission and consumption.

        \item \textbf{The Affective Computing Research Community} \\
        As a research project, it is expected that \progname{}'s development
        and testing is shared with the broader academic community so that
        members can compare other designs with, reuse, and/or build on
        \progname{}'s design, implementation, and/or testing artefacts.

    \end{enumerate}
\end{itemize}

\subsection{Purpose of Document}\label{sec:doc_purpose}
This document represents the specifications of \progname{} as a self-contained
system that interfaces with external systems. It describes all of \progname{}'s
necessary functionality, and the underlying assumptions and models derived from
affective science, psychology, and emotion theory literature.

This document's purpose is to guide the design of \progname{}'s architecture
and interfaces, as well as the development of its test plan. After initial
development, the document's purpose is to aid in the use, maintenance, and
further development of \progname{}. This includes activities that change
\progname{}'s existing components, such as those described in the likely
changes (Section~\ref{sec_Expand}), or add new components.

\subsection{Characteristics of Intended Reader}\label{sec:doc_reader}
The intended reader of this document must have an undergraduate-level 
understanding of affective science and/or affective psychology. They must also  
know the differences between types of affect, including emotion 
(\citepg{broekens2021emotion}{349}; 
\citepg{scherer2000psychological}{138--140}), ``core'' 
affect~\citep[p.~170]{barrett2009affect}, mood~\citep{oxfordMood}, and
personality~\citep{oxfordPersonality}. It would be beneficial if the reader is 
familiar with Plutchik's Psycho-evolutionary Synthesis 
(PES)~\citep{robert1980emotion}, Oatley and Johnson-Laird's Communicative 
Theory of Emotions (CTE)~\citep{oatley1992best, oatley1987towards}, and 
Mehrabian's Pleasure-Arousal-Dominance affective space model 
(PAD)~\citep{mehrabian1980basic, mehrabian1996pleasure}.

The intended reader must also have an understanding of type, function, 
sequence, and set notation. They must also understand mathematical states, 
logical quantification, and intervals over a numerical domain.

\subsection{Scope of Requirements}\label{sec:doc_reqscope}
\progname{} calculates a discrete emotion type/kind/category and associated
intensity for a game entity---initia-lly, a NPC---from user-defined information
about the entity's goals, plans to achieve certain world states, attention to
other entities, social attachments to other entities and game states.
\progname{} uses this information to update the entity's internal ``emotion''
state at a given time step. \progname{} also provides the means to decay the
intensities of an ``excited'' emotion state back to user-defined default values
such that it can ``calm down''. \progname{} also provides the means for a user
to query an emotion state for its equivalent representation as a point in PAD
space.

\progname{} focuses on \textit{emotion generation} alone. Therefore, it does
not provide the functionality to calculate other types of affect such as
``core'' affect, mood, and personality. However, providing an emotion state's
coordinates in PAD space can help \progname{} interface with other
user-provided components that do. The use of PAD space as a common
representation for different types of affect and their interactions has proven
successful in other CMEs~\citep{broekens2004scalable, gebhard2005alma,
masuyama2018personality}).

The focus on emotion generation also means that \progname{} does not define
most cognitive entities and functions at all. However, there are some that are
necessary for CTE's functions---goals, plans, and optional attention and social
attachment components/functions. \progname{} provides Application Programming
Interfaces (APIs) for these components. It uses abstract types and/or defines
them with elements from the provided APIs where possible. This maximizes
\progname{}'s portability as it gives video game designers/developers as much
freedom as possible to use an agent architecture of their choice. Providing
inputs to the attention and social attachment APIs are optional because
\progname{} can still generate six of the eight ``primary'' emotion types
defined in PES (\textit{Anger}, \textit{Fear}, \textit{Sadness}, \textit{Joy},
\textit{Surprise}, and \textit{Disgust}) without them. Using the API for
attention allows \progname{} to generate \textit{Interest}, and the API for
social attachment allows \progname{} to generate \textit{Acceptance}.

\progname{}'s users are responsible for providing it with definitions for a
game entity's goals and plans, and for defining the default emotion state
intensities and decay rates for each emotion type it contains. Users can
optionally provide information for the game entity's attention and social
attachments to other game entities. Users also have the option to define new
emotion kinds using the provided ``primary'' ones and emotion intensity values.

Due to its intended use in game development, \progname{} design is unconcerned
with strict adherence to ``realistic'' emotion generation. This means that its
usage scope is limited to the scope of games, which have been used for
pedagogical purposes. It is left to the game designers/developers to determine
if the level of realism provided by \progname{} is appropriate for their
application.

\subsection{Organization of Document}\label{sec:doc_org}
The rest of this document is organized as follows:
\begin{itemize}
    \item A general description of \progname{} (Section~\ref{sec_genDesc}),
    including the system context and constraints, as well as characteristics of
    the intended end users

    \item The problem description (Section~\ref{Sec_pd}), which defines
    terminology necessary for understanding subsequent models and requirements,
    a description of the physical system that \progname{} exists in, and the
    goals that \progname{} must achieve through its requirements

    \item A solution characteristics descriptions (Section~\ref{Sec_scs}),
    documenting \progname{}'s models, types, assumptions, data constraints and
    definitions, and properties of a correct solution

    \item \progname{}'s functional and non-functional requirements
    (Section~\ref{sec_Reqs})

    \item Future changes, both likely and unlikely, to \progname{}'s solution
    characteristics specification (Section~\ref{sec_Changes})

    \item Traceability matrices and graphs (Section~\ref{sec_trace}), visually
    describing the dependencies between document components, for reference
    if/when making changes to \progname{}'s specifications

    \item Suggestions for other CME designs, components, and extensions
    (Appendices~\ref{appendix_extensions}, \ref{appendix_cognitive},
    \ref{appendix_lazarus}), including preliminary assumptions, models, and
    type definitions
\end{itemize}