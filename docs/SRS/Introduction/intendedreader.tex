\subsection{Characteristics of Intended Reader}\label{sec:doc_reader}
The intended reader of this document must have an undergraduate-level 
understanding of affective science and/or affective psychology. They must also  
know the differences between types of affect, including emotion 
(\citepg{broekens2021emotion}{349}; 
\citepg{scherer2000psychological}{138--140}), ``core'' 
affect~\citep[p.~170]{barrett2009affect}, mood~\citep{oxfordMood}, and
personality~\citep{oxfordPersonality}. It would be beneficial if the reader is 
familiar with Plutchik's Psycho-evolutionary Synthesis 
(PES)~\citep{robert1980emotion}, Oatley and Johnson-Laird's Communicative 
Theory of Emotions (CTE)~\citep{oatley1992best, oatley1987towards}, and 
Mehrabian's Pleasure-Arousal-Dominance affective space model 
(PAD)~\citep{mehrabian1980basic, mehrabian1996pleasure}.

The intended reader must also have an understanding of type, function, 
sequence, and set notation. They must also understand mathematical states, 
logical quantification, and intervals over a numerical domain.