\subsection{Stakeholders}\label{sec:doc_stakeholder}
\begin{itemize}
    \item \textbf{Primary}
    \begin{enumerate}

        \item \textbf{Video Game Designers/Developers} \\
        Game designers/developers are concerned with making enjoyable
        experiences for their players, so \progname{} must aid them in this.
        Designers/developers interact with \progname{} directly to give their
        NPCs ``emotion''.

        A common issue raised by designers/developers is that adding emotions
        to their game is an involved and complicated
        process~\citep{broekens2016emotional}, implying that \progname{} must
        be simple to use in order for it to have the potential to be effective
        as a development tool. It must also help designers/developers see how
        ``emotional'' NPCs contribute to the player experience (PX), and to
        novel game design.

        Video game designers/developers are not expected to have a background
        in Affective Science or Psychology, or an understanding of Emotion
        Theories.

        \item \textbf{Video Game Players} \\
        Players do not interact with \progname{} directly, instead seeing its
        effects through a game's NPCs. As a key stakeholder, any feature of
        \progname{} that takes away from their entertainment must be seriously
        evaluated for redevelopment or removal. Video Game
        Designers/Developers, too, view their players as primary stakeholders
        of their games for similar reasons. This enforces the need to view
        players as primary stakeholders.

        Video game players are not expected to have a background in Affective
        Science or Psychology, or an understanding of Emotion Theories.

    \end{enumerate}

    \item \textbf{Secondary}
    \begin{enumerate}
        \item \textbf{Geneva M. Smith} \\
        The main developer of \progname{}, Ms. Smith sets time constraints on
        its development. The initial prioritization of \progname{} components
        to develop are set by the time and skills available to her.

        Ms. Smith's background is in Software Engineering (SE) and
        Human-Computer Interaction (HCI).

        \item \textbf{Dr. Jacques Carette} \\
        An employee of McMaster University, Dr. Carette is the direct
        supervisor of Geneva M. Smith and part of her supervisory committee. He
        contributes to \progname{}'s design and the prioritization of
        development tasks.

        Dr. Carette requires \progname{} to be easily explainable to
        non-experts of affective science and psychology, and for it to follow
        software engineering best practices. Dr. Carette also requires that
        \progname{}'s development be serialized to enable the communication of
        progress and creation of academic papers.

        Dr. Carette's background is in Mathematics and Computer Science (CS).

        \item \textbf{Drs Spencer Smith, Alan Wassyng, and Denise Geiskkovitch}
        \\
        Members of Geneva M. Smith's supervisory committee and employees of
        McMaster University. Drs Smith, Wassyng, and Geiskkovitch evaluate her
        development of \progname{}, and provide advice and feedback.

        Drs Smith, Wassyng, and Geiskkovitch require \progname{} to be easily
        explainable to non-experts of affective science and psychology, and for
        it to follow software engineering best practices.

        Dr. Smith's background is SE and Civil Engineering, Dr. Wassyng's is
        SE, and Dr. Geiskkovitch's is HCI.

        \item \textbf{McMaster University} \\
        \progname{}'s development is being conducted at McMaster and must
        adhere to its Academic Integrity and Research Ethics rules and
        regulations.

        \item \textbf{Conference and Journal Selection Committees} \\
        As a research project, it is expected that any findings resulting from
        \progname{}'s development and testing is shared with the broader
        academic community. This is commonly done via field-specific conference
        and journal papers that researchers submit for evaluation by a
        committee of experts in the field. Therefore, \progname{} must be
        documented in a traceable and easily explainable manner to accelerate
        its translation into suitable formats for submission and consumption.

        \item \textbf{The Affective Computing Research Community} \\
        As a research project, it is expected that \progname{}'s development
        and testing is shared with the broader academic community so that
        members can compare other designs with, reuse, and/or build on
        \progname{}'s design, implementation, and/or testing artefacts.

    \end{enumerate}
\end{itemize}