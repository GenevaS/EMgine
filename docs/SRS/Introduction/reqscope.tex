\subsection{Scope of Requirements}\label{sec:doc_reqscope}
\progname{} calculates a discrete emotion type/kind/category and associated
intensity for a game entity---initially, a NPC---from user-defined information
about the entity's goals, plans to achieve certain world states, attention to
other entities, social attachments to other entities and game states.
\progname{} uses this information to update the entity's internal ``emotion''
state at a given time step. \progname{} also provides the means to decay the
intensities of an ``excited'' emotion state back to user-defined default values
such that it can ``calm down''. \progname{} also provides the means for a user
to query an emotion state for its equivalent representation as a point in PAD
space.

\progname{} focuses on \textit{emotion generation} alone. Therefore, it does
not provide the functionality to calculate other types of affect such as
``core'' affect, mood, and personality. However, providing an emotion state's
coordinates in PAD space can help \progname{} interface with other
user-provided components that do. The use of PAD space as a common
representation for different types of affect and their interactions has proven
successful in other CMEs~\citep{broekens2004scalable, gebhard2005alma,
masuyama2018personality}).

The focus on emotion generation also means that \progname{} does not define
most cognitive entities and functions at all. However, there are some that are
necessary for CTE's functions---goals, plans, and optional attention and social
attachment components/functions. \progname{} provides Application Programming
Interfaces (APIs) for these components. It uses abstract types and/or defines
them with elements from the provided APIs where possible. This maximizes
\progname{}'s portability as it gives video game designers/developers as much
freedom as possible to use an agent architecture of their choice. Providing
inputs to the attention and social attachment APIs are optional because
\progname{} can still generate six of the eight ``primary'' emotion types
defined in PES (\textit{Anger}, \textit{Fear}, \textit{Sadness}, \textit{Joy},
\textit{Surprise}, and \textit{Disgust}) without them. Using the API for
attention allows \progname{} to generate \textit{Interest}, and the API for
social attachment allows \progname{} to generate \textit{Acceptance}.

\progname{}'s users are responsible for providing it with definitions for a
game entity's goals and plans, and for defining the default emotion state
intensities and decay rates for each emotion type it contains. Users can
optionally provide information for the game entity's attention and social
attachments to other game entities. Users also have the option to define new
emotion kinds using the provided ``primary'' ones and emotion intensity values.

Due to its intended use in game development, \progname{} design is unconcerned
with strict adherence to ``realistic'' emotion generation. This means that its
usage scope is limited to the scope of games, which have been used for
pedagogical purposes. It is left to the game designers/developers to determine
if the level of realism provided by \progname{} is appropriate for their
application.