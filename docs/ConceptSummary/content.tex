\section{\progname{} Concept Summary}\label{appendix:concept}
\progname{} is a Computational Model of Emotion (CME) for Non-Player Characters
(NPCs) to enhance their believability, with the goal of improving long-term
player engagement. \progname{} is for \textit{emotion generation}, accepting
user-defined information from a game environment to determines what emotion and
intensity a NPC is ``experiencing''. How the emotion is expressed and what
other effects it could have on game entities is left for game
designers/developers to decide.

You can find all Software Development Artifacts (SDAs) related to \progname{}
at:
\begin{center}
    \url{https://github.com/GenevaS/EMgine}
\end{center}

\paragraph{Design Goals} \progname{} aims to provide a feasible and easy-to-use
method for game designers/developers to include emotion in their NPCs, which
they perceive to be challenging with the current tools and
restrictions~\citep{broekens2016emotional}. \progname{} should:
\begin{itemize}

    \item Be modular and portable such that game designers/developers can use
    it in their regular development environment

    \item Be flexible to allow game designers/developers the freedom to choose
    when, where, and which of \progname{}'s components to use

    \item Not require knowledge of affective science, psychology, and/or
    emotion theories to use

    \item Be efficient with respect to computational resources

    \item Demonstrate that it positively impacts the player experience

\end{itemize}

\paragraph{Background: Computational Model of Emotion} A \textit{Computational
Model of Emotion (CME)} is a software system that is influenced by emotion
research, embodying at least one emotion theory as the basis for its stimuli
evaluation, emotion elicitation, and emotional behaviour generation
mechanisms~\citep[p.~2, 14]{osuna2020seperspective}. This theoretical
foundation helps define a CME's mechanisms, components, phases, and
architecture which software engineering techniques and methods can
implement~\citep[p.~139]{osuna2021toward}.

There are two types of CME, differing in requirements and validation:
\textit{research-oriented} and \textit{domain-specific} or \textit{applied}
(\citepg{hudlicka2019modeling}{130--131};
\citepg{osuna2020seperspective}{4--6}). Research-oriented models emulate
structures, processes, and mechanisms in order to understand their design and
structure in biological agents. In software engineering terms,
research-oriented systems are white-box models because they must have
explainable behaviours and mechanisms that lead to affective phenomena. In
contrast, domain-specific models aim to produce specific aspects of affective
phenomena and need only mimic the processes that produce them. They are
black-box models, where the transformation of inputs into affective phenomena
is not important, as long as it has the desired effects. Generally, the degree
of realism in the CME's mechanisms and behaviours is proportional to its design
complexity~\citep[p.~83]{rosis2003from}. This implies that domain-specific CMEs
are unlikely to be exactly alike, even if they target the same domain. This
also means that any emotion theory that satisfies the CME's needs is a viable
choice.

An emotion generation engine that strives to create a more engaging player
experience via believable NPCs is a \textit{domain-specific} CME. This imposes
fewer design constraints than a research-oriented system as it is not a strict
model of affective phenomena~\citep[p.~233]{sloman2005architectural}.
Validation also differs from research-oriented systems, as it is defined by how
well it engages a player rather than how closely it resembles true affective
phenomena~\citep[p.~131]{hudlicka2019modeling}. This bodes well for converting
informal emotion theories into the formal domain of software, which necessarily
includes making assumptions and design decisions that existing research might
not support (\citepg{marsella2010computational}{21, 23};
\citepg{hudlicka2019modeling}{130}). A software engineering approach would
systematically account for these factors, producing a well-designed CME.

\paragraph{Background: Player Engagement} \textit{Engagement} is a quality of
the user experience describing a positive human-computer
interaction~\citep[p.~1094]{o2013examining}. It is difficult to define, as
people use it interchangeably with related concepts like flow and immersion
(\citepg{brockmyer2009development}{624}; \citepg{turner2014figure}{33};
\citepg{glas2015definitions}{944}; \citepg{cairns2016engagement}{81};
\citepg{doherty2018engagement}{99:4}). It can also have a subtly different
meaning in a given context. This makes sense given that it overlaps with
different elements of each concept, such as attention.

Engagement is a multidimensional concept encompassing cognitive, emotional, and
behavioural influences measuring the user's temporal, cognitive, or emotional
investment in the interaction (\citepg{o2010development}{62--63};
\citepg{o2016theoretical}{22}). It is not an ``all or nothing'' quality, and
can fluctuate over the course of the
interaction~\citep[p.~19]{o2016theoretical}. Engagement is both a process and
product of an interaction, existing both during and after human-computer
interactions (\citepg{o2016theoretical}{22}). This dual nature of engagement
results in a cyclical relationship---the process of engagement can lead to the
product of engagement, which in turn influences the likelihood that the user
will re-enter the process later~\citep[p.~945]{o2008user}.

Engagement is difficult to measure due to its subjectivity and dual
process-product nature, but it has common elements that makes it possible to
compare experiences between users~\citep[p.~41]{calvillo2015assessing}. These
are broadly categorized as aesthetics, perceived usability, focused attention
and reward.

Engagement with the game is a fundamental goal of games, having a key role
in player satisfa\-/ction---``the degree to which the player feels gratified
with his or her experience while playing a video
game''~\citep[p.~1220]{phan2016development}. Players have a disposition towards
being, and expect to be, engaged when they
play~\citep[p.~84]{cairns2016engagement}. This could be because playing a game
is a voluntary activity done for pleasure (\citepg{poels2007always}{86--87};
\citepg{yannakakis2015emotion}{459}) in which they are an active
participant~\citep[p.~94]{mayra2007fundamental}. Emotional engagement is one
way to engage players, caused by the player's personal feelings aroused by an
in-game event, character, asset attributes, or another player which causes them
to want to continue playing~\citep[p.~406--407]{schonau2012sure}.

\paragraph{Background: Believable Character} A \textit{believable character}
``...allows the audience to suspend their disbelief and...provides a convincing
portrayal of the personality they expect or come to expect [from the
character]''~\cite[p.~1]{loyall1997believable}. Believability is not limited to
``smart'' or ``normal'' characters because it depends on the situational
context and the character's personality~\citep{lisetti2015and,
loyall1997believable, reilly1996believable}. In short: an NPC must behave
reasonably within the context of the game world. Generally, NPCs are believable
when they~\citep{lankoski2007gameplay, loyall1997believable,
warpefelt2013analyzing}:
\begin{itemize}
    \item Appear to be self-motivated,

    \item Are aware of what is happening around them, and

    \item React in ways appropriate for their surrounding context while
    adhering to their personality.
\end{itemize}

\textit{Emotion}, transient responses to changes in the
environment~\citep{lazarus1991emotion}, is one element of believable character
design~\citep{de2015beyond, emmerich2018m, gard2000building,
    lankoski2007gameplay, lisetti2015and, loyall1997believable,
    paiva2005learning,
    warpefelt2013analyzing}. It is an established aspect of believability in
animation~\citep{thomas1981illusion}, and game designers have acknowledged its
importance in NPC design~\citep{hudlicka2009foundations, yannakakis2015emotion}.

Characters with emotion address the core features of believability because they
convey a character's goals and desires (\textit{self-motivated}) by showing
their \textit{awareness} of, \textit{responsiveness} to, and care
(\textit{personality}-driven) for their surroundings~\citep{bates1994,
    broekens2021emotion, reilly1996believable}. It follows that one way to
    improve
an NPC's believability is to have them react emotionally to their
surroundings~\citep{togelius2013assessing, yannakakis2015emotion}.